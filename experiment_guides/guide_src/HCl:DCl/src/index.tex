\maketitle% this prints the handout title, author, and date

\begin{abstract}
\noindent
This experiment is concerned with the rotational fine structure of the infrared vibrational spectrum of a linear molecule such as \ch{HCl}. By interpreting the details of this spectrum, it is possible to obtain the moment of inertia of the molecule and thus the internuclear separation. In addition, the pure vibrational frequency determines the force constant that is a measure of the bond strength. By also investigating \ch{DCl}, the isotope effect can be observed.\thanks{Transcribed (with corrections) from \textcite{nibler14}.}
\end{abstract}

\section{Introduction} % (fold)
\label{sec:intro}

The infrared region of the spectrum extends from the long-wavelength end of the visible region near \qty{1}{\um} out to the microwave region at approximately \qty{1000}{\um}.
It is common practice to specify infrared frequencies in wavenumber units of ``inverse centimeters'', \unit{\wn}. 
The wavenumber of a photon, \( \widetilde{\nu} \), is calculated as: 
\begin{equation}
  \widetilde{\nu} = 1/\lambda = \nu/c \, ,
\end{equation} 
where \( c \) is the speed of light in \unit{\cm \per \s}, \( \nu \) is the frequency in \unit{\Hz}, and \( \lambda \) is the wavelength in \unit{\cm}.
Thus, this region extends from \qty{10000}{\wn} down to \qty{10}{\wn}. 
Although considerable work is now being done in the far-infrared region (below \qty{400}{\wn}), the spectral range from \SIrange{4000}{400}{\wn} has received the greatest attention because the vibrational frequencies of most molecules lie in this region. 

% section intro (end)

\section{Theory} % (fold)
\label{sec:theory}

Almost all infrared work makes use of absorption techniques in which radiation from a source emitting broadband infrared radiation is passed through a sample of the material to be studied. 
When the frequency of this radiation is the same as the vibrational frequency of the molecule, the molecule may be vibrationally excited.
This results in loss of energy from the radiation at that frequency and gives rise to an absorption band. 
The spectrum of a polyatomic molecule generally consists of several such bands arising from different vibrational motions of the molecule. 
This experiment involves diatomic molecules, which have only one vibrational mode. 

The simplest model of a \emph{vibrating} diatomic molecule is a harmonic oscillator, for which the potential energy depends quadratically on the change in internuclear distance. 
The allowed energy levels of a harmonic oscillator, as calculated from quantum mechanics~\autocite{atkins94}, are
\begin{equation}
	E\br{v} = h \nu \br{v + \tfrac{1}{2}} \, ,
	\label{eq:ho_energies}
\end{equation}
where \( v \) is the vibrational quantum number (having integer values \( 0, 1, 2, \ldots \)), \( \nu \) is the vibrational frequency in \unit{\Hz}, and \( h \) is the Planck constant, \qty{6.626e-34}{\J\s}. 

The simplest model of a \emph{rotating} diatomic molecule is a rigid rotor, or ``dumbell'', model in which the two atoms of mass \( m_1 \) and \( m_2 \) are considered to be joined by a rigid, weightless rod. 
The allowed energy levels for a rigid rotor may be shown by quantum mechanics~\autocite{atkins94} to be
\begin{equation}
	E\br{J} = \frac{h^2}{8 \pi^2 I} J \br{J + 1} \, ,
	\label{eq:vib_energies}
\end{equation}
where the rotational quantum number, \( J \), again takes integer values \( 0, 1, 2, \ldots \).
The quantity \( I \) is the moment of inertia, which is related to the internuclear distance, \( r \), and the reduced mass, \( \mu \),\sidenote{
Recall that \( \mu \) is defined by the equation \( \mu^{-1} = m_1^{-1} + m_2^{-1} = \br{m_1 m_2}/\br{m_1 + m_2} \).
} by
\begin{equation}
	I = \mu r^2 \, .
	\label{eq:mom_of_intert}
\end{equation}

Since a real molecule is undergoing both rotation and vibration simultaneously, a first approximation to its energy levels, \( E\br{v, J} \) would be the sum of \cref{eq:ho_energies,eq:vib_energies}. 
A more complete expression for the energy levels of a diatomic molecule is given below~\autocite{herzberg89,kerr82}, with the levels expressed as \emph{term values}, \( T \), in frequency units of \unit{\wn} rather than as energy values, \( E \), in joules (\unit{\J}):
\begin{equation}
	\begin{aligned}
	T\br{v, J} =\,& \frac{E\br{v, J}}{h c} = \\
		& \quad \widetilde{\nu}_e\br{v + \tfrac{1}{2}} - \widetilde{\nu}_e \chi_e\br{v + \tfrac{1}{2}}^2 + B_e J \br{J + 1} \\
			& \quad - D_J J^2 \br{J + 1}^2 - \alpha_e \br{v + \tfrac{1}{2}} J \br{J + 1} \, ,
	\end{aligned}
	\label{eq:vib_rot_energies}
\end{equation}
where \( c \) is the speed of light in \unit{\cm \per \s}, \( \widetilde{\nu}_e \) is the frequency in \unit{\wn} for the molecule vibrating about its equilibrium internuclear separation, \( r_e \), in \unit{\angstrom}, and the equilibrium rotation frequency is defined as
\begin{equation}
	B_e = \frac{h}{8 \pi^2 I_e c} \, .
	\label{eq:b_e}
\end{equation}

The first and third terms on the right-hand side of \cref{eq:vib_rot_energies} are the harmonic-oscillator and rigid-rotor terms with \( r = r_e \).
The second term (containing the \emph{anharmonicity constant}, \( \nu_e\chi_e \)) takes into account the effect of anharmonicity on the vibrational levels. 
Since the real potential, \( U\br{r} \), for a molecule differs from a harmonic potential, \( U_\mtext{HO} \), the real vibrational levels are not quite those given by \cref{eq:ho_energies} and a correction term is required.\sidenote{See \cref{fig:potential_curves}} 
The fourth term (involving the \emph{centrifugal distortion constant}, \( D_J \)) accounts for the effects of centrifugal stretching on the rotational levels. 
Since a chemical bond is not truly rigid but more like a stiff spring, it stretches somewhat when the molecule rotates. 
Such an effect is only important for high \( J \) values, since the constant \( D_J \) is usually very small. 
The last term in \cref{eq:vib_rot_energies}, involving \( \alpha_e \), accounts for interactions between vibration and rotation. 
During a vibration, the internuclear distance \( r \) changes; this changes the moment of inertia and affects the rotation of the molecule. 
The constant \( \alpha_e \) is also quite small, but this term should not be neglected. 

\begin{figure}[tb]
  \centering
    \includegraphics[width=0.95\textwidth]{images/potentials.tikz}
  \caption{Schematic diagram showing potential energy curves as a function of internuclear separation, \( r \), for a diatomic molecule.
		The curve for a harmonic oscillator is depicted in red while the Morse potential is drawn in blue.
		Vibrational levels are shown for each.
    Note that \( 0 \) on the \emph{y}-axis corresponds to the dissociated system, not any physical representation of ``zero energy.''
    This is important when considering your computational results later. 
	}
  \label{fig:potential_curves}
\end{figure}

\subsection{Selection Rules} % (fold)
\label{sub:selection_rules}

The harmonic-oscillator, rigid-rotor selection rules are \( \increment v = \pm 1 \) and \( \increment J = \pm 1\); that is, infrared emission or absorption can occur only when these ``allowed'' transitions take place~\autocite{herzberg89}.
For an \emph{anharmonic} diatomic molecule, the \( \increment J = \pm 1 \) selection rule is still valid, but weak transitions corresponding to \( \increment v = \pm2, \pm3, \) etc. (overtones) can now be observed~\autocite{herzberg89}. 
Since we are interested in the most intense absorption band (the ``fundamental''), we are concerned with transitions from various \( J'' \) levels of the vibrational ground state \( \br{v'' = 0} \) to \( J' \) levels in the first excited vibrational state \( \br{v' = 1} \). 
From the still-valid selection rule, we know that the transition must be from \( J'' \) to \( J' = J'' \pm 1 \). 
Since \( \increment E = h \nu = h c \widetilde{\nu} \), the frequency \( \widetilde{\nu} \) (in wavenumbers) for this transition will be \( \br{T\br{v', J'} - T\br{v'', J''}} \). Applying the selection rule, we find from \cref{eq:vib_rot_energies} that
\begin{align}
	\widetilde{\nu}_R &= \widetilde{\nu}_0 + \br{2 B_e - 3 \alpha_e} + \br{2 B_e - 4 \alpha_e} J'' - \alpha_e J''^2 	&	J'' &= 0, 1, 2, \ldots 
	\label{eq:r_branch} 
	\\
	\widetilde{\nu}_P &= \widetilde{\nu}_0 -													 \br{2 B_e - 2 \alpha_e} J'' - \alpha_e J''^2 	&	J'' &= 0, 1, 2, \ldots 
	\label{eq:p_branch}
\end{align}
for \( \increment J = +1 \) and \( \increment J = -1 \), respectively, where the \( D_J \) term has been omitted and \( \widetilde{\nu}_0 \), the frequency of the \emph{forbidden} transition from \( {\br{v'',J''} = \br{0, 0}} \) to \( {v' = 1} \), \( {J' = 0} \) is defined as
\begin{equation}
	\widetilde{\nu}_0 = \widetilde{\nu}_e - 2 \widetilde{\nu}_e \chi_e \, .
	\label{eq:forbid_transition}
\end{equation}

\begin{figure}[tb]
  \centering
    %% Creator: Matplotlib, PGF backend
%%
%% To include the figure in your LaTeX document, write
%%   \input{<filename>.pgf}
%%
%% Make sure the required packages are loaded in your preamble
%%   \usepackage{pgf}
%%
%% and, on pdftex
%%   \usepackage[utf8]{inputenc}\DeclareUnicodeCharacter{2212}{-}
%%
%% or, on luatex and xetex
%%   \usepackage{unicode-math}
%%
%% Figures using additional raster images can only be included by \input if
%% they are in the same directory as the main LaTeX file. For loading figures
%% from other directories you can use the `import` package
%%   \usepackage{import}
%%
%% and then include the figures with
%%   \import{<path to file>}{<filename>.pgf}
%%
%% Matplotlib used the following preamble
%%
\begingroup%
\makeatletter%
\begin{pgfpicture}%
\pgfpathrectangle{\pgfpointorigin}{\pgfqpoint{3.936625in}{4.918380in}}%
\pgfusepath{use as bounding box, clip}%
\begin{pgfscope}%
\pgfsetbuttcap%
\pgfsetmiterjoin%
\definecolor{currentfill}{rgb}{1.000000,1.000000,1.000000}%
\pgfsetfillcolor{currentfill}%
\pgfsetlinewidth{0.000000pt}%
\definecolor{currentstroke}{rgb}{1.000000,1.000000,1.000000}%
\pgfsetstrokecolor{currentstroke}%
\pgfsetdash{}{0pt}%
\pgfpathmoveto{\pgfqpoint{0.000000in}{0.000000in}}%
\pgfpathlineto{\pgfqpoint{3.936625in}{0.000000in}}%
\pgfpathlineto{\pgfqpoint{3.936625in}{4.918380in}}%
\pgfpathlineto{\pgfqpoint{0.000000in}{4.918380in}}%
\pgfpathclose%
\pgfusepath{fill}%
\end{pgfscope}%
\begin{pgfscope}%
\pgfsetbuttcap%
\pgfsetmiterjoin%
\definecolor{currentfill}{rgb}{1.000000,1.000000,1.000000}%
\pgfsetfillcolor{currentfill}%
\pgfsetlinewidth{0.000000pt}%
\definecolor{currentstroke}{rgb}{0.000000,0.000000,0.000000}%
\pgfsetstrokecolor{currentstroke}%
\pgfsetstrokeopacity{0.000000}%
\pgfsetdash{}{0pt}%
\pgfpathmoveto{\pgfqpoint{0.689290in}{3.154025in}}%
\pgfpathlineto{\pgfqpoint{3.755070in}{3.154025in}}%
\pgfpathlineto{\pgfqpoint{3.755070in}{4.818380in}}%
\pgfpathlineto{\pgfqpoint{0.689290in}{4.818380in}}%
\pgfpathclose%
\pgfusepath{fill}%
\end{pgfscope}%
\begin{pgfscope}%
\pgfpathrectangle{\pgfqpoint{0.689290in}{3.154025in}}{\pgfqpoint{3.065780in}{1.664355in}}%
\pgfusepath{clip}%
\pgfsetbuttcap%
\pgfsetroundjoin%
\pgfsetlinewidth{0.501875pt}%
\definecolor{currentstroke}{rgb}{0.700000,0.700000,0.700000}%
\pgfsetstrokecolor{currentstroke}%
\pgfsetstrokeopacity{0.200000}%
\pgfsetdash{{1.850000pt}{0.800000pt}}{0.000000pt}%
\pgfpathmoveto{\pgfqpoint{1.109402in}{3.154025in}}%
\pgfpathlineto{\pgfqpoint{1.109402in}{4.818380in}}%
\pgfusepath{stroke}%
\end{pgfscope}%
\begin{pgfscope}%
\pgfpathrectangle{\pgfqpoint{0.689290in}{3.154025in}}{\pgfqpoint{3.065780in}{1.664355in}}%
\pgfusepath{clip}%
\pgfsetbuttcap%
\pgfsetroundjoin%
\pgfsetlinewidth{0.501875pt}%
\definecolor{currentstroke}{rgb}{0.700000,0.700000,0.700000}%
\pgfsetstrokecolor{currentstroke}%
\pgfsetstrokeopacity{0.200000}%
\pgfsetdash{{1.850000pt}{0.800000pt}}{0.000000pt}%
\pgfpathmoveto{\pgfqpoint{1.765745in}{3.154025in}}%
\pgfpathlineto{\pgfqpoint{1.765745in}{4.818380in}}%
\pgfusepath{stroke}%
\end{pgfscope}%
\begin{pgfscope}%
\pgfpathrectangle{\pgfqpoint{0.689290in}{3.154025in}}{\pgfqpoint{3.065780in}{1.664355in}}%
\pgfusepath{clip}%
\pgfsetbuttcap%
\pgfsetroundjoin%
\pgfsetlinewidth{0.501875pt}%
\definecolor{currentstroke}{rgb}{0.700000,0.700000,0.700000}%
\pgfsetstrokecolor{currentstroke}%
\pgfsetstrokeopacity{0.200000}%
\pgfsetdash{{1.850000pt}{0.800000pt}}{0.000000pt}%
\pgfpathmoveto{\pgfqpoint{2.422088in}{3.154025in}}%
\pgfpathlineto{\pgfqpoint{2.422088in}{4.818380in}}%
\pgfusepath{stroke}%
\end{pgfscope}%
\begin{pgfscope}%
\pgfpathrectangle{\pgfqpoint{0.689290in}{3.154025in}}{\pgfqpoint{3.065780in}{1.664355in}}%
\pgfusepath{clip}%
\pgfsetbuttcap%
\pgfsetroundjoin%
\pgfsetlinewidth{0.501875pt}%
\definecolor{currentstroke}{rgb}{0.700000,0.700000,0.700000}%
\pgfsetstrokecolor{currentstroke}%
\pgfsetstrokeopacity{0.200000}%
\pgfsetdash{{1.850000pt}{0.800000pt}}{0.000000pt}%
\pgfpathmoveto{\pgfqpoint{3.078432in}{3.154025in}}%
\pgfpathlineto{\pgfqpoint{3.078432in}{4.818380in}}%
\pgfusepath{stroke}%
\end{pgfscope}%
\begin{pgfscope}%
\pgfpathrectangle{\pgfqpoint{0.689290in}{3.154025in}}{\pgfqpoint{3.065780in}{1.664355in}}%
\pgfusepath{clip}%
\pgfsetbuttcap%
\pgfsetroundjoin%
\pgfsetlinewidth{0.501875pt}%
\definecolor{currentstroke}{rgb}{0.700000,0.700000,0.700000}%
\pgfsetstrokecolor{currentstroke}%
\pgfsetstrokeopacity{0.200000}%
\pgfsetdash{{1.850000pt}{0.800000pt}}{0.000000pt}%
\pgfpathmoveto{\pgfqpoint{3.734775in}{3.154025in}}%
\pgfpathlineto{\pgfqpoint{3.734775in}{4.818380in}}%
\pgfusepath{stroke}%
\end{pgfscope}%
\begin{pgfscope}%
\pgfpathrectangle{\pgfqpoint{0.689290in}{3.154025in}}{\pgfqpoint{3.065780in}{1.664355in}}%
\pgfusepath{clip}%
\pgfsetbuttcap%
\pgfsetroundjoin%
\pgfsetlinewidth{0.501875pt}%
\definecolor{currentstroke}{rgb}{0.700000,0.700000,0.700000}%
\pgfsetstrokecolor{currentstroke}%
\pgfsetstrokeopacity{0.200000}%
\pgfsetdash{{1.850000pt}{0.800000pt}}{0.000000pt}%
\pgfpathmoveto{\pgfqpoint{0.689290in}{3.301391in}}%
\pgfpathlineto{\pgfqpoint{3.755070in}{3.301391in}}%
\pgfusepath{stroke}%
\end{pgfscope}%
\begin{pgfscope}%
\pgfsetbuttcap%
\pgfsetroundjoin%
\definecolor{currentfill}{rgb}{0.000000,0.000000,0.000000}%
\pgfsetfillcolor{currentfill}%
\pgfsetlinewidth{1.003750pt}%
\definecolor{currentstroke}{rgb}{0.000000,0.000000,0.000000}%
\pgfsetstrokecolor{currentstroke}%
\pgfsetdash{}{0pt}%
\pgfsys@defobject{currentmarker}{\pgfqpoint{-0.083333in}{0.000000in}}{\pgfqpoint{0.000000in}{0.000000in}}{%
\pgfpathmoveto{\pgfqpoint{0.000000in}{0.000000in}}%
\pgfpathlineto{\pgfqpoint{-0.083333in}{0.000000in}}%
\pgfusepath{stroke,fill}%
}%
\begin{pgfscope}%
\pgfsys@transformshift{0.689290in}{3.301391in}%
\pgfsys@useobject{currentmarker}{}%
\end{pgfscope}%
\end{pgfscope}%
\begin{pgfscope}%
\pgfpathrectangle{\pgfqpoint{0.689290in}{3.154025in}}{\pgfqpoint{3.065780in}{1.664355in}}%
\pgfusepath{clip}%
\pgfsetbuttcap%
\pgfsetroundjoin%
\pgfsetlinewidth{0.501875pt}%
\definecolor{currentstroke}{rgb}{0.700000,0.700000,0.700000}%
\pgfsetstrokecolor{currentstroke}%
\pgfsetstrokeopacity{0.200000}%
\pgfsetdash{{1.850000pt}{0.800000pt}}{0.000000pt}%
\pgfpathmoveto{\pgfqpoint{0.689290in}{3.333555in}}%
\pgfpathlineto{\pgfqpoint{3.755070in}{3.333555in}}%
\pgfusepath{stroke}%
\end{pgfscope}%
\begin{pgfscope}%
\pgfsetbuttcap%
\pgfsetroundjoin%
\definecolor{currentfill}{rgb}{0.000000,0.000000,0.000000}%
\pgfsetfillcolor{currentfill}%
\pgfsetlinewidth{1.003750pt}%
\definecolor{currentstroke}{rgb}{0.000000,0.000000,0.000000}%
\pgfsetstrokecolor{currentstroke}%
\pgfsetdash{}{0pt}%
\pgfsys@defobject{currentmarker}{\pgfqpoint{-0.083333in}{0.000000in}}{\pgfqpoint{0.000000in}{0.000000in}}{%
\pgfpathmoveto{\pgfqpoint{0.000000in}{0.000000in}}%
\pgfpathlineto{\pgfqpoint{-0.083333in}{0.000000in}}%
\pgfusepath{stroke,fill}%
}%
\begin{pgfscope}%
\pgfsys@transformshift{0.689290in}{3.333555in}%
\pgfsys@useobject{currentmarker}{}%
\end{pgfscope}%
\end{pgfscope}%
\begin{pgfscope}%
\definecolor{textcolor}{rgb}{0.000000,0.000000,0.000000}%
\pgfsetstrokecolor{textcolor}%
\pgfsetfillcolor{textcolor}%
\pgftext[x=0.345158in, y=3.304620in, left, base]{\color{textcolor}\rmfamily\fontsize{6.000000}{7.200000}\selectfont 0.544}%
\end{pgfscope}%
\begin{pgfscope}%
\pgfpathrectangle{\pgfqpoint{0.689290in}{3.154025in}}{\pgfqpoint{3.065780in}{1.664355in}}%
\pgfusepath{clip}%
\pgfsetbuttcap%
\pgfsetroundjoin%
\pgfsetlinewidth{0.501875pt}%
\definecolor{currentstroke}{rgb}{0.700000,0.700000,0.700000}%
\pgfsetstrokecolor{currentstroke}%
\pgfsetstrokeopacity{0.200000}%
\pgfsetdash{{1.850000pt}{0.800000pt}}{0.000000pt}%
\pgfpathmoveto{\pgfqpoint{0.689290in}{3.397864in}}%
\pgfpathlineto{\pgfqpoint{3.755070in}{3.397864in}}%
\pgfusepath{stroke}%
\end{pgfscope}%
\begin{pgfscope}%
\pgfsetbuttcap%
\pgfsetroundjoin%
\definecolor{currentfill}{rgb}{0.000000,0.000000,0.000000}%
\pgfsetfillcolor{currentfill}%
\pgfsetlinewidth{1.003750pt}%
\definecolor{currentstroke}{rgb}{0.000000,0.000000,0.000000}%
\pgfsetstrokecolor{currentstroke}%
\pgfsetdash{}{0pt}%
\pgfsys@defobject{currentmarker}{\pgfqpoint{-0.083333in}{0.000000in}}{\pgfqpoint{0.000000in}{0.000000in}}{%
\pgfpathmoveto{\pgfqpoint{0.000000in}{0.000000in}}%
\pgfpathlineto{\pgfqpoint{-0.083333in}{0.000000in}}%
\pgfusepath{stroke,fill}%
}%
\begin{pgfscope}%
\pgfsys@transformshift{0.689290in}{3.397864in}%
\pgfsys@useobject{currentmarker}{}%
\end{pgfscope}%
\end{pgfscope}%
\begin{pgfscope}%
\definecolor{textcolor}{rgb}{0.000000,0.000000,0.000000}%
\pgfsetstrokecolor{textcolor}%
\pgfsetfillcolor{textcolor}%
\pgftext[x=0.345158in, y=3.368929in, left, base]{\color{textcolor}\rmfamily\fontsize{6.000000}{7.200000}\selectfont 0.549}%
\end{pgfscope}%
\begin{pgfscope}%
\pgfpathrectangle{\pgfqpoint{0.689290in}{3.154025in}}{\pgfqpoint{3.065780in}{1.664355in}}%
\pgfusepath{clip}%
\pgfsetbuttcap%
\pgfsetroundjoin%
\pgfsetlinewidth{0.501875pt}%
\definecolor{currentstroke}{rgb}{0.700000,0.700000,0.700000}%
\pgfsetstrokecolor{currentstroke}%
\pgfsetstrokeopacity{0.200000}%
\pgfsetdash{{1.850000pt}{0.800000pt}}{0.000000pt}%
\pgfpathmoveto{\pgfqpoint{0.689290in}{3.494276in}}%
\pgfpathlineto{\pgfqpoint{3.755070in}{3.494276in}}%
\pgfusepath{stroke}%
\end{pgfscope}%
\begin{pgfscope}%
\pgfsetbuttcap%
\pgfsetroundjoin%
\definecolor{currentfill}{rgb}{0.000000,0.000000,0.000000}%
\pgfsetfillcolor{currentfill}%
\pgfsetlinewidth{1.003750pt}%
\definecolor{currentstroke}{rgb}{0.000000,0.000000,0.000000}%
\pgfsetstrokecolor{currentstroke}%
\pgfsetdash{}{0pt}%
\pgfsys@defobject{currentmarker}{\pgfqpoint{-0.083333in}{0.000000in}}{\pgfqpoint{0.000000in}{0.000000in}}{%
\pgfpathmoveto{\pgfqpoint{0.000000in}{0.000000in}}%
\pgfpathlineto{\pgfqpoint{-0.083333in}{0.000000in}}%
\pgfusepath{stroke,fill}%
}%
\begin{pgfscope}%
\pgfsys@transformshift{0.689290in}{3.494276in}%
\pgfsys@useobject{currentmarker}{}%
\end{pgfscope}%
\end{pgfscope}%
\begin{pgfscope}%
\definecolor{textcolor}{rgb}{0.000000,0.000000,0.000000}%
\pgfsetstrokecolor{textcolor}%
\pgfsetfillcolor{textcolor}%
\pgftext[x=0.345158in, y=3.465341in, left, base]{\color{textcolor}\rmfamily\fontsize{6.000000}{7.200000}\selectfont 0.557}%
\end{pgfscope}%
\begin{pgfscope}%
\pgfpathrectangle{\pgfqpoint{0.689290in}{3.154025in}}{\pgfqpoint{3.065780in}{1.664355in}}%
\pgfusepath{clip}%
\pgfsetbuttcap%
\pgfsetroundjoin%
\pgfsetlinewidth{0.501875pt}%
\definecolor{currentstroke}{rgb}{0.700000,0.700000,0.700000}%
\pgfsetstrokecolor{currentstroke}%
\pgfsetstrokeopacity{0.200000}%
\pgfsetdash{{1.850000pt}{0.800000pt}}{0.000000pt}%
\pgfpathmoveto{\pgfqpoint{0.689290in}{3.622732in}}%
\pgfpathlineto{\pgfqpoint{3.755070in}{3.622732in}}%
\pgfusepath{stroke}%
\end{pgfscope}%
\begin{pgfscope}%
\pgfsetbuttcap%
\pgfsetroundjoin%
\definecolor{currentfill}{rgb}{0.000000,0.000000,0.000000}%
\pgfsetfillcolor{currentfill}%
\pgfsetlinewidth{1.003750pt}%
\definecolor{currentstroke}{rgb}{0.000000,0.000000,0.000000}%
\pgfsetstrokecolor{currentstroke}%
\pgfsetdash{}{0pt}%
\pgfsys@defobject{currentmarker}{\pgfqpoint{-0.083333in}{0.000000in}}{\pgfqpoint{0.000000in}{0.000000in}}{%
\pgfpathmoveto{\pgfqpoint{0.000000in}{0.000000in}}%
\pgfpathlineto{\pgfqpoint{-0.083333in}{0.000000in}}%
\pgfusepath{stroke,fill}%
}%
\begin{pgfscope}%
\pgfsys@transformshift{0.689290in}{3.622732in}%
\pgfsys@useobject{currentmarker}{}%
\end{pgfscope}%
\end{pgfscope}%
\begin{pgfscope}%
\definecolor{textcolor}{rgb}{0.000000,0.000000,0.000000}%
\pgfsetstrokecolor{textcolor}%
\pgfsetfillcolor{textcolor}%
\pgftext[x=0.345158in, y=3.593797in, left, base]{\color{textcolor}\rmfamily\fontsize{6.000000}{7.200000}\selectfont 0.567}%
\end{pgfscope}%
\begin{pgfscope}%
\pgfpathrectangle{\pgfqpoint{0.689290in}{3.154025in}}{\pgfqpoint{3.065780in}{1.664355in}}%
\pgfusepath{clip}%
\pgfsetbuttcap%
\pgfsetroundjoin%
\pgfsetlinewidth{0.501875pt}%
\definecolor{currentstroke}{rgb}{0.700000,0.700000,0.700000}%
\pgfsetstrokecolor{currentstroke}%
\pgfsetstrokeopacity{0.200000}%
\pgfsetdash{{1.850000pt}{0.800000pt}}{0.000000pt}%
\pgfpathmoveto{\pgfqpoint{0.689290in}{3.783149in}}%
\pgfpathlineto{\pgfqpoint{3.755070in}{3.783149in}}%
\pgfusepath{stroke}%
\end{pgfscope}%
\begin{pgfscope}%
\pgfsetbuttcap%
\pgfsetroundjoin%
\definecolor{currentfill}{rgb}{0.000000,0.000000,0.000000}%
\pgfsetfillcolor{currentfill}%
\pgfsetlinewidth{1.003750pt}%
\definecolor{currentstroke}{rgb}{0.000000,0.000000,0.000000}%
\pgfsetstrokecolor{currentstroke}%
\pgfsetdash{}{0pt}%
\pgfsys@defobject{currentmarker}{\pgfqpoint{-0.083333in}{0.000000in}}{\pgfqpoint{0.000000in}{0.000000in}}{%
\pgfpathmoveto{\pgfqpoint{0.000000in}{0.000000in}}%
\pgfpathlineto{\pgfqpoint{-0.083333in}{0.000000in}}%
\pgfusepath{stroke,fill}%
}%
\begin{pgfscope}%
\pgfsys@transformshift{0.689290in}{3.783149in}%
\pgfsys@useobject{currentmarker}{}%
\end{pgfscope}%
\end{pgfscope}%
\begin{pgfscope}%
\definecolor{textcolor}{rgb}{0.000000,0.000000,0.000000}%
\pgfsetstrokecolor{textcolor}%
\pgfsetfillcolor{textcolor}%
\pgftext[x=0.345158in, y=3.754214in, left, base]{\color{textcolor}\rmfamily\fontsize{6.000000}{7.200000}\selectfont 0.579}%
\end{pgfscope}%
\begin{pgfscope}%
\pgfpathrectangle{\pgfqpoint{0.689290in}{3.154025in}}{\pgfqpoint{3.065780in}{1.664355in}}%
\pgfusepath{clip}%
\pgfsetbuttcap%
\pgfsetroundjoin%
\pgfsetlinewidth{0.501875pt}%
\definecolor{currentstroke}{rgb}{0.700000,0.700000,0.700000}%
\pgfsetstrokecolor{currentstroke}%
\pgfsetstrokeopacity{0.200000}%
\pgfsetdash{{1.850000pt}{0.800000pt}}{0.000000pt}%
\pgfpathmoveto{\pgfqpoint{0.689290in}{3.975427in}}%
\pgfpathlineto{\pgfqpoint{3.755070in}{3.975427in}}%
\pgfusepath{stroke}%
\end{pgfscope}%
\begin{pgfscope}%
\pgfsetbuttcap%
\pgfsetroundjoin%
\definecolor{currentfill}{rgb}{0.000000,0.000000,0.000000}%
\pgfsetfillcolor{currentfill}%
\pgfsetlinewidth{1.003750pt}%
\definecolor{currentstroke}{rgb}{0.000000,0.000000,0.000000}%
\pgfsetstrokecolor{currentstroke}%
\pgfsetdash{}{0pt}%
\pgfsys@defobject{currentmarker}{\pgfqpoint{-0.083333in}{0.000000in}}{\pgfqpoint{0.000000in}{0.000000in}}{%
\pgfpathmoveto{\pgfqpoint{0.000000in}{0.000000in}}%
\pgfpathlineto{\pgfqpoint{-0.083333in}{0.000000in}}%
\pgfusepath{stroke,fill}%
}%
\begin{pgfscope}%
\pgfsys@transformshift{0.689290in}{3.975427in}%
\pgfsys@useobject{currentmarker}{}%
\end{pgfscope}%
\end{pgfscope}%
\begin{pgfscope}%
\definecolor{textcolor}{rgb}{0.000000,0.000000,0.000000}%
\pgfsetstrokecolor{textcolor}%
\pgfsetfillcolor{textcolor}%
\pgftext[x=0.345158in, y=3.946492in, left, base]{\color{textcolor}\rmfamily\fontsize{6.000000}{7.200000}\selectfont 0.594}%
\end{pgfscope}%
\begin{pgfscope}%
\pgfpathrectangle{\pgfqpoint{0.689290in}{3.154025in}}{\pgfqpoint{3.065780in}{1.664355in}}%
\pgfusepath{clip}%
\pgfsetbuttcap%
\pgfsetroundjoin%
\pgfsetlinewidth{0.501875pt}%
\definecolor{currentstroke}{rgb}{0.700000,0.700000,0.700000}%
\pgfsetstrokecolor{currentstroke}%
\pgfsetstrokeopacity{0.200000}%
\pgfsetdash{{1.850000pt}{0.800000pt}}{0.000000pt}%
\pgfpathmoveto{\pgfqpoint{0.689290in}{4.199444in}}%
\pgfpathlineto{\pgfqpoint{3.755070in}{4.199444in}}%
\pgfusepath{stroke}%
\end{pgfscope}%
\begin{pgfscope}%
\pgfsetbuttcap%
\pgfsetroundjoin%
\definecolor{currentfill}{rgb}{0.000000,0.000000,0.000000}%
\pgfsetfillcolor{currentfill}%
\pgfsetlinewidth{1.003750pt}%
\definecolor{currentstroke}{rgb}{0.000000,0.000000,0.000000}%
\pgfsetstrokecolor{currentstroke}%
\pgfsetdash{}{0pt}%
\pgfsys@defobject{currentmarker}{\pgfqpoint{-0.083333in}{0.000000in}}{\pgfqpoint{0.000000in}{0.000000in}}{%
\pgfpathmoveto{\pgfqpoint{0.000000in}{0.000000in}}%
\pgfpathlineto{\pgfqpoint{-0.083333in}{0.000000in}}%
\pgfusepath{stroke,fill}%
}%
\begin{pgfscope}%
\pgfsys@transformshift{0.689290in}{4.199444in}%
\pgfsys@useobject{currentmarker}{}%
\end{pgfscope}%
\end{pgfscope}%
\begin{pgfscope}%
\definecolor{textcolor}{rgb}{0.000000,0.000000,0.000000}%
\pgfsetstrokecolor{textcolor}%
\pgfsetfillcolor{textcolor}%
\pgftext[x=0.345158in, y=4.170509in, left, base]{\color{textcolor}\rmfamily\fontsize{6.000000}{7.200000}\selectfont 0.612}%
\end{pgfscope}%
\begin{pgfscope}%
\pgfpathrectangle{\pgfqpoint{0.689290in}{3.154025in}}{\pgfqpoint{3.065780in}{1.664355in}}%
\pgfusepath{clip}%
\pgfsetbuttcap%
\pgfsetroundjoin%
\pgfsetlinewidth{0.501875pt}%
\definecolor{currentstroke}{rgb}{0.700000,0.700000,0.700000}%
\pgfsetstrokecolor{currentstroke}%
\pgfsetstrokeopacity{0.200000}%
\pgfsetdash{{1.850000pt}{0.800000pt}}{0.000000pt}%
\pgfpathmoveto{\pgfqpoint{0.689290in}{4.455059in}}%
\pgfpathlineto{\pgfqpoint{3.755070in}{4.455059in}}%
\pgfusepath{stroke}%
\end{pgfscope}%
\begin{pgfscope}%
\pgfsetbuttcap%
\pgfsetroundjoin%
\definecolor{currentfill}{rgb}{0.000000,0.000000,0.000000}%
\pgfsetfillcolor{currentfill}%
\pgfsetlinewidth{1.003750pt}%
\definecolor{currentstroke}{rgb}{0.000000,0.000000,0.000000}%
\pgfsetstrokecolor{currentstroke}%
\pgfsetdash{}{0pt}%
\pgfsys@defobject{currentmarker}{\pgfqpoint{-0.083333in}{0.000000in}}{\pgfqpoint{0.000000in}{0.000000in}}{%
\pgfpathmoveto{\pgfqpoint{0.000000in}{0.000000in}}%
\pgfpathlineto{\pgfqpoint{-0.083333in}{0.000000in}}%
\pgfusepath{stroke,fill}%
}%
\begin{pgfscope}%
\pgfsys@transformshift{0.689290in}{4.455059in}%
\pgfsys@useobject{currentmarker}{}%
\end{pgfscope}%
\end{pgfscope}%
\begin{pgfscope}%
\definecolor{textcolor}{rgb}{0.000000,0.000000,0.000000}%
\pgfsetstrokecolor{textcolor}%
\pgfsetfillcolor{textcolor}%
\pgftext[x=0.345158in, y=4.426124in, left, base]{\color{textcolor}\rmfamily\fontsize{6.000000}{7.200000}\selectfont 0.632}%
\end{pgfscope}%
\begin{pgfscope}%
\pgfpathrectangle{\pgfqpoint{0.689290in}{3.154025in}}{\pgfqpoint{3.065780in}{1.664355in}}%
\pgfusepath{clip}%
\pgfsetbuttcap%
\pgfsetroundjoin%
\pgfsetlinewidth{0.501875pt}%
\definecolor{currentstroke}{rgb}{0.700000,0.700000,0.700000}%
\pgfsetstrokecolor{currentstroke}%
\pgfsetstrokeopacity{0.200000}%
\pgfsetdash{{1.850000pt}{0.800000pt}}{0.000000pt}%
\pgfpathmoveto{\pgfqpoint{0.689290in}{4.742109in}}%
\pgfpathlineto{\pgfqpoint{3.755070in}{4.742109in}}%
\pgfusepath{stroke}%
\end{pgfscope}%
\begin{pgfscope}%
\pgfsetbuttcap%
\pgfsetroundjoin%
\definecolor{currentfill}{rgb}{0.000000,0.000000,0.000000}%
\pgfsetfillcolor{currentfill}%
\pgfsetlinewidth{1.003750pt}%
\definecolor{currentstroke}{rgb}{0.000000,0.000000,0.000000}%
\pgfsetstrokecolor{currentstroke}%
\pgfsetdash{}{0pt}%
\pgfsys@defobject{currentmarker}{\pgfqpoint{-0.083333in}{0.000000in}}{\pgfqpoint{0.000000in}{0.000000in}}{%
\pgfpathmoveto{\pgfqpoint{0.000000in}{0.000000in}}%
\pgfpathlineto{\pgfqpoint{-0.083333in}{0.000000in}}%
\pgfusepath{stroke,fill}%
}%
\begin{pgfscope}%
\pgfsys@transformshift{0.689290in}{4.742109in}%
\pgfsys@useobject{currentmarker}{}%
\end{pgfscope}%
\end{pgfscope}%
\begin{pgfscope}%
\definecolor{textcolor}{rgb}{0.000000,0.000000,0.000000}%
\pgfsetstrokecolor{textcolor}%
\pgfsetfillcolor{textcolor}%
\pgftext[x=0.345158in, y=4.713174in, left, base]{\color{textcolor}\rmfamily\fontsize{6.000000}{7.200000}\selectfont 0.654}%
\end{pgfscope}%
\begin{pgfscope}%
\pgfpathrectangle{\pgfqpoint{0.689290in}{3.154025in}}{\pgfqpoint{3.065780in}{1.664355in}}%
\pgfusepath{clip}%
\pgfsetbuttcap%
\pgfsetroundjoin%
\pgfsetlinewidth{1.003750pt}%
\definecolor{currentstroke}{rgb}{0.000000,0.000000,0.000000}%
\pgfsetstrokecolor{currentstroke}%
\pgfsetdash{}{0pt}%
\pgfpathmoveto{\pgfqpoint{0.858842in}{3.301391in}}%
\pgfpathlineto{\pgfqpoint{3.465868in}{3.301391in}}%
\pgfusepath{stroke}%
\end{pgfscope}%
\begin{pgfscope}%
\pgfpathrectangle{\pgfqpoint{0.689290in}{3.154025in}}{\pgfqpoint{3.065780in}{1.664355in}}%
\pgfusepath{clip}%
\pgfsetbuttcap%
\pgfsetroundjoin%
\pgfsetlinewidth{1.003750pt}%
\definecolor{currentstroke}{rgb}{0.000000,0.000000,0.000000}%
\pgfsetstrokecolor{currentstroke}%
\pgfsetdash{}{0pt}%
\pgfpathmoveto{\pgfqpoint{0.858842in}{3.333555in}}%
\pgfpathlineto{\pgfqpoint{3.465868in}{3.333555in}}%
\pgfusepath{stroke}%
\end{pgfscope}%
\begin{pgfscope}%
\pgfpathrectangle{\pgfqpoint{0.689290in}{3.154025in}}{\pgfqpoint{3.065780in}{1.664355in}}%
\pgfusepath{clip}%
\pgfsetbuttcap%
\pgfsetroundjoin%
\pgfsetlinewidth{1.003750pt}%
\definecolor{currentstroke}{rgb}{0.000000,0.000000,0.000000}%
\pgfsetstrokecolor{currentstroke}%
\pgfsetdash{}{0pt}%
\pgfpathmoveto{\pgfqpoint{0.858842in}{3.397864in}}%
\pgfpathlineto{\pgfqpoint{3.465868in}{3.397864in}}%
\pgfusepath{stroke}%
\end{pgfscope}%
\begin{pgfscope}%
\pgfpathrectangle{\pgfqpoint{0.689290in}{3.154025in}}{\pgfqpoint{3.065780in}{1.664355in}}%
\pgfusepath{clip}%
\pgfsetbuttcap%
\pgfsetroundjoin%
\pgfsetlinewidth{1.003750pt}%
\definecolor{currentstroke}{rgb}{0.000000,0.000000,0.000000}%
\pgfsetstrokecolor{currentstroke}%
\pgfsetdash{}{0pt}%
\pgfpathmoveto{\pgfqpoint{0.858842in}{3.494276in}}%
\pgfpathlineto{\pgfqpoint{3.465868in}{3.494276in}}%
\pgfusepath{stroke}%
\end{pgfscope}%
\begin{pgfscope}%
\pgfpathrectangle{\pgfqpoint{0.689290in}{3.154025in}}{\pgfqpoint{3.065780in}{1.664355in}}%
\pgfusepath{clip}%
\pgfsetbuttcap%
\pgfsetroundjoin%
\pgfsetlinewidth{1.003750pt}%
\definecolor{currentstroke}{rgb}{0.000000,0.000000,0.000000}%
\pgfsetstrokecolor{currentstroke}%
\pgfsetdash{}{0pt}%
\pgfpathmoveto{\pgfqpoint{0.858842in}{3.622732in}}%
\pgfpathlineto{\pgfqpoint{3.465868in}{3.622732in}}%
\pgfusepath{stroke}%
\end{pgfscope}%
\begin{pgfscope}%
\pgfpathrectangle{\pgfqpoint{0.689290in}{3.154025in}}{\pgfqpoint{3.065780in}{1.664355in}}%
\pgfusepath{clip}%
\pgfsetbuttcap%
\pgfsetroundjoin%
\pgfsetlinewidth{1.003750pt}%
\definecolor{currentstroke}{rgb}{0.000000,0.000000,0.000000}%
\pgfsetstrokecolor{currentstroke}%
\pgfsetdash{}{0pt}%
\pgfpathmoveto{\pgfqpoint{0.858842in}{3.783149in}}%
\pgfpathlineto{\pgfqpoint{3.465868in}{3.783149in}}%
\pgfusepath{stroke}%
\end{pgfscope}%
\begin{pgfscope}%
\pgfpathrectangle{\pgfqpoint{0.689290in}{3.154025in}}{\pgfqpoint{3.065780in}{1.664355in}}%
\pgfusepath{clip}%
\pgfsetbuttcap%
\pgfsetroundjoin%
\pgfsetlinewidth{1.003750pt}%
\definecolor{currentstroke}{rgb}{0.000000,0.000000,0.000000}%
\pgfsetstrokecolor{currentstroke}%
\pgfsetdash{}{0pt}%
\pgfpathmoveto{\pgfqpoint{0.858842in}{3.975427in}}%
\pgfpathlineto{\pgfqpoint{3.465868in}{3.975427in}}%
\pgfusepath{stroke}%
\end{pgfscope}%
\begin{pgfscope}%
\pgfpathrectangle{\pgfqpoint{0.689290in}{3.154025in}}{\pgfqpoint{3.065780in}{1.664355in}}%
\pgfusepath{clip}%
\pgfsetbuttcap%
\pgfsetroundjoin%
\pgfsetlinewidth{1.003750pt}%
\definecolor{currentstroke}{rgb}{0.000000,0.000000,0.000000}%
\pgfsetstrokecolor{currentstroke}%
\pgfsetdash{}{0pt}%
\pgfpathmoveto{\pgfqpoint{0.858842in}{4.199444in}}%
\pgfpathlineto{\pgfqpoint{3.465868in}{4.199444in}}%
\pgfusepath{stroke}%
\end{pgfscope}%
\begin{pgfscope}%
\pgfpathrectangle{\pgfqpoint{0.689290in}{3.154025in}}{\pgfqpoint{3.065780in}{1.664355in}}%
\pgfusepath{clip}%
\pgfsetbuttcap%
\pgfsetroundjoin%
\pgfsetlinewidth{1.003750pt}%
\definecolor{currentstroke}{rgb}{0.000000,0.000000,0.000000}%
\pgfsetstrokecolor{currentstroke}%
\pgfsetdash{}{0pt}%
\pgfpathmoveto{\pgfqpoint{0.858842in}{4.455059in}}%
\pgfpathlineto{\pgfqpoint{3.465868in}{4.455059in}}%
\pgfusepath{stroke}%
\end{pgfscope}%
\begin{pgfscope}%
\pgfpathrectangle{\pgfqpoint{0.689290in}{3.154025in}}{\pgfqpoint{3.065780in}{1.664355in}}%
\pgfusepath{clip}%
\pgfsetbuttcap%
\pgfsetroundjoin%
\pgfsetlinewidth{1.003750pt}%
\definecolor{currentstroke}{rgb}{0.000000,0.000000,0.000000}%
\pgfsetstrokecolor{currentstroke}%
\pgfsetdash{}{0pt}%
\pgfpathmoveto{\pgfqpoint{0.858842in}{4.742109in}}%
\pgfpathlineto{\pgfqpoint{3.465868in}{4.742109in}}%
\pgfusepath{stroke}%
\end{pgfscope}%
\begin{pgfscope}%
\pgfsetroundcap%
\pgfsetroundjoin%
\pgfsetlinewidth{1.003750pt}%
\definecolor{currentstroke}{rgb}{0.000000,0.000000,0.000000}%
\pgfsetstrokecolor{currentstroke}%
\pgfsetdash{}{0pt}%
\pgfpathmoveto{\pgfqpoint{0.612646in}{3.112416in}}%
\pgfpathlineto{\pgfqpoint{0.765935in}{3.195633in}}%
\pgfusepath{stroke}%
\end{pgfscope}%
\begin{pgfscope}%
\pgfsetrectcap%
\pgfsetmiterjoin%
\pgfsetlinewidth{0.803000pt}%
\definecolor{currentstroke}{rgb}{0.000000,0.000000,0.000000}%
\pgfsetstrokecolor{currentstroke}%
\pgfsetdash{}{0pt}%
\pgfpathmoveto{\pgfqpoint{0.689290in}{3.154025in}}%
\pgfpathlineto{\pgfqpoint{0.689290in}{4.818380in}}%
\pgfusepath{stroke}%
\end{pgfscope}%
\begin{pgfscope}%
\pgfsetroundcap%
\pgfsetroundjoin%
\pgfsetlinewidth{1.003750pt}%
\definecolor{currentstroke}{rgb}{0.000000,0.000000,0.000000}%
\pgfsetstrokecolor{currentstroke}%
\pgfsetdash{}{0pt}%
\pgfpathmoveto{\pgfqpoint{0.532207in}{3.266946in}}%
\pgfpathquadraticcurveto{\pgfqpoint{0.558657in}{3.278579in}}{\pgfqpoint{0.585107in}{3.290211in}}%
\pgfusepath{stroke}%
\end{pgfscope}%
\begin{pgfscope}%
\pgfsetbuttcap%
\pgfsetmiterjoin%
\definecolor{currentfill}{rgb}{1.000000,1.000000,1.000000}%
\pgfsetfillcolor{currentfill}%
\pgfsetlinewidth{1.003750pt}%
\definecolor{currentstroke}{rgb}{1.000000,1.000000,1.000000}%
\pgfsetstrokecolor{currentstroke}%
\pgfsetdash{}{0pt}%
\pgfpathmoveto{\pgfqpoint{0.354974in}{3.207872in}}%
\pgfpathlineto{\pgfqpoint{0.571328in}{3.207872in}}%
\pgfpathlineto{\pgfqpoint{0.571328in}{3.265280in}}%
\pgfpathlineto{\pgfqpoint{0.354974in}{3.265280in}}%
\pgfpathclose%
\pgfusepath{stroke,fill}%
\end{pgfscope}%
\begin{pgfscope}%
\definecolor{textcolor}{rgb}{0.000000,0.000000,0.000000}%
\pgfsetstrokecolor{textcolor}%
\pgfsetfillcolor{textcolor}%
\pgftext[x=0.346640in,y=3.273613in,left,top]{\color{textcolor}\rmfamily\fontsize{6.000000}{7.200000}\selectfont 0.542}%
\end{pgfscope}%
\begin{pgfscope}%
\definecolor{textcolor}{rgb}{0.000000,0.000000,0.000000}%
\pgfsetstrokecolor{textcolor}%
\pgfsetfillcolor{textcolor}%
\pgftext[x=3.701958in,y=3.208028in,left,base]{\color{textcolor}\rmfamily\fontsize{6.000000}{7.200000}\selectfont 0}%
\end{pgfscope}%
\begin{pgfscope}%
\definecolor{textcolor}{rgb}{0.000000,0.000000,0.000000}%
\pgfsetstrokecolor{textcolor}%
\pgfsetfillcolor{textcolor}%
\pgftext[x=3.701958in,y=3.304367in,left,base]{\color{textcolor}\rmfamily\fontsize{6.000000}{7.200000}\selectfont 1}%
\end{pgfscope}%
\begin{pgfscope}%
\definecolor{textcolor}{rgb}{0.000000,0.000000,0.000000}%
\pgfsetstrokecolor{textcolor}%
\pgfsetfillcolor{textcolor}%
\pgftext[x=3.701958in,y=3.368997in,left,base]{\color{textcolor}\rmfamily\fontsize{6.000000}{7.200000}\selectfont 2}%
\end{pgfscope}%
\begin{pgfscope}%
\definecolor{textcolor}{rgb}{0.000000,0.000000,0.000000}%
\pgfsetstrokecolor{textcolor}%
\pgfsetfillcolor{textcolor}%
\pgftext[x=3.701958in,y=3.465891in,left,base]{\color{textcolor}\rmfamily\fontsize{6.000000}{7.200000}\selectfont 3}%
\end{pgfscope}%
\begin{pgfscope}%
\definecolor{textcolor}{rgb}{0.000000,0.000000,0.000000}%
\pgfsetstrokecolor{textcolor}%
\pgfsetfillcolor{textcolor}%
\pgftext[x=3.701958in,y=3.594989in,left,base]{\color{textcolor}\rmfamily\fontsize{6.000000}{7.200000}\selectfont 4}%
\end{pgfscope}%
\begin{pgfscope}%
\definecolor{textcolor}{rgb}{0.000000,0.000000,0.000000}%
\pgfsetstrokecolor{textcolor}%
\pgfsetfillcolor{textcolor}%
\pgftext[x=3.701958in,y=3.756208in,left,base]{\color{textcolor}\rmfamily\fontsize{6.000000}{7.200000}\selectfont 5}%
\end{pgfscope}%
\begin{pgfscope}%
\definecolor{textcolor}{rgb}{0.000000,0.000000,0.000000}%
\pgfsetstrokecolor{textcolor}%
\pgfsetfillcolor{textcolor}%
\pgftext[x=3.701958in,y=3.949448in,left,base]{\color{textcolor}\rmfamily\fontsize{6.000000}{7.200000}\selectfont 6}%
\end{pgfscope}%
\begin{pgfscope}%
\definecolor{textcolor}{rgb}{0.000000,0.000000,0.000000}%
\pgfsetstrokecolor{textcolor}%
\pgfsetfillcolor{textcolor}%
\pgftext[x=3.701958in,y=4.174585in,left,base]{\color{textcolor}\rmfamily\fontsize{6.000000}{7.200000}\selectfont 7}%
\end{pgfscope}%
\begin{pgfscope}%
\definecolor{textcolor}{rgb}{0.000000,0.000000,0.000000}%
\pgfsetstrokecolor{textcolor}%
\pgfsetfillcolor{textcolor}%
\pgftext[x=3.701958in,y=4.431478in,left,base]{\color{textcolor}\rmfamily\fontsize{6.000000}{7.200000}\selectfont 8}%
\end{pgfscope}%
\begin{pgfscope}%
\definecolor{textcolor}{rgb}{0.000000,0.000000,0.000000}%
\pgfsetstrokecolor{textcolor}%
\pgfsetfillcolor{textcolor}%
\pgftext[x=3.701958in,y=4.719963in,left,base]{\color{textcolor}\rmfamily\fontsize{6.000000}{7.200000}\selectfont 9}%
\end{pgfscope}%
\begin{pgfscope}%
\definecolor{textcolor}{rgb}{0.000000,0.000000,0.000000}%
\pgfsetstrokecolor{textcolor}%
\pgfsetfillcolor{textcolor}%
\pgftext[x=3.511618in,y=4.719963in,left,base]{\color{textcolor}\rmfamily\fontsize{6.000000}{7.200000}\selectfont \(\displaystyle J\,'\!=\)}%
\end{pgfscope}%
\begin{pgfscope}%
\definecolor{textcolor}{rgb}{0.000000,0.000000,0.000000}%
\pgfsetstrokecolor{textcolor}%
\pgfsetfillcolor{textcolor}%
\pgftext[x=0.812703in, y=3.766945in, left, base,rotate=90.000000]{\color{textcolor}\rmfamily\fontsize{6.000000}{7.200000}\selectfont \(\displaystyle v\,'\!=1\)}%
\end{pgfscope}%
\begin{pgfscope}%
\pgfsetbuttcap%
\pgfsetmiterjoin%
\definecolor{currentfill}{rgb}{1.000000,1.000000,1.000000}%
\pgfsetfillcolor{currentfill}%
\pgfsetlinewidth{0.000000pt}%
\definecolor{currentstroke}{rgb}{0.000000,0.000000,0.000000}%
\pgfsetstrokecolor{currentstroke}%
\pgfsetstrokeopacity{0.000000}%
\pgfsetdash{}{0pt}%
\pgfpathmoveto{\pgfqpoint{0.689290in}{1.439739in}}%
\pgfpathlineto{\pgfqpoint{3.755070in}{1.439739in}}%
\pgfpathlineto{\pgfqpoint{3.755070in}{3.104094in}}%
\pgfpathlineto{\pgfqpoint{0.689290in}{3.104094in}}%
\pgfpathclose%
\pgfusepath{fill}%
\end{pgfscope}%
\begin{pgfscope}%
\pgfsetroundcap%
\pgfsetroundjoin%
\pgfsetlinewidth{1.254687pt}%
\definecolor{currentstroke}{rgb}{0.000000,0.000000,1.000000}%
\pgfsetstrokecolor{currentstroke}%
\pgfsetdash{}{0pt}%
\pgfpathmoveto{\pgfqpoint{2.188635in}{1.606946in}}%
\pgfpathquadraticcurveto{\pgfqpoint{2.188635in}{2.454169in}}{\pgfqpoint{2.188635in}{3.281981in}}%
\pgfusepath{stroke}%
\end{pgfscope}%
\begin{pgfscope}%
\pgfsetroundcap%
\pgfsetroundjoin%
\definecolor{currentfill}{rgb}{0.000000,0.000000,1.000000}%
\pgfsetfillcolor{currentfill}%
\pgfsetlinewidth{1.254687pt}%
\definecolor{currentstroke}{rgb}{0.000000,0.000000,1.000000}%
\pgfsetstrokecolor{currentstroke}%
\pgfsetdash{}{0pt}%
\pgfpathmoveto{\pgfqpoint{2.160858in}{3.226425in}}%
\pgfpathlineto{\pgfqpoint{2.188635in}{3.281981in}}%
\pgfpathlineto{\pgfqpoint{2.216413in}{3.226425in}}%
\pgfpathlineto{\pgfqpoint{2.160858in}{3.226425in}}%
\pgfpathclose%
\pgfusepath{stroke,fill}%
\end{pgfscope}%
\begin{pgfscope}%
\pgfsetroundcap%
\pgfsetroundjoin%
\pgfsetlinewidth{1.254687pt}%
\definecolor{currentstroke}{rgb}{0.000000,0.000000,1.000000}%
\pgfsetstrokecolor{currentstroke}%
\pgfsetdash{}{0pt}%
\pgfpathmoveto{\pgfqpoint{2.047659in}{1.667182in}}%
\pgfpathquadraticcurveto{\pgfqpoint{2.047659in}{2.500369in}}{\pgfqpoint{2.047659in}{3.314145in}}%
\pgfusepath{stroke}%
\end{pgfscope}%
\begin{pgfscope}%
\pgfsetroundcap%
\pgfsetroundjoin%
\definecolor{currentfill}{rgb}{0.000000,0.000000,1.000000}%
\pgfsetfillcolor{currentfill}%
\pgfsetlinewidth{1.254687pt}%
\definecolor{currentstroke}{rgb}{0.000000,0.000000,1.000000}%
\pgfsetstrokecolor{currentstroke}%
\pgfsetdash{}{0pt}%
\pgfpathmoveto{\pgfqpoint{2.019881in}{3.258590in}}%
\pgfpathlineto{\pgfqpoint{2.047659in}{3.314145in}}%
\pgfpathlineto{\pgfqpoint{2.075436in}{3.258590in}}%
\pgfpathlineto{\pgfqpoint{2.019881in}{3.258590in}}%
\pgfpathclose%
\pgfusepath{stroke,fill}%
\end{pgfscope}%
\begin{pgfscope}%
\pgfsetroundcap%
\pgfsetroundjoin%
\pgfsetlinewidth{1.254687pt}%
\definecolor{currentstroke}{rgb}{0.000000,0.000000,1.000000}%
\pgfsetstrokecolor{currentstroke}%
\pgfsetdash{}{0pt}%
\pgfpathmoveto{\pgfqpoint{1.902817in}{1.757489in}}%
\pgfpathquadraticcurveto{\pgfqpoint{1.902817in}{2.577677in}}{\pgfqpoint{1.902817in}{3.378454in}}%
\pgfusepath{stroke}%
\end{pgfscope}%
\begin{pgfscope}%
\pgfsetroundcap%
\pgfsetroundjoin%
\definecolor{currentfill}{rgb}{0.000000,0.000000,1.000000}%
\pgfsetfillcolor{currentfill}%
\pgfsetlinewidth{1.254687pt}%
\definecolor{currentstroke}{rgb}{0.000000,0.000000,1.000000}%
\pgfsetstrokecolor{currentstroke}%
\pgfsetdash{}{0pt}%
\pgfpathmoveto{\pgfqpoint{1.875039in}{3.322898in}}%
\pgfpathlineto{\pgfqpoint{1.902817in}{3.378454in}}%
\pgfpathlineto{\pgfqpoint{1.930595in}{3.322898in}}%
\pgfpathlineto{\pgfqpoint{1.875039in}{3.322898in}}%
\pgfpathclose%
\pgfusepath{stroke,fill}%
\end{pgfscope}%
\begin{pgfscope}%
\pgfsetroundcap%
\pgfsetroundjoin%
\pgfsetlinewidth{1.254687pt}%
\definecolor{currentstroke}{rgb}{0.000000,0.000000,1.000000}%
\pgfsetstrokecolor{currentstroke}%
\pgfsetdash{}{0pt}%
\pgfpathmoveto{\pgfqpoint{1.754194in}{1.877813in}}%
\pgfpathquadraticcurveto{\pgfqpoint{1.754194in}{2.686045in}}{\pgfqpoint{1.754194in}{3.474866in}}%
\pgfusepath{stroke}%
\end{pgfscope}%
\begin{pgfscope}%
\pgfsetroundcap%
\pgfsetroundjoin%
\definecolor{currentfill}{rgb}{0.000000,0.000000,1.000000}%
\pgfsetfillcolor{currentfill}%
\pgfsetlinewidth{1.254687pt}%
\definecolor{currentstroke}{rgb}{0.000000,0.000000,1.000000}%
\pgfsetstrokecolor{currentstroke}%
\pgfsetdash{}{0pt}%
\pgfpathmoveto{\pgfqpoint{1.726416in}{3.419311in}}%
\pgfpathlineto{\pgfqpoint{1.754194in}{3.474866in}}%
\pgfpathlineto{\pgfqpoint{1.781972in}{3.419311in}}%
\pgfpathlineto{\pgfqpoint{1.726416in}{3.419311in}}%
\pgfpathclose%
\pgfusepath{stroke,fill}%
\end{pgfscope}%
\begin{pgfscope}%
\pgfsetroundcap%
\pgfsetroundjoin%
\pgfsetlinewidth{1.254687pt}%
\definecolor{currentstroke}{rgb}{0.000000,0.000000,1.000000}%
\pgfsetstrokecolor{currentstroke}%
\pgfsetdash{}{0pt}%
\pgfpathmoveto{\pgfqpoint{1.601874in}{2.028079in}}%
\pgfpathquadraticcurveto{\pgfqpoint{1.601874in}{2.825406in}}{\pgfqpoint{1.601874in}{3.603321in}}%
\pgfusepath{stroke}%
\end{pgfscope}%
\begin{pgfscope}%
\pgfsetroundcap%
\pgfsetroundjoin%
\definecolor{currentfill}{rgb}{0.000000,0.000000,1.000000}%
\pgfsetfillcolor{currentfill}%
\pgfsetlinewidth{1.254687pt}%
\definecolor{currentstroke}{rgb}{0.000000,0.000000,1.000000}%
\pgfsetstrokecolor{currentstroke}%
\pgfsetdash{}{0pt}%
\pgfpathmoveto{\pgfqpoint{1.574096in}{3.547766in}}%
\pgfpathlineto{\pgfqpoint{1.601874in}{3.603321in}}%
\pgfpathlineto{\pgfqpoint{1.629652in}{3.547766in}}%
\pgfpathlineto{\pgfqpoint{1.574096in}{3.547766in}}%
\pgfpathclose%
\pgfusepath{stroke,fill}%
\end{pgfscope}%
\begin{pgfscope}%
\pgfsetroundcap%
\pgfsetroundjoin%
\pgfsetlinewidth{1.254687pt}%
\definecolor{currentstroke}{rgb}{0.000000,0.000000,1.000000}%
\pgfsetstrokecolor{currentstroke}%
\pgfsetdash{}{0pt}%
\pgfpathmoveto{\pgfqpoint{1.445940in}{2.208197in}}%
\pgfpathquadraticcurveto{\pgfqpoint{1.445940in}{2.995673in}}{\pgfqpoint{1.445940in}{3.763739in}}%
\pgfusepath{stroke}%
\end{pgfscope}%
\begin{pgfscope}%
\pgfsetroundcap%
\pgfsetroundjoin%
\definecolor{currentfill}{rgb}{0.000000,0.000000,1.000000}%
\pgfsetfillcolor{currentfill}%
\pgfsetlinewidth{1.254687pt}%
\definecolor{currentstroke}{rgb}{0.000000,0.000000,1.000000}%
\pgfsetstrokecolor{currentstroke}%
\pgfsetdash{}{0pt}%
\pgfpathmoveto{\pgfqpoint{1.418162in}{3.708183in}}%
\pgfpathlineto{\pgfqpoint{1.445940in}{3.763739in}}%
\pgfpathlineto{\pgfqpoint{1.473718in}{3.708183in}}%
\pgfpathlineto{\pgfqpoint{1.418162in}{3.708183in}}%
\pgfpathclose%
\pgfusepath{stroke,fill}%
\end{pgfscope}%
\begin{pgfscope}%
\pgfsetroundcap%
\pgfsetroundjoin%
\pgfsetlinewidth{1.254687pt}%
\definecolor{currentstroke}{rgb}{0.000000,0.000000,1.000000}%
\pgfsetstrokecolor{currentstroke}%
\pgfsetdash{}{0pt}%
\pgfpathmoveto{\pgfqpoint{1.286476in}{2.418055in}}%
\pgfpathquadraticcurveto{\pgfqpoint{1.286476in}{3.196741in}}{\pgfqpoint{1.286476in}{3.956017in}}%
\pgfusepath{stroke}%
\end{pgfscope}%
\begin{pgfscope}%
\pgfsetroundcap%
\pgfsetroundjoin%
\definecolor{currentfill}{rgb}{0.000000,0.000000,1.000000}%
\pgfsetfillcolor{currentfill}%
\pgfsetlinewidth{1.254687pt}%
\definecolor{currentstroke}{rgb}{0.000000,0.000000,1.000000}%
\pgfsetstrokecolor{currentstroke}%
\pgfsetdash{}{0pt}%
\pgfpathmoveto{\pgfqpoint{1.258698in}{3.900461in}}%
\pgfpathlineto{\pgfqpoint{1.286476in}{3.956017in}}%
\pgfpathlineto{\pgfqpoint{1.314254in}{3.900461in}}%
\pgfpathlineto{\pgfqpoint{1.258698in}{3.900461in}}%
\pgfpathclose%
\pgfusepath{stroke,fill}%
\end{pgfscope}%
\begin{pgfscope}%
\pgfsetroundcap%
\pgfsetroundjoin%
\pgfsetlinewidth{1.254687pt}%
\definecolor{currentstroke}{rgb}{0.000000,0.000000,1.000000}%
\pgfsetstrokecolor{currentstroke}%
\pgfsetdash{}{0pt}%
\pgfpathmoveto{\pgfqpoint{1.123566in}{2.657524in}}%
\pgfpathquadraticcurveto{\pgfqpoint{1.123566in}{3.428484in}}{\pgfqpoint{1.123566in}{4.180034in}}%
\pgfusepath{stroke}%
\end{pgfscope}%
\begin{pgfscope}%
\pgfsetroundcap%
\pgfsetroundjoin%
\definecolor{currentfill}{rgb}{0.000000,0.000000,1.000000}%
\pgfsetfillcolor{currentfill}%
\pgfsetlinewidth{1.254687pt}%
\definecolor{currentstroke}{rgb}{0.000000,0.000000,1.000000}%
\pgfsetstrokecolor{currentstroke}%
\pgfsetdash{}{0pt}%
\pgfpathmoveto{\pgfqpoint{1.095788in}{4.124478in}}%
\pgfpathlineto{\pgfqpoint{1.123566in}{4.180034in}}%
\pgfpathlineto{\pgfqpoint{1.151344in}{4.124478in}}%
\pgfpathlineto{\pgfqpoint{1.095788in}{4.124478in}}%
\pgfpathclose%
\pgfusepath{stroke,fill}%
\end{pgfscope}%
\begin{pgfscope}%
\pgfsetroundcap%
\pgfsetroundjoin%
\pgfsetlinewidth{1.254687pt}%
\definecolor{currentstroke}{rgb}{0.000000,0.000000,1.000000}%
\pgfsetstrokecolor{currentstroke}%
\pgfsetdash{}{0pt}%
\pgfpathmoveto{\pgfqpoint{0.957293in}{2.926458in}}%
\pgfpathquadraticcurveto{\pgfqpoint{0.957293in}{3.690758in}}{\pgfqpoint{0.957293in}{4.435649in}}%
\pgfusepath{stroke}%
\end{pgfscope}%
\begin{pgfscope}%
\pgfsetroundcap%
\pgfsetroundjoin%
\definecolor{currentfill}{rgb}{0.000000,0.000000,1.000000}%
\pgfsetfillcolor{currentfill}%
\pgfsetlinewidth{1.254687pt}%
\definecolor{currentstroke}{rgb}{0.000000,0.000000,1.000000}%
\pgfsetstrokecolor{currentstroke}%
\pgfsetdash{}{0pt}%
\pgfpathmoveto{\pgfqpoint{0.929516in}{4.380093in}}%
\pgfpathlineto{\pgfqpoint{0.957293in}{4.435649in}}%
\pgfpathlineto{\pgfqpoint{0.985071in}{4.380093in}}%
\pgfpathlineto{\pgfqpoint{0.929516in}{4.380093in}}%
\pgfpathclose%
\pgfusepath{stroke,fill}%
\end{pgfscope}%
\begin{pgfscope}%
\pgfsetroundcap%
\pgfsetroundjoin%
\pgfsetlinewidth{1.254687pt}%
\definecolor{currentstroke}{rgb}{0.000000,0.500000,0.000000}%
\pgfsetstrokecolor{currentstroke}%
\pgfsetdash{}{0pt}%
\pgfpathmoveto{\pgfqpoint{2.458660in}{1.576819in}}%
\pgfpathquadraticcurveto{\pgfqpoint{2.458660in}{2.455187in}}{\pgfqpoint{2.458660in}{3.314145in}}%
\pgfusepath{stroke}%
\end{pgfscope}%
\begin{pgfscope}%
\pgfsetroundcap%
\pgfsetroundjoin%
\definecolor{currentfill}{rgb}{0.000000,0.500000,0.000000}%
\pgfsetfillcolor{currentfill}%
\pgfsetlinewidth{1.254687pt}%
\definecolor{currentstroke}{rgb}{0.000000,0.500000,0.000000}%
\pgfsetstrokecolor{currentstroke}%
\pgfsetdash{}{0pt}%
\pgfpathmoveto{\pgfqpoint{2.430882in}{3.258590in}}%
\pgfpathlineto{\pgfqpoint{2.458660in}{3.314145in}}%
\pgfpathlineto{\pgfqpoint{2.486437in}{3.258590in}}%
\pgfpathlineto{\pgfqpoint{2.430882in}{3.258590in}}%
\pgfpathclose%
\pgfusepath{stroke,fill}%
\end{pgfscope}%
\begin{pgfscope}%
\pgfsetroundcap%
\pgfsetroundjoin%
\pgfsetlinewidth{1.254687pt}%
\definecolor{currentstroke}{rgb}{0.000000,0.500000,0.000000}%
\pgfsetstrokecolor{currentstroke}%
\pgfsetdash{}{0pt}%
\pgfpathmoveto{\pgfqpoint{2.587539in}{1.606946in}}%
\pgfpathquadraticcurveto{\pgfqpoint{2.587539in}{2.502405in}}{\pgfqpoint{2.587539in}{3.378454in}}%
\pgfusepath{stroke}%
\end{pgfscope}%
\begin{pgfscope}%
\pgfsetroundcap%
\pgfsetroundjoin%
\definecolor{currentfill}{rgb}{0.000000,0.500000,0.000000}%
\pgfsetfillcolor{currentfill}%
\pgfsetlinewidth{1.254687pt}%
\definecolor{currentstroke}{rgb}{0.000000,0.500000,0.000000}%
\pgfsetstrokecolor{currentstroke}%
\pgfsetdash{}{0pt}%
\pgfpathmoveto{\pgfqpoint{2.559762in}{3.322898in}}%
\pgfpathlineto{\pgfqpoint{2.587539in}{3.378454in}}%
\pgfpathlineto{\pgfqpoint{2.615317in}{3.322898in}}%
\pgfpathlineto{\pgfqpoint{2.559762in}{3.322898in}}%
\pgfpathclose%
\pgfusepath{stroke,fill}%
\end{pgfscope}%
\begin{pgfscope}%
\pgfsetroundcap%
\pgfsetroundjoin%
\pgfsetlinewidth{1.254687pt}%
\definecolor{currentstroke}{rgb}{0.000000,0.500000,0.000000}%
\pgfsetstrokecolor{currentstroke}%
\pgfsetdash{}{0pt}%
\pgfpathmoveto{\pgfqpoint{2.712219in}{1.667182in}}%
\pgfpathquadraticcurveto{\pgfqpoint{2.712219in}{2.580729in}}{\pgfqpoint{2.712219in}{3.474866in}}%
\pgfusepath{stroke}%
\end{pgfscope}%
\begin{pgfscope}%
\pgfsetroundcap%
\pgfsetroundjoin%
\definecolor{currentfill}{rgb}{0.000000,0.500000,0.000000}%
\pgfsetfillcolor{currentfill}%
\pgfsetlinewidth{1.254687pt}%
\definecolor{currentstroke}{rgb}{0.000000,0.500000,0.000000}%
\pgfsetstrokecolor{currentstroke}%
\pgfsetdash{}{0pt}%
\pgfpathmoveto{\pgfqpoint{2.684442in}{3.419311in}}%
\pgfpathlineto{\pgfqpoint{2.712219in}{3.474866in}}%
\pgfpathlineto{\pgfqpoint{2.739997in}{3.419311in}}%
\pgfpathlineto{\pgfqpoint{2.684442in}{3.419311in}}%
\pgfpathclose%
\pgfusepath{stroke,fill}%
\end{pgfscope}%
\begin{pgfscope}%
\pgfsetroundcap%
\pgfsetroundjoin%
\pgfsetlinewidth{1.254687pt}%
\definecolor{currentstroke}{rgb}{0.000000,0.500000,0.000000}%
\pgfsetstrokecolor{currentstroke}%
\pgfsetdash{}{0pt}%
\pgfpathmoveto{\pgfqpoint{2.832616in}{1.757489in}}%
\pgfpathquadraticcurveto{\pgfqpoint{2.832616in}{2.690110in}}{\pgfqpoint{2.832616in}{3.603321in}}%
\pgfusepath{stroke}%
\end{pgfscope}%
\begin{pgfscope}%
\pgfsetroundcap%
\pgfsetroundjoin%
\definecolor{currentfill}{rgb}{0.000000,0.500000,0.000000}%
\pgfsetfillcolor{currentfill}%
\pgfsetlinewidth{1.254687pt}%
\definecolor{currentstroke}{rgb}{0.000000,0.500000,0.000000}%
\pgfsetstrokecolor{currentstroke}%
\pgfsetdash{}{0pt}%
\pgfpathmoveto{\pgfqpoint{2.804838in}{3.547766in}}%
\pgfpathlineto{\pgfqpoint{2.832616in}{3.603321in}}%
\pgfpathlineto{\pgfqpoint{2.860394in}{3.547766in}}%
\pgfpathlineto{\pgfqpoint{2.804838in}{3.547766in}}%
\pgfpathclose%
\pgfusepath{stroke,fill}%
\end{pgfscope}%
\begin{pgfscope}%
\pgfsetroundcap%
\pgfsetroundjoin%
\pgfsetlinewidth{1.254687pt}%
\definecolor{currentstroke}{rgb}{0.000000,0.500000,0.000000}%
\pgfsetstrokecolor{currentstroke}%
\pgfsetdash{}{0pt}%
\pgfpathmoveto{\pgfqpoint{2.948645in}{1.877813in}}%
\pgfpathquadraticcurveto{\pgfqpoint{2.948645in}{2.830481in}}{\pgfqpoint{2.948645in}{3.763739in}}%
\pgfusepath{stroke}%
\end{pgfscope}%
\begin{pgfscope}%
\pgfsetroundcap%
\pgfsetroundjoin%
\definecolor{currentfill}{rgb}{0.000000,0.500000,0.000000}%
\pgfsetfillcolor{currentfill}%
\pgfsetlinewidth{1.254687pt}%
\definecolor{currentstroke}{rgb}{0.000000,0.500000,0.000000}%
\pgfsetstrokecolor{currentstroke}%
\pgfsetdash{}{0pt}%
\pgfpathmoveto{\pgfqpoint{2.920868in}{3.708183in}}%
\pgfpathlineto{\pgfqpoint{2.948645in}{3.763739in}}%
\pgfpathlineto{\pgfqpoint{2.976423in}{3.708183in}}%
\pgfpathlineto{\pgfqpoint{2.920868in}{3.708183in}}%
\pgfpathclose%
\pgfusepath{stroke,fill}%
\end{pgfscope}%
\begin{pgfscope}%
\pgfsetroundcap%
\pgfsetroundjoin%
\pgfsetlinewidth{1.254687pt}%
\definecolor{currentstroke}{rgb}{0.000000,0.500000,0.000000}%
\pgfsetstrokecolor{currentstroke}%
\pgfsetdash{}{0pt}%
\pgfpathmoveto{\pgfqpoint{3.060224in}{2.028079in}}%
\pgfpathquadraticcurveto{\pgfqpoint{3.060224in}{3.001753in}}{\pgfqpoint{3.060224in}{3.956017in}}%
\pgfusepath{stroke}%
\end{pgfscope}%
\begin{pgfscope}%
\pgfsetroundcap%
\pgfsetroundjoin%
\definecolor{currentfill}{rgb}{0.000000,0.500000,0.000000}%
\pgfsetfillcolor{currentfill}%
\pgfsetlinewidth{1.254687pt}%
\definecolor{currentstroke}{rgb}{0.000000,0.500000,0.000000}%
\pgfsetstrokecolor{currentstroke}%
\pgfsetdash{}{0pt}%
\pgfpathmoveto{\pgfqpoint{3.032446in}{3.900461in}}%
\pgfpathlineto{\pgfqpoint{3.060224in}{3.956017in}}%
\pgfpathlineto{\pgfqpoint{3.088002in}{3.900461in}}%
\pgfpathlineto{\pgfqpoint{3.032446in}{3.900461in}}%
\pgfpathclose%
\pgfusepath{stroke,fill}%
\end{pgfscope}%
\begin{pgfscope}%
\pgfsetroundcap%
\pgfsetroundjoin%
\pgfsetlinewidth{1.254687pt}%
\definecolor{currentstroke}{rgb}{0.000000,0.500000,0.000000}%
\pgfsetstrokecolor{currentstroke}%
\pgfsetdash{}{0pt}%
\pgfpathmoveto{\pgfqpoint{3.167268in}{2.208197in}}%
\pgfpathquadraticcurveto{\pgfqpoint{3.167268in}{3.203820in}}{\pgfqpoint{3.167268in}{4.180034in}}%
\pgfusepath{stroke}%
\end{pgfscope}%
\begin{pgfscope}%
\pgfsetroundcap%
\pgfsetroundjoin%
\definecolor{currentfill}{rgb}{0.000000,0.500000,0.000000}%
\pgfsetfillcolor{currentfill}%
\pgfsetlinewidth{1.254687pt}%
\definecolor{currentstroke}{rgb}{0.000000,0.500000,0.000000}%
\pgfsetstrokecolor{currentstroke}%
\pgfsetdash{}{0pt}%
\pgfpathmoveto{\pgfqpoint{3.139490in}{4.124478in}}%
\pgfpathlineto{\pgfqpoint{3.167268in}{4.180034in}}%
\pgfpathlineto{\pgfqpoint{3.195046in}{4.124478in}}%
\pgfpathlineto{\pgfqpoint{3.139490in}{4.124478in}}%
\pgfpathclose%
\pgfusepath{stroke,fill}%
\end{pgfscope}%
\begin{pgfscope}%
\pgfsetroundcap%
\pgfsetroundjoin%
\pgfsetlinewidth{1.254687pt}%
\definecolor{currentstroke}{rgb}{0.000000,0.500000,0.000000}%
\pgfsetstrokecolor{currentstroke}%
\pgfsetdash{}{0pt}%
\pgfpathmoveto{\pgfqpoint{3.269693in}{2.418055in}}%
\pgfpathquadraticcurveto{\pgfqpoint{3.269693in}{3.436557in}}{\pgfqpoint{3.269693in}{4.435649in}}%
\pgfusepath{stroke}%
\end{pgfscope}%
\begin{pgfscope}%
\pgfsetroundcap%
\pgfsetroundjoin%
\definecolor{currentfill}{rgb}{0.000000,0.500000,0.000000}%
\pgfsetfillcolor{currentfill}%
\pgfsetlinewidth{1.254687pt}%
\definecolor{currentstroke}{rgb}{0.000000,0.500000,0.000000}%
\pgfsetstrokecolor{currentstroke}%
\pgfsetdash{}{0pt}%
\pgfpathmoveto{\pgfqpoint{3.241916in}{4.380093in}}%
\pgfpathlineto{\pgfqpoint{3.269693in}{4.435649in}}%
\pgfpathlineto{\pgfqpoint{3.297471in}{4.380093in}}%
\pgfpathlineto{\pgfqpoint{3.241916in}{4.380093in}}%
\pgfpathclose%
\pgfusepath{stroke,fill}%
\end{pgfscope}%
\begin{pgfscope}%
\pgfsetroundcap%
\pgfsetroundjoin%
\pgfsetlinewidth{1.254687pt}%
\definecolor{currentstroke}{rgb}{0.000000,0.500000,0.000000}%
\pgfsetstrokecolor{currentstroke}%
\pgfsetdash{}{0pt}%
\pgfpathmoveto{\pgfqpoint{3.367417in}{2.657524in}}%
\pgfpathquadraticcurveto{\pgfqpoint{3.367417in}{3.699817in}}{\pgfqpoint{3.367417in}{4.722699in}}%
\pgfusepath{stroke}%
\end{pgfscope}%
\begin{pgfscope}%
\pgfsetroundcap%
\pgfsetroundjoin%
\definecolor{currentfill}{rgb}{0.000000,0.500000,0.000000}%
\pgfsetfillcolor{currentfill}%
\pgfsetlinewidth{1.254687pt}%
\definecolor{currentstroke}{rgb}{0.000000,0.500000,0.000000}%
\pgfsetstrokecolor{currentstroke}%
\pgfsetdash{}{0pt}%
\pgfpathmoveto{\pgfqpoint{3.339639in}{4.667143in}}%
\pgfpathlineto{\pgfqpoint{3.367417in}{4.722699in}}%
\pgfpathlineto{\pgfqpoint{3.395194in}{4.667143in}}%
\pgfpathlineto{\pgfqpoint{3.339639in}{4.667143in}}%
\pgfpathclose%
\pgfusepath{stroke,fill}%
\end{pgfscope}%
\begin{pgfscope}%
\pgfpathrectangle{\pgfqpoint{0.689290in}{1.439739in}}{\pgfqpoint{3.065780in}{1.664355in}}%
\pgfusepath{clip}%
\pgfsetbuttcap%
\pgfsetroundjoin%
\pgfsetlinewidth{0.501875pt}%
\definecolor{currentstroke}{rgb}{0.700000,0.700000,0.700000}%
\pgfsetstrokecolor{currentstroke}%
\pgfsetstrokeopacity{0.200000}%
\pgfsetdash{{1.850000pt}{0.800000pt}}{0.000000pt}%
\pgfpathmoveto{\pgfqpoint{1.109402in}{1.439739in}}%
\pgfpathlineto{\pgfqpoint{1.109402in}{3.104094in}}%
\pgfusepath{stroke}%
\end{pgfscope}%
\begin{pgfscope}%
\pgfpathrectangle{\pgfqpoint{0.689290in}{1.439739in}}{\pgfqpoint{3.065780in}{1.664355in}}%
\pgfusepath{clip}%
\pgfsetbuttcap%
\pgfsetroundjoin%
\pgfsetlinewidth{0.501875pt}%
\definecolor{currentstroke}{rgb}{0.700000,0.700000,0.700000}%
\pgfsetstrokecolor{currentstroke}%
\pgfsetstrokeopacity{0.200000}%
\pgfsetdash{{1.850000pt}{0.800000pt}}{0.000000pt}%
\pgfpathmoveto{\pgfqpoint{1.765745in}{1.439739in}}%
\pgfpathlineto{\pgfqpoint{1.765745in}{3.104094in}}%
\pgfusepath{stroke}%
\end{pgfscope}%
\begin{pgfscope}%
\pgfpathrectangle{\pgfqpoint{0.689290in}{1.439739in}}{\pgfqpoint{3.065780in}{1.664355in}}%
\pgfusepath{clip}%
\pgfsetbuttcap%
\pgfsetroundjoin%
\pgfsetlinewidth{0.501875pt}%
\definecolor{currentstroke}{rgb}{0.700000,0.700000,0.700000}%
\pgfsetstrokecolor{currentstroke}%
\pgfsetstrokeopacity{0.200000}%
\pgfsetdash{{1.850000pt}{0.800000pt}}{0.000000pt}%
\pgfpathmoveto{\pgfqpoint{2.422088in}{1.439739in}}%
\pgfpathlineto{\pgfqpoint{2.422088in}{3.104094in}}%
\pgfusepath{stroke}%
\end{pgfscope}%
\begin{pgfscope}%
\pgfpathrectangle{\pgfqpoint{0.689290in}{1.439739in}}{\pgfqpoint{3.065780in}{1.664355in}}%
\pgfusepath{clip}%
\pgfsetbuttcap%
\pgfsetroundjoin%
\pgfsetlinewidth{0.501875pt}%
\definecolor{currentstroke}{rgb}{0.700000,0.700000,0.700000}%
\pgfsetstrokecolor{currentstroke}%
\pgfsetstrokeopacity{0.200000}%
\pgfsetdash{{1.850000pt}{0.800000pt}}{0.000000pt}%
\pgfpathmoveto{\pgfqpoint{3.078432in}{1.439739in}}%
\pgfpathlineto{\pgfqpoint{3.078432in}{3.104094in}}%
\pgfusepath{stroke}%
\end{pgfscope}%
\begin{pgfscope}%
\pgfpathrectangle{\pgfqpoint{0.689290in}{1.439739in}}{\pgfqpoint{3.065780in}{1.664355in}}%
\pgfusepath{clip}%
\pgfsetbuttcap%
\pgfsetroundjoin%
\pgfsetlinewidth{0.501875pt}%
\definecolor{currentstroke}{rgb}{0.700000,0.700000,0.700000}%
\pgfsetstrokecolor{currentstroke}%
\pgfsetstrokeopacity{0.200000}%
\pgfsetdash{{1.850000pt}{0.800000pt}}{0.000000pt}%
\pgfpathmoveto{\pgfqpoint{3.734775in}{1.439739in}}%
\pgfpathlineto{\pgfqpoint{3.734775in}{3.104094in}}%
\pgfusepath{stroke}%
\end{pgfscope}%
\begin{pgfscope}%
\pgfpathrectangle{\pgfqpoint{0.689290in}{1.439739in}}{\pgfqpoint{3.065780in}{1.664355in}}%
\pgfusepath{clip}%
\pgfsetbuttcap%
\pgfsetroundjoin%
\pgfsetlinewidth{0.501875pt}%
\definecolor{currentstroke}{rgb}{0.700000,0.700000,0.700000}%
\pgfsetstrokecolor{currentstroke}%
\pgfsetstrokeopacity{0.200000}%
\pgfsetdash{{1.850000pt}{0.800000pt}}{0.000000pt}%
\pgfpathmoveto{\pgfqpoint{0.689290in}{1.576819in}}%
\pgfpathlineto{\pgfqpoint{3.755070in}{1.576819in}}%
\pgfusepath{stroke}%
\end{pgfscope}%
\begin{pgfscope}%
\pgfsetbuttcap%
\pgfsetroundjoin%
\definecolor{currentfill}{rgb}{0.000000,0.000000,0.000000}%
\pgfsetfillcolor{currentfill}%
\pgfsetlinewidth{1.003750pt}%
\definecolor{currentstroke}{rgb}{0.000000,0.000000,0.000000}%
\pgfsetstrokecolor{currentstroke}%
\pgfsetdash{}{0pt}%
\pgfsys@defobject{currentmarker}{\pgfqpoint{-0.083333in}{0.000000in}}{\pgfqpoint{0.000000in}{0.000000in}}{%
\pgfpathmoveto{\pgfqpoint{0.000000in}{0.000000in}}%
\pgfpathlineto{\pgfqpoint{-0.083333in}{0.000000in}}%
\pgfusepath{stroke,fill}%
}%
\begin{pgfscope}%
\pgfsys@transformshift{0.689290in}{1.576819in}%
\pgfsys@useobject{currentmarker}{}%
\end{pgfscope}%
\end{pgfscope}%
\begin{pgfscope}%
\pgfpathrectangle{\pgfqpoint{0.689290in}{1.439739in}}{\pgfqpoint{3.065780in}{1.664355in}}%
\pgfusepath{clip}%
\pgfsetbuttcap%
\pgfsetroundjoin%
\pgfsetlinewidth{0.501875pt}%
\definecolor{currentstroke}{rgb}{0.700000,0.700000,0.700000}%
\pgfsetstrokecolor{currentstroke}%
\pgfsetstrokeopacity{0.200000}%
\pgfsetdash{{1.850000pt}{0.800000pt}}{0.000000pt}%
\pgfpathmoveto{\pgfqpoint{0.689290in}{1.606946in}}%
\pgfpathlineto{\pgfqpoint{3.755070in}{1.606946in}}%
\pgfusepath{stroke}%
\end{pgfscope}%
\begin{pgfscope}%
\pgfsetbuttcap%
\pgfsetroundjoin%
\definecolor{currentfill}{rgb}{0.000000,0.000000,0.000000}%
\pgfsetfillcolor{currentfill}%
\pgfsetlinewidth{1.003750pt}%
\definecolor{currentstroke}{rgb}{0.000000,0.000000,0.000000}%
\pgfsetstrokecolor{currentstroke}%
\pgfsetdash{}{0pt}%
\pgfsys@defobject{currentmarker}{\pgfqpoint{-0.083333in}{0.000000in}}{\pgfqpoint{0.000000in}{0.000000in}}{%
\pgfpathmoveto{\pgfqpoint{0.000000in}{0.000000in}}%
\pgfpathlineto{\pgfqpoint{-0.083333in}{0.000000in}}%
\pgfusepath{stroke,fill}%
}%
\begin{pgfscope}%
\pgfsys@transformshift{0.689290in}{1.606946in}%
\pgfsys@useobject{currentmarker}{}%
\end{pgfscope}%
\end{pgfscope}%
\begin{pgfscope}%
\definecolor{textcolor}{rgb}{0.000000,0.000000,0.000000}%
\pgfsetstrokecolor{textcolor}%
\pgfsetfillcolor{textcolor}%
\pgftext[x=0.345158in, y=1.578011in, left, base]{\color{textcolor}\rmfamily\fontsize{6.000000}{7.200000}\selectfont 0.186}%
\end{pgfscope}%
\begin{pgfscope}%
\pgfpathrectangle{\pgfqpoint{0.689290in}{1.439739in}}{\pgfqpoint{3.065780in}{1.664355in}}%
\pgfusepath{clip}%
\pgfsetbuttcap%
\pgfsetroundjoin%
\pgfsetlinewidth{0.501875pt}%
\definecolor{currentstroke}{rgb}{0.700000,0.700000,0.700000}%
\pgfsetstrokecolor{currentstroke}%
\pgfsetstrokeopacity{0.200000}%
\pgfsetdash{{1.850000pt}{0.800000pt}}{0.000000pt}%
\pgfpathmoveto{\pgfqpoint{0.689290in}{1.667182in}}%
\pgfpathlineto{\pgfqpoint{3.755070in}{1.667182in}}%
\pgfusepath{stroke}%
\end{pgfscope}%
\begin{pgfscope}%
\pgfsetbuttcap%
\pgfsetroundjoin%
\definecolor{currentfill}{rgb}{0.000000,0.000000,0.000000}%
\pgfsetfillcolor{currentfill}%
\pgfsetlinewidth{1.003750pt}%
\definecolor{currentstroke}{rgb}{0.000000,0.000000,0.000000}%
\pgfsetstrokecolor{currentstroke}%
\pgfsetdash{}{0pt}%
\pgfsys@defobject{currentmarker}{\pgfqpoint{-0.083333in}{0.000000in}}{\pgfqpoint{0.000000in}{0.000000in}}{%
\pgfpathmoveto{\pgfqpoint{0.000000in}{0.000000in}}%
\pgfpathlineto{\pgfqpoint{-0.083333in}{0.000000in}}%
\pgfusepath{stroke,fill}%
}%
\begin{pgfscope}%
\pgfsys@transformshift{0.689290in}{1.667182in}%
\pgfsys@useobject{currentmarker}{}%
\end{pgfscope}%
\end{pgfscope}%
\begin{pgfscope}%
\definecolor{textcolor}{rgb}{0.000000,0.000000,0.000000}%
\pgfsetstrokecolor{textcolor}%
\pgfsetfillcolor{textcolor}%
\pgftext[x=0.345158in, y=1.638247in, left, base]{\color{textcolor}\rmfamily\fontsize{6.000000}{7.200000}\selectfont 0.192}%
\end{pgfscope}%
\begin{pgfscope}%
\pgfpathrectangle{\pgfqpoint{0.689290in}{1.439739in}}{\pgfqpoint{3.065780in}{1.664355in}}%
\pgfusepath{clip}%
\pgfsetbuttcap%
\pgfsetroundjoin%
\pgfsetlinewidth{0.501875pt}%
\definecolor{currentstroke}{rgb}{0.700000,0.700000,0.700000}%
\pgfsetstrokecolor{currentstroke}%
\pgfsetstrokeopacity{0.200000}%
\pgfsetdash{{1.850000pt}{0.800000pt}}{0.000000pt}%
\pgfpathmoveto{\pgfqpoint{0.689290in}{1.757489in}}%
\pgfpathlineto{\pgfqpoint{3.755070in}{1.757489in}}%
\pgfusepath{stroke}%
\end{pgfscope}%
\begin{pgfscope}%
\pgfsetbuttcap%
\pgfsetroundjoin%
\definecolor{currentfill}{rgb}{0.000000,0.000000,0.000000}%
\pgfsetfillcolor{currentfill}%
\pgfsetlinewidth{1.003750pt}%
\definecolor{currentstroke}{rgb}{0.000000,0.000000,0.000000}%
\pgfsetstrokecolor{currentstroke}%
\pgfsetdash{}{0pt}%
\pgfsys@defobject{currentmarker}{\pgfqpoint{-0.083333in}{0.000000in}}{\pgfqpoint{0.000000in}{0.000000in}}{%
\pgfpathmoveto{\pgfqpoint{0.000000in}{0.000000in}}%
\pgfpathlineto{\pgfqpoint{-0.083333in}{0.000000in}}%
\pgfusepath{stroke,fill}%
}%
\begin{pgfscope}%
\pgfsys@transformshift{0.689290in}{1.757489in}%
\pgfsys@useobject{currentmarker}{}%
\end{pgfscope}%
\end{pgfscope}%
\begin{pgfscope}%
\definecolor{textcolor}{rgb}{0.000000,0.000000,0.000000}%
\pgfsetstrokecolor{textcolor}%
\pgfsetfillcolor{textcolor}%
\pgftext[x=0.345158in, y=1.728554in, left, base]{\color{textcolor}\rmfamily\fontsize{6.000000}{7.200000}\selectfont 0.199}%
\end{pgfscope}%
\begin{pgfscope}%
\pgfpathrectangle{\pgfqpoint{0.689290in}{1.439739in}}{\pgfqpoint{3.065780in}{1.664355in}}%
\pgfusepath{clip}%
\pgfsetbuttcap%
\pgfsetroundjoin%
\pgfsetlinewidth{0.501875pt}%
\definecolor{currentstroke}{rgb}{0.700000,0.700000,0.700000}%
\pgfsetstrokecolor{currentstroke}%
\pgfsetstrokeopacity{0.200000}%
\pgfsetdash{{1.850000pt}{0.800000pt}}{0.000000pt}%
\pgfpathmoveto{\pgfqpoint{0.689290in}{1.877813in}}%
\pgfpathlineto{\pgfqpoint{3.755070in}{1.877813in}}%
\pgfusepath{stroke}%
\end{pgfscope}%
\begin{pgfscope}%
\pgfsetbuttcap%
\pgfsetroundjoin%
\definecolor{currentfill}{rgb}{0.000000,0.000000,0.000000}%
\pgfsetfillcolor{currentfill}%
\pgfsetlinewidth{1.003750pt}%
\definecolor{currentstroke}{rgb}{0.000000,0.000000,0.000000}%
\pgfsetstrokecolor{currentstroke}%
\pgfsetdash{}{0pt}%
\pgfsys@defobject{currentmarker}{\pgfqpoint{-0.083333in}{0.000000in}}{\pgfqpoint{0.000000in}{0.000000in}}{%
\pgfpathmoveto{\pgfqpoint{0.000000in}{0.000000in}}%
\pgfpathlineto{\pgfqpoint{-0.083333in}{0.000000in}}%
\pgfusepath{stroke,fill}%
}%
\begin{pgfscope}%
\pgfsys@transformshift{0.689290in}{1.877813in}%
\pgfsys@useobject{currentmarker}{}%
\end{pgfscope}%
\end{pgfscope}%
\begin{pgfscope}%
\definecolor{textcolor}{rgb}{0.000000,0.000000,0.000000}%
\pgfsetstrokecolor{textcolor}%
\pgfsetfillcolor{textcolor}%
\pgftext[x=0.345158in, y=1.848878in, left, base]{\color{textcolor}\rmfamily\fontsize{6.000000}{7.200000}\selectfont 0.210}%
\end{pgfscope}%
\begin{pgfscope}%
\pgfpathrectangle{\pgfqpoint{0.689290in}{1.439739in}}{\pgfqpoint{3.065780in}{1.664355in}}%
\pgfusepath{clip}%
\pgfsetbuttcap%
\pgfsetroundjoin%
\pgfsetlinewidth{0.501875pt}%
\definecolor{currentstroke}{rgb}{0.700000,0.700000,0.700000}%
\pgfsetstrokecolor{currentstroke}%
\pgfsetstrokeopacity{0.200000}%
\pgfsetdash{{1.850000pt}{0.800000pt}}{0.000000pt}%
\pgfpathmoveto{\pgfqpoint{0.689290in}{2.028079in}}%
\pgfpathlineto{\pgfqpoint{3.755070in}{2.028079in}}%
\pgfusepath{stroke}%
\end{pgfscope}%
\begin{pgfscope}%
\pgfsetbuttcap%
\pgfsetroundjoin%
\definecolor{currentfill}{rgb}{0.000000,0.000000,0.000000}%
\pgfsetfillcolor{currentfill}%
\pgfsetlinewidth{1.003750pt}%
\definecolor{currentstroke}{rgb}{0.000000,0.000000,0.000000}%
\pgfsetstrokecolor{currentstroke}%
\pgfsetdash{}{0pt}%
\pgfsys@defobject{currentmarker}{\pgfqpoint{-0.083333in}{0.000000in}}{\pgfqpoint{0.000000in}{0.000000in}}{%
\pgfpathmoveto{\pgfqpoint{0.000000in}{0.000000in}}%
\pgfpathlineto{\pgfqpoint{-0.083333in}{0.000000in}}%
\pgfusepath{stroke,fill}%
}%
\begin{pgfscope}%
\pgfsys@transformshift{0.689290in}{2.028079in}%
\pgfsys@useobject{currentmarker}{}%
\end{pgfscope}%
\end{pgfscope}%
\begin{pgfscope}%
\definecolor{textcolor}{rgb}{0.000000,0.000000,0.000000}%
\pgfsetstrokecolor{textcolor}%
\pgfsetfillcolor{textcolor}%
\pgftext[x=0.345158in, y=1.999144in, left, base]{\color{textcolor}\rmfamily\fontsize{6.000000}{7.200000}\selectfont 0.223}%
\end{pgfscope}%
\begin{pgfscope}%
\pgfpathrectangle{\pgfqpoint{0.689290in}{1.439739in}}{\pgfqpoint{3.065780in}{1.664355in}}%
\pgfusepath{clip}%
\pgfsetbuttcap%
\pgfsetroundjoin%
\pgfsetlinewidth{0.501875pt}%
\definecolor{currentstroke}{rgb}{0.700000,0.700000,0.700000}%
\pgfsetstrokecolor{currentstroke}%
\pgfsetstrokeopacity{0.200000}%
\pgfsetdash{{1.850000pt}{0.800000pt}}{0.000000pt}%
\pgfpathmoveto{\pgfqpoint{0.689290in}{2.208197in}}%
\pgfpathlineto{\pgfqpoint{3.755070in}{2.208197in}}%
\pgfusepath{stroke}%
\end{pgfscope}%
\begin{pgfscope}%
\pgfsetbuttcap%
\pgfsetroundjoin%
\definecolor{currentfill}{rgb}{0.000000,0.000000,0.000000}%
\pgfsetfillcolor{currentfill}%
\pgfsetlinewidth{1.003750pt}%
\definecolor{currentstroke}{rgb}{0.000000,0.000000,0.000000}%
\pgfsetstrokecolor{currentstroke}%
\pgfsetdash{}{0pt}%
\pgfsys@defobject{currentmarker}{\pgfqpoint{-0.083333in}{0.000000in}}{\pgfqpoint{0.000000in}{0.000000in}}{%
\pgfpathmoveto{\pgfqpoint{0.000000in}{0.000000in}}%
\pgfpathlineto{\pgfqpoint{-0.083333in}{0.000000in}}%
\pgfusepath{stroke,fill}%
}%
\begin{pgfscope}%
\pgfsys@transformshift{0.689290in}{2.208197in}%
\pgfsys@useobject{currentmarker}{}%
\end{pgfscope}%
\end{pgfscope}%
\begin{pgfscope}%
\definecolor{textcolor}{rgb}{0.000000,0.000000,0.000000}%
\pgfsetstrokecolor{textcolor}%
\pgfsetfillcolor{textcolor}%
\pgftext[x=0.345158in, y=2.179262in, left, base]{\color{textcolor}\rmfamily\fontsize{6.000000}{7.200000}\selectfont 0.238}%
\end{pgfscope}%
\begin{pgfscope}%
\pgfpathrectangle{\pgfqpoint{0.689290in}{1.439739in}}{\pgfqpoint{3.065780in}{1.664355in}}%
\pgfusepath{clip}%
\pgfsetbuttcap%
\pgfsetroundjoin%
\pgfsetlinewidth{0.501875pt}%
\definecolor{currentstroke}{rgb}{0.700000,0.700000,0.700000}%
\pgfsetstrokecolor{currentstroke}%
\pgfsetstrokeopacity{0.200000}%
\pgfsetdash{{1.850000pt}{0.800000pt}}{0.000000pt}%
\pgfpathmoveto{\pgfqpoint{0.689290in}{2.418055in}}%
\pgfpathlineto{\pgfqpoint{3.755070in}{2.418055in}}%
\pgfusepath{stroke}%
\end{pgfscope}%
\begin{pgfscope}%
\pgfsetbuttcap%
\pgfsetroundjoin%
\definecolor{currentfill}{rgb}{0.000000,0.000000,0.000000}%
\pgfsetfillcolor{currentfill}%
\pgfsetlinewidth{1.003750pt}%
\definecolor{currentstroke}{rgb}{0.000000,0.000000,0.000000}%
\pgfsetstrokecolor{currentstroke}%
\pgfsetdash{}{0pt}%
\pgfsys@defobject{currentmarker}{\pgfqpoint{-0.083333in}{0.000000in}}{\pgfqpoint{0.000000in}{0.000000in}}{%
\pgfpathmoveto{\pgfqpoint{0.000000in}{0.000000in}}%
\pgfpathlineto{\pgfqpoint{-0.083333in}{0.000000in}}%
\pgfusepath{stroke,fill}%
}%
\begin{pgfscope}%
\pgfsys@transformshift{0.689290in}{2.418055in}%
\pgfsys@useobject{currentmarker}{}%
\end{pgfscope}%
\end{pgfscope}%
\begin{pgfscope}%
\definecolor{textcolor}{rgb}{0.000000,0.000000,0.000000}%
\pgfsetstrokecolor{textcolor}%
\pgfsetfillcolor{textcolor}%
\pgftext[x=0.345158in, y=2.389119in, left, base]{\color{textcolor}\rmfamily\fontsize{6.000000}{7.200000}\selectfont 0.256}%
\end{pgfscope}%
\begin{pgfscope}%
\pgfpathrectangle{\pgfqpoint{0.689290in}{1.439739in}}{\pgfqpoint{3.065780in}{1.664355in}}%
\pgfusepath{clip}%
\pgfsetbuttcap%
\pgfsetroundjoin%
\pgfsetlinewidth{0.501875pt}%
\definecolor{currentstroke}{rgb}{0.700000,0.700000,0.700000}%
\pgfsetstrokecolor{currentstroke}%
\pgfsetstrokeopacity{0.200000}%
\pgfsetdash{{1.850000pt}{0.800000pt}}{0.000000pt}%
\pgfpathmoveto{\pgfqpoint{0.689290in}{2.657524in}}%
\pgfpathlineto{\pgfqpoint{3.755070in}{2.657524in}}%
\pgfusepath{stroke}%
\end{pgfscope}%
\begin{pgfscope}%
\pgfsetbuttcap%
\pgfsetroundjoin%
\definecolor{currentfill}{rgb}{0.000000,0.000000,0.000000}%
\pgfsetfillcolor{currentfill}%
\pgfsetlinewidth{1.003750pt}%
\definecolor{currentstroke}{rgb}{0.000000,0.000000,0.000000}%
\pgfsetstrokecolor{currentstroke}%
\pgfsetdash{}{0pt}%
\pgfsys@defobject{currentmarker}{\pgfqpoint{-0.083333in}{0.000000in}}{\pgfqpoint{0.000000in}{0.000000in}}{%
\pgfpathmoveto{\pgfqpoint{0.000000in}{0.000000in}}%
\pgfpathlineto{\pgfqpoint{-0.083333in}{0.000000in}}%
\pgfusepath{stroke,fill}%
}%
\begin{pgfscope}%
\pgfsys@transformshift{0.689290in}{2.657524in}%
\pgfsys@useobject{currentmarker}{}%
\end{pgfscope}%
\end{pgfscope}%
\begin{pgfscope}%
\definecolor{textcolor}{rgb}{0.000000,0.000000,0.000000}%
\pgfsetstrokecolor{textcolor}%
\pgfsetfillcolor{textcolor}%
\pgftext[x=0.345158in, y=2.628589in, left, base]{\color{textcolor}\rmfamily\fontsize{6.000000}{7.200000}\selectfont 0.277}%
\end{pgfscope}%
\begin{pgfscope}%
\pgfpathrectangle{\pgfqpoint{0.689290in}{1.439739in}}{\pgfqpoint{3.065780in}{1.664355in}}%
\pgfusepath{clip}%
\pgfsetbuttcap%
\pgfsetroundjoin%
\pgfsetlinewidth{0.501875pt}%
\definecolor{currentstroke}{rgb}{0.700000,0.700000,0.700000}%
\pgfsetstrokecolor{currentstroke}%
\pgfsetstrokeopacity{0.200000}%
\pgfsetdash{{1.850000pt}{0.800000pt}}{0.000000pt}%
\pgfpathmoveto{\pgfqpoint{0.689290in}{2.926458in}}%
\pgfpathlineto{\pgfqpoint{3.755070in}{2.926458in}}%
\pgfusepath{stroke}%
\end{pgfscope}%
\begin{pgfscope}%
\pgfsetbuttcap%
\pgfsetroundjoin%
\definecolor{currentfill}{rgb}{0.000000,0.000000,0.000000}%
\pgfsetfillcolor{currentfill}%
\pgfsetlinewidth{1.003750pt}%
\definecolor{currentstroke}{rgb}{0.000000,0.000000,0.000000}%
\pgfsetstrokecolor{currentstroke}%
\pgfsetdash{}{0pt}%
\pgfsys@defobject{currentmarker}{\pgfqpoint{-0.083333in}{0.000000in}}{\pgfqpoint{0.000000in}{0.000000in}}{%
\pgfpathmoveto{\pgfqpoint{0.000000in}{0.000000in}}%
\pgfpathlineto{\pgfqpoint{-0.083333in}{0.000000in}}%
\pgfusepath{stroke,fill}%
}%
\begin{pgfscope}%
\pgfsys@transformshift{0.689290in}{2.926458in}%
\pgfsys@useobject{currentmarker}{}%
\end{pgfscope}%
\end{pgfscope}%
\begin{pgfscope}%
\definecolor{textcolor}{rgb}{0.000000,0.000000,0.000000}%
\pgfsetstrokecolor{textcolor}%
\pgfsetfillcolor{textcolor}%
\pgftext[x=0.345158in, y=2.897522in, left, base]{\color{textcolor}\rmfamily\fontsize{6.000000}{7.200000}\selectfont 0.300}%
\end{pgfscope}%
\begin{pgfscope}%
\pgfpathrectangle{\pgfqpoint{0.689290in}{1.439739in}}{\pgfqpoint{3.065780in}{1.664355in}}%
\pgfusepath{clip}%
\pgfsetbuttcap%
\pgfsetroundjoin%
\pgfsetlinewidth{1.003750pt}%
\definecolor{currentstroke}{rgb}{0.000000,0.000000,0.000000}%
\pgfsetstrokecolor{currentstroke}%
\pgfsetdash{}{0pt}%
\pgfpathmoveto{\pgfqpoint{0.858842in}{1.576819in}}%
\pgfpathlineto{\pgfqpoint{3.465868in}{1.576819in}}%
\pgfusepath{stroke}%
\end{pgfscope}%
\begin{pgfscope}%
\pgfpathrectangle{\pgfqpoint{0.689290in}{1.439739in}}{\pgfqpoint{3.065780in}{1.664355in}}%
\pgfusepath{clip}%
\pgfsetbuttcap%
\pgfsetroundjoin%
\pgfsetlinewidth{1.003750pt}%
\definecolor{currentstroke}{rgb}{0.000000,0.000000,0.000000}%
\pgfsetstrokecolor{currentstroke}%
\pgfsetdash{}{0pt}%
\pgfpathmoveto{\pgfqpoint{0.858842in}{1.606946in}}%
\pgfpathlineto{\pgfqpoint{3.465868in}{1.606946in}}%
\pgfusepath{stroke}%
\end{pgfscope}%
\begin{pgfscope}%
\pgfpathrectangle{\pgfqpoint{0.689290in}{1.439739in}}{\pgfqpoint{3.065780in}{1.664355in}}%
\pgfusepath{clip}%
\pgfsetbuttcap%
\pgfsetroundjoin%
\pgfsetlinewidth{1.003750pt}%
\definecolor{currentstroke}{rgb}{0.000000,0.000000,0.000000}%
\pgfsetstrokecolor{currentstroke}%
\pgfsetdash{}{0pt}%
\pgfpathmoveto{\pgfqpoint{0.858842in}{1.667182in}}%
\pgfpathlineto{\pgfqpoint{3.465868in}{1.667182in}}%
\pgfusepath{stroke}%
\end{pgfscope}%
\begin{pgfscope}%
\pgfpathrectangle{\pgfqpoint{0.689290in}{1.439739in}}{\pgfqpoint{3.065780in}{1.664355in}}%
\pgfusepath{clip}%
\pgfsetbuttcap%
\pgfsetroundjoin%
\pgfsetlinewidth{1.003750pt}%
\definecolor{currentstroke}{rgb}{0.000000,0.000000,0.000000}%
\pgfsetstrokecolor{currentstroke}%
\pgfsetdash{}{0pt}%
\pgfpathmoveto{\pgfqpoint{0.858842in}{1.757489in}}%
\pgfpathlineto{\pgfqpoint{3.465868in}{1.757489in}}%
\pgfusepath{stroke}%
\end{pgfscope}%
\begin{pgfscope}%
\pgfpathrectangle{\pgfqpoint{0.689290in}{1.439739in}}{\pgfqpoint{3.065780in}{1.664355in}}%
\pgfusepath{clip}%
\pgfsetbuttcap%
\pgfsetroundjoin%
\pgfsetlinewidth{1.003750pt}%
\definecolor{currentstroke}{rgb}{0.000000,0.000000,0.000000}%
\pgfsetstrokecolor{currentstroke}%
\pgfsetdash{}{0pt}%
\pgfpathmoveto{\pgfqpoint{0.858842in}{1.877813in}}%
\pgfpathlineto{\pgfqpoint{3.465868in}{1.877813in}}%
\pgfusepath{stroke}%
\end{pgfscope}%
\begin{pgfscope}%
\pgfpathrectangle{\pgfqpoint{0.689290in}{1.439739in}}{\pgfqpoint{3.065780in}{1.664355in}}%
\pgfusepath{clip}%
\pgfsetbuttcap%
\pgfsetroundjoin%
\pgfsetlinewidth{1.003750pt}%
\definecolor{currentstroke}{rgb}{0.000000,0.000000,0.000000}%
\pgfsetstrokecolor{currentstroke}%
\pgfsetdash{}{0pt}%
\pgfpathmoveto{\pgfqpoint{0.858842in}{2.028079in}}%
\pgfpathlineto{\pgfqpoint{3.465868in}{2.028079in}}%
\pgfusepath{stroke}%
\end{pgfscope}%
\begin{pgfscope}%
\pgfpathrectangle{\pgfqpoint{0.689290in}{1.439739in}}{\pgfqpoint{3.065780in}{1.664355in}}%
\pgfusepath{clip}%
\pgfsetbuttcap%
\pgfsetroundjoin%
\pgfsetlinewidth{1.003750pt}%
\definecolor{currentstroke}{rgb}{0.000000,0.000000,0.000000}%
\pgfsetstrokecolor{currentstroke}%
\pgfsetdash{}{0pt}%
\pgfpathmoveto{\pgfqpoint{0.858842in}{2.208197in}}%
\pgfpathlineto{\pgfqpoint{3.465868in}{2.208197in}}%
\pgfusepath{stroke}%
\end{pgfscope}%
\begin{pgfscope}%
\pgfpathrectangle{\pgfqpoint{0.689290in}{1.439739in}}{\pgfqpoint{3.065780in}{1.664355in}}%
\pgfusepath{clip}%
\pgfsetbuttcap%
\pgfsetroundjoin%
\pgfsetlinewidth{1.003750pt}%
\definecolor{currentstroke}{rgb}{0.000000,0.000000,0.000000}%
\pgfsetstrokecolor{currentstroke}%
\pgfsetdash{}{0pt}%
\pgfpathmoveto{\pgfqpoint{0.858842in}{2.418055in}}%
\pgfpathlineto{\pgfqpoint{3.465868in}{2.418055in}}%
\pgfusepath{stroke}%
\end{pgfscope}%
\begin{pgfscope}%
\pgfpathrectangle{\pgfqpoint{0.689290in}{1.439739in}}{\pgfqpoint{3.065780in}{1.664355in}}%
\pgfusepath{clip}%
\pgfsetbuttcap%
\pgfsetroundjoin%
\pgfsetlinewidth{1.003750pt}%
\definecolor{currentstroke}{rgb}{0.000000,0.000000,0.000000}%
\pgfsetstrokecolor{currentstroke}%
\pgfsetdash{}{0pt}%
\pgfpathmoveto{\pgfqpoint{0.858842in}{2.657524in}}%
\pgfpathlineto{\pgfqpoint{3.465868in}{2.657524in}}%
\pgfusepath{stroke}%
\end{pgfscope}%
\begin{pgfscope}%
\pgfpathrectangle{\pgfqpoint{0.689290in}{1.439739in}}{\pgfqpoint{3.065780in}{1.664355in}}%
\pgfusepath{clip}%
\pgfsetbuttcap%
\pgfsetroundjoin%
\pgfsetlinewidth{1.003750pt}%
\definecolor{currentstroke}{rgb}{0.000000,0.000000,0.000000}%
\pgfsetstrokecolor{currentstroke}%
\pgfsetdash{}{0pt}%
\pgfpathmoveto{\pgfqpoint{0.858842in}{2.926458in}}%
\pgfpathlineto{\pgfqpoint{3.465868in}{2.926458in}}%
\pgfusepath{stroke}%
\end{pgfscope}%
\begin{pgfscope}%
\pgfsetroundcap%
\pgfsetroundjoin%
\pgfsetlinewidth{1.003750pt}%
\definecolor{currentstroke}{rgb}{0.000000,0.000000,0.000000}%
\pgfsetstrokecolor{currentstroke}%
\pgfsetdash{}{0pt}%
\pgfpathmoveto{\pgfqpoint{0.612646in}{3.062485in}}%
\pgfpathlineto{\pgfqpoint{0.765935in}{3.145703in}}%
\pgfusepath{stroke}%
\end{pgfscope}%
\begin{pgfscope}%
\pgfsetrectcap%
\pgfsetmiterjoin%
\pgfsetlinewidth{0.803000pt}%
\definecolor{currentstroke}{rgb}{0.000000,0.000000,0.000000}%
\pgfsetstrokecolor{currentstroke}%
\pgfsetdash{}{0pt}%
\pgfpathmoveto{\pgfqpoint{0.689290in}{1.439739in}}%
\pgfpathlineto{\pgfqpoint{0.689290in}{3.104094in}}%
\pgfusepath{stroke}%
\end{pgfscope}%
\begin{pgfscope}%
\pgfsetrectcap%
\pgfsetmiterjoin%
\pgfsetlinewidth{0.803000pt}%
\definecolor{currentstroke}{rgb}{0.000000,0.000000,0.000000}%
\pgfsetstrokecolor{currentstroke}%
\pgfsetdash{}{0pt}%
\pgfpathmoveto{\pgfqpoint{0.689290in}{1.439739in}}%
\pgfpathlineto{\pgfqpoint{3.755070in}{1.439739in}}%
\pgfusepath{stroke}%
\end{pgfscope}%
\begin{pgfscope}%
\pgfsetroundcap%
\pgfsetroundjoin%
\pgfsetlinewidth{1.003750pt}%
\definecolor{currentstroke}{rgb}{0.000000,0.000000,0.000000}%
\pgfsetstrokecolor{currentstroke}%
\pgfsetdash{}{0pt}%
\pgfpathmoveto{\pgfqpoint{0.532207in}{1.542375in}}%
\pgfpathquadraticcurveto{\pgfqpoint{0.558657in}{1.554007in}}{\pgfqpoint{0.585107in}{1.565639in}}%
\pgfusepath{stroke}%
\end{pgfscope}%
\begin{pgfscope}%
\pgfsetbuttcap%
\pgfsetmiterjoin%
\definecolor{currentfill}{rgb}{1.000000,1.000000,1.000000}%
\pgfsetfillcolor{currentfill}%
\pgfsetlinewidth{1.003750pt}%
\definecolor{currentstroke}{rgb}{1.000000,1.000000,1.000000}%
\pgfsetstrokecolor{currentstroke}%
\pgfsetdash{}{0pt}%
\pgfpathmoveto{\pgfqpoint{0.354974in}{1.483301in}}%
\pgfpathlineto{\pgfqpoint{0.571328in}{1.483301in}}%
\pgfpathlineto{\pgfqpoint{0.571328in}{1.540708in}}%
\pgfpathlineto{\pgfqpoint{0.354974in}{1.540708in}}%
\pgfpathclose%
\pgfusepath{stroke,fill}%
\end{pgfscope}%
\begin{pgfscope}%
\definecolor{textcolor}{rgb}{0.000000,0.000000,0.000000}%
\pgfsetstrokecolor{textcolor}%
\pgfsetfillcolor{textcolor}%
\pgftext[x=0.346640in,y=1.549042in,left,top]{\color{textcolor}\rmfamily\fontsize{6.000000}{7.200000}\selectfont 0.184}%
\end{pgfscope}%
\begin{pgfscope}%
\definecolor{textcolor}{rgb}{0.000000,0.000000,0.000000}%
\pgfsetstrokecolor{textcolor}%
\pgfsetfillcolor{textcolor}%
\pgftext[x=3.701958in,y=1.492515in,left,base]{\color{textcolor}\rmfamily\fontsize{6.000000}{7.200000}\selectfont 0}%
\end{pgfscope}%
\begin{pgfscope}%
\definecolor{textcolor}{rgb}{0.000000,0.000000,0.000000}%
\pgfsetstrokecolor{textcolor}%
\pgfsetfillcolor{textcolor}%
\pgftext[x=3.701958in,y=1.581288in,left,base]{\color{textcolor}\rmfamily\fontsize{6.000000}{7.200000}\selectfont 1}%
\end{pgfscope}%
\begin{pgfscope}%
\definecolor{textcolor}{rgb}{0.000000,0.000000,0.000000}%
\pgfsetstrokecolor{textcolor}%
\pgfsetfillcolor{textcolor}%
\pgftext[x=3.701958in,y=1.642427in,left,base]{\color{textcolor}\rmfamily\fontsize{6.000000}{7.200000}\selectfont 2}%
\end{pgfscope}%
\begin{pgfscope}%
\definecolor{textcolor}{rgb}{0.000000,0.000000,0.000000}%
\pgfsetstrokecolor{textcolor}%
\pgfsetfillcolor{textcolor}%
\pgftext[x=3.701958in,y=1.734089in,left,base]{\color{textcolor}\rmfamily\fontsize{6.000000}{7.200000}\selectfont 3}%
\end{pgfscope}%
\begin{pgfscope}%
\definecolor{textcolor}{rgb}{0.000000,0.000000,0.000000}%
\pgfsetstrokecolor{textcolor}%
\pgfsetfillcolor{textcolor}%
\pgftext[x=3.701958in,y=1.856218in,left,base]{\color{textcolor}\rmfamily\fontsize{6.000000}{7.200000}\selectfont 4}%
\end{pgfscope}%
\begin{pgfscope}%
\definecolor{textcolor}{rgb}{0.000000,0.000000,0.000000}%
\pgfsetstrokecolor{textcolor}%
\pgfsetfillcolor{textcolor}%
\pgftext[x=3.701958in,y=2.008738in,left,base]{\color{textcolor}\rmfamily\fontsize{6.000000}{7.200000}\selectfont 5}%
\end{pgfscope}%
\begin{pgfscope}%
\definecolor{textcolor}{rgb}{0.000000,0.000000,0.000000}%
\pgfsetstrokecolor{textcolor}%
\pgfsetfillcolor{textcolor}%
\pgftext[x=3.701958in,y=2.191558in,left,base]{\color{textcolor}\rmfamily\fontsize{6.000000}{7.200000}\selectfont 6}%
\end{pgfscope}%
\begin{pgfscope}%
\definecolor{textcolor}{rgb}{0.000000,0.000000,0.000000}%
\pgfsetstrokecolor{textcolor}%
\pgfsetfillcolor{textcolor}%
\pgftext[x=3.701958in,y=2.404563in,left,base]{\color{textcolor}\rmfamily\fontsize{6.000000}{7.200000}\selectfont 7}%
\end{pgfscope}%
\begin{pgfscope}%
\definecolor{textcolor}{rgb}{0.000000,0.000000,0.000000}%
\pgfsetstrokecolor{textcolor}%
\pgfsetfillcolor{textcolor}%
\pgftext[x=3.701958in,y=2.647625in,left,base]{\color{textcolor}\rmfamily\fontsize{6.000000}{7.200000}\selectfont 8}%
\end{pgfscope}%
\begin{pgfscope}%
\definecolor{textcolor}{rgb}{0.000000,0.000000,0.000000}%
\pgfsetstrokecolor{textcolor}%
\pgfsetfillcolor{textcolor}%
\pgftext[x=3.701958in,y=2.920592in,left,base]{\color{textcolor}\rmfamily\fontsize{6.000000}{7.200000}\selectfont 9}%
\end{pgfscope}%
\begin{pgfscope}%
\definecolor{textcolor}{rgb}{0.000000,0.000000,0.000000}%
\pgfsetstrokecolor{textcolor}%
\pgfsetfillcolor{textcolor}%
\pgftext[x=3.498492in,y=2.920592in,left,base]{\color{textcolor}\rmfamily\fontsize{6.000000}{7.200000}\selectfont \(\displaystyle J\,''\!=\)}%
\end{pgfscope}%
\begin{pgfscope}%
\definecolor{textcolor}{rgb}{0.000000,0.000000,0.000000}%
\pgfsetstrokecolor{textcolor}%
\pgfsetfillcolor{textcolor}%
\pgftext[x=0.812703in, y=2.011876in, left, base,rotate=90.000000]{\color{textcolor}\rmfamily\fontsize{6.000000}{7.200000}\selectfont \(\displaystyle v\,''\!=0\)}%
\end{pgfscope}%
\begin{pgfscope}%
\pgfsetbuttcap%
\pgfsetmiterjoin%
\definecolor{currentfill}{rgb}{1.000000,1.000000,1.000000}%
\pgfsetfillcolor{currentfill}%
\pgfsetlinewidth{0.000000pt}%
\definecolor{currentstroke}{rgb}{0.000000,0.000000,0.000000}%
\pgfsetstrokecolor{currentstroke}%
\pgfsetstrokeopacity{0.000000}%
\pgfsetdash{}{0pt}%
\pgfpathmoveto{\pgfqpoint{0.689290in}{0.468380in}}%
\pgfpathlineto{\pgfqpoint{3.755070in}{0.468380in}}%
\pgfpathlineto{\pgfqpoint{3.755070in}{1.313040in}}%
\pgfpathlineto{\pgfqpoint{0.689290in}{1.313040in}}%
\pgfpathclose%
\pgfusepath{fill}%
\end{pgfscope}%
\begin{pgfscope}%
\pgfpathrectangle{\pgfqpoint{0.689290in}{0.468380in}}{\pgfqpoint{3.065780in}{0.844660in}}%
\pgfusepath{clip}%
\pgfsetbuttcap%
\pgfsetroundjoin%
\pgfsetlinewidth{0.501875pt}%
\definecolor{currentstroke}{rgb}{0.700000,0.700000,0.700000}%
\pgfsetstrokecolor{currentstroke}%
\pgfsetstrokeopacity{0.200000}%
\pgfsetdash{{1.850000pt}{0.800000pt}}{0.000000pt}%
\pgfpathmoveto{\pgfqpoint{1.109402in}{0.468380in}}%
\pgfpathlineto{\pgfqpoint{1.109402in}{1.313040in}}%
\pgfusepath{stroke}%
\end{pgfscope}%
\begin{pgfscope}%
\pgfsetbuttcap%
\pgfsetroundjoin%
\definecolor{currentfill}{rgb}{0.000000,0.000000,0.000000}%
\pgfsetfillcolor{currentfill}%
\pgfsetlinewidth{1.003750pt}%
\definecolor{currentstroke}{rgb}{0.000000,0.000000,0.000000}%
\pgfsetstrokecolor{currentstroke}%
\pgfsetdash{}{0pt}%
\pgfsys@defobject{currentmarker}{\pgfqpoint{0.000000in}{-0.083333in}}{\pgfqpoint{0.000000in}{0.000000in}}{%
\pgfpathmoveto{\pgfqpoint{0.000000in}{0.000000in}}%
\pgfpathlineto{\pgfqpoint{0.000000in}{-0.083333in}}%
\pgfusepath{stroke,fill}%
}%
\begin{pgfscope}%
\pgfsys@transformshift{1.109402in}{0.468380in}%
\pgfsys@useobject{currentmarker}{}%
\end{pgfscope}%
\end{pgfscope}%
\begin{pgfscope}%
\definecolor{textcolor}{rgb}{0.000000,0.000000,0.000000}%
\pgfsetstrokecolor{textcolor}%
\pgfsetfillcolor{textcolor}%
\pgftext[x=1.109402in,y=0.357268in,,top]{\color{textcolor}\rmfamily\fontsize{6.000000}{7.200000}\selectfont \(\displaystyle 2700\)}%
\end{pgfscope}%
\begin{pgfscope}%
\pgfpathrectangle{\pgfqpoint{0.689290in}{0.468380in}}{\pgfqpoint{3.065780in}{0.844660in}}%
\pgfusepath{clip}%
\pgfsetbuttcap%
\pgfsetroundjoin%
\pgfsetlinewidth{0.501875pt}%
\definecolor{currentstroke}{rgb}{0.700000,0.700000,0.700000}%
\pgfsetstrokecolor{currentstroke}%
\pgfsetstrokeopacity{0.200000}%
\pgfsetdash{{1.850000pt}{0.800000pt}}{0.000000pt}%
\pgfpathmoveto{\pgfqpoint{1.765745in}{0.468380in}}%
\pgfpathlineto{\pgfqpoint{1.765745in}{1.313040in}}%
\pgfusepath{stroke}%
\end{pgfscope}%
\begin{pgfscope}%
\pgfsetbuttcap%
\pgfsetroundjoin%
\definecolor{currentfill}{rgb}{0.000000,0.000000,0.000000}%
\pgfsetfillcolor{currentfill}%
\pgfsetlinewidth{1.003750pt}%
\definecolor{currentstroke}{rgb}{0.000000,0.000000,0.000000}%
\pgfsetstrokecolor{currentstroke}%
\pgfsetdash{}{0pt}%
\pgfsys@defobject{currentmarker}{\pgfqpoint{0.000000in}{-0.083333in}}{\pgfqpoint{0.000000in}{0.000000in}}{%
\pgfpathmoveto{\pgfqpoint{0.000000in}{0.000000in}}%
\pgfpathlineto{\pgfqpoint{0.000000in}{-0.083333in}}%
\pgfusepath{stroke,fill}%
}%
\begin{pgfscope}%
\pgfsys@transformshift{1.765745in}{0.468380in}%
\pgfsys@useobject{currentmarker}{}%
\end{pgfscope}%
\end{pgfscope}%
\begin{pgfscope}%
\definecolor{textcolor}{rgb}{0.000000,0.000000,0.000000}%
\pgfsetstrokecolor{textcolor}%
\pgfsetfillcolor{textcolor}%
\pgftext[x=1.765745in,y=0.357268in,,top]{\color{textcolor}\rmfamily\fontsize{6.000000}{7.200000}\selectfont \(\displaystyle 2800\)}%
\end{pgfscope}%
\begin{pgfscope}%
\pgfpathrectangle{\pgfqpoint{0.689290in}{0.468380in}}{\pgfqpoint{3.065780in}{0.844660in}}%
\pgfusepath{clip}%
\pgfsetbuttcap%
\pgfsetroundjoin%
\pgfsetlinewidth{0.501875pt}%
\definecolor{currentstroke}{rgb}{0.700000,0.700000,0.700000}%
\pgfsetstrokecolor{currentstroke}%
\pgfsetstrokeopacity{0.200000}%
\pgfsetdash{{1.850000pt}{0.800000pt}}{0.000000pt}%
\pgfpathmoveto{\pgfqpoint{2.422088in}{0.468380in}}%
\pgfpathlineto{\pgfqpoint{2.422088in}{1.313040in}}%
\pgfusepath{stroke}%
\end{pgfscope}%
\begin{pgfscope}%
\pgfsetbuttcap%
\pgfsetroundjoin%
\definecolor{currentfill}{rgb}{0.000000,0.000000,0.000000}%
\pgfsetfillcolor{currentfill}%
\pgfsetlinewidth{1.003750pt}%
\definecolor{currentstroke}{rgb}{0.000000,0.000000,0.000000}%
\pgfsetstrokecolor{currentstroke}%
\pgfsetdash{}{0pt}%
\pgfsys@defobject{currentmarker}{\pgfqpoint{0.000000in}{-0.083333in}}{\pgfqpoint{0.000000in}{0.000000in}}{%
\pgfpathmoveto{\pgfqpoint{0.000000in}{0.000000in}}%
\pgfpathlineto{\pgfqpoint{0.000000in}{-0.083333in}}%
\pgfusepath{stroke,fill}%
}%
\begin{pgfscope}%
\pgfsys@transformshift{2.422088in}{0.468380in}%
\pgfsys@useobject{currentmarker}{}%
\end{pgfscope}%
\end{pgfscope}%
\begin{pgfscope}%
\definecolor{textcolor}{rgb}{0.000000,0.000000,0.000000}%
\pgfsetstrokecolor{textcolor}%
\pgfsetfillcolor{textcolor}%
\pgftext[x=2.422088in,y=0.357268in,,top]{\color{textcolor}\rmfamily\fontsize{6.000000}{7.200000}\selectfont \(\displaystyle 2900\)}%
\end{pgfscope}%
\begin{pgfscope}%
\pgfpathrectangle{\pgfqpoint{0.689290in}{0.468380in}}{\pgfqpoint{3.065780in}{0.844660in}}%
\pgfusepath{clip}%
\pgfsetbuttcap%
\pgfsetroundjoin%
\pgfsetlinewidth{0.501875pt}%
\definecolor{currentstroke}{rgb}{0.700000,0.700000,0.700000}%
\pgfsetstrokecolor{currentstroke}%
\pgfsetstrokeopacity{0.200000}%
\pgfsetdash{{1.850000pt}{0.800000pt}}{0.000000pt}%
\pgfpathmoveto{\pgfqpoint{3.078432in}{0.468380in}}%
\pgfpathlineto{\pgfqpoint{3.078432in}{1.313040in}}%
\pgfusepath{stroke}%
\end{pgfscope}%
\begin{pgfscope}%
\pgfsetbuttcap%
\pgfsetroundjoin%
\definecolor{currentfill}{rgb}{0.000000,0.000000,0.000000}%
\pgfsetfillcolor{currentfill}%
\pgfsetlinewidth{1.003750pt}%
\definecolor{currentstroke}{rgb}{0.000000,0.000000,0.000000}%
\pgfsetstrokecolor{currentstroke}%
\pgfsetdash{}{0pt}%
\pgfsys@defobject{currentmarker}{\pgfqpoint{0.000000in}{-0.083333in}}{\pgfqpoint{0.000000in}{0.000000in}}{%
\pgfpathmoveto{\pgfqpoint{0.000000in}{0.000000in}}%
\pgfpathlineto{\pgfqpoint{0.000000in}{-0.083333in}}%
\pgfusepath{stroke,fill}%
}%
\begin{pgfscope}%
\pgfsys@transformshift{3.078432in}{0.468380in}%
\pgfsys@useobject{currentmarker}{}%
\end{pgfscope}%
\end{pgfscope}%
\begin{pgfscope}%
\definecolor{textcolor}{rgb}{0.000000,0.000000,0.000000}%
\pgfsetstrokecolor{textcolor}%
\pgfsetfillcolor{textcolor}%
\pgftext[x=3.078432in,y=0.357268in,,top]{\color{textcolor}\rmfamily\fontsize{6.000000}{7.200000}\selectfont \(\displaystyle 3000\)}%
\end{pgfscope}%
\begin{pgfscope}%
\pgfpathrectangle{\pgfqpoint{0.689290in}{0.468380in}}{\pgfqpoint{3.065780in}{0.844660in}}%
\pgfusepath{clip}%
\pgfsetbuttcap%
\pgfsetroundjoin%
\pgfsetlinewidth{0.501875pt}%
\definecolor{currentstroke}{rgb}{0.700000,0.700000,0.700000}%
\pgfsetstrokecolor{currentstroke}%
\pgfsetstrokeopacity{0.200000}%
\pgfsetdash{{1.850000pt}{0.800000pt}}{0.000000pt}%
\pgfpathmoveto{\pgfqpoint{3.734775in}{0.468380in}}%
\pgfpathlineto{\pgfqpoint{3.734775in}{1.313040in}}%
\pgfusepath{stroke}%
\end{pgfscope}%
\begin{pgfscope}%
\pgfsetbuttcap%
\pgfsetroundjoin%
\definecolor{currentfill}{rgb}{0.000000,0.000000,0.000000}%
\pgfsetfillcolor{currentfill}%
\pgfsetlinewidth{1.003750pt}%
\definecolor{currentstroke}{rgb}{0.000000,0.000000,0.000000}%
\pgfsetstrokecolor{currentstroke}%
\pgfsetdash{}{0pt}%
\pgfsys@defobject{currentmarker}{\pgfqpoint{0.000000in}{-0.083333in}}{\pgfqpoint{0.000000in}{0.000000in}}{%
\pgfpathmoveto{\pgfqpoint{0.000000in}{0.000000in}}%
\pgfpathlineto{\pgfqpoint{0.000000in}{-0.083333in}}%
\pgfusepath{stroke,fill}%
}%
\begin{pgfscope}%
\pgfsys@transformshift{3.734775in}{0.468380in}%
\pgfsys@useobject{currentmarker}{}%
\end{pgfscope}%
\end{pgfscope}%
\begin{pgfscope}%
\definecolor{textcolor}{rgb}{0.000000,0.000000,0.000000}%
\pgfsetstrokecolor{textcolor}%
\pgfsetfillcolor{textcolor}%
\pgftext[x=3.734775in,y=0.357268in,,top]{\color{textcolor}\rmfamily\fontsize{6.000000}{7.200000}\selectfont \(\displaystyle 3100\)}%
\end{pgfscope}%
\begin{pgfscope}%
\definecolor{textcolor}{rgb}{0.000000,0.000000,0.000000}%
\pgfsetstrokecolor{textcolor}%
\pgfsetfillcolor{textcolor}%
\pgftext[x=2.222180in,y=0.227639in,,top]{\color{textcolor}\rmfamily\fontsize{8.000000}{9.600000}\selectfont Wavenumber (cm\(\displaystyle ^{-1}\))}%
\end{pgfscope}%
\begin{pgfscope}%
\pgfpathrectangle{\pgfqpoint{0.689290in}{0.468380in}}{\pgfqpoint{3.065780in}{0.844660in}}%
\pgfusepath{clip}%
\pgfsetbuttcap%
\pgfsetroundjoin%
\pgfsetlinewidth{0.501875pt}%
\definecolor{currentstroke}{rgb}{0.700000,0.700000,0.700000}%
\pgfsetstrokecolor{currentstroke}%
\pgfsetstrokeopacity{0.200000}%
\pgfsetdash{{1.850000pt}{0.800000pt}}{0.000000pt}%
\pgfpathmoveto{\pgfqpoint{0.689290in}{0.488981in}}%
\pgfpathlineto{\pgfqpoint{3.755070in}{0.488981in}}%
\pgfusepath{stroke}%
\end{pgfscope}%
\begin{pgfscope}%
\pgfsetbuttcap%
\pgfsetroundjoin%
\definecolor{currentfill}{rgb}{0.000000,0.000000,0.000000}%
\pgfsetfillcolor{currentfill}%
\pgfsetlinewidth{1.003750pt}%
\definecolor{currentstroke}{rgb}{0.000000,0.000000,0.000000}%
\pgfsetstrokecolor{currentstroke}%
\pgfsetdash{}{0pt}%
\pgfsys@defobject{currentmarker}{\pgfqpoint{-0.083333in}{0.000000in}}{\pgfqpoint{0.000000in}{0.000000in}}{%
\pgfpathmoveto{\pgfqpoint{0.000000in}{0.000000in}}%
\pgfpathlineto{\pgfqpoint{-0.083333in}{0.000000in}}%
\pgfusepath{stroke,fill}%
}%
\begin{pgfscope}%
\pgfsys@transformshift{0.689290in}{0.488981in}%
\pgfsys@useobject{currentmarker}{}%
\end{pgfscope}%
\end{pgfscope}%
\begin{pgfscope}%
\definecolor{textcolor}{rgb}{0.000000,0.000000,0.000000}%
\pgfsetstrokecolor{textcolor}%
\pgfsetfillcolor{textcolor}%
\pgftext[x=0.393768in, y=0.460046in, left, base]{\color{textcolor}\rmfamily\fontsize{6.000000}{7.200000}\selectfont \(\displaystyle 0.00\)}%
\end{pgfscope}%
\begin{pgfscope}%
\pgfpathrectangle{\pgfqpoint{0.689290in}{0.468380in}}{\pgfqpoint{3.065780in}{0.844660in}}%
\pgfusepath{clip}%
\pgfsetbuttcap%
\pgfsetroundjoin%
\pgfsetlinewidth{0.501875pt}%
\definecolor{currentstroke}{rgb}{0.700000,0.700000,0.700000}%
\pgfsetstrokecolor{currentstroke}%
\pgfsetstrokeopacity{0.200000}%
\pgfsetdash{{1.850000pt}{0.800000pt}}{0.000000pt}%
\pgfpathmoveto{\pgfqpoint{0.689290in}{0.694996in}}%
\pgfpathlineto{\pgfqpoint{3.755070in}{0.694996in}}%
\pgfusepath{stroke}%
\end{pgfscope}%
\begin{pgfscope}%
\pgfsetbuttcap%
\pgfsetroundjoin%
\definecolor{currentfill}{rgb}{0.000000,0.000000,0.000000}%
\pgfsetfillcolor{currentfill}%
\pgfsetlinewidth{1.003750pt}%
\definecolor{currentstroke}{rgb}{0.000000,0.000000,0.000000}%
\pgfsetstrokecolor{currentstroke}%
\pgfsetdash{}{0pt}%
\pgfsys@defobject{currentmarker}{\pgfqpoint{-0.083333in}{0.000000in}}{\pgfqpoint{0.000000in}{0.000000in}}{%
\pgfpathmoveto{\pgfqpoint{0.000000in}{0.000000in}}%
\pgfpathlineto{\pgfqpoint{-0.083333in}{0.000000in}}%
\pgfusepath{stroke,fill}%
}%
\begin{pgfscope}%
\pgfsys@transformshift{0.689290in}{0.694996in}%
\pgfsys@useobject{currentmarker}{}%
\end{pgfscope}%
\end{pgfscope}%
\begin{pgfscope}%
\definecolor{textcolor}{rgb}{0.000000,0.000000,0.000000}%
\pgfsetstrokecolor{textcolor}%
\pgfsetfillcolor{textcolor}%
\pgftext[x=0.393768in, y=0.666061in, left, base]{\color{textcolor}\rmfamily\fontsize{6.000000}{7.200000}\selectfont \(\displaystyle 0.05\)}%
\end{pgfscope}%
\begin{pgfscope}%
\pgfpathrectangle{\pgfqpoint{0.689290in}{0.468380in}}{\pgfqpoint{3.065780in}{0.844660in}}%
\pgfusepath{clip}%
\pgfsetbuttcap%
\pgfsetroundjoin%
\pgfsetlinewidth{0.501875pt}%
\definecolor{currentstroke}{rgb}{0.700000,0.700000,0.700000}%
\pgfsetstrokecolor{currentstroke}%
\pgfsetstrokeopacity{0.200000}%
\pgfsetdash{{1.850000pt}{0.800000pt}}{0.000000pt}%
\pgfpathmoveto{\pgfqpoint{0.689290in}{0.901010in}}%
\pgfpathlineto{\pgfqpoint{3.755070in}{0.901010in}}%
\pgfusepath{stroke}%
\end{pgfscope}%
\begin{pgfscope}%
\pgfsetbuttcap%
\pgfsetroundjoin%
\definecolor{currentfill}{rgb}{0.000000,0.000000,0.000000}%
\pgfsetfillcolor{currentfill}%
\pgfsetlinewidth{1.003750pt}%
\definecolor{currentstroke}{rgb}{0.000000,0.000000,0.000000}%
\pgfsetstrokecolor{currentstroke}%
\pgfsetdash{}{0pt}%
\pgfsys@defobject{currentmarker}{\pgfqpoint{-0.083333in}{0.000000in}}{\pgfqpoint{0.000000in}{0.000000in}}{%
\pgfpathmoveto{\pgfqpoint{0.000000in}{0.000000in}}%
\pgfpathlineto{\pgfqpoint{-0.083333in}{0.000000in}}%
\pgfusepath{stroke,fill}%
}%
\begin{pgfscope}%
\pgfsys@transformshift{0.689290in}{0.901010in}%
\pgfsys@useobject{currentmarker}{}%
\end{pgfscope}%
\end{pgfscope}%
\begin{pgfscope}%
\definecolor{textcolor}{rgb}{0.000000,0.000000,0.000000}%
\pgfsetstrokecolor{textcolor}%
\pgfsetfillcolor{textcolor}%
\pgftext[x=0.393768in, y=0.872075in, left, base]{\color{textcolor}\rmfamily\fontsize{6.000000}{7.200000}\selectfont \(\displaystyle 0.10\)}%
\end{pgfscope}%
\begin{pgfscope}%
\pgfpathrectangle{\pgfqpoint{0.689290in}{0.468380in}}{\pgfqpoint{3.065780in}{0.844660in}}%
\pgfusepath{clip}%
\pgfsetbuttcap%
\pgfsetroundjoin%
\pgfsetlinewidth{0.501875pt}%
\definecolor{currentstroke}{rgb}{0.700000,0.700000,0.700000}%
\pgfsetstrokecolor{currentstroke}%
\pgfsetstrokeopacity{0.200000}%
\pgfsetdash{{1.850000pt}{0.800000pt}}{0.000000pt}%
\pgfpathmoveto{\pgfqpoint{0.689290in}{1.107025in}}%
\pgfpathlineto{\pgfqpoint{3.755070in}{1.107025in}}%
\pgfusepath{stroke}%
\end{pgfscope}%
\begin{pgfscope}%
\pgfsetbuttcap%
\pgfsetroundjoin%
\definecolor{currentfill}{rgb}{0.000000,0.000000,0.000000}%
\pgfsetfillcolor{currentfill}%
\pgfsetlinewidth{1.003750pt}%
\definecolor{currentstroke}{rgb}{0.000000,0.000000,0.000000}%
\pgfsetstrokecolor{currentstroke}%
\pgfsetdash{}{0pt}%
\pgfsys@defobject{currentmarker}{\pgfqpoint{-0.083333in}{0.000000in}}{\pgfqpoint{0.000000in}{0.000000in}}{%
\pgfpathmoveto{\pgfqpoint{0.000000in}{0.000000in}}%
\pgfpathlineto{\pgfqpoint{-0.083333in}{0.000000in}}%
\pgfusepath{stroke,fill}%
}%
\begin{pgfscope}%
\pgfsys@transformshift{0.689290in}{1.107025in}%
\pgfsys@useobject{currentmarker}{}%
\end{pgfscope}%
\end{pgfscope}%
\begin{pgfscope}%
\definecolor{textcolor}{rgb}{0.000000,0.000000,0.000000}%
\pgfsetstrokecolor{textcolor}%
\pgfsetfillcolor{textcolor}%
\pgftext[x=0.393768in, y=1.078090in, left, base]{\color{textcolor}\rmfamily\fontsize{6.000000}{7.200000}\selectfont \(\displaystyle 0.15\)}%
\end{pgfscope}%
\begin{pgfscope}%
\pgfpathrectangle{\pgfqpoint{0.689290in}{0.468380in}}{\pgfqpoint{3.065780in}{0.844660in}}%
\pgfusepath{clip}%
\pgfsetbuttcap%
\pgfsetroundjoin%
\pgfsetlinewidth{0.501875pt}%
\definecolor{currentstroke}{rgb}{0.700000,0.700000,0.700000}%
\pgfsetstrokecolor{currentstroke}%
\pgfsetstrokeopacity{0.200000}%
\pgfsetdash{{1.850000pt}{0.800000pt}}{0.000000pt}%
\pgfpathmoveto{\pgfqpoint{0.689290in}{1.313040in}}%
\pgfpathlineto{\pgfqpoint{3.755070in}{1.313040in}}%
\pgfusepath{stroke}%
\end{pgfscope}%
\begin{pgfscope}%
\pgfsetbuttcap%
\pgfsetroundjoin%
\definecolor{currentfill}{rgb}{0.000000,0.000000,0.000000}%
\pgfsetfillcolor{currentfill}%
\pgfsetlinewidth{1.003750pt}%
\definecolor{currentstroke}{rgb}{0.000000,0.000000,0.000000}%
\pgfsetstrokecolor{currentstroke}%
\pgfsetdash{}{0pt}%
\pgfsys@defobject{currentmarker}{\pgfqpoint{-0.083333in}{0.000000in}}{\pgfqpoint{0.000000in}{0.000000in}}{%
\pgfpathmoveto{\pgfqpoint{0.000000in}{0.000000in}}%
\pgfpathlineto{\pgfqpoint{-0.083333in}{0.000000in}}%
\pgfusepath{stroke,fill}%
}%
\begin{pgfscope}%
\pgfsys@transformshift{0.689290in}{1.313040in}%
\pgfsys@useobject{currentmarker}{}%
\end{pgfscope}%
\end{pgfscope}%
\begin{pgfscope}%
\definecolor{textcolor}{rgb}{0.000000,0.000000,0.000000}%
\pgfsetstrokecolor{textcolor}%
\pgfsetfillcolor{textcolor}%
\pgftext[x=0.393768in, y=1.284105in, left, base]{\color{textcolor}\rmfamily\fontsize{6.000000}{7.200000}\selectfont \(\displaystyle 0.20\)}%
\end{pgfscope}%
\begin{pgfscope}%
\definecolor{textcolor}{rgb}{0.000000,0.000000,0.000000}%
\pgfsetstrokecolor{textcolor}%
\pgfsetfillcolor{textcolor}%
\pgftext[x=0.198766in,y=0.890710in,,bottom,rotate=90.000000]{\color{textcolor}\rmfamily\fontsize{8.000000}{9.600000}\selectfont Absorbance}%
\end{pgfscope}%
\begin{pgfscope}%
\pgfpathrectangle{\pgfqpoint{0.689290in}{0.468380in}}{\pgfqpoint{3.065780in}{0.844660in}}%
\pgfusepath{clip}%
\pgfsetroundcap%
\pgfsetroundjoin%
\pgfsetlinewidth{1.003750pt}%
\definecolor{currentstroke}{rgb}{0.000000,0.000000,0.000000}%
\pgfsetstrokecolor{currentstroke}%
\pgfsetdash{}{0pt}%
\pgfpathmoveto{\pgfqpoint{0.787742in}{0.488981in}}%
\pgfpathlineto{\pgfqpoint{0.919667in}{0.489311in}}%
\pgfpathlineto{\pgfqpoint{0.925574in}{0.490064in}}%
\pgfpathlineto{\pgfqpoint{0.930168in}{0.491353in}}%
\pgfpathlineto{\pgfqpoint{0.934106in}{0.493196in}}%
\pgfpathlineto{\pgfqpoint{0.938045in}{0.495827in}}%
\pgfpathlineto{\pgfqpoint{0.943295in}{0.500342in}}%
\pgfpathlineto{\pgfqpoint{0.950515in}{0.506538in}}%
\pgfpathlineto{\pgfqpoint{0.953797in}{0.508343in}}%
\pgfpathlineto{\pgfqpoint{0.956422in}{0.508998in}}%
\pgfpathlineto{\pgfqpoint{0.959047in}{0.508864in}}%
\pgfpathlineto{\pgfqpoint{0.961673in}{0.507956in}}%
\pgfpathlineto{\pgfqpoint{0.964955in}{0.505901in}}%
\pgfpathlineto{\pgfqpoint{0.969549in}{0.501955in}}%
\pgfpathlineto{\pgfqpoint{0.977425in}{0.495170in}}%
\pgfpathlineto{\pgfqpoint{0.981363in}{0.492716in}}%
\pgfpathlineto{\pgfqpoint{0.985301in}{0.491041in}}%
\pgfpathlineto{\pgfqpoint{0.989896in}{0.489899in}}%
\pgfpathlineto{\pgfqpoint{0.995803in}{0.489252in}}%
\pgfpathlineto{\pgfqpoint{1.006304in}{0.489000in}}%
\pgfpathlineto{\pgfqpoint{1.075220in}{0.489031in}}%
\pgfpathlineto{\pgfqpoint{1.084409in}{0.489494in}}%
\pgfpathlineto{\pgfqpoint{1.089003in}{0.490351in}}%
\pgfpathlineto{\pgfqpoint{1.092942in}{0.491866in}}%
\pgfpathlineto{\pgfqpoint{1.096223in}{0.493991in}}%
\pgfpathlineto{\pgfqpoint{1.099505in}{0.497155in}}%
\pgfpathlineto{\pgfqpoint{1.102787in}{0.501508in}}%
\pgfpathlineto{\pgfqpoint{1.106725in}{0.508236in}}%
\pgfpathlineto{\pgfqpoint{1.117883in}{0.528908in}}%
\pgfpathlineto{\pgfqpoint{1.120508in}{0.531658in}}%
\pgfpathlineto{\pgfqpoint{1.122477in}{0.532681in}}%
\pgfpathlineto{\pgfqpoint{1.124446in}{0.532733in}}%
\pgfpathlineto{\pgfqpoint{1.126415in}{0.531811in}}%
\pgfpathlineto{\pgfqpoint{1.128384in}{0.529975in}}%
\pgfpathlineto{\pgfqpoint{1.131009in}{0.526320in}}%
\pgfpathlineto{\pgfqpoint{1.134948in}{0.519093in}}%
\pgfpathlineto{\pgfqpoint{1.142824in}{0.503931in}}%
\pgfpathlineto{\pgfqpoint{1.146762in}{0.498185in}}%
\pgfpathlineto{\pgfqpoint{1.150043in}{0.494716in}}%
\pgfpathlineto{\pgfqpoint{1.153325in}{0.492338in}}%
\pgfpathlineto{\pgfqpoint{1.156607in}{0.490827in}}%
\pgfpathlineto{\pgfqpoint{1.160545in}{0.489810in}}%
\pgfpathlineto{\pgfqpoint{1.166452in}{0.489192in}}%
\pgfpathlineto{\pgfqpoint{1.177610in}{0.488990in}}%
\pgfpathlineto{\pgfqpoint{1.240619in}{0.489172in}}%
\pgfpathlineto{\pgfqpoint{1.246526in}{0.489814in}}%
\pgfpathlineto{\pgfqpoint{1.250464in}{0.490966in}}%
\pgfpathlineto{\pgfqpoint{1.253746in}{0.492800in}}%
\pgfpathlineto{\pgfqpoint{1.256371in}{0.495144in}}%
\pgfpathlineto{\pgfqpoint{1.258996in}{0.498538in}}%
\pgfpathlineto{\pgfqpoint{1.261622in}{0.503220in}}%
\pgfpathlineto{\pgfqpoint{1.264247in}{0.509364in}}%
\pgfpathlineto{\pgfqpoint{1.267529in}{0.519149in}}%
\pgfpathlineto{\pgfqpoint{1.271467in}{0.533450in}}%
\pgfpathlineto{\pgfqpoint{1.279343in}{0.562743in}}%
\pgfpathlineto{\pgfqpoint{1.281968in}{0.569583in}}%
\pgfpathlineto{\pgfqpoint{1.283937in}{0.572894in}}%
\pgfpathlineto{\pgfqpoint{1.285250in}{0.574106in}}%
\pgfpathlineto{\pgfqpoint{1.286563in}{0.574476in}}%
\pgfpathlineto{\pgfqpoint{1.287876in}{0.573994in}}%
\pgfpathlineto{\pgfqpoint{1.289188in}{0.572673in}}%
\pgfpathlineto{\pgfqpoint{1.291157in}{0.569211in}}%
\pgfpathlineto{\pgfqpoint{1.293126in}{0.564182in}}%
\pgfpathlineto{\pgfqpoint{1.296408in}{0.553198in}}%
\pgfpathlineto{\pgfqpoint{1.307566in}{0.512502in}}%
\pgfpathlineto{\pgfqpoint{1.310848in}{0.504237in}}%
\pgfpathlineto{\pgfqpoint{1.313473in}{0.499297in}}%
\pgfpathlineto{\pgfqpoint{1.316098in}{0.495683in}}%
\pgfpathlineto{\pgfqpoint{1.318724in}{0.493164in}}%
\pgfpathlineto{\pgfqpoint{1.322005in}{0.491175in}}%
\pgfpathlineto{\pgfqpoint{1.325287in}{0.490062in}}%
\pgfpathlineto{\pgfqpoint{1.329882in}{0.489342in}}%
\pgfpathlineto{\pgfqpoint{1.337758in}{0.489023in}}%
\pgfpathlineto{\pgfqpoint{1.373857in}{0.488981in}}%
\pgfpathlineto{\pgfqpoint{1.400110in}{0.489315in}}%
\pgfpathlineto{\pgfqpoint{1.404705in}{0.490046in}}%
\pgfpathlineto{\pgfqpoint{1.407986in}{0.491244in}}%
\pgfpathlineto{\pgfqpoint{1.410612in}{0.492936in}}%
\pgfpathlineto{\pgfqpoint{1.413237in}{0.495621in}}%
\pgfpathlineto{\pgfqpoint{1.415862in}{0.499694in}}%
\pgfpathlineto{\pgfqpoint{1.418488in}{0.505586in}}%
\pgfpathlineto{\pgfqpoint{1.421113in}{0.513710in}}%
\pgfpathlineto{\pgfqpoint{1.423739in}{0.524364in}}%
\pgfpathlineto{\pgfqpoint{1.427020in}{0.541323in}}%
\pgfpathlineto{\pgfqpoint{1.430958in}{0.566087in}}%
\pgfpathlineto{\pgfqpoint{1.438835in}{0.616717in}}%
\pgfpathlineto{\pgfqpoint{1.441460in}{0.628504in}}%
\pgfpathlineto{\pgfqpoint{1.443429in}{0.634190in}}%
\pgfpathlineto{\pgfqpoint{1.444742in}{0.636257in}}%
\pgfpathlineto{\pgfqpoint{1.446054in}{0.636866in}}%
\pgfpathlineto{\pgfqpoint{1.447367in}{0.636001in}}%
\pgfpathlineto{\pgfqpoint{1.448680in}{0.633686in}}%
\pgfpathlineto{\pgfqpoint{1.450649in}{0.627657in}}%
\pgfpathlineto{\pgfqpoint{1.452618in}{0.618923in}}%
\pgfpathlineto{\pgfqpoint{1.455899in}{0.599885in}}%
\pgfpathlineto{\pgfqpoint{1.467057in}{0.529530in}}%
\pgfpathlineto{\pgfqpoint{1.470339in}{0.515269in}}%
\pgfpathlineto{\pgfqpoint{1.472964in}{0.506748in}}%
\pgfpathlineto{\pgfqpoint{1.475590in}{0.500519in}}%
\pgfpathlineto{\pgfqpoint{1.478215in}{0.496180in}}%
\pgfpathlineto{\pgfqpoint{1.480840in}{0.493296in}}%
\pgfpathlineto{\pgfqpoint{1.483466in}{0.491466in}}%
\pgfpathlineto{\pgfqpoint{1.486748in}{0.490160in}}%
\pgfpathlineto{\pgfqpoint{1.490686in}{0.489425in}}%
\pgfpathlineto{\pgfqpoint{1.497249in}{0.489052in}}%
\pgfpathlineto{\pgfqpoint{1.519565in}{0.488981in}}%
\pgfpathlineto{\pgfqpoint{1.554351in}{0.489302in}}%
\pgfpathlineto{\pgfqpoint{1.558945in}{0.490053in}}%
\pgfpathlineto{\pgfqpoint{1.562227in}{0.491334in}}%
\pgfpathlineto{\pgfqpoint{1.564852in}{0.493201in}}%
\pgfpathlineto{\pgfqpoint{1.566821in}{0.495351in}}%
\pgfpathlineto{\pgfqpoint{1.568791in}{0.498382in}}%
\pgfpathlineto{\pgfqpoint{1.570760in}{0.502548in}}%
\pgfpathlineto{\pgfqpoint{1.572729in}{0.508124in}}%
\pgfpathlineto{\pgfqpoint{1.575354in}{0.518235in}}%
\pgfpathlineto{\pgfqpoint{1.577979in}{0.531932in}}%
\pgfpathlineto{\pgfqpoint{1.580605in}{0.549570in}}%
\pgfpathlineto{\pgfqpoint{1.583886in}{0.577034in}}%
\pgfpathlineto{\pgfqpoint{1.588481in}{0.622774in}}%
\pgfpathlineto{\pgfqpoint{1.594388in}{0.680344in}}%
\pgfpathlineto{\pgfqpoint{1.597013in}{0.699217in}}%
\pgfpathlineto{\pgfqpoint{1.598982in}{0.708739in}}%
\pgfpathlineto{\pgfqpoint{1.600295in}{0.712513in}}%
\pgfpathlineto{\pgfqpoint{1.601608in}{0.714090in}}%
\pgfpathlineto{\pgfqpoint{1.602264in}{0.714037in}}%
\pgfpathlineto{\pgfqpoint{1.602920in}{0.713422in}}%
\pgfpathlineto{\pgfqpoint{1.604233in}{0.710530in}}%
\pgfpathlineto{\pgfqpoint{1.605546in}{0.705499in}}%
\pgfpathlineto{\pgfqpoint{1.607515in}{0.694280in}}%
\pgfpathlineto{\pgfqpoint{1.610140in}{0.673642in}}%
\pgfpathlineto{\pgfqpoint{1.614078in}{0.635128in}}%
\pgfpathlineto{\pgfqpoint{1.620642in}{0.570006in}}%
\pgfpathlineto{\pgfqpoint{1.623923in}{0.543912in}}%
\pgfpathlineto{\pgfqpoint{1.627205in}{0.523966in}}%
\pgfpathlineto{\pgfqpoint{1.629830in}{0.512293in}}%
\pgfpathlineto{\pgfqpoint{1.632456in}{0.503906in}}%
\pgfpathlineto{\pgfqpoint{1.635081in}{0.498162in}}%
\pgfpathlineto{\pgfqpoint{1.637707in}{0.494407in}}%
\pgfpathlineto{\pgfqpoint{1.640332in}{0.492062in}}%
\pgfpathlineto{\pgfqpoint{1.642957in}{0.490662in}}%
\pgfpathlineto{\pgfqpoint{1.646239in}{0.489726in}}%
\pgfpathlineto{\pgfqpoint{1.651490in}{0.489159in}}%
\pgfpathlineto{\pgfqpoint{1.662648in}{0.488986in}}%
\pgfpathlineto{\pgfqpoint{1.705310in}{0.489271in}}%
\pgfpathlineto{\pgfqpoint{1.709904in}{0.489986in}}%
\pgfpathlineto{\pgfqpoint{1.713186in}{0.491244in}}%
\pgfpathlineto{\pgfqpoint{1.715811in}{0.493125in}}%
\pgfpathlineto{\pgfqpoint{1.717780in}{0.495334in}}%
\pgfpathlineto{\pgfqpoint{1.719749in}{0.498504in}}%
\pgfpathlineto{\pgfqpoint{1.721719in}{0.502940in}}%
\pgfpathlineto{\pgfqpoint{1.723688in}{0.508985in}}%
\pgfpathlineto{\pgfqpoint{1.725657in}{0.517011in}}%
\pgfpathlineto{\pgfqpoint{1.728282in}{0.531421in}}%
\pgfpathlineto{\pgfqpoint{1.730907in}{0.550718in}}%
\pgfpathlineto{\pgfqpoint{1.733533in}{0.575269in}}%
\pgfpathlineto{\pgfqpoint{1.736814in}{0.612938in}}%
\pgfpathlineto{\pgfqpoint{1.742722in}{0.692240in}}%
\pgfpathlineto{\pgfqpoint{1.746660in}{0.741548in}}%
\pgfpathlineto{\pgfqpoint{1.749285in}{0.766662in}}%
\pgfpathlineto{\pgfqpoint{1.751254in}{0.779400in}}%
\pgfpathlineto{\pgfqpoint{1.752567in}{0.784497in}}%
\pgfpathlineto{\pgfqpoint{1.753879in}{0.786691in}}%
\pgfpathlineto{\pgfqpoint{1.754536in}{0.786676in}}%
\pgfpathlineto{\pgfqpoint{1.755192in}{0.785918in}}%
\pgfpathlineto{\pgfqpoint{1.756505in}{0.782200in}}%
\pgfpathlineto{\pgfqpoint{1.757817in}{0.775647in}}%
\pgfpathlineto{\pgfqpoint{1.759786in}{0.760944in}}%
\pgfpathlineto{\pgfqpoint{1.762412in}{0.733787in}}%
\pgfpathlineto{\pgfqpoint{1.766350in}{0.682943in}}%
\pgfpathlineto{\pgfqpoint{1.772913in}{0.596714in}}%
\pgfpathlineto{\pgfqpoint{1.776195in}{0.562087in}}%
\pgfpathlineto{\pgfqpoint{1.779477in}{0.535584in}}%
\pgfpathlineto{\pgfqpoint{1.782102in}{0.520058in}}%
\pgfpathlineto{\pgfqpoint{1.784727in}{0.508892in}}%
\pgfpathlineto{\pgfqpoint{1.787353in}{0.501237in}}%
\pgfpathlineto{\pgfqpoint{1.789978in}{0.496230in}}%
\pgfpathlineto{\pgfqpoint{1.792604in}{0.493100in}}%
\pgfpathlineto{\pgfqpoint{1.795229in}{0.491230in}}%
\pgfpathlineto{\pgfqpoint{1.798511in}{0.489979in}}%
\pgfpathlineto{\pgfqpoint{1.802449in}{0.489327in}}%
\pgfpathlineto{\pgfqpoint{1.809669in}{0.489020in}}%
\pgfpathlineto{\pgfqpoint{1.843798in}{0.488995in}}%
\pgfpathlineto{\pgfqpoint{1.854956in}{0.489415in}}%
\pgfpathlineto{\pgfqpoint{1.858894in}{0.490219in}}%
\pgfpathlineto{\pgfqpoint{1.861520in}{0.491350in}}%
\pgfpathlineto{\pgfqpoint{1.864145in}{0.493338in}}%
\pgfpathlineto{\pgfqpoint{1.866114in}{0.495683in}}%
\pgfpathlineto{\pgfqpoint{1.868083in}{0.499061in}}%
\pgfpathlineto{\pgfqpoint{1.870052in}{0.503805in}}%
\pgfpathlineto{\pgfqpoint{1.872021in}{0.510296in}}%
\pgfpathlineto{\pgfqpoint{1.873990in}{0.518947in}}%
\pgfpathlineto{\pgfqpoint{1.876616in}{0.534551in}}%
\pgfpathlineto{\pgfqpoint{1.879241in}{0.555565in}}%
\pgfpathlineto{\pgfqpoint{1.881866in}{0.582453in}}%
\pgfpathlineto{\pgfqpoint{1.885148in}{0.624001in}}%
\pgfpathlineto{\pgfqpoint{1.890399in}{0.702522in}}%
\pgfpathlineto{\pgfqpoint{1.894993in}{0.768672in}}%
\pgfpathlineto{\pgfqpoint{1.897619in}{0.797842in}}%
\pgfpathlineto{\pgfqpoint{1.899588in}{0.813078in}}%
\pgfpathlineto{\pgfqpoint{1.900900in}{0.819494in}}%
\pgfpathlineto{\pgfqpoint{1.902213in}{0.822683in}}%
\pgfpathlineto{\pgfqpoint{1.902869in}{0.823034in}}%
\pgfpathlineto{\pgfqpoint{1.903526in}{0.822550in}}%
\pgfpathlineto{\pgfqpoint{1.904838in}{0.819100in}}%
\pgfpathlineto{\pgfqpoint{1.906151in}{0.812434in}}%
\pgfpathlineto{\pgfqpoint{1.908120in}{0.796860in}}%
\pgfpathlineto{\pgfqpoint{1.910745in}{0.767340in}}%
\pgfpathlineto{\pgfqpoint{1.914683in}{0.710990in}}%
\pgfpathlineto{\pgfqpoint{1.921903in}{0.605061in}}%
\pgfpathlineto{\pgfqpoint{1.925185in}{0.567202in}}%
\pgfpathlineto{\pgfqpoint{1.927810in}{0.543510in}}%
\pgfpathlineto{\pgfqpoint{1.930436in}{0.525504in}}%
\pgfpathlineto{\pgfqpoint{1.933061in}{0.512484in}}%
\pgfpathlineto{\pgfqpoint{1.935686in}{0.503513in}}%
\pgfpathlineto{\pgfqpoint{1.938312in}{0.497614in}}%
\pgfpathlineto{\pgfqpoint{1.940937in}{0.493908in}}%
\pgfpathlineto{\pgfqpoint{1.943563in}{0.491683in}}%
\pgfpathlineto{\pgfqpoint{1.946188in}{0.490405in}}%
\pgfpathlineto{\pgfqpoint{1.949470in}{0.489585in}}%
\pgfpathlineto{\pgfqpoint{1.954720in}{0.489116in}}%
\pgfpathlineto{\pgfqpoint{1.967847in}{0.488983in}}%
\pgfpathlineto{\pgfqpoint{1.999352in}{0.489326in}}%
\pgfpathlineto{\pgfqpoint{2.003946in}{0.490159in}}%
\pgfpathlineto{\pgfqpoint{2.007228in}{0.491605in}}%
\pgfpathlineto{\pgfqpoint{2.009853in}{0.493744in}}%
\pgfpathlineto{\pgfqpoint{2.011822in}{0.496235in}}%
\pgfpathlineto{\pgfqpoint{2.013791in}{0.499784in}}%
\pgfpathlineto{\pgfqpoint{2.015760in}{0.504710in}}%
\pgfpathlineto{\pgfqpoint{2.017729in}{0.511375in}}%
\pgfpathlineto{\pgfqpoint{2.020355in}{0.523613in}}%
\pgfpathlineto{\pgfqpoint{2.022980in}{0.540440in}}%
\pgfpathlineto{\pgfqpoint{2.025606in}{0.562444in}}%
\pgfpathlineto{\pgfqpoint{2.028887in}{0.597349in}}%
\pgfpathlineto{\pgfqpoint{2.032825in}{0.648082in}}%
\pgfpathlineto{\pgfqpoint{2.040045in}{0.743606in}}%
\pgfpathlineto{\pgfqpoint{2.042670in}{0.769262in}}%
\pgfpathlineto{\pgfqpoint{2.044639in}{0.782382in}}%
\pgfpathlineto{\pgfqpoint{2.045952in}{0.787712in}}%
\pgfpathlineto{\pgfqpoint{2.047265in}{0.790111in}}%
\pgfpathlineto{\pgfqpoint{2.047921in}{0.790186in}}%
\pgfpathlineto{\pgfqpoint{2.048578in}{0.789510in}}%
\pgfpathlineto{\pgfqpoint{2.049890in}{0.785925in}}%
\pgfpathlineto{\pgfqpoint{2.051203in}{0.779464in}}%
\pgfpathlineto{\pgfqpoint{2.053172in}{0.764814in}}%
\pgfpathlineto{\pgfqpoint{2.055797in}{0.737568in}}%
\pgfpathlineto{\pgfqpoint{2.059735in}{0.686295in}}%
\pgfpathlineto{\pgfqpoint{2.066955in}{0.591244in}}%
\pgfpathlineto{\pgfqpoint{2.070237in}{0.557616in}}%
\pgfpathlineto{\pgfqpoint{2.072862in}{0.536675in}}%
\pgfpathlineto{\pgfqpoint{2.075488in}{0.520824in}}%
\pgfpathlineto{\pgfqpoint{2.078113in}{0.509407in}}%
\pgfpathlineto{\pgfqpoint{2.080738in}{0.501570in}}%
\pgfpathlineto{\pgfqpoint{2.083364in}{0.496435in}}%
\pgfpathlineto{\pgfqpoint{2.085989in}{0.493222in}}%
\pgfpathlineto{\pgfqpoint{2.088614in}{0.491299in}}%
\pgfpathlineto{\pgfqpoint{2.091896in}{0.490011in}}%
\pgfpathlineto{\pgfqpoint{2.095834in}{0.489339in}}%
\pgfpathlineto{\pgfqpoint{2.103054in}{0.489022in}}%
\pgfpathlineto{\pgfqpoint{2.133902in}{0.489012in}}%
\pgfpathlineto{\pgfqpoint{2.143091in}{0.489429in}}%
\pgfpathlineto{\pgfqpoint{2.147029in}{0.490194in}}%
\pgfpathlineto{\pgfqpoint{2.150311in}{0.491578in}}%
\pgfpathlineto{\pgfqpoint{2.152936in}{0.493546in}}%
\pgfpathlineto{\pgfqpoint{2.155562in}{0.496689in}}%
\pgfpathlineto{\pgfqpoint{2.157531in}{0.500104in}}%
\pgfpathlineto{\pgfqpoint{2.160156in}{0.506493in}}%
\pgfpathlineto{\pgfqpoint{2.162781in}{0.515472in}}%
\pgfpathlineto{\pgfqpoint{2.165407in}{0.527483in}}%
\pgfpathlineto{\pgfqpoint{2.168032in}{0.542746in}}%
\pgfpathlineto{\pgfqpoint{2.171314in}{0.566132in}}%
\pgfpathlineto{\pgfqpoint{2.177221in}{0.615237in}}%
\pgfpathlineto{\pgfqpoint{2.181159in}{0.645657in}}%
\pgfpathlineto{\pgfqpoint{2.183784in}{0.661084in}}%
\pgfpathlineto{\pgfqpoint{2.185753in}{0.668859in}}%
\pgfpathlineto{\pgfqpoint{2.187066in}{0.671935in}}%
\pgfpathlineto{\pgfqpoint{2.188379in}{0.673212in}}%
\pgfpathlineto{\pgfqpoint{2.189035in}{0.673162in}}%
\pgfpathlineto{\pgfqpoint{2.190348in}{0.671686in}}%
\pgfpathlineto{\pgfqpoint{2.191660in}{0.668419in}}%
\pgfpathlineto{\pgfqpoint{2.193629in}{0.660384in}}%
\pgfpathlineto{\pgfqpoint{2.196255in}{0.644681in}}%
\pgfpathlineto{\pgfqpoint{2.199537in}{0.619507in}}%
\pgfpathlineto{\pgfqpoint{2.208069in}{0.550573in}}%
\pgfpathlineto{\pgfqpoint{2.211351in}{0.530211in}}%
\pgfpathlineto{\pgfqpoint{2.213976in}{0.517572in}}%
\pgfpathlineto{\pgfqpoint{2.216601in}{0.508030in}}%
\pgfpathlineto{\pgfqpoint{2.219227in}{0.501175in}}%
\pgfpathlineto{\pgfqpoint{2.221852in}{0.496480in}}%
\pgfpathlineto{\pgfqpoint{2.224478in}{0.493413in}}%
\pgfpathlineto{\pgfqpoint{2.227103in}{0.491497in}}%
\pgfpathlineto{\pgfqpoint{2.230385in}{0.490153in}}%
\pgfpathlineto{\pgfqpoint{2.234323in}{0.489413in}}%
\pgfpathlineto{\pgfqpoint{2.240886in}{0.489048in}}%
\pgfpathlineto{\pgfqpoint{2.264515in}{0.488981in}}%
\pgfpathlineto{\pgfqpoint{2.411535in}{0.489305in}}%
\pgfpathlineto{\pgfqpoint{2.416130in}{0.490052in}}%
\pgfpathlineto{\pgfqpoint{2.419412in}{0.491315in}}%
\pgfpathlineto{\pgfqpoint{2.422037in}{0.493141in}}%
\pgfpathlineto{\pgfqpoint{2.424662in}{0.496105in}}%
\pgfpathlineto{\pgfqpoint{2.426631in}{0.499369in}}%
\pgfpathlineto{\pgfqpoint{2.428600in}{0.503792in}}%
\pgfpathlineto{\pgfqpoint{2.431226in}{0.511932in}}%
\pgfpathlineto{\pgfqpoint{2.433851in}{0.523150in}}%
\pgfpathlineto{\pgfqpoint{2.436476in}{0.537858in}}%
\pgfpathlineto{\pgfqpoint{2.439758in}{0.561260in}}%
\pgfpathlineto{\pgfqpoint{2.443696in}{0.595412in}}%
\pgfpathlineto{\pgfqpoint{2.450916in}{0.660242in}}%
\pgfpathlineto{\pgfqpoint{2.453541in}{0.677872in}}%
\pgfpathlineto{\pgfqpoint{2.455510in}{0.687008in}}%
\pgfpathlineto{\pgfqpoint{2.456823in}{0.690805in}}%
\pgfpathlineto{\pgfqpoint{2.458136in}{0.692627in}}%
\pgfpathlineto{\pgfqpoint{2.458792in}{0.692779in}}%
\pgfpathlineto{\pgfqpoint{2.459448in}{0.692422in}}%
\pgfpathlineto{\pgfqpoint{2.460761in}{0.690194in}}%
\pgfpathlineto{\pgfqpoint{2.462074in}{0.686011in}}%
\pgfpathlineto{\pgfqpoint{2.464043in}{0.676352in}}%
\pgfpathlineto{\pgfqpoint{2.466668in}{0.658179in}}%
\pgfpathlineto{\pgfqpoint{2.470606in}{0.623679in}}%
\pgfpathlineto{\pgfqpoint{2.477826in}{0.559173in}}%
\pgfpathlineto{\pgfqpoint{2.481108in}{0.536208in}}%
\pgfpathlineto{\pgfqpoint{2.483733in}{0.521863in}}%
\pgfpathlineto{\pgfqpoint{2.486359in}{0.510978in}}%
\pgfpathlineto{\pgfqpoint{2.488984in}{0.503119in}}%
\pgfpathlineto{\pgfqpoint{2.491609in}{0.497712in}}%
\pgfpathlineto{\pgfqpoint{2.494235in}{0.494161in}}%
\pgfpathlineto{\pgfqpoint{2.496860in}{0.491934in}}%
\pgfpathlineto{\pgfqpoint{2.499485in}{0.490598in}}%
\pgfpathlineto{\pgfqpoint{2.502767in}{0.489701in}}%
\pgfpathlineto{\pgfqpoint{2.508018in}{0.489154in}}%
\pgfpathlineto{\pgfqpoint{2.519832in}{0.488986in}}%
\pgfpathlineto{\pgfqpoint{2.538210in}{0.489297in}}%
\pgfpathlineto{\pgfqpoint{2.542804in}{0.490089in}}%
\pgfpathlineto{\pgfqpoint{2.545429in}{0.491128in}}%
\pgfpathlineto{\pgfqpoint{2.548055in}{0.492979in}}%
\pgfpathlineto{\pgfqpoint{2.550024in}{0.495188in}}%
\pgfpathlineto{\pgfqpoint{2.551993in}{0.498404in}}%
\pgfpathlineto{\pgfqpoint{2.553962in}{0.502967in}}%
\pgfpathlineto{\pgfqpoint{2.555931in}{0.509279in}}%
\pgfpathlineto{\pgfqpoint{2.557900in}{0.517783in}}%
\pgfpathlineto{\pgfqpoint{2.560525in}{0.533327in}}%
\pgfpathlineto{\pgfqpoint{2.563151in}{0.554583in}}%
\pgfpathlineto{\pgfqpoint{2.565776in}{0.582222in}}%
\pgfpathlineto{\pgfqpoint{2.569058in}{0.625767in}}%
\pgfpathlineto{\pgfqpoint{2.573652in}{0.699576in}}%
\pgfpathlineto{\pgfqpoint{2.579559in}{0.795337in}}%
\pgfpathlineto{\pgfqpoint{2.582185in}{0.828093in}}%
\pgfpathlineto{\pgfqpoint{2.584154in}{0.845457in}}%
\pgfpathlineto{\pgfqpoint{2.585466in}{0.852947in}}%
\pgfpathlineto{\pgfqpoint{2.586779in}{0.856896in}}%
\pgfpathlineto{\pgfqpoint{2.587435in}{0.857502in}}%
\pgfpathlineto{\pgfqpoint{2.588092in}{0.857187in}}%
\pgfpathlineto{\pgfqpoint{2.588748in}{0.855954in}}%
\pgfpathlineto{\pgfqpoint{2.590061in}{0.850777in}}%
\pgfpathlineto{\pgfqpoint{2.591374in}{0.842124in}}%
\pgfpathlineto{\pgfqpoint{2.593343in}{0.823204in}}%
\pgfpathlineto{\pgfqpoint{2.595968in}{0.788864in}}%
\pgfpathlineto{\pgfqpoint{2.599906in}{0.725438in}}%
\pgfpathlineto{\pgfqpoint{2.606469in}{0.619266in}}%
\pgfpathlineto{\pgfqpoint{2.609751in}{0.577035in}}%
\pgfpathlineto{\pgfqpoint{2.613033in}{0.544888in}}%
\pgfpathlineto{\pgfqpoint{2.615658in}{0.526142in}}%
\pgfpathlineto{\pgfqpoint{2.618284in}{0.512714in}}%
\pgfpathlineto{\pgfqpoint{2.620909in}{0.503543in}}%
\pgfpathlineto{\pgfqpoint{2.623534in}{0.497566in}}%
\pgfpathlineto{\pgfqpoint{2.626160in}{0.493844in}}%
\pgfpathlineto{\pgfqpoint{2.628785in}{0.491627in}}%
\pgfpathlineto{\pgfqpoint{2.631410in}{0.490365in}}%
\pgfpathlineto{\pgfqpoint{2.635349in}{0.489466in}}%
\pgfpathlineto{\pgfqpoint{2.641256in}{0.489066in}}%
\pgfpathlineto{\pgfqpoint{2.657008in}{0.489046in}}%
\pgfpathlineto{\pgfqpoint{2.664228in}{0.489547in}}%
\pgfpathlineto{\pgfqpoint{2.667509in}{0.490349in}}%
\pgfpathlineto{\pgfqpoint{2.670135in}{0.491630in}}%
\pgfpathlineto{\pgfqpoint{2.672104in}{0.493218in}}%
\pgfpathlineto{\pgfqpoint{2.674073in}{0.495607in}}%
\pgfpathlineto{\pgfqpoint{2.676042in}{0.499113in}}%
\pgfpathlineto{\pgfqpoint{2.678011in}{0.504129in}}%
\pgfpathlineto{\pgfqpoint{2.679980in}{0.511123in}}%
\pgfpathlineto{\pgfqpoint{2.681949in}{0.520628in}}%
\pgfpathlineto{\pgfqpoint{2.683918in}{0.533206in}}%
\pgfpathlineto{\pgfqpoint{2.686543in}{0.555700in}}%
\pgfpathlineto{\pgfqpoint{2.689169in}{0.585688in}}%
\pgfpathlineto{\pgfqpoint{2.691794in}{0.623659in}}%
\pgfpathlineto{\pgfqpoint{2.695076in}{0.681586in}}%
\pgfpathlineto{\pgfqpoint{2.706890in}{0.905128in}}%
\pgfpathlineto{\pgfqpoint{2.708859in}{0.926312in}}%
\pgfpathlineto{\pgfqpoint{2.710172in}{0.935414in}}%
\pgfpathlineto{\pgfqpoint{2.711484in}{0.940171in}}%
\pgfpathlineto{\pgfqpoint{2.712141in}{0.940871in}}%
\pgfpathlineto{\pgfqpoint{2.712797in}{0.940442in}}%
\pgfpathlineto{\pgfqpoint{2.713453in}{0.938887in}}%
\pgfpathlineto{\pgfqpoint{2.714766in}{0.932455in}}%
\pgfpathlineto{\pgfqpoint{2.716079in}{0.921765in}}%
\pgfpathlineto{\pgfqpoint{2.718048in}{0.898461in}}%
\pgfpathlineto{\pgfqpoint{2.720673in}{0.856248in}}%
\pgfpathlineto{\pgfqpoint{2.724611in}{0.778404in}}%
\pgfpathlineto{\pgfqpoint{2.731175in}{0.648297in}}%
\pgfpathlineto{\pgfqpoint{2.734456in}{0.596605in}}%
\pgfpathlineto{\pgfqpoint{2.737738in}{0.557280in}}%
\pgfpathlineto{\pgfqpoint{2.740363in}{0.534362in}}%
\pgfpathlineto{\pgfqpoint{2.742989in}{0.517952in}}%
\pgfpathlineto{\pgfqpoint{2.745614in}{0.506751in}}%
\pgfpathlineto{\pgfqpoint{2.747583in}{0.500978in}}%
\pgfpathlineto{\pgfqpoint{2.749552in}{0.496900in}}%
\pgfpathlineto{\pgfqpoint{2.751521in}{0.494092in}}%
\pgfpathlineto{\pgfqpoint{2.754147in}{0.491734in}}%
\pgfpathlineto{\pgfqpoint{2.756772in}{0.490406in}}%
\pgfpathlineto{\pgfqpoint{2.760054in}{0.489572in}}%
\pgfpathlineto{\pgfqpoint{2.765305in}{0.489109in}}%
\pgfpathlineto{\pgfqpoint{2.777119in}{0.489042in}}%
\pgfpathlineto{\pgfqpoint{2.784338in}{0.489496in}}%
\pgfpathlineto{\pgfqpoint{2.788277in}{0.490464in}}%
\pgfpathlineto{\pgfqpoint{2.790902in}{0.491838in}}%
\pgfpathlineto{\pgfqpoint{2.792871in}{0.493532in}}%
\pgfpathlineto{\pgfqpoint{2.794840in}{0.496068in}}%
\pgfpathlineto{\pgfqpoint{2.796809in}{0.499771in}}%
\pgfpathlineto{\pgfqpoint{2.798778in}{0.505044in}}%
\pgfpathlineto{\pgfqpoint{2.800747in}{0.512363in}}%
\pgfpathlineto{\pgfqpoint{2.802716in}{0.522258in}}%
\pgfpathlineto{\pgfqpoint{2.804685in}{0.535288in}}%
\pgfpathlineto{\pgfqpoint{2.807310in}{0.558447in}}%
\pgfpathlineto{\pgfqpoint{2.809936in}{0.589104in}}%
\pgfpathlineto{\pgfqpoint{2.813218in}{0.638450in}}%
\pgfpathlineto{\pgfqpoint{2.817156in}{0.711590in}}%
\pgfpathlineto{\pgfqpoint{2.825688in}{0.876482in}}%
\pgfpathlineto{\pgfqpoint{2.828313in}{0.911094in}}%
\pgfpathlineto{\pgfqpoint{2.830283in}{0.927406in}}%
\pgfpathlineto{\pgfqpoint{2.831595in}{0.933043in}}%
\pgfpathlineto{\pgfqpoint{2.832252in}{0.934217in}}%
\pgfpathlineto{\pgfqpoint{2.832908in}{0.934278in}}%
\pgfpathlineto{\pgfqpoint{2.833564in}{0.933228in}}%
\pgfpathlineto{\pgfqpoint{2.834877in}{0.927831in}}%
\pgfpathlineto{\pgfqpoint{2.836190in}{0.918186in}}%
\pgfpathlineto{\pgfqpoint{2.838159in}{0.896403in}}%
\pgfpathlineto{\pgfqpoint{2.840784in}{0.855996in}}%
\pgfpathlineto{\pgfqpoint{2.844722in}{0.780101in}}%
\pgfpathlineto{\pgfqpoint{2.851286in}{0.650985in}}%
\pgfpathlineto{\pgfqpoint{2.854567in}{0.599017in}}%
\pgfpathlineto{\pgfqpoint{2.857849in}{0.559192in}}%
\pgfpathlineto{\pgfqpoint{2.860474in}{0.535836in}}%
\pgfpathlineto{\pgfqpoint{2.863100in}{0.519023in}}%
\pgfpathlineto{\pgfqpoint{2.865725in}{0.507488in}}%
\pgfpathlineto{\pgfqpoint{2.868350in}{0.499935in}}%
\pgfpathlineto{\pgfqpoint{2.870319in}{0.496181in}}%
\pgfpathlineto{\pgfqpoint{2.872945in}{0.492955in}}%
\pgfpathlineto{\pgfqpoint{2.875570in}{0.491088in}}%
\pgfpathlineto{\pgfqpoint{2.878852in}{0.489882in}}%
\pgfpathlineto{\pgfqpoint{2.883446in}{0.489230in}}%
\pgfpathlineto{\pgfqpoint{2.891979in}{0.489031in}}%
\pgfpathlineto{\pgfqpoint{2.900511in}{0.489430in}}%
\pgfpathlineto{\pgfqpoint{2.904449in}{0.490267in}}%
\pgfpathlineto{\pgfqpoint{2.907075in}{0.491453in}}%
\pgfpathlineto{\pgfqpoint{2.909700in}{0.493546in}}%
\pgfpathlineto{\pgfqpoint{2.911669in}{0.496025in}}%
\pgfpathlineto{\pgfqpoint{2.913638in}{0.499609in}}%
\pgfpathlineto{\pgfqpoint{2.915607in}{0.504660in}}%
\pgfpathlineto{\pgfqpoint{2.917576in}{0.511595in}}%
\pgfpathlineto{\pgfqpoint{2.919545in}{0.520873in}}%
\pgfpathlineto{\pgfqpoint{2.922171in}{0.537683in}}%
\pgfpathlineto{\pgfqpoint{2.924796in}{0.560437in}}%
\pgfpathlineto{\pgfqpoint{2.927421in}{0.589712in}}%
\pgfpathlineto{\pgfqpoint{2.930703in}{0.635246in}}%
\pgfpathlineto{\pgfqpoint{2.935297in}{0.710956in}}%
\pgfpathlineto{\pgfqpoint{2.940548in}{0.796735in}}%
\pgfpathlineto{\pgfqpoint{2.943174in}{0.830250in}}%
\pgfpathlineto{\pgfqpoint{2.945143in}{0.848206in}}%
\pgfpathlineto{\pgfqpoint{2.946455in}{0.856081in}}%
\pgfpathlineto{\pgfqpoint{2.947768in}{0.860395in}}%
\pgfpathlineto{\pgfqpoint{2.948424in}{0.861173in}}%
\pgfpathlineto{\pgfqpoint{2.949081in}{0.861022in}}%
\pgfpathlineto{\pgfqpoint{2.949737in}{0.859941in}}%
\pgfpathlineto{\pgfqpoint{2.951050in}{0.855035in}}%
\pgfpathlineto{\pgfqpoint{2.952362in}{0.846599in}}%
\pgfpathlineto{\pgfqpoint{2.954331in}{0.827894in}}%
\pgfpathlineto{\pgfqpoint{2.956957in}{0.793615in}}%
\pgfpathlineto{\pgfqpoint{2.960895in}{0.729829in}}%
\pgfpathlineto{\pgfqpoint{2.967458in}{0.622279in}}%
\pgfpathlineto{\pgfqpoint{2.970740in}{0.579273in}}%
\pgfpathlineto{\pgfqpoint{2.974022in}{0.546437in}}%
\pgfpathlineto{\pgfqpoint{2.976647in}{0.527240in}}%
\pgfpathlineto{\pgfqpoint{2.979272in}{0.513459in}}%
\pgfpathlineto{\pgfqpoint{2.981898in}{0.504027in}}%
\pgfpathlineto{\pgfqpoint{2.984523in}{0.497867in}}%
\pgfpathlineto{\pgfqpoint{2.987149in}{0.494023in}}%
\pgfpathlineto{\pgfqpoint{2.989774in}{0.491730in}}%
\pgfpathlineto{\pgfqpoint{2.992399in}{0.490421in}}%
\pgfpathlineto{\pgfqpoint{2.995681in}{0.489589in}}%
\pgfpathlineto{\pgfqpoint{3.000932in}{0.489125in}}%
\pgfpathlineto{\pgfqpoint{3.010121in}{0.489173in}}%
\pgfpathlineto{\pgfqpoint{3.015371in}{0.489770in}}%
\pgfpathlineto{\pgfqpoint{3.018653in}{0.490776in}}%
\pgfpathlineto{\pgfqpoint{3.021278in}{0.492295in}}%
\pgfpathlineto{\pgfqpoint{3.023904in}{0.494860in}}%
\pgfpathlineto{\pgfqpoint{3.025873in}{0.497785in}}%
\pgfpathlineto{\pgfqpoint{3.027842in}{0.501871in}}%
\pgfpathlineto{\pgfqpoint{3.029811in}{0.507434in}}%
\pgfpathlineto{\pgfqpoint{3.031780in}{0.514810in}}%
\pgfpathlineto{\pgfqpoint{3.034405in}{0.528031in}}%
\pgfpathlineto{\pgfqpoint{3.037031in}{0.545706in}}%
\pgfpathlineto{\pgfqpoint{3.039656in}{0.568151in}}%
\pgfpathlineto{\pgfqpoint{3.042938in}{0.602510in}}%
\pgfpathlineto{\pgfqpoint{3.049501in}{0.682521in}}%
\pgfpathlineto{\pgfqpoint{3.053439in}{0.725385in}}%
\pgfpathlineto{\pgfqpoint{3.056065in}{0.745939in}}%
\pgfpathlineto{\pgfqpoint{3.058034in}{0.755432in}}%
\pgfpathlineto{\pgfqpoint{3.059346in}{0.758564in}}%
\pgfpathlineto{\pgfqpoint{3.060003in}{0.759129in}}%
\pgfpathlineto{\pgfqpoint{3.060659in}{0.759019in}}%
\pgfpathlineto{\pgfqpoint{3.061315in}{0.758235in}}%
\pgfpathlineto{\pgfqpoint{3.062628in}{0.754674in}}%
\pgfpathlineto{\pgfqpoint{3.063941in}{0.748551in}}%
\pgfpathlineto{\pgfqpoint{3.065910in}{0.734974in}}%
\pgfpathlineto{\pgfqpoint{3.068535in}{0.710094in}}%
\pgfpathlineto{\pgfqpoint{3.072473in}{0.663797in}}%
\pgfpathlineto{\pgfqpoint{3.079037in}{0.585734in}}%
\pgfpathlineto{\pgfqpoint{3.082318in}{0.554519in}}%
\pgfpathlineto{\pgfqpoint{3.085600in}{0.530685in}}%
\pgfpathlineto{\pgfqpoint{3.088225in}{0.516751in}}%
\pgfpathlineto{\pgfqpoint{3.090851in}{0.506748in}}%
\pgfpathlineto{\pgfqpoint{3.093476in}{0.499902in}}%
\pgfpathlineto{\pgfqpoint{3.096102in}{0.495431in}}%
\pgfpathlineto{\pgfqpoint{3.098727in}{0.492641in}}%
\pgfpathlineto{\pgfqpoint{3.101352in}{0.490977in}}%
\pgfpathlineto{\pgfqpoint{3.104634in}{0.489867in}}%
\pgfpathlineto{\pgfqpoint{3.109228in}{0.489245in}}%
\pgfpathlineto{\pgfqpoint{3.117105in}{0.489120in}}%
\pgfpathlineto{\pgfqpoint{3.123668in}{0.489677in}}%
\pgfpathlineto{\pgfqpoint{3.127606in}{0.490779in}}%
\pgfpathlineto{\pgfqpoint{3.130231in}{0.492205in}}%
\pgfpathlineto{\pgfqpoint{3.132857in}{0.494536in}}%
\pgfpathlineto{\pgfqpoint{3.135482in}{0.498179in}}%
\pgfpathlineto{\pgfqpoint{3.138108in}{0.503614in}}%
\pgfpathlineto{\pgfqpoint{3.140733in}{0.511347in}}%
\pgfpathlineto{\pgfqpoint{3.143358in}{0.521827in}}%
\pgfpathlineto{\pgfqpoint{3.145984in}{0.535327in}}%
\pgfpathlineto{\pgfqpoint{3.149265in}{0.556353in}}%
\pgfpathlineto{\pgfqpoint{3.153860in}{0.591390in}}%
\pgfpathlineto{\pgfqpoint{3.159767in}{0.635531in}}%
\pgfpathlineto{\pgfqpoint{3.162392in}{0.650020in}}%
\pgfpathlineto{\pgfqpoint{3.164361in}{0.657342in}}%
\pgfpathlineto{\pgfqpoint{3.165674in}{0.660253in}}%
\pgfpathlineto{\pgfqpoint{3.166987in}{0.661481in}}%
\pgfpathlineto{\pgfqpoint{3.167643in}{0.661450in}}%
\pgfpathlineto{\pgfqpoint{3.168956in}{0.660100in}}%
\pgfpathlineto{\pgfqpoint{3.170268in}{0.657071in}}%
\pgfpathlineto{\pgfqpoint{3.172237in}{0.649588in}}%
\pgfpathlineto{\pgfqpoint{3.174863in}{0.634928in}}%
\pgfpathlineto{\pgfqpoint{3.178145in}{0.611388in}}%
\pgfpathlineto{\pgfqpoint{3.187333in}{0.542624in}}%
\pgfpathlineto{\pgfqpoint{3.190615in}{0.524460in}}%
\pgfpathlineto{\pgfqpoint{3.193240in}{0.513348in}}%
\pgfpathlineto{\pgfqpoint{3.195866in}{0.505060in}}%
\pgfpathlineto{\pgfqpoint{3.198491in}{0.499175in}}%
\pgfpathlineto{\pgfqpoint{3.201117in}{0.495191in}}%
\pgfpathlineto{\pgfqpoint{3.203742in}{0.492615in}}%
\pgfpathlineto{\pgfqpoint{3.206367in}{0.491025in}}%
\pgfpathlineto{\pgfqpoint{3.209649in}{0.489924in}}%
\pgfpathlineto{\pgfqpoint{3.214243in}{0.489280in}}%
\pgfpathlineto{\pgfqpoint{3.221463in}{0.489130in}}%
\pgfpathlineto{\pgfqpoint{3.228027in}{0.489619in}}%
\pgfpathlineto{\pgfqpoint{3.231965in}{0.490553in}}%
\pgfpathlineto{\pgfqpoint{3.235246in}{0.492105in}}%
\pgfpathlineto{\pgfqpoint{3.237872in}{0.494156in}}%
\pgfpathlineto{\pgfqpoint{3.240497in}{0.497218in}}%
\pgfpathlineto{\pgfqpoint{3.243123in}{0.501577in}}%
\pgfpathlineto{\pgfqpoint{3.245748in}{0.507490in}}%
\pgfpathlineto{\pgfqpoint{3.248373in}{0.515111in}}%
\pgfpathlineto{\pgfqpoint{3.251655in}{0.526992in}}%
\pgfpathlineto{\pgfqpoint{3.256249in}{0.546814in}}%
\pgfpathlineto{\pgfqpoint{3.262156in}{0.571844in}}%
\pgfpathlineto{\pgfqpoint{3.264782in}{0.580086in}}%
\pgfpathlineto{\pgfqpoint{3.266751in}{0.584268in}}%
\pgfpathlineto{\pgfqpoint{3.268064in}{0.585942in}}%
\pgfpathlineto{\pgfqpoint{3.269376in}{0.586663in}}%
\pgfpathlineto{\pgfqpoint{3.270689in}{0.586411in}}%
\pgfpathlineto{\pgfqpoint{3.272002in}{0.585193in}}%
\pgfpathlineto{\pgfqpoint{3.273314in}{0.583044in}}%
\pgfpathlineto{\pgfqpoint{3.275283in}{0.578222in}}%
\pgfpathlineto{\pgfqpoint{3.277909in}{0.569313in}}%
\pgfpathlineto{\pgfqpoint{3.281847in}{0.552632in}}%
\pgfpathlineto{\pgfqpoint{3.288410in}{0.524338in}}%
\pgfpathlineto{\pgfqpoint{3.291692in}{0.512975in}}%
\pgfpathlineto{\pgfqpoint{3.294974in}{0.504277in}}%
\pgfpathlineto{\pgfqpoint{3.297599in}{0.499181in}}%
\pgfpathlineto{\pgfqpoint{3.300224in}{0.495517in}}%
\pgfpathlineto{\pgfqpoint{3.302850in}{0.493004in}}%
\pgfpathlineto{\pgfqpoint{3.305475in}{0.491361in}}%
\pgfpathlineto{\pgfqpoint{3.308757in}{0.490150in}}%
\pgfpathlineto{\pgfqpoint{3.313351in}{0.489379in}}%
\pgfpathlineto{\pgfqpoint{3.319915in}{0.489117in}}%
\pgfpathlineto{\pgfqpoint{3.327791in}{0.489505in}}%
\pgfpathlineto{\pgfqpoint{3.332385in}{0.490384in}}%
\pgfpathlineto{\pgfqpoint{3.335667in}{0.491629in}}%
\pgfpathlineto{\pgfqpoint{3.338949in}{0.493678in}}%
\pgfpathlineto{\pgfqpoint{3.342230in}{0.496811in}}%
\pgfpathlineto{\pgfqpoint{3.345512in}{0.501241in}}%
\pgfpathlineto{\pgfqpoint{3.348794in}{0.507015in}}%
\pgfpathlineto{\pgfqpoint{3.353388in}{0.516850in}}%
\pgfpathlineto{\pgfqpoint{3.359952in}{0.530948in}}%
\pgfpathlineto{\pgfqpoint{3.362577in}{0.535072in}}%
\pgfpathlineto{\pgfqpoint{3.364546in}{0.537148in}}%
\pgfpathlineto{\pgfqpoint{3.366515in}{0.538197in}}%
\pgfpathlineto{\pgfqpoint{3.368484in}{0.538150in}}%
\pgfpathlineto{\pgfqpoint{3.370453in}{0.537011in}}%
\pgfpathlineto{\pgfqpoint{3.372422in}{0.534854in}}%
\pgfpathlineto{\pgfqpoint{3.375048in}{0.530644in}}%
\pgfpathlineto{\pgfqpoint{3.378986in}{0.522437in}}%
\pgfpathlineto{\pgfqpoint{3.386862in}{0.505449in}}%
\pgfpathlineto{\pgfqpoint{3.390143in}{0.500003in}}%
\pgfpathlineto{\pgfqpoint{3.393425in}{0.495910in}}%
\pgfpathlineto{\pgfqpoint{3.396707in}{0.493074in}}%
\pgfpathlineto{\pgfqpoint{3.399989in}{0.491252in}}%
\pgfpathlineto{\pgfqpoint{3.403927in}{0.490012in}}%
\pgfpathlineto{\pgfqpoint{3.409177in}{0.489294in}}%
\pgfpathlineto{\pgfqpoint{3.418366in}{0.489007in}}%
\pgfpathlineto{\pgfqpoint{3.459716in}{0.488981in}}%
\pgfpathlineto{\pgfqpoint{3.459716in}{0.488981in}}%
\pgfusepath{stroke}%
\end{pgfscope}%
\begin{pgfscope}%
\pgfsetrectcap%
\pgfsetmiterjoin%
\pgfsetlinewidth{0.803000pt}%
\definecolor{currentstroke}{rgb}{0.000000,0.000000,0.000000}%
\pgfsetstrokecolor{currentstroke}%
\pgfsetdash{}{0pt}%
\pgfpathmoveto{\pgfqpoint{0.689290in}{0.468380in}}%
\pgfpathlineto{\pgfqpoint{0.689290in}{1.313040in}}%
\pgfusepath{stroke}%
\end{pgfscope}%
\begin{pgfscope}%
\pgfsetrectcap%
\pgfsetmiterjoin%
\pgfsetlinewidth{0.803000pt}%
\definecolor{currentstroke}{rgb}{0.000000,0.000000,0.000000}%
\pgfsetstrokecolor{currentstroke}%
\pgfsetdash{}{0pt}%
\pgfpathmoveto{\pgfqpoint{0.689290in}{0.468380in}}%
\pgfpathlineto{\pgfqpoint{3.755070in}{0.468380in}}%
\pgfusepath{stroke}%
\end{pgfscope}%
\begin{pgfscope}%
\pgfsetroundcap%
\pgfsetroundjoin%
\pgfsetlinewidth{1.254687pt}%
\definecolor{currentstroke}{rgb}{0.000000,0.000000,1.000000}%
\pgfsetstrokecolor{currentstroke}%
\pgfsetdash{}{0pt}%
\pgfpathmoveto{\pgfqpoint{2.188635in}{0.844273in}}%
\pgfpathquadraticcurveto{\pgfqpoint{2.188635in}{0.772648in}}{\pgfqpoint{2.188635in}{0.720435in}}%
\pgfusepath{stroke}%
\end{pgfscope}%
\begin{pgfscope}%
\pgfsetroundcap%
\pgfsetroundjoin%
\definecolor{currentfill}{rgb}{0.000000,0.000000,1.000000}%
\pgfsetfillcolor{currentfill}%
\pgfsetlinewidth{1.254687pt}%
\definecolor{currentstroke}{rgb}{0.000000,0.000000,1.000000}%
\pgfsetstrokecolor{currentstroke}%
\pgfsetdash{}{0pt}%
\pgfpathmoveto{\pgfqpoint{2.205302in}{0.753768in}}%
\pgfpathlineto{\pgfqpoint{2.188635in}{0.720435in}}%
\pgfpathlineto{\pgfqpoint{2.171969in}{0.753768in}}%
\pgfpathlineto{\pgfqpoint{2.205302in}{0.753768in}}%
\pgfpathclose%
\pgfusepath{stroke,fill}%
\end{pgfscope}%
\begin{pgfscope}%
\pgfsetbuttcap%
\pgfsetmiterjoin%
\definecolor{currentfill}{rgb}{1.000000,1.000000,1.000000}%
\pgfsetfillcolor{currentfill}%
\pgfsetlinewidth{1.003750pt}%
\definecolor{currentstroke}{rgb}{1.000000,1.000000,1.000000}%
\pgfsetstrokecolor{currentstroke}%
\pgfsetdash{}{0pt}%
\pgfpathmoveto{\pgfqpoint{2.221969in}{0.887595in}}%
\pgfpathlineto{\pgfqpoint{2.221969in}{1.070309in}}%
\pgfpathlineto{\pgfqpoint{2.155302in}{1.070309in}}%
\pgfpathlineto{\pgfqpoint{2.155302in}{0.887595in}}%
\pgfpathclose%
\pgfusepath{stroke,fill}%
\end{pgfscope}%
\begin{pgfscope}%
\definecolor{textcolor}{rgb}{0.000000,0.000000,0.000000}%
\pgfsetstrokecolor{textcolor}%
\pgfsetfillcolor{textcolor}%
\pgftext[x=2.209469in, y=0.879262in, left, base,rotate=90.000000]{\color{textcolor}\rmfamily\fontsize{6.000000}{7.200000}\selectfont P(1)}%
\end{pgfscope}%
\begin{pgfscope}%
\pgfsetroundcap%
\pgfsetroundjoin%
\pgfsetlinewidth{1.254687pt}%
\definecolor{currentstroke}{rgb}{0.000000,0.000000,1.000000}%
\pgfsetstrokecolor{currentstroke}%
\pgfsetdash{}{0pt}%
\pgfpathmoveto{\pgfqpoint{2.047659in}{0.961272in}}%
\pgfpathquadraticcurveto{\pgfqpoint{2.047659in}{0.889648in}}{\pgfqpoint{2.047659in}{0.837434in}}%
\pgfusepath{stroke}%
\end{pgfscope}%
\begin{pgfscope}%
\pgfsetroundcap%
\pgfsetroundjoin%
\definecolor{currentfill}{rgb}{0.000000,0.000000,1.000000}%
\pgfsetfillcolor{currentfill}%
\pgfsetlinewidth{1.254687pt}%
\definecolor{currentstroke}{rgb}{0.000000,0.000000,1.000000}%
\pgfsetstrokecolor{currentstroke}%
\pgfsetdash{}{0pt}%
\pgfpathmoveto{\pgfqpoint{2.064325in}{0.870767in}}%
\pgfpathlineto{\pgfqpoint{2.047659in}{0.837434in}}%
\pgfpathlineto{\pgfqpoint{2.030992in}{0.870767in}}%
\pgfpathlineto{\pgfqpoint{2.064325in}{0.870767in}}%
\pgfpathclose%
\pgfusepath{stroke,fill}%
\end{pgfscope}%
\begin{pgfscope}%
\pgfsetbuttcap%
\pgfsetmiterjoin%
\definecolor{currentfill}{rgb}{1.000000,1.000000,1.000000}%
\pgfsetfillcolor{currentfill}%
\pgfsetlinewidth{1.003750pt}%
\definecolor{currentstroke}{rgb}{1.000000,1.000000,1.000000}%
\pgfsetstrokecolor{currentstroke}%
\pgfsetdash{}{0pt}%
\pgfpathmoveto{\pgfqpoint{2.080992in}{1.004595in}}%
\pgfpathlineto{\pgfqpoint{2.080992in}{1.187308in}}%
\pgfpathlineto{\pgfqpoint{2.014325in}{1.187308in}}%
\pgfpathlineto{\pgfqpoint{2.014325in}{1.004595in}}%
\pgfpathclose%
\pgfusepath{stroke,fill}%
\end{pgfscope}%
\begin{pgfscope}%
\definecolor{textcolor}{rgb}{0.000000,0.000000,0.000000}%
\pgfsetstrokecolor{textcolor}%
\pgfsetfillcolor{textcolor}%
\pgftext[x=2.068492in, y=0.996261in, left, base,rotate=90.000000]{\color{textcolor}\rmfamily\fontsize{6.000000}{7.200000}\selectfont P(2)}%
\end{pgfscope}%
\begin{pgfscope}%
\pgfsetroundcap%
\pgfsetroundjoin%
\pgfsetlinewidth{1.254687pt}%
\definecolor{currentstroke}{rgb}{0.000000,0.000000,1.000000}%
\pgfsetstrokecolor{currentstroke}%
\pgfsetdash{}{0pt}%
\pgfpathmoveto{\pgfqpoint{1.902817in}{0.994062in}}%
\pgfpathquadraticcurveto{\pgfqpoint{1.902817in}{0.922438in}}{\pgfqpoint{1.902817in}{0.870224in}}%
\pgfusepath{stroke}%
\end{pgfscope}%
\begin{pgfscope}%
\pgfsetroundcap%
\pgfsetroundjoin%
\definecolor{currentfill}{rgb}{0.000000,0.000000,1.000000}%
\pgfsetfillcolor{currentfill}%
\pgfsetlinewidth{1.254687pt}%
\definecolor{currentstroke}{rgb}{0.000000,0.000000,1.000000}%
\pgfsetstrokecolor{currentstroke}%
\pgfsetdash{}{0pt}%
\pgfpathmoveto{\pgfqpoint{1.919484in}{0.903557in}}%
\pgfpathlineto{\pgfqpoint{1.902817in}{0.870224in}}%
\pgfpathlineto{\pgfqpoint{1.886150in}{0.903557in}}%
\pgfpathlineto{\pgfqpoint{1.919484in}{0.903557in}}%
\pgfpathclose%
\pgfusepath{stroke,fill}%
\end{pgfscope}%
\begin{pgfscope}%
\pgfsetbuttcap%
\pgfsetmiterjoin%
\definecolor{currentfill}{rgb}{1.000000,1.000000,1.000000}%
\pgfsetfillcolor{currentfill}%
\pgfsetlinewidth{1.003750pt}%
\definecolor{currentstroke}{rgb}{1.000000,1.000000,1.000000}%
\pgfsetstrokecolor{currentstroke}%
\pgfsetdash{}{0pt}%
\pgfpathmoveto{\pgfqpoint{1.936150in}{1.037384in}}%
\pgfpathlineto{\pgfqpoint{1.936150in}{1.220098in}}%
\pgfpathlineto{\pgfqpoint{1.869484in}{1.220098in}}%
\pgfpathlineto{\pgfqpoint{1.869484in}{1.037384in}}%
\pgfpathclose%
\pgfusepath{stroke,fill}%
\end{pgfscope}%
\begin{pgfscope}%
\definecolor{textcolor}{rgb}{0.000000,0.000000,0.000000}%
\pgfsetstrokecolor{textcolor}%
\pgfsetfillcolor{textcolor}%
\pgftext[x=1.923650in, y=1.029051in, left, base,rotate=90.000000]{\color{textcolor}\rmfamily\fontsize{6.000000}{7.200000}\selectfont P(3)}%
\end{pgfscope}%
\begin{pgfscope}%
\pgfsetroundcap%
\pgfsetroundjoin%
\pgfsetlinewidth{1.254687pt}%
\definecolor{currentstroke}{rgb}{0.000000,0.000000,1.000000}%
\pgfsetstrokecolor{currentstroke}%
\pgfsetdash{}{0pt}%
\pgfpathmoveto{\pgfqpoint{1.754194in}{0.957803in}}%
\pgfpathquadraticcurveto{\pgfqpoint{1.754194in}{0.886178in}}{\pgfqpoint{1.754194in}{0.833964in}}%
\pgfusepath{stroke}%
\end{pgfscope}%
\begin{pgfscope}%
\pgfsetroundcap%
\pgfsetroundjoin%
\definecolor{currentfill}{rgb}{0.000000,0.000000,1.000000}%
\pgfsetfillcolor{currentfill}%
\pgfsetlinewidth{1.254687pt}%
\definecolor{currentstroke}{rgb}{0.000000,0.000000,1.000000}%
\pgfsetstrokecolor{currentstroke}%
\pgfsetdash{}{0pt}%
\pgfpathmoveto{\pgfqpoint{1.770861in}{0.867298in}}%
\pgfpathlineto{\pgfqpoint{1.754194in}{0.833964in}}%
\pgfpathlineto{\pgfqpoint{1.737528in}{0.867298in}}%
\pgfpathlineto{\pgfqpoint{1.770861in}{0.867298in}}%
\pgfpathclose%
\pgfusepath{stroke,fill}%
\end{pgfscope}%
\begin{pgfscope}%
\pgfsetbuttcap%
\pgfsetmiterjoin%
\definecolor{currentfill}{rgb}{1.000000,1.000000,1.000000}%
\pgfsetfillcolor{currentfill}%
\pgfsetlinewidth{1.003750pt}%
\definecolor{currentstroke}{rgb}{1.000000,1.000000,1.000000}%
\pgfsetstrokecolor{currentstroke}%
\pgfsetdash{}{0pt}%
\pgfpathmoveto{\pgfqpoint{1.787528in}{1.001125in}}%
\pgfpathlineto{\pgfqpoint{1.787528in}{1.183839in}}%
\pgfpathlineto{\pgfqpoint{1.720861in}{1.183839in}}%
\pgfpathlineto{\pgfqpoint{1.720861in}{1.001125in}}%
\pgfpathclose%
\pgfusepath{stroke,fill}%
\end{pgfscope}%
\begin{pgfscope}%
\definecolor{textcolor}{rgb}{0.000000,0.000000,0.000000}%
\pgfsetstrokecolor{textcolor}%
\pgfsetfillcolor{textcolor}%
\pgftext[x=1.775028in, y=0.992792in, left, base,rotate=90.000000]{\color{textcolor}\rmfamily\fontsize{6.000000}{7.200000}\selectfont P(4)}%
\end{pgfscope}%
\begin{pgfscope}%
\pgfsetroundcap%
\pgfsetroundjoin%
\pgfsetlinewidth{1.254687pt}%
\definecolor{currentstroke}{rgb}{0.000000,0.000000,1.000000}%
\pgfsetstrokecolor{currentstroke}%
\pgfsetdash{}{0pt}%
\pgfpathmoveto{\pgfqpoint{1.601874in}{0.885162in}}%
\pgfpathquadraticcurveto{\pgfqpoint{1.601874in}{0.813537in}}{\pgfqpoint{1.601874in}{0.761323in}}%
\pgfusepath{stroke}%
\end{pgfscope}%
\begin{pgfscope}%
\pgfsetroundcap%
\pgfsetroundjoin%
\definecolor{currentfill}{rgb}{0.000000,0.000000,1.000000}%
\pgfsetfillcolor{currentfill}%
\pgfsetlinewidth{1.254687pt}%
\definecolor{currentstroke}{rgb}{0.000000,0.000000,1.000000}%
\pgfsetstrokecolor{currentstroke}%
\pgfsetdash{}{0pt}%
\pgfpathmoveto{\pgfqpoint{1.618541in}{0.794657in}}%
\pgfpathlineto{\pgfqpoint{1.601874in}{0.761323in}}%
\pgfpathlineto{\pgfqpoint{1.585207in}{0.794657in}}%
\pgfpathlineto{\pgfqpoint{1.618541in}{0.794657in}}%
\pgfpathclose%
\pgfusepath{stroke,fill}%
\end{pgfscope}%
\begin{pgfscope}%
\pgfsetbuttcap%
\pgfsetmiterjoin%
\definecolor{currentfill}{rgb}{1.000000,1.000000,1.000000}%
\pgfsetfillcolor{currentfill}%
\pgfsetlinewidth{1.003750pt}%
\definecolor{currentstroke}{rgb}{1.000000,1.000000,1.000000}%
\pgfsetstrokecolor{currentstroke}%
\pgfsetdash{}{0pt}%
\pgfpathmoveto{\pgfqpoint{1.635207in}{0.928484in}}%
\pgfpathlineto{\pgfqpoint{1.635207in}{1.111198in}}%
\pgfpathlineto{\pgfqpoint{1.568541in}{1.111198in}}%
\pgfpathlineto{\pgfqpoint{1.568541in}{0.928484in}}%
\pgfpathclose%
\pgfusepath{stroke,fill}%
\end{pgfscope}%
\begin{pgfscope}%
\definecolor{textcolor}{rgb}{0.000000,0.000000,0.000000}%
\pgfsetstrokecolor{textcolor}%
\pgfsetfillcolor{textcolor}%
\pgftext[x=1.622707in, y=0.920151in, left, base,rotate=90.000000]{\color{textcolor}\rmfamily\fontsize{6.000000}{7.200000}\selectfont P(5)}%
\end{pgfscope}%
\begin{pgfscope}%
\pgfsetroundcap%
\pgfsetroundjoin%
\pgfsetlinewidth{1.254687pt}%
\definecolor{currentstroke}{rgb}{0.000000,0.000000,1.000000}%
\pgfsetstrokecolor{currentstroke}%
\pgfsetdash{}{0pt}%
\pgfpathmoveto{\pgfqpoint{1.445940in}{0.807897in}}%
\pgfpathquadraticcurveto{\pgfqpoint{1.445940in}{0.736273in}}{\pgfqpoint{1.445940in}{0.684059in}}%
\pgfusepath{stroke}%
\end{pgfscope}%
\begin{pgfscope}%
\pgfsetroundcap%
\pgfsetroundjoin%
\definecolor{currentfill}{rgb}{0.000000,0.000000,1.000000}%
\pgfsetfillcolor{currentfill}%
\pgfsetlinewidth{1.254687pt}%
\definecolor{currentstroke}{rgb}{0.000000,0.000000,1.000000}%
\pgfsetstrokecolor{currentstroke}%
\pgfsetdash{}{0pt}%
\pgfpathmoveto{\pgfqpoint{1.462607in}{0.717392in}}%
\pgfpathlineto{\pgfqpoint{1.445940in}{0.684059in}}%
\pgfpathlineto{\pgfqpoint{1.429273in}{0.717392in}}%
\pgfpathlineto{\pgfqpoint{1.462607in}{0.717392in}}%
\pgfpathclose%
\pgfusepath{stroke,fill}%
\end{pgfscope}%
\begin{pgfscope}%
\pgfsetbuttcap%
\pgfsetmiterjoin%
\definecolor{currentfill}{rgb}{1.000000,1.000000,1.000000}%
\pgfsetfillcolor{currentfill}%
\pgfsetlinewidth{1.003750pt}%
\definecolor{currentstroke}{rgb}{1.000000,1.000000,1.000000}%
\pgfsetstrokecolor{currentstroke}%
\pgfsetdash{}{0pt}%
\pgfpathmoveto{\pgfqpoint{1.479273in}{0.851220in}}%
\pgfpathlineto{\pgfqpoint{1.479273in}{1.033933in}}%
\pgfpathlineto{\pgfqpoint{1.412607in}{1.033933in}}%
\pgfpathlineto{\pgfqpoint{1.412607in}{0.851220in}}%
\pgfpathclose%
\pgfusepath{stroke,fill}%
\end{pgfscope}%
\begin{pgfscope}%
\definecolor{textcolor}{rgb}{0.000000,0.000000,0.000000}%
\pgfsetstrokecolor{textcolor}%
\pgfsetfillcolor{textcolor}%
\pgftext[x=1.466773in, y=0.842886in, left, base,rotate=90.000000]{\color{textcolor}\rmfamily\fontsize{6.000000}{7.200000}\selectfont P(6)}%
\end{pgfscope}%
\begin{pgfscope}%
\pgfsetroundcap%
\pgfsetroundjoin%
\pgfsetlinewidth{1.254687pt}%
\definecolor{currentstroke}{rgb}{0.000000,0.000000,1.000000}%
\pgfsetstrokecolor{currentstroke}%
\pgfsetdash{}{0pt}%
\pgfpathmoveto{\pgfqpoint{1.286476in}{0.745503in}}%
\pgfpathquadraticcurveto{\pgfqpoint{1.286476in}{0.673879in}}{\pgfqpoint{1.286476in}{0.621665in}}%
\pgfusepath{stroke}%
\end{pgfscope}%
\begin{pgfscope}%
\pgfsetroundcap%
\pgfsetroundjoin%
\definecolor{currentfill}{rgb}{0.000000,0.000000,1.000000}%
\pgfsetfillcolor{currentfill}%
\pgfsetlinewidth{1.254687pt}%
\definecolor{currentstroke}{rgb}{0.000000,0.000000,1.000000}%
\pgfsetstrokecolor{currentstroke}%
\pgfsetdash{}{0pt}%
\pgfpathmoveto{\pgfqpoint{1.303143in}{0.654999in}}%
\pgfpathlineto{\pgfqpoint{1.286476in}{0.621665in}}%
\pgfpathlineto{\pgfqpoint{1.269810in}{0.654999in}}%
\pgfpathlineto{\pgfqpoint{1.303143in}{0.654999in}}%
\pgfpathclose%
\pgfusepath{stroke,fill}%
\end{pgfscope}%
\begin{pgfscope}%
\pgfsetbuttcap%
\pgfsetmiterjoin%
\definecolor{currentfill}{rgb}{1.000000,1.000000,1.000000}%
\pgfsetfillcolor{currentfill}%
\pgfsetlinewidth{1.003750pt}%
\definecolor{currentstroke}{rgb}{1.000000,1.000000,1.000000}%
\pgfsetstrokecolor{currentstroke}%
\pgfsetdash{}{0pt}%
\pgfpathmoveto{\pgfqpoint{1.319810in}{0.788826in}}%
\pgfpathlineto{\pgfqpoint{1.319810in}{0.971539in}}%
\pgfpathlineto{\pgfqpoint{1.253143in}{0.971539in}}%
\pgfpathlineto{\pgfqpoint{1.253143in}{0.788826in}}%
\pgfpathclose%
\pgfusepath{stroke,fill}%
\end{pgfscope}%
\begin{pgfscope}%
\definecolor{textcolor}{rgb}{0.000000,0.000000,0.000000}%
\pgfsetstrokecolor{textcolor}%
\pgfsetfillcolor{textcolor}%
\pgftext[x=1.307310in, y=0.780493in, left, base,rotate=90.000000]{\color{textcolor}\rmfamily\fontsize{6.000000}{7.200000}\selectfont P(7)}%
\end{pgfscope}%
\begin{pgfscope}%
\pgfsetroundcap%
\pgfsetroundjoin%
\pgfsetlinewidth{1.254687pt}%
\definecolor{currentstroke}{rgb}{0.000000,0.000000,1.000000}%
\pgfsetstrokecolor{currentstroke}%
\pgfsetdash{}{0pt}%
\pgfpathmoveto{\pgfqpoint{1.123566in}{0.703857in}}%
\pgfpathquadraticcurveto{\pgfqpoint{1.123566in}{0.632233in}}{\pgfqpoint{1.123566in}{0.580019in}}%
\pgfusepath{stroke}%
\end{pgfscope}%
\begin{pgfscope}%
\pgfsetroundcap%
\pgfsetroundjoin%
\definecolor{currentfill}{rgb}{0.000000,0.000000,1.000000}%
\pgfsetfillcolor{currentfill}%
\pgfsetlinewidth{1.254687pt}%
\definecolor{currentstroke}{rgb}{0.000000,0.000000,1.000000}%
\pgfsetstrokecolor{currentstroke}%
\pgfsetdash{}{0pt}%
\pgfpathmoveto{\pgfqpoint{1.140233in}{0.613353in}}%
\pgfpathlineto{\pgfqpoint{1.123566in}{0.580019in}}%
\pgfpathlineto{\pgfqpoint{1.106899in}{0.613353in}}%
\pgfpathlineto{\pgfqpoint{1.140233in}{0.613353in}}%
\pgfpathclose%
\pgfusepath{stroke,fill}%
\end{pgfscope}%
\begin{pgfscope}%
\pgfsetbuttcap%
\pgfsetmiterjoin%
\definecolor{currentfill}{rgb}{1.000000,1.000000,1.000000}%
\pgfsetfillcolor{currentfill}%
\pgfsetlinewidth{1.003750pt}%
\definecolor{currentstroke}{rgb}{1.000000,1.000000,1.000000}%
\pgfsetstrokecolor{currentstroke}%
\pgfsetdash{}{0pt}%
\pgfpathmoveto{\pgfqpoint{1.156899in}{0.747180in}}%
\pgfpathlineto{\pgfqpoint{1.156899in}{0.929893in}}%
\pgfpathlineto{\pgfqpoint{1.090233in}{0.929893in}}%
\pgfpathlineto{\pgfqpoint{1.090233in}{0.747180in}}%
\pgfpathclose%
\pgfusepath{stroke,fill}%
\end{pgfscope}%
\begin{pgfscope}%
\definecolor{textcolor}{rgb}{0.000000,0.000000,0.000000}%
\pgfsetstrokecolor{textcolor}%
\pgfsetfillcolor{textcolor}%
\pgftext[x=1.144399in, y=0.738847in, left, base,rotate=90.000000]{\color{textcolor}\rmfamily\fontsize{6.000000}{7.200000}\selectfont P(8)}%
\end{pgfscope}%
\begin{pgfscope}%
\pgfsetroundcap%
\pgfsetroundjoin%
\pgfsetlinewidth{1.254687pt}%
\definecolor{currentstroke}{rgb}{0.000000,0.000000,1.000000}%
\pgfsetstrokecolor{currentstroke}%
\pgfsetdash{}{0pt}%
\pgfpathmoveto{\pgfqpoint{0.957293in}{0.680067in}}%
\pgfpathquadraticcurveto{\pgfqpoint{0.957293in}{0.608443in}}{\pgfqpoint{0.957293in}{0.556229in}}%
\pgfusepath{stroke}%
\end{pgfscope}%
\begin{pgfscope}%
\pgfsetroundcap%
\pgfsetroundjoin%
\definecolor{currentfill}{rgb}{0.000000,0.000000,1.000000}%
\pgfsetfillcolor{currentfill}%
\pgfsetlinewidth{1.254687pt}%
\definecolor{currentstroke}{rgb}{0.000000,0.000000,1.000000}%
\pgfsetstrokecolor{currentstroke}%
\pgfsetdash{}{0pt}%
\pgfpathmoveto{\pgfqpoint{0.973960in}{0.589563in}}%
\pgfpathlineto{\pgfqpoint{0.957293in}{0.556229in}}%
\pgfpathlineto{\pgfqpoint{0.940627in}{0.589563in}}%
\pgfpathlineto{\pgfqpoint{0.973960in}{0.589563in}}%
\pgfpathclose%
\pgfusepath{stroke,fill}%
\end{pgfscope}%
\begin{pgfscope}%
\pgfsetbuttcap%
\pgfsetmiterjoin%
\definecolor{currentfill}{rgb}{1.000000,1.000000,1.000000}%
\pgfsetfillcolor{currentfill}%
\pgfsetlinewidth{1.003750pt}%
\definecolor{currentstroke}{rgb}{1.000000,1.000000,1.000000}%
\pgfsetstrokecolor{currentstroke}%
\pgfsetdash{}{0pt}%
\pgfpathmoveto{\pgfqpoint{0.990627in}{0.723390in}}%
\pgfpathlineto{\pgfqpoint{0.990627in}{0.906103in}}%
\pgfpathlineto{\pgfqpoint{0.923960in}{0.906103in}}%
\pgfpathlineto{\pgfqpoint{0.923960in}{0.723390in}}%
\pgfpathclose%
\pgfusepath{stroke,fill}%
\end{pgfscope}%
\begin{pgfscope}%
\definecolor{textcolor}{rgb}{0.000000,0.000000,0.000000}%
\pgfsetstrokecolor{textcolor}%
\pgfsetfillcolor{textcolor}%
\pgftext[x=0.978127in, y=0.715057in, left, base,rotate=90.000000]{\color{textcolor}\rmfamily\fontsize{6.000000}{7.200000}\selectfont P(9)}%
\end{pgfscope}%
\begin{pgfscope}%
\pgfsetroundcap%
\pgfsetroundjoin%
\pgfsetlinewidth{1.254687pt}%
\definecolor{currentstroke}{rgb}{0.000000,0.500000,0.000000}%
\pgfsetstrokecolor{currentstroke}%
\pgfsetdash{}{0pt}%
\pgfpathmoveto{\pgfqpoint{2.458660in}{0.863812in}}%
\pgfpathquadraticcurveto{\pgfqpoint{2.458660in}{0.792189in}}{\pgfqpoint{2.458660in}{0.739976in}}%
\pgfusepath{stroke}%
\end{pgfscope}%
\begin{pgfscope}%
\pgfsetroundcap%
\pgfsetroundjoin%
\definecolor{currentfill}{rgb}{0.000000,0.500000,0.000000}%
\pgfsetfillcolor{currentfill}%
\pgfsetlinewidth{1.254687pt}%
\definecolor{currentstroke}{rgb}{0.000000,0.500000,0.000000}%
\pgfsetstrokecolor{currentstroke}%
\pgfsetdash{}{0pt}%
\pgfpathmoveto{\pgfqpoint{2.475326in}{0.773310in}}%
\pgfpathlineto{\pgfqpoint{2.458660in}{0.739976in}}%
\pgfpathlineto{\pgfqpoint{2.441993in}{0.773310in}}%
\pgfpathlineto{\pgfqpoint{2.475326in}{0.773310in}}%
\pgfpathclose%
\pgfusepath{stroke,fill}%
\end{pgfscope}%
\begin{pgfscope}%
\pgfsetbuttcap%
\pgfsetmiterjoin%
\definecolor{currentfill}{rgb}{1.000000,1.000000,1.000000}%
\pgfsetfillcolor{currentfill}%
\pgfsetlinewidth{1.003750pt}%
\definecolor{currentstroke}{rgb}{1.000000,1.000000,1.000000}%
\pgfsetstrokecolor{currentstroke}%
\pgfsetdash{}{0pt}%
\pgfpathmoveto{\pgfqpoint{2.491993in}{0.907137in}}%
\pgfpathlineto{\pgfqpoint{2.491993in}{1.095252in}}%
\pgfpathlineto{\pgfqpoint{2.425326in}{1.095252in}}%
\pgfpathlineto{\pgfqpoint{2.425326in}{0.907137in}}%
\pgfpathclose%
\pgfusepath{stroke,fill}%
\end{pgfscope}%
\begin{pgfscope}%
\definecolor{textcolor}{rgb}{0.000000,0.000000,0.000000}%
\pgfsetstrokecolor{textcolor}%
\pgfsetfillcolor{textcolor}%
\pgftext[x=2.479493in, y=0.898804in, left, base,rotate=90.000000]{\color{textcolor}\rmfamily\fontsize{6.000000}{7.200000}\selectfont R(0)}%
\end{pgfscope}%
\begin{pgfscope}%
\pgfsetroundcap%
\pgfsetroundjoin%
\pgfsetlinewidth{1.254687pt}%
\definecolor{currentstroke}{rgb}{0.000000,0.500000,0.000000}%
\pgfsetstrokecolor{currentstroke}%
\pgfsetdash{}{0pt}%
\pgfpathmoveto{\pgfqpoint{2.587539in}{1.028536in}}%
\pgfpathquadraticcurveto{\pgfqpoint{2.587539in}{0.956913in}}{\pgfqpoint{2.587539in}{0.904700in}}%
\pgfusepath{stroke}%
\end{pgfscope}%
\begin{pgfscope}%
\pgfsetroundcap%
\pgfsetroundjoin%
\definecolor{currentfill}{rgb}{0.000000,0.500000,0.000000}%
\pgfsetfillcolor{currentfill}%
\pgfsetlinewidth{1.254687pt}%
\definecolor{currentstroke}{rgb}{0.000000,0.500000,0.000000}%
\pgfsetstrokecolor{currentstroke}%
\pgfsetdash{}{0pt}%
\pgfpathmoveto{\pgfqpoint{2.604206in}{0.938034in}}%
\pgfpathlineto{\pgfqpoint{2.587539in}{0.904700in}}%
\pgfpathlineto{\pgfqpoint{2.570873in}{0.938034in}}%
\pgfpathlineto{\pgfqpoint{2.604206in}{0.938034in}}%
\pgfpathclose%
\pgfusepath{stroke,fill}%
\end{pgfscope}%
\begin{pgfscope}%
\pgfsetbuttcap%
\pgfsetmiterjoin%
\definecolor{currentfill}{rgb}{1.000000,1.000000,1.000000}%
\pgfsetfillcolor{currentfill}%
\pgfsetlinewidth{1.003750pt}%
\definecolor{currentstroke}{rgb}{1.000000,1.000000,1.000000}%
\pgfsetstrokecolor{currentstroke}%
\pgfsetdash{}{0pt}%
\pgfpathmoveto{\pgfqpoint{2.620873in}{1.071861in}}%
\pgfpathlineto{\pgfqpoint{2.620873in}{1.259976in}}%
\pgfpathlineto{\pgfqpoint{2.554206in}{1.259976in}}%
\pgfpathlineto{\pgfqpoint{2.554206in}{1.071861in}}%
\pgfpathclose%
\pgfusepath{stroke,fill}%
\end{pgfscope}%
\begin{pgfscope}%
\definecolor{textcolor}{rgb}{0.000000,0.000000,0.000000}%
\pgfsetstrokecolor{textcolor}%
\pgfsetfillcolor{textcolor}%
\pgftext[x=2.608373in, y=1.063528in, left, base,rotate=90.000000]{\color{textcolor}\rmfamily\fontsize{6.000000}{7.200000}\selectfont R(1)}%
\end{pgfscope}%
\begin{pgfscope}%
\pgfsetroundcap%
\pgfsetroundjoin%
\pgfsetlinewidth{1.254687pt}%
\definecolor{currentstroke}{rgb}{0.000000,0.500000,0.000000}%
\pgfsetstrokecolor{currentstroke}%
\pgfsetdash{}{0pt}%
\pgfpathmoveto{\pgfqpoint{2.712219in}{1.111902in}}%
\pgfpathquadraticcurveto{\pgfqpoint{2.712219in}{1.040279in}}{\pgfqpoint{2.712219in}{0.988066in}}%
\pgfusepath{stroke}%
\end{pgfscope}%
\begin{pgfscope}%
\pgfsetroundcap%
\pgfsetroundjoin%
\definecolor{currentfill}{rgb}{0.000000,0.500000,0.000000}%
\pgfsetfillcolor{currentfill}%
\pgfsetlinewidth{1.254687pt}%
\definecolor{currentstroke}{rgb}{0.000000,0.500000,0.000000}%
\pgfsetstrokecolor{currentstroke}%
\pgfsetdash{}{0pt}%
\pgfpathmoveto{\pgfqpoint{2.728886in}{1.021400in}}%
\pgfpathlineto{\pgfqpoint{2.712219in}{0.988066in}}%
\pgfpathlineto{\pgfqpoint{2.695553in}{1.021400in}}%
\pgfpathlineto{\pgfqpoint{2.728886in}{1.021400in}}%
\pgfpathclose%
\pgfusepath{stroke,fill}%
\end{pgfscope}%
\begin{pgfscope}%
\pgfsetbuttcap%
\pgfsetmiterjoin%
\definecolor{currentfill}{rgb}{1.000000,1.000000,1.000000}%
\pgfsetfillcolor{currentfill}%
\pgfsetlinewidth{1.003750pt}%
\definecolor{currentstroke}{rgb}{1.000000,1.000000,1.000000}%
\pgfsetstrokecolor{currentstroke}%
\pgfsetdash{}{0pt}%
\pgfpathmoveto{\pgfqpoint{2.745553in}{1.155227in}}%
\pgfpathlineto{\pgfqpoint{2.745553in}{1.343342in}}%
\pgfpathlineto{\pgfqpoint{2.678886in}{1.343342in}}%
\pgfpathlineto{\pgfqpoint{2.678886in}{1.155227in}}%
\pgfpathclose%
\pgfusepath{stroke,fill}%
\end{pgfscope}%
\begin{pgfscope}%
\definecolor{textcolor}{rgb}{0.000000,0.000000,0.000000}%
\pgfsetstrokecolor{textcolor}%
\pgfsetfillcolor{textcolor}%
\pgftext[x=2.733053in, y=1.146894in, left, base,rotate=90.000000]{\color{textcolor}\rmfamily\fontsize{6.000000}{7.200000}\selectfont R(2)}%
\end{pgfscope}%
\begin{pgfscope}%
\pgfsetroundcap%
\pgfsetroundjoin%
\pgfsetlinewidth{1.254687pt}%
\definecolor{currentstroke}{rgb}{0.000000,0.500000,0.000000}%
\pgfsetstrokecolor{currentstroke}%
\pgfsetdash{}{0pt}%
\pgfpathmoveto{\pgfqpoint{2.832616in}{1.105411in}}%
\pgfpathquadraticcurveto{\pgfqpoint{2.832616in}{1.033788in}}{\pgfqpoint{2.832616in}{0.981575in}}%
\pgfusepath{stroke}%
\end{pgfscope}%
\begin{pgfscope}%
\pgfsetroundcap%
\pgfsetroundjoin%
\definecolor{currentfill}{rgb}{0.000000,0.500000,0.000000}%
\pgfsetfillcolor{currentfill}%
\pgfsetlinewidth{1.254687pt}%
\definecolor{currentstroke}{rgb}{0.000000,0.500000,0.000000}%
\pgfsetstrokecolor{currentstroke}%
\pgfsetdash{}{0pt}%
\pgfpathmoveto{\pgfqpoint{2.849283in}{1.014908in}}%
\pgfpathlineto{\pgfqpoint{2.832616in}{0.981575in}}%
\pgfpathlineto{\pgfqpoint{2.815949in}{1.014908in}}%
\pgfpathlineto{\pgfqpoint{2.849283in}{1.014908in}}%
\pgfpathclose%
\pgfusepath{stroke,fill}%
\end{pgfscope}%
\begin{pgfscope}%
\pgfsetbuttcap%
\pgfsetmiterjoin%
\definecolor{currentfill}{rgb}{1.000000,1.000000,1.000000}%
\pgfsetfillcolor{currentfill}%
\pgfsetlinewidth{1.003750pt}%
\definecolor{currentstroke}{rgb}{1.000000,1.000000,1.000000}%
\pgfsetstrokecolor{currentstroke}%
\pgfsetdash{}{0pt}%
\pgfpathmoveto{\pgfqpoint{2.865949in}{1.148736in}}%
\pgfpathlineto{\pgfqpoint{2.865949in}{1.336851in}}%
\pgfpathlineto{\pgfqpoint{2.799283in}{1.336851in}}%
\pgfpathlineto{\pgfqpoint{2.799283in}{1.148736in}}%
\pgfpathclose%
\pgfusepath{stroke,fill}%
\end{pgfscope}%
\begin{pgfscope}%
\definecolor{textcolor}{rgb}{0.000000,0.000000,0.000000}%
\pgfsetstrokecolor{textcolor}%
\pgfsetfillcolor{textcolor}%
\pgftext[x=2.853449in, y=1.140403in, left, base,rotate=90.000000]{\color{textcolor}\rmfamily\fontsize{6.000000}{7.200000}\selectfont R(3)}%
\end{pgfscope}%
\begin{pgfscope}%
\pgfsetroundcap%
\pgfsetroundjoin%
\pgfsetlinewidth{1.254687pt}%
\definecolor{currentstroke}{rgb}{0.000000,0.500000,0.000000}%
\pgfsetstrokecolor{currentstroke}%
\pgfsetdash{}{0pt}%
\pgfpathmoveto{\pgfqpoint{2.948645in}{1.032249in}}%
\pgfpathquadraticcurveto{\pgfqpoint{2.948645in}{0.960626in}}{\pgfqpoint{2.948645in}{0.908413in}}%
\pgfusepath{stroke}%
\end{pgfscope}%
\begin{pgfscope}%
\pgfsetroundcap%
\pgfsetroundjoin%
\definecolor{currentfill}{rgb}{0.000000,0.500000,0.000000}%
\pgfsetfillcolor{currentfill}%
\pgfsetlinewidth{1.254687pt}%
\definecolor{currentstroke}{rgb}{0.000000,0.500000,0.000000}%
\pgfsetstrokecolor{currentstroke}%
\pgfsetdash{}{0pt}%
\pgfpathmoveto{\pgfqpoint{2.965312in}{0.941746in}}%
\pgfpathlineto{\pgfqpoint{2.948645in}{0.908413in}}%
\pgfpathlineto{\pgfqpoint{2.931979in}{0.941746in}}%
\pgfpathlineto{\pgfqpoint{2.965312in}{0.941746in}}%
\pgfpathclose%
\pgfusepath{stroke,fill}%
\end{pgfscope}%
\begin{pgfscope}%
\pgfsetbuttcap%
\pgfsetmiterjoin%
\definecolor{currentfill}{rgb}{1.000000,1.000000,1.000000}%
\pgfsetfillcolor{currentfill}%
\pgfsetlinewidth{1.003750pt}%
\definecolor{currentstroke}{rgb}{1.000000,1.000000,1.000000}%
\pgfsetstrokecolor{currentstroke}%
\pgfsetdash{}{0pt}%
\pgfpathmoveto{\pgfqpoint{2.981979in}{1.075574in}}%
\pgfpathlineto{\pgfqpoint{2.981979in}{1.263689in}}%
\pgfpathlineto{\pgfqpoint{2.915312in}{1.263689in}}%
\pgfpathlineto{\pgfqpoint{2.915312in}{1.075574in}}%
\pgfpathclose%
\pgfusepath{stroke,fill}%
\end{pgfscope}%
\begin{pgfscope}%
\definecolor{textcolor}{rgb}{0.000000,0.000000,0.000000}%
\pgfsetstrokecolor{textcolor}%
\pgfsetfillcolor{textcolor}%
\pgftext[x=2.969479in, y=1.067241in, left, base,rotate=90.000000]{\color{textcolor}\rmfamily\fontsize{6.000000}{7.200000}\selectfont R(4)}%
\end{pgfscope}%
\begin{pgfscope}%
\pgfsetroundcap%
\pgfsetroundjoin%
\pgfsetlinewidth{1.254687pt}%
\definecolor{currentstroke}{rgb}{0.000000,0.500000,0.000000}%
\pgfsetstrokecolor{currentstroke}%
\pgfsetdash{}{0pt}%
\pgfpathmoveto{\pgfqpoint{3.060224in}{0.930190in}}%
\pgfpathquadraticcurveto{\pgfqpoint{3.060224in}{0.858567in}}{\pgfqpoint{3.060224in}{0.806354in}}%
\pgfusepath{stroke}%
\end{pgfscope}%
\begin{pgfscope}%
\pgfsetroundcap%
\pgfsetroundjoin%
\definecolor{currentfill}{rgb}{0.000000,0.500000,0.000000}%
\pgfsetfillcolor{currentfill}%
\pgfsetlinewidth{1.254687pt}%
\definecolor{currentstroke}{rgb}{0.000000,0.500000,0.000000}%
\pgfsetstrokecolor{currentstroke}%
\pgfsetdash{}{0pt}%
\pgfpathmoveto{\pgfqpoint{3.076891in}{0.839687in}}%
\pgfpathlineto{\pgfqpoint{3.060224in}{0.806354in}}%
\pgfpathlineto{\pgfqpoint{3.043557in}{0.839687in}}%
\pgfpathlineto{\pgfqpoint{3.076891in}{0.839687in}}%
\pgfpathclose%
\pgfusepath{stroke,fill}%
\end{pgfscope}%
\begin{pgfscope}%
\pgfsetbuttcap%
\pgfsetmiterjoin%
\definecolor{currentfill}{rgb}{1.000000,1.000000,1.000000}%
\pgfsetfillcolor{currentfill}%
\pgfsetlinewidth{1.003750pt}%
\definecolor{currentstroke}{rgb}{1.000000,1.000000,1.000000}%
\pgfsetstrokecolor{currentstroke}%
\pgfsetdash{}{0pt}%
\pgfpathmoveto{\pgfqpoint{3.093557in}{0.973515in}}%
\pgfpathlineto{\pgfqpoint{3.093557in}{1.161630in}}%
\pgfpathlineto{\pgfqpoint{3.026891in}{1.161630in}}%
\pgfpathlineto{\pgfqpoint{3.026891in}{0.973515in}}%
\pgfpathclose%
\pgfusepath{stroke,fill}%
\end{pgfscope}%
\begin{pgfscope}%
\definecolor{textcolor}{rgb}{0.000000,0.000000,0.000000}%
\pgfsetstrokecolor{textcolor}%
\pgfsetfillcolor{textcolor}%
\pgftext[x=3.081057in, y=0.965182in, left, base,rotate=90.000000]{\color{textcolor}\rmfamily\fontsize{6.000000}{7.200000}\selectfont R(5)}%
\end{pgfscope}%
\begin{pgfscope}%
\pgfsetroundcap%
\pgfsetroundjoin%
\pgfsetlinewidth{1.254687pt}%
\definecolor{currentstroke}{rgb}{0.000000,0.500000,0.000000}%
\pgfsetstrokecolor{currentstroke}%
\pgfsetdash{}{0pt}%
\pgfpathmoveto{\pgfqpoint{3.167268in}{0.832543in}}%
\pgfpathquadraticcurveto{\pgfqpoint{3.167268in}{0.760920in}}{\pgfqpoint{3.167268in}{0.708707in}}%
\pgfusepath{stroke}%
\end{pgfscope}%
\begin{pgfscope}%
\pgfsetroundcap%
\pgfsetroundjoin%
\definecolor{currentfill}{rgb}{0.000000,0.500000,0.000000}%
\pgfsetfillcolor{currentfill}%
\pgfsetlinewidth{1.254687pt}%
\definecolor{currentstroke}{rgb}{0.000000,0.500000,0.000000}%
\pgfsetstrokecolor{currentstroke}%
\pgfsetdash{}{0pt}%
\pgfpathmoveto{\pgfqpoint{3.183934in}{0.742040in}}%
\pgfpathlineto{\pgfqpoint{3.167268in}{0.708707in}}%
\pgfpathlineto{\pgfqpoint{3.150601in}{0.742040in}}%
\pgfpathlineto{\pgfqpoint{3.183934in}{0.742040in}}%
\pgfpathclose%
\pgfusepath{stroke,fill}%
\end{pgfscope}%
\begin{pgfscope}%
\pgfsetbuttcap%
\pgfsetmiterjoin%
\definecolor{currentfill}{rgb}{1.000000,1.000000,1.000000}%
\pgfsetfillcolor{currentfill}%
\pgfsetlinewidth{1.003750pt}%
\definecolor{currentstroke}{rgb}{1.000000,1.000000,1.000000}%
\pgfsetstrokecolor{currentstroke}%
\pgfsetdash{}{0pt}%
\pgfpathmoveto{\pgfqpoint{3.200601in}{0.875868in}}%
\pgfpathlineto{\pgfqpoint{3.200601in}{1.063983in}}%
\pgfpathlineto{\pgfqpoint{3.133934in}{1.063983in}}%
\pgfpathlineto{\pgfqpoint{3.133934in}{0.875868in}}%
\pgfpathclose%
\pgfusepath{stroke,fill}%
\end{pgfscope}%
\begin{pgfscope}%
\definecolor{textcolor}{rgb}{0.000000,0.000000,0.000000}%
\pgfsetstrokecolor{textcolor}%
\pgfsetfillcolor{textcolor}%
\pgftext[x=3.188101in, y=0.867535in, left, base,rotate=90.000000]{\color{textcolor}\rmfamily\fontsize{6.000000}{7.200000}\selectfont R(6)}%
\end{pgfscope}%
\begin{pgfscope}%
\pgfsetroundcap%
\pgfsetroundjoin%
\pgfsetlinewidth{1.254687pt}%
\definecolor{currentstroke}{rgb}{0.000000,0.500000,0.000000}%
\pgfsetstrokecolor{currentstroke}%
\pgfsetdash{}{0pt}%
\pgfpathmoveto{\pgfqpoint{3.269693in}{0.757714in}}%
\pgfpathquadraticcurveto{\pgfqpoint{3.269693in}{0.686091in}}{\pgfqpoint{3.269693in}{0.633879in}}%
\pgfusepath{stroke}%
\end{pgfscope}%
\begin{pgfscope}%
\pgfsetroundcap%
\pgfsetroundjoin%
\definecolor{currentfill}{rgb}{0.000000,0.500000,0.000000}%
\pgfsetfillcolor{currentfill}%
\pgfsetlinewidth{1.254687pt}%
\definecolor{currentstroke}{rgb}{0.000000,0.500000,0.000000}%
\pgfsetstrokecolor{currentstroke}%
\pgfsetdash{}{0pt}%
\pgfpathmoveto{\pgfqpoint{3.286360in}{0.667212in}}%
\pgfpathlineto{\pgfqpoint{3.269693in}{0.633879in}}%
\pgfpathlineto{\pgfqpoint{3.253027in}{0.667212in}}%
\pgfpathlineto{\pgfqpoint{3.286360in}{0.667212in}}%
\pgfpathclose%
\pgfusepath{stroke,fill}%
\end{pgfscope}%
\begin{pgfscope}%
\pgfsetbuttcap%
\pgfsetmiterjoin%
\definecolor{currentfill}{rgb}{1.000000,1.000000,1.000000}%
\pgfsetfillcolor{currentfill}%
\pgfsetlinewidth{1.003750pt}%
\definecolor{currentstroke}{rgb}{1.000000,1.000000,1.000000}%
\pgfsetstrokecolor{currentstroke}%
\pgfsetdash{}{0pt}%
\pgfpathmoveto{\pgfqpoint{3.303027in}{0.801040in}}%
\pgfpathlineto{\pgfqpoint{3.303027in}{0.989155in}}%
\pgfpathlineto{\pgfqpoint{3.236360in}{0.989155in}}%
\pgfpathlineto{\pgfqpoint{3.236360in}{0.801040in}}%
\pgfpathclose%
\pgfusepath{stroke,fill}%
\end{pgfscope}%
\begin{pgfscope}%
\definecolor{textcolor}{rgb}{0.000000,0.000000,0.000000}%
\pgfsetstrokecolor{textcolor}%
\pgfsetfillcolor{textcolor}%
\pgftext[x=3.290527in, y=0.792706in, left, base,rotate=90.000000]{\color{textcolor}\rmfamily\fontsize{6.000000}{7.200000}\selectfont R(7)}%
\end{pgfscope}%
\begin{pgfscope}%
\pgfsetroundcap%
\pgfsetroundjoin%
\pgfsetlinewidth{1.254687pt}%
\definecolor{currentstroke}{rgb}{0.000000,0.500000,0.000000}%
\pgfsetstrokecolor{currentstroke}%
\pgfsetdash{}{0pt}%
\pgfpathmoveto{\pgfqpoint{3.367417in}{0.709336in}}%
\pgfpathquadraticcurveto{\pgfqpoint{3.367417in}{0.637713in}}{\pgfqpoint{3.367417in}{0.585500in}}%
\pgfusepath{stroke}%
\end{pgfscope}%
\begin{pgfscope}%
\pgfsetroundcap%
\pgfsetroundjoin%
\definecolor{currentfill}{rgb}{0.000000,0.500000,0.000000}%
\pgfsetfillcolor{currentfill}%
\pgfsetlinewidth{1.254687pt}%
\definecolor{currentstroke}{rgb}{0.000000,0.500000,0.000000}%
\pgfsetstrokecolor{currentstroke}%
\pgfsetdash{}{0pt}%
\pgfpathmoveto{\pgfqpoint{3.384083in}{0.618833in}}%
\pgfpathlineto{\pgfqpoint{3.367417in}{0.585500in}}%
\pgfpathlineto{\pgfqpoint{3.350750in}{0.618833in}}%
\pgfpathlineto{\pgfqpoint{3.384083in}{0.618833in}}%
\pgfpathclose%
\pgfusepath{stroke,fill}%
\end{pgfscope}%
\begin{pgfscope}%
\pgfsetbuttcap%
\pgfsetmiterjoin%
\definecolor{currentfill}{rgb}{1.000000,1.000000,1.000000}%
\pgfsetfillcolor{currentfill}%
\pgfsetlinewidth{1.003750pt}%
\definecolor{currentstroke}{rgb}{1.000000,1.000000,1.000000}%
\pgfsetstrokecolor{currentstroke}%
\pgfsetdash{}{0pt}%
\pgfpathmoveto{\pgfqpoint{3.400750in}{0.752661in}}%
\pgfpathlineto{\pgfqpoint{3.400750in}{0.940776in}}%
\pgfpathlineto{\pgfqpoint{3.334083in}{0.940776in}}%
\pgfpathlineto{\pgfqpoint{3.334083in}{0.752661in}}%
\pgfpathclose%
\pgfusepath{stroke,fill}%
\end{pgfscope}%
\begin{pgfscope}%
\definecolor{textcolor}{rgb}{0.000000,0.000000,0.000000}%
\pgfsetstrokecolor{textcolor}%
\pgfsetfillcolor{textcolor}%
\pgftext[x=3.388250in, y=0.744328in, left, base,rotate=90.000000]{\color{textcolor}\rmfamily\fontsize{6.000000}{7.200000}\selectfont R(8)}%
\end{pgfscope}%
\begin{pgfscope}%
\definecolor{textcolor}{rgb}{0.000000,0.000000,0.000000}%
\pgfsetstrokecolor{textcolor}%
\pgfsetfillcolor{textcolor}%
\pgftext[x=0.198908in, y=2.999500in, left, base,rotate=90.000000]{\color{textcolor}\rmfamily\fontsize{8.000000}{9.600000}\selectfont Energy (eV)}%
\end{pgfscope}%
\end{pgfpicture}%
\makeatother%
\endgroup%

    \caption{Rotational energy levels for the ground vibrational state \( {(v'' = 0)} \) and the first excited vibrational state \( {(v' = 1)} \) in a diatomic molecule.
      The vertical arrows indicated allowed transitions in the \emph{R} and \emph{P} branches; the numbers above the arrows index the value \( J'' \) of the lower state.
      Transitions in the \emph{Q} branch \( {(\increment J = 0)} \) are not shown, since they are not infrared-active.
      }
    \label{fig:typ_spectrum}
\end{figure}

The two series of lines given in \cref{eq:r_branch,eq:p_branch} are called \emph{R} and \emph{P} branches, respectively. 
These allowed transitions are indicated on the energy-level diagram given in \cref{fig:typ_spectrum}. 
If \( \alpha_e \) were negligible, \cref{eq:r_branch,eq:p_branch} would predict a series of equally spaced lines separated by \( 2 B_e \) except for a missing line at \( \widetilde{\nu}_0 \). 
The effect of interaction between rotation and vibration (non-zero \( \alpha_e \)) is to squeeze lines in the \emph{R} branch closer together and spread the lines in the \emph{P} branch farther apart, as shown for a typical spectrum in \cref{fig:typ_spectrum}. 

For convenience, let us introduce a new quantity \( m \), where \( {m = J'' + 1} \) for the \emph{R} branch and \( {m = -J''} \) for the \emph{P} branch, as shown in \cref{fig:typ_spectrum}. 
It is now possible to replace \cref{eq:r_branch,eq:p_branch} with a single equation:
\begin{equation}
	\widetilde{\nu}\br{m} = \widetilde{\nu}_0 + \br{2 B_e - 2 \alpha_e} m - \alpha_e m^2 \, ,
	\label{eq:both_branches1}
\end{equation}
where \( m \) takes all integral values and \( {m = 0} \) yields the frequency \( \widetilde{\nu}_0 \) of the forbidden ``purely vibrational'' transition. 
If one retains the \( D_J \) term of \cref{eq:vib_rot_energies} (which assumes \( {D'' = D' = D_J} \)), \cref{eq:both_branches1} takes the form
\begin{equation}
	\widetilde{\nu}\br{m} = \widetilde{\nu}_0 + \br{2 B_e - 2 \alpha_e} m - \alpha_e m^2 - 4 D_J m^3 \, .
	\label{eq:both_branches2}
\end{equation}
Thus, a multiple linear regression can be performed to determine \( \widetilde{\nu}_0 \), \( B_e \), \( \alpha_e \), and \( D_J \).

% subsection selection_rules (end)

\subsection{Isotope Effect} % (fold)
\label{sub:isotope_effect}

When an isotopic substitution is made in a diatomic molecule, the equilibrium bond length, \( r_e \), and the force constant, \( k \), are unchanged, since they depend only on the behavior of the bonding electrons. 
However, the reduced mass, \( \mu \), does change, and this will affect the rotation and vibration of the molecule. 
In the case of rotation, the isotope effect can be easily stated. From the definitions of \( B_e \) and \( I \), we see that 
\begin{equation}
	\frac{B_e^*}{B_e} = \frac{\mu\hphantom{^*}}{\mu^*} \, ,
	\label{eq:eff_mass_isotope_relation}
\end{equation}
where an asterisk is used to distinguish one isotopologue\sidenote{two molecules which differ only by the nuclear isotopes which make up the molecule} from another. 

For a harmonic oscillator model, the frequency \( \widetilde{\nu}_e \) in wavenumbers is given by 
\begin{equation}
	\widetilde{\nu}_e = \frac{1}{2 \pi c} \sqrt{\frac{k}{\mu}} \, ,
	\label{eq:ho_freq}
\end{equation}
which leads to the relation 
\begin{equation}
	\frac{\widetilde{\nu}_e^*}{\widetilde{\nu}_e} = \sqrt{\frac{\mu\hphantom{^*}}{\mu^*}}\, .
	\label{eq:ho_freq_relation}
\end{equation}
The ratio \( \widetilde{\nu}_e^* / \widetilde{\nu}_e \) differs slightly from this harmonic ratio due to deviation of the true potential function from a quadratic form, as depicted in \cref{fig:potential_curves}. 
A closer approximation to the solid curve can be obtained by adding cubic and higher anharmonic terms to the interatomic potential \( U\br{r} \), like so:
\begin{equation}
	U\br{r} = \frac{1}{2} k \br{r - r_e}^2 + c_1 \br{r - r_e}^3 + c_2 \br{r - r_e}^4 + \ldots
	\label{eq:anharmonic_expansion}
\end{equation}
Although somewhat complicated, it can be shown~\autocite{herzberg89,levine75} that the \( c_1 \) and \( c_2 \) terms yield, as the first correction to the energy levels, precisely the \( -\widetilde{\nu}_e \chi_e \br{v + {1}/{2}}^2 \) term given in \cref{eq:vib_rot_energies}. 
A similar conclusion is reached if \( U\br{r} \) is taken to have the Morse potential form given by\sidenote{
	The parameter \( a \) (in units of \unit{1/\angstrom}) is determined by equating \( k_e \) to the curvature of the Morse potential at \( r = r_e \), yielding
		\[
		  a = \sqrt{\frac{k_e}{2 D_e}} \, .
		\]
	This three-parameter function has the desired values of \( U\br{r = r_e} = -D_e \) and \( U\br{r = \infty} = 0 \). 
	It provides a very good approximation to the real potential energy curve at all distances except \( r \ll r_e \), a region of no practical significance. 
} 
\begin{equation}
	U\br{r} = D_e \br{\br{1 - \exp\br{-a\br{r - r_e}}}^2 - 1} \, .
	\label{eq_morse_potential}
\end{equation}
In both cases, the mass dependence of \( \widetilde{\nu}_e \chi_e \) is found to be greater than for \( \widetilde{\nu}_e \) and is
\begin{equation}
	\frac{\widetilde{\nu}_e^* \chi_e^*}{\widetilde{\nu}_e \chi_e } = \frac{\mu\hphantom{^*}}{\mu^*} \, .
	\label{eq:ve_isotope_relation}
\end{equation}
\Cref{eq:ho_freq_relation,eq:ve_isotope_relation} are useful in obtaining the \( \widetilde{\nu}_0^* \) counterpart of \cref{eq:forbid_transition},
\begin{equation}
	\widetilde{\nu}_0^* = \widetilde{\nu}_e^* - 2\widetilde{\nu}_e^* \chi_e^* = \widetilde{\nu}_e \sqrt{\frac{\mu\hphantom{^*}}{\mu^*}} - 2 \widetilde{\nu}_e \chi_e \frac{\mu\hphantom{^*}}{\mu^*} \, ,
	\label{eq:forbidden_isotope_transition} 
\end{equation}
and it is seen that a measurement of \( \widetilde{\nu}_0 \) for \ch{HCl} and \ch{DCl} suffices for a determination of \( \widetilde{\nu}_e \) and \( \widetilde{\nu}_e \chi_e \). 
Alternatively or course, the latter constants can be determined from overtone vibrations \( \br{\increment v > 0} \) of a single isotopic form.~\Footcite[See Exp. 39 in][]{nibler14} 
However, such overtones generally have low intensity and the transitions may fall outside the range of many infrared instruments, so the isotopic shift method is used in the present experiment. 

Since \ch{HCl} gas is a mixture of \ch{H^{35}Cl} and \ch{H^{37}Cl} molecules, a chlorine isotope effect will also be present. 
Since the ratio of the reduced masses is only \num{1.0015}, high resolution is required to detect this effect. 
\ch{HCl} is predominantly \ch{H^{35}Cl}. 
If a single peak is seen in the \ch{HCl} manifold, it can be assumed that the bands are due to \ch{H^{37}Cl}. 
At a resolution of \qty{<1}{\wn}, the band should consist of a series of doublets, with the larger peak due to \ch{H^{35}Cl} and the smaller peak due to \ch{H^{37}Cl}.
If deuterium is substituted for hydrogen, the ratio of the reduced masses, \( \mu\br{\ch{D^{35}Cl}} / \mu\br{\ch{H^{35}Cl}} \), is \num{1.946} and the isotope effect is quite large. 

% subsection isotope_effect (end)

\subsection[Vibrational Partition Function]{Vibrational Partition Function~\autocite{levine95,mcquarrie73,lewis61}} % (fold)
\label{sub:vibrational_partition_function}

The thermodynamic quantities for an ideal gas can usually be expressed as a sum of translational, rotational, and vibrational contributions. 
We shall consider here the heat capacity at constant volume. 
At room temperature and above, the translational and rotational contributions to \( C_v \) are constants that are independent of temperature. 
For \ch{HCl} and \ch{DCl},\sidenote{diatomic, and thus linear, molecules} the molar quantities are 
\begin{align}
	\begin{split}
		\widetilde{C}_v\br{\mtext{trans}}	&=	\frac{3}{2}R \\
		\widetilde{C}_v\br{\mtext{rot}}		&=	R \, .
	\end{split}
	\label{eq:c_v_rot_vib}
\end{align}
The vibrational contribution to \( \widetilde{C}_v \) varies with temperature and can be calculated from the vibrational partition function, \( q_\mtext{vib} \) using 
\begin{equation}
	\widetilde{C}_v\br{\mtext{vib}} = R \pdv{}{T} \br{T^2 \pdv{\ln{q_\mtext{vib}}}{T}} \, .
	\label{eq:vib_part_func1}
\end{equation}
The partition function, \( q_\mtext{vib} \), of \ch{HCl} or \ch{DCl} is well-approximated by the harmonic-oscillator partition function, \( q_\mtext{HO} \). 
Since the energy levels of a harmonic oscillator are given by \( \br{v + \tfrac{1}{2}} h \nu \), one obtains~\autocite{levine95}
\begin{equation}
	q_\mtext{HO} = \sum_{v = 0}^\infty \exp \br{\frac{-\br{v + \tfrac{1}{2}} h \nu}{k T}} = \frac{e^{-h \nu/\br{2kT}}}{{1 - e^{-h \nu / \br{k T}}}} \, .
	\label{eq:qho_part_func}
\end{equation}
Combining \cref{eq:vib_part_func1,eq:qho_part_func}, we find 
\begin{equation}
	\widetilde{C}_v\br{\mtext{vib}} = R \frac{u^2 e^{-u/2}}{\br{1 - e^{-u}}^2} \, ,
	\label{eq:vib_part_func2}
\end{equation}
where \( u = h \nu / \br{kT} = h c \widetilde{\nu} / \br{k T} = 143.88 \widetilde{\nu} / T \). 

% subsection vibrational_partition_function (end)

% section theory (end)


\pagebreak

\section{Safety Precautions} %(fold)
\label{sec:safety}

\begin{itemize}
	\item \ch{HCl} is a strong acid should be handled cautiously. 
	\item All chemical work should be done in the fume hood and gloves should be worn at all times. 
	\item Always wear safety glasses in the laboratory.
	\item The cell used to obtain the infrared spectra is expensive and difficult to replace. 
	Handle it with extreme care. 
	The windows are made from sapphire.
	Though they are sturdy and moisture-resistant, you should avoid handling them. 
	Do not touch the lenses with your fingers and avoid impacts with hard objects. 
	If the cell filled with HCl breaks outside the fume hood, notify the instructor immediately and clear the area. 
\end{itemize}
% section safety (end)


\section{Procedure} % (fold)
\label{sec:procedure}

This experiment must be performed with an IR spectrometer capable of medium to high resolution (\( \qty{>2}{\wn} \) resolution). 
A high-quality grating spectrometer and most FTIR instruments are generally capable of this, and FTIR instruments don't require calibration with external gas standards. 

The infrared gas cell is constructed from a short (\qtyrange{5}{10}{\cm}) length of large-diameter (\qtyrange{2}{4}{\cm}) Pyrex tubing with a vacuum stopcock attached. 
Infrared-transparent windows are clamped agains O-rings at the ends of the cell. 
In the present experiment, inert sapphire windows with transmission down to \qty{1600}{\wn} are particularly convenient and relatively inexpensive. 
When the spectrum of interest extends much below \qty{\sim 2000}{\wn}, other materials are necessary (\emph{e.g.,} \ch{KBr}, \ch{NaCl}, or \ch{ZnSe}). 
Most of the salt windows become ``foggy'' on prolonged exposure to a moist atmosphere and should be protected\sidenote{e.g., stored in a desiccator} when not in use. 
The majority of the semiconductor windows are relatively soft and are therefore easily scratched. 

\subsection{Required Equipment} % (fold)
\label{sub:required_equipment}

\begin{itemize}
	\item \qty{10}{\mL} two-neck flask (conical or round-bottom) 
	\item \qtyrange{200}{400}{\mL} beaker
	\item small vacuum filtration flask or side-arm test tube
	\item Tygon tubing
	\item disposable pipettes
	\item ring stands and clamps
	\item (4) one-hole rubber stoppers
	\item \qty{10}{\cm} glass spectroscopy cell
	\item dry nitrogen gas cylinder
\end{itemize}

% subsection required_equipment (end)

\subsection{Filling the Cell and Recording the Spectra} % (fold)
\label{sub:filling_the_cell}

% A 50-mL two-neck round-bottom flask and a 10-cm air condenser should be washed thoroughly with soap and water followed by several rinses with distilled water.
% Then, the glassware  should be dried in an oven for approximately \qty{10}{\minute} at a temperature near \qty{110}{\celsius}.

% By this time, the glassware in the oven should be dry. Remove the dry glassware and place on a piece of paper towel to cool for another \qty{10}{\minute} (or until cool to the touch).

% Mix \qty{\sim5}{\mL} of \ch{D2O} with \qty{\sim5}{\mL} concentrated \ch{HCl},\sidenote{this is easiest to do in a small beaker}, then pour the mixture into the round-bottom flask via the second neck.
% Finally, insert another stopper with a hole into the second neck of the flask.
%
% Suspend the round-bottom flask in a water bath held around \qty{80}{\celsius}. 
% Gaseous \ch{HCl} will slowly evolve from the solution and bubble through the indicator solution. 
% This procedure should take \qty{\sim20}{\min}. 

Before beginning the synthesis, power on the FTIR. 
Prior to starting, purge the 10-cm gas cell (Thorlabs) with nitrogen gas to clean out the cell.
This can be accomplished by attaching the tubing from the nitrogen tank regulator to one of the glass ports on the gas cell. 
Open both stopcocks on the gas cell and slowly open the needle valve on the nitrogen tank\sidenote{Ask the instructor if you are unfamiliar with the use of a regulator.} until a steady stream of gas is exiting the open port of the gas cell. 
After allowing the gas to flow for \qty{2}{\minute}, close the stopcock on the gas cell port releasing the gas and immediately close the stopcock on the port receiving the flowing gas, effectively sealing the gas cell. 
Immediately close the knob above the regulator nozzle, then close the needle valve controlling the tank's gas flow. 
Remove the tubing from the gas cell port and take the gas cell to the FTIR to obtain a background spectrum. 
Return the gas cell to the fume hood for the gas synthesis. 

Prepare an indicator solution by dissolving \numrange{2}{4} drops of bromcresol green in DI water in a medium beaker.
There should be enough liquid in the beaker to submerge the end of the tubing.  
Attach Tygon tubing to the two stems of the gas cell, lubricating with glycerin to help the tubing slide on the glass. 
Take the end of one tube and connect it to your water trap. The tubing exiting your water trap should be submerged in your indicator solution.\sidenote{This will fill two roles: as a bubbler, it will prevent atmospheric gases from diffusing back into the cell; and as an indicator, it will show when sufficient \ch{HCl} has passed through the cell. \ch{HCl} is highly soluble in water, so the gaseous \ch{HCl} will quickly acidify the water.} 

To set up the rest of the apparatus, clamp a \qty{50}{\mL} two-neck flask to a ring stand. 
Place \qty{\sim5}{\mL} of \ch{D2O} with \qty{\sim5}{\mL} concentrated \ch{HCl} in the round-bottom flask.
Prepare a gas dryer (\qty{10}{\cm}) by placing a small plug of glass wool at the bottom of an air condenser and filling the bulb and tube of the condenser with \ch{CaCl2} (approximately \qty{12}{mesh}). 
Insert the stem of the condenser into a stopper, again using glycerin for lubrication, and insert the stopper into the neck of the round-bottom flask. 
Insert the end of the second tube (attached to the gas cell) into another stopper with a hole, then insert that stopper into the top of the air condenser. 
With a disposable plastic pipette, obtain \qty{\sim 2}{\mL} of concentrated \ch{H2SO4}.
Insert the pipette (\textbf{carefully} so as not to prematurely inject it) into the hole of the second stopper in the round-bottom flask.
Once you have verified that all of the holes are plugged, squeeze \numrange{1}{2} drops of sulfuric acid into the solution.
You should see vigorous bubbling in the flask.
At this point, open the stoppers on the gas cell.
Continue to \emph{slowly} add sulfuric acid drop-wise to the flask, watching the bubbles evolving in the indicator. 
You will probably need to add the entire pipette. 
When the indicator changes color from blue to green/yellow, close the stopcocks on the gas cell and remove the tubing. 
While the cell is still in the hood, \emph{briefly} vent the cell to decrease the pressure inside the cell (we want to let out the excess \ch{HCl} vapor, but not let atmospheric gases diffuse in). 
Lower gas pressure in the cell will result in narrower spectral lines due to decreased collision broadening. 
Carefully remove the tubes from the gas cell and proceed to the spectrometer for analysis. 

An arrangement for filling the cell is shown in \cref{fig:cell_filling}.

\begin{marginfigure}%[ht!]
  \centering
	\includegraphics[width=.9\textwidth]{images/new_exp_portrait.png}
  \caption{Experimental apparatus. (1) Two-neck flask containing a mixture of \ch{D2O} and concentrated (\qty{12}{\molar}) hydrochloric acid. (2) Condenser containing a drying agent. (3) Gas spectrometry cell. (4) Trap preventing indicator from sucking back into gas cell. (5) Beaker containing indicator solution. Tygon tubes are indicated in blue, red, and green. Blue tube leads from the condenser to the gas cell. Red tube runs from the cell to the trap. Green tube runs from the trap to the indicator.}
  \label{fig:cell_filling}
\end{marginfigure}

Make sure the experimental settings in the FTIR software are appropriate. 
The most important change to make is changing the resolution to the smallest possible value. 
This will allow you to distinguish the individual rotational transitions in the spectrum. 


% subsection filling_the_cell (end)

\subsection{Theoretical Calculations} % (fold)
\label{sub:theoretical_calculations}

The following work will be done during the second week of this lab. 
You will need to calculate the physical properties of \ch{HCl} and \ch{DCl} using Psi4~\autocite{psi42012}, including the potential energy, equilibrium bond length, and the predicted vibrational and rotational energies for the molecule.

Using Psi4, calculate the potential well for the \ch{HCl} molecule. 
We performed a nearly identical calculation when doing the bond scan exercise on \ch{HF}. 
Because of the methods used for calculation, the mass of the nuclei doesn't influence geometry, so results will be the same for each of the four isotopologues. 
You will need to modify the scan parameters in \cref{lst:bond_calc} for the Python input printed below (\Verb{start}, \Verb{stop}, and \Verb{step-size}).
The selected basis set and levels of theory make this a quick calculation, so you can set the scan parameters quite large. 
I recommend scanning from \qtyrange{0.6}{3.5}{\angstrom} in steps less than \qty{0.05}{\angstrom}. 
With this resolution, you should obtain a plot that gives good fidelity around the minimum while allowing you to see the behavior at large bond lengths. 
\begin{figure}
\begin{pyinput}
method_list = ['hf', 'remp2', 'pbe0']
basis = '6-311+g(d,p)'

hcl = psi4.geometry("""
H
Cl  1  r
""")

r_vals = np.arange(start, stop, step-size)
hcl_bond = dict()

for method in method_list:
    energies = []
    for dist in r_vals:
        hcl.r = dist
        hcl.update_geometry()
        e = psi4.energy(method+'/'+basis)
        energies.append(e)
    hcl_bond[method] = np.array(energies)
\end{pyinput}
	\label{lst:bond_calc}
	\caption{Python/Psi4 code to calculate potential energy curves for the \ch{H-Cl} bond.}
\end{figure}

Once you have calculated the potential energy surface for the molecule, you'll need to perform a frequency analysis of the \ch{HCl} molecule. 
You can use the \Verb{psi4.diatomic.anharmonicity()} function to calculate all the required constants for the molecule. 
The function takes a list of bond lengths and a list of energies as inputs, and outputs a dictionary of constants. 
The function takes an optional argument, \Verb{mol=}, which allows you to define a specific molecule used for computation. 
In the example shown below, if you leave \Verb{mol=} blank (in other words, accepting the default value), Psi4 will use the most recently defined molecule in your code.
This can sometimes lead to unexpected results… better to explicitly define your molecule.   
For your lists of bond lengths and energies, use the \Verb{'pbe0'} data from the bond analysis performed previously so you don't need to recalculate the potential energy curve for each isotopologue (the values wouldn't change anyway, since Psi4 doesn't account for isotopic differences in energy calculations). 
You need to get anharmonicity constants for \ch{H^{35}Cl}, \ch{H^{37}Cl}, \ch{D^{35}Cl}, and \ch{D^{37}Cl}, so setting up a loop and saving the results in a dictionary will reduce the amount of repetition in your code.
\Cref{lst:anharm_calc} shows an example of the calculation for \ch{H^{35}Cl}; use this as a starting point for the calculations in your loop.

Notice how isotopes are defined in Psi4. 
You'll need to look up appropriate isotopic masses for your four atoms and use that data to define the four required molecules. 
Notice that the \(z\)-matrix for the isotope has the separation between the two atoms set to \num{1.0}. There is no need to create this molecule in its equilibrium geometry, as the anharmonicity calculation will calculated it and return the equilibrium value as one of the dictionary entries. 
Note that the units are not given with the constants. In general, Psi4 provides energies in hartrees, bond lengths in Ångstroms, vibrational constants in wavenumbers, and rotational constants in hertz. 
\begin{figure}
\begin{pyinput}
h35cl = psi4.geometry("""
Cl@34.968852682
H@1.0078 1 1.0
""")

# The next step might be a good place to wrap your loop to calculate your four molecules 
hcl_consts['h35cl'] = psi4.diatomic.anharmonicity(r_vals, hcl_bond['pbe0'], mol=h35cl)
\end{pyinput}
	\label{lst:anharm_calc}
	\caption{Python code for calculating anharmonicity parameters for a diatomic molecule.}
\end{figure}

% subsection theoretical_calculations (end)

% section procedure (end)

\section{Data Analysis} % (fold)
\label{sec:data_analysis}

\begin{enumerate}
	\item Import your \ch{HCl} and \ch{DCl} spectra and index the lines with the appropriate \( m \) values, as seen in \cref{fig:typ_spectrum}. 
	If the \ch{^{35}Cl}/\ch{^{37}Cl} splitting is seen, index both sets, but be sure to separate them appropriately.
	Make a table of these \( m \) values and the corresponding frequencies, \( \widetilde{\nu}\br{m} \). 
	Express the frequencies in \unit{\wn} to the tenth of a \unit{\wn}. 
	Then, list the differences between adjacent lines, \( \increment{\widetilde{\nu}}\br{m} \), which will be roughly \( 2 B_e \), but will vary with \( m \). 
	\item Plot \( \increment{\widetilde{\nu}}\br{m} \) against \( m \).
	If any values seem out of line, check the calculation for that cell. 
	\item Carry out a polynomial least-squares fit to the data using \cref{eq:both_branches1} to determine \( \widetilde{\nu}_0 \), \( B_e \), \( \alpha_e \), and their standard errors. 
	Repeat this fitting procedure using \cref{eq:both_branches2}, noting that high \( m \) transitions will be the most important ones in determining \( D_J \), due to its \( m^3 \) dependence. 
	\item Using your values of \( \widetilde{\nu}_0 \) for \ch{HCl} and \ch{DCl}, determine \( \widetilde{\nu}_e \) and \( \widetilde{\nu}_e \chi_e \) for the various molecules. From \( \widetilde{\nu}_e \), calculate \( k \) for \ch{HCl}. 
	\item Calculate \( I_e \), the moment of inertia, and \( r_e \), the internuclear distance, for both \ch{HCl} and \ch{DCl}. 
	Tabulate your experimental and theoretical results, along with your estimates  of the experimental uncertainty. 
	Compare your results with literature values found in \textcite{kerr82}. 
	\item Using your value of \( \widetilde{\nu}_0 \) for \ch{HCl}, calculate \( \widetilde{C}_v\br{\text{vib}} \) at \qty{298}{\K} and at \qty{1000}{\K} from \cref{eq:vib_part_func2}. 
	Compare the spectroscopic value \( \widetilde{C}_v = 5/2 R + \widetilde{C}_v(\text{vib}) \) with the experimental \( \widetilde{C}_v \) value obtained from directly measured values~\autocite{lewis61,spencer48} of \( \widetilde{C}_p \) and the expression \( \widetilde{C}_v = \widetilde{C}_p - R \): \( \widetilde{C}_v = \qty{20.80}{\J \per \mole\K} \) at \qty{298}{K} and \qty{23.20}{\J \per \mol\K} at \qty{1000}{\K}. 
\end{enumerate}
% section data_analysis (end)

\section{Questions and Further Thoughts} % (fold)
\label{sec:questions_and_further_thoughts}

\begin{enumerate}
	\item Compute the ratio of \( B_e^*/B_e \) and compare with the rigid-rotor prediction of \cref{eq:eff_mass_isotope_relation}. 
	Based on your experimental data, how constant is \( r_e \) for \ch{HCl} and \ch{DCl}?
	\item Compute \( B_v = B_e - \alpha_e\br{v + \tfrac{1}{2}} \) for the \( v = 0 \), \num{1}, and \num{2} levels of \ch{HCl} and \ch{DCl} and, from these, obtain average \( r_v \) values for these levels. 
	Comment in your report on the changes in these distances. 
	\item Compare your \( \widetilde{\nu}_0^*/\widetilde{\nu}_0 \) ratio with the ratio \( \sqrt{\mu/\mu^*} \) expected for a harmonic oscillator. How anharmonic is the \ch{HCl} molecule (\emph{i.e.,} how large is \( \nu_e\chi_e \))?
	\item Use your values of \( \widetilde{\nu}_e \) and \( \widetilde{\nu}_e \chi_e \) and \cref{eq:vib_rot_energies} to predict the frequencies of the first overtone transitions of \ch{HCl} and \ch{DCl} (ignore the rotational terms). 
	Did you observe any evidence of these overtones in your spectra? 
	\item Do your spectra show any evidence of a \ch{^{35}Cl}/\ch{^{37}Cl} isotope effect? 
	Use \cref{eq:eff_mass_isotope_relation} to calculate the splitting expected for this effect for several \emph{P} and \emph{R} branch transitions in \ch{HCl} and \ch{DCl}. 
	\item Compare your experimental results with the values deduced from your theoretical calculations. 
	For each calculation, report the values for total energy \( (D_e) \), vibration frequency \( (\widetilde{\nu}_0) \), bond length \( (r_e) \), rotation constant \( B_e \), and centrifugal distortion constant \( D_J \). 
	Additionally, you should plot one of the potential energy curves output by Psi4 against the theoretical Morse potential for the molecule. You will need to think carefully about how to match up units for the two curves. 
\end{enumerate}

% section questions_and_further_thoughts (end)

\section{Lab Report Guidelines} % (fold)
\label{sec:lab_report_guidelines}

As previously stated, your lab report should consist of the following parts:
\begin{description}
	\item[Title, Author and Date]
	\item[Introduction and Objective] A paragraph describing what we hope to find in this experiment, and how.
	\item[Experimental Procedure] This should be a very brief general outline of the procedure, written out as a paragraph or two. Give the make and model for any major instruments you used, as well as any important settings. For IR spectroscopy, this especially means the spectral resolution and number of scans.
	\item[Results and Discussion] This should include answers to the questions in the section ``Questions and Further Thoughts''. This should not be a separate section, but should instead be included organically in the discussion as a way of filling it out.
	\item[Conclusion]
	\item[References]
	\item[Appendix] Include examples of any calculations that you did by hand. 
	Provide digital copies of any additional inputs used to generate your information. 
\end{description}
% section lab_report_guidelines (end)
