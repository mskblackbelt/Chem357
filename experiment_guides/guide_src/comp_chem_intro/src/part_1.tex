%!TEX root = ./Intro_to_CC.tex

\section{Problem I: The hydrogen atom} % (fold)
\label{sec:problemI}

In this exercise, we will look at different basis sets using the hydrogen atom. 
The hydrogen atom is the only non-trivial system for which the exact analytic solution is known. 
By the end of the first exercise, we will see how various computational methods compare to each other and to the exact solution.  
From a technical perspective, we will learn how to compose input files, run basic Psi4 calculations, search for energy in the Gaussian output, and perform basis set convergence tests.

\subsection*{Getting started}

\textbf{Tasks}
\begin{enumerate}
  \item We'll begin by setting some options for the calculation. 
  Because we're dealing with a hydrogen atom, we have to use an ``unrestricted Hartree-Fock''  method (the unpaired electron must be properly accounted for), hence \Verb{"uhf"}. 
  We'll set the basis set to a very minimal set called \href{https://en.wikipedia.org/wiki/STO-nG_basis_sets}{``STO-3G''} (each available atomic orbital is represented by three contracted gaussian functions). 
  Finally, we define the molecule geometry and assign it a name. 
  For this \emph{very} simple system, setting the basis, method, and geometry is as simple as: 
\begin{pyinput}
h_atom = psi4.geometry("H")
basis = 'sto-3g'
psi4.set_options({
                  'reference': 'uhf'
                })
\end{pyinput}
  With a single atom, there's no need to identify connectivity or location of the atom in space.
  We use the \Verb{set_options()} method and create a dictionary of options we'd like to set.  
  \item Now that we've defined a molecule and basis, we'll tell Psi4 what sort of calculation we'd like to run. 
  To begin, we'll perform a simple energy calculation and ask Psi4 to return the energy and the wavefunction for the system. 
\begin{pyinput}
energy, wfn = psi4.energy('hf'+basis, return_wfn=True)
nbf = wfn.basisset().nbf()
print(f'{energy=}\t{nbf=}')
\end{pyinput}
    
    The option \pyinline{return_wfn=True} tells Psi4 to return information about the wavefunction for the system. 
    From that, we pull out the number of basis functions (\Verb{nbf}) being used in our calculation and print out the energy and number of basis functions. 
    
    This is the computed electronic energy of the H atom using Hartree-Fock theory in the STO-3G basis set. 
    Compare it with the exact result for the hydrogen atom (\(\qty{0.5}{\hartree} \approx  \qty{13.6057}{\eV} \approx \qty{313.7545}{\kcal\per\mol}\)).
    
  \item Redo the calculation with your original basis set (STO-3G) and the following basis sets: cc-pVDZ, cc-pVTZ, cc-pVQZ by creating a list of the basis set names (as strings), then setting up a loop to perform the calculation and save the energy and number of basis sets to a pair of lists.\sidenote{
    You can find a rough overview of \href{https://en.wikipedia.org/wiki/Basis_set_\%28chemistry\%29}{basis sets at Wikipedia} and a \href{https://psicode.org/psi4manual/master/basissets.html}{Psi4-specific overview} is also available.
    } 
  Then, plot the total energy as function of the basis set size. 
  At which basis set does the energy converge to the exact solution? 
\end{enumerate}

\subsection*{Method performance}

In this step, we'll repeat the previous calculations with different methods by wrapping your previous work in another loop. This loop should run over the following list of methods: HF, SVWN, PBE, and PBE0. 
As before, we create a list containing these methods (as strings), and run a loop inside a loop. 
In order to save our information in a nice structure of recalling later (as we want to keep the methods separate for plotting), we're going to store our results in a dictionary object. 
This means we need to initialize an empty dictionary in which to save our data. This has been set up for you in the notebook template. 
Spend some time looking over the nested loops to make sure you see what's being done in each loop. 

In the next cell, we set up a small loop to plot the data showing the convergence of different methods to the exact value of \qty{0.5}{Ha}.
Do all of them converge correctly to the same solution? 
The details of the listed theoretical methods to evaluate electron--electron interactions and why they converge to different values for the apparently trivial one-electron system are beyond this tutorial and will be covered in lecture later this semester. 

% section problemI (end)