%!TEX root = ./Intro_to_CC.tex

\section{Problem II: Hydrofluoric acid (HF) -- bond length and dipole moment} \label{sec:problemII}

\subsection*{The hydrogen fluoride molecule (HF)}

In the exercise, we will calculate the binding curve, atomization energy (\( \increment H_\mtext{at} \)), and dipole moment for the hydrogen fluoride (HF) molecule with two methods. 
From a technical perspective, this exercise teaches how to use Python loops to perform repeated computations with Psi4. 

\begin{enumerate} 
  \item The first task of this exercise will be to find learn a bit about the output data from a calculation
  Start by defining a new hydrogen fluoride molecule. 
  This is again done for you in the first cell. 
  The input block also has space for you to specify the method and the basis set used for computation. 
  In this exercise, use \Verb{HF} (Hartree-Fock) and the \Verb{6-31G(d,p)} basis set. 
  
  Notice the geometry is defined in terms of the atomic connectivities (the first atom is the "center", labeled 1, other atoms are labeled sequentially with connectivity assigned by number. The line ``\Verb{H 1 r}'' means that a hydrogen atom is placed distance \(r\) away from atom \num{1}.) The geometry could alternatively be defined explicitly in Cartesian coordinates like so: 
  \begin{geominput}[title=Cartesian geometry definition]
  F 0.0 0.0 0.0
  H 0.0 0.0 r
  \end{geominput}
  We use the variable \Verb{r} as a placeholder so we can define this distance later in our code.\sidenote{This variable can be anything we like, as long as it's a valid variable name in Python (\emph{i.e.,} \Verb{interatomic_distance} would be perfectly acceptable).}
  Before we start making multiple calculations, let's make a single one and figure out which computational variables we'd like to collect. 
  We can set the distance variable to a value with \pyinline{hf_bond.r = 1.0}, then perform our energy calculation (again, making sure to return and store the wavefunction data, just as before.)
  
  Once we have this, you can print out all of the data saved in the wavefunction variable with \pyinline{wfn.variables()}.\sidenote{Assuming you named the wavefunction variable \Verb{wfn}.}
  This will print out all of the properties returned by the current calculation. 
  The data are returned as dictionary items, with the dictionary key in capital letters and the value given after the colon. 
  An individual property can be recalled using \pyinline{wfn.variable('name')}, where \Verb{'name'} is the (case-insensitive) name of the key. 
  Go ahead and print out the value for the property \Verb{'current dipole'}. 
  Notice that the value is returned as a vector: a three-element array of \(x\), \(y\), and \(z\) components. 
  Recall that, by convention, the principle bond axis of a molecule is defined as \(z\). In a simple diatomic, this is the \emph{only} bond, and so the entirety of the dipole should lie along the \(z\)-axis. 
  You should verify this by making sure the \(x\) and \(y\) components of the dipole array are zero. 
  If they are, you can take the following easy shortcut: rather than needing to calculate the total length of the vector (``taking the norm''),\sidenote{Easily done using the NumPy function \Verb{np.linalg.norm}.} you can just grab the \(z\) component as the full magnitude of the dipole vector. 
  In the rest of this exercise, we'll be using the energy and dipole values from this calculation. 

  \item The next step of this exercise will be to find the equilibrium bond distance of hydrogen fluoride (HF) from a series of calculations. 
  Create a loop to calculate the energy of the molecule at a range of  \(r\) values between \qtyrange{0.7}{1.3}{\bohr}.\sidenote{
    The easiest way to generate this list of values is to use the \pyinline{np.arange(start, stop, step)} function from NumPy. 
    This function takes (up to) three arguments: start, stop, and step. 
    If only a single argument is given, it acts just like the \pyinline{range()} function and generates integer values from 0 up to (but not including) the (stop) value given. 
    If two values are given, it goes from the start to stop in unit steps, and providing all three values goes from start to stop in the requested step size.}
  A good step size is \num{0.02} (this is the atomic unit of length, \(\unit{\bohr} \simeq \qty{52.9}{\pm}\)).
  Save the energy of each calculation as an element in a list, then plot the list of \(r\) values against the list of energies. 
  The plot should have a distinct minimum. 
  To extract the minimum value from the energy list, use the \pyinline{np.min()} and \pyinline{np.argmin()} functions from NumPy. 
  Both functions take a list as the input. 
  The first returns the minimum value of the list, while the second returns the index of the minimum value. 
  Use the index of the minimum to find the corresponding value in the list of \(r\) values. 
  Which bond length corresponds to the lowest energy? How does the bond length compare to the experimental bond length of \qty{0.917}{\angstrom}?

  \item To compare with experimental values, we compute the atomization energy (\( \increment H_\mtext{at} \)).
  In order to calculate \( \increment H_\mtext{at} \), we will also need the total energy of the isolated H and F atoms. Compute the total energies for the single atoms using the methods \Verb{HF} and \Verb{6-31G(d,p)} basis set. 
  You can look back to our work on the lone hydrogen atom in Problem~I to guide your work. 

  %\begin{tip}
  %\textbf{Note:}\newline
  %Atoms are highly symmetric systems, often with multiple degenerate solutions. In the case of fluorine, for example, the unpaired p-electron might sit in the $p_x$, $p_y$, or $p_z$ orbital. All three solutions are equivalent. If the calculation is started unbiased, it might converge to a superposition of these three cases, which is a saddle point on the potential energy surface and results in partial electron occupations. Although in DFT non-integer occupations are in principle allowed, one should be very suspicious when obtaining such a solution for non-metallic systems. Typically, solutions exist that are lower in energy. They can be found by breaking the inherent symmetry of the problem, for example by applying a small external field at the beginning of the SCF cycle.
  %\end{tip}
  
  %To break the inherent symmetry of an atom and ensure integer occupation, set the keyword \texttt{switch\_external\_pert 10 safe}. This means that for 10 iterations, a small external field in the z-direction is applied and then switched off. Usually, this is sufficient to perturb the SCF out of the symmetric solution and towards the correct electronic structure.
  
  Next, calculate the atomization energy (\( \increment H_\mtext{at} \)) of HF by subtracting the free-atom energies from the predicted total energy of HF (\emph{i.e.,} the minimum total energy found when varying bond distances).
  
  \begin{equation}
   \increment H_\mtext{at}= E^\mtext{HF}_\mtext{tot} - E^\mtext{H}_\mtext{atom} - E^\mtext{F}_\mtext{atom}
  \end{equation}
  
  How does this compare to the experimental value of \(\increment H_\mtext{at}=\qty{135.2}{\kcal\per\mol}\) (\qty{5.86}{\eV})? 
  
  \item Now, let us look at the dipole moment. 
  How does the dipole at the equilibrium distance compare with the experimental value of \qty{1.82}{\debye}?\sidenote[][-5\baselineskip]{
    Psi4 returns values in terms of atomic units. 
    The debye (\unit{\debye}) is the most commonly used dipole unit, and is one of the few remaining relics in use from the cgs system. 
    The SI unit of dipole is the coulomb-meter (\unit{\coulomb\meter}), but atomic dipoles are on the order of \qty{1e-30}{\coulomb\meter}, or a quectocoulomb-meter \unit{\quecto\coulomb\meter} (no, I'm not making that prefix up, it's one of the four newest SI prefixes), making it an unwieldy quantity. 
    The true \emph{atomic} unit of the dipole is \unit{\dipole}, or the product of the charge on one electron multiplied by the Bohr radius.
  } 
  Plot the dipole moment vs. the bond distance. You will find a (mostly) linear correspondence. 
  Do you expect this trend to continue at large distances?  Why or why not?
  
  \item Next, repeat the bond length determination using \Verb{PBE0} method and same basis set. 
  In addition, you need to compute energies for hydrogen (H) and fluorine (F) again using new method. 
  How does the optimal bond length, atomization energy and dipole moment change?
  In the lab report, prepare a plot of both dissociation curves, a plot of both dipole moments and a comparison of the two computed atomization energies (and the experimental value). 
  
\end{enumerate}

%\subsection*{Optional: Charge partition schemes.}
%Chemical reactivity and many physical properties are often explained in terms of atomic charges. However, atomic charges are not physical observables, since no unique operator exists to determine this quantity. They rather depend on the chosen charge partition scheme. The charge partition schemes that is probably most commonly used is Mulliken \cite{mulliken1955epa}. In FHI-aims, you can request it by specifying
%\texttt{output mulliken}  
%in \controlin{}.
%
%For the equilibrium structure of HF, compute the atomic charges with this scheme using \texttt{pbe0}. Use the charges to calculate the dipole moment $\mu$ in the point dipole approximation. In this approximation, for a two atom system, the dipole moment is defined as:
%\begin{equation}
%\mu = q \cdot \left|  \vec{r_H} - \vec{r_F} \right|
%\end{equation}
%where $q$ is the atomic partial charge and $\vec{r_H}$ and $\vec{r_F}$ are the atomic positions of the atoms. The absolute value of the difference $ \left|  \vec{r_H} - \vec{r_F} \right|$ is the distance between hydrogen and fluorine.  Compare the dipole moment to the one computed by FHI-aims. How do they compare?
%
%\section{Problem III: Molecular oxygen - a critical look \textcolor{red}{!}} \label{sec:problemIII}
%
%An important part of every calculation is to always look critically at the output and ensure that the result is reasonable. For some systems, defaults may not be adequate, or assumptions which commonly work well may prove to be wrong. A prime example is the the treatment of spin in systems with degenerate orbitals, such as O$_2$.
%
%\task{[\textbf{Educational Objectives:}] 
%  \begin{itemize}
%\item  Do not trust default settings blindly
%\item See the effect of an incorrect spin-treatment
%  \end{itemize}%
%}
%%
%\noindent
%
%\textbf{Tasks}
%\begin{enumerate}
%\item Set up a calculation for O$_2$ similar to the previous exercise, but leave out the \texttt{ spin} keyword in \controlin{}.  Look at the output file to find out which spin treatment is used by default.
%\item Look at the output file and find the occupation numbers of the Kohn-Sham orbitals.  Does this make sense here?
%\item Using \texttt{pbe} and \texttt{pbe0}, calculate the binding curve in the interval  $[0.8 , 1.6 ]$~\AA~ with a stepwidth of 0.1~\AA~ as before. Also, calculate the atomization energy ($\increment H_{at}$). 
%\item Repeat the calculation with \texttt{spin collinear} and \texttt{default\_initial\_moment 2.0} - is Hund's rule now fulfilled? Compare the results: Do both spin settings yield the same equilibrium bond length? Calculate the difference in the total energy at the equilibrium bond length. Which one is lower? How does it compare to the experimental value of 1.21 \AA? How does the atomization energy ($\increment H_{at}$) compare to the experimental value of 5.18 eV?
%\end{enumerate}
%
%Note: The oxygen atom is a notoriously difficult case to converge. If you have problems with it, try using the linear mixer for the SCF by including  the keyword \texttt{mixer linear} in \controlin{}. The linear mixer is guaranteed to converge, but usually requires much more iterations than the default (Pulay) mixing scheme.
