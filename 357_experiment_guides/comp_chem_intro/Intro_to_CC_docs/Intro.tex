%!TEX root = ./Intro_to_CC.tex

\section{A first look at the Shell and Gaussian16}

The work in this lab will take place partly in a Jupyter notebook and partly in the Terminal program.
You can open Terminal from the JupyterLab launcher or from the menu under \menu{File>New}. 
Prior to beginning work, you should familiarize yourself with the Bash shell by running through the exercises from DigitalOcean available at \url{https://www.digitalocean.com/community/tutorial_series/getting-started-with-linux} (skip the part on Linux Permissions, as it isn't relevant).

The most common Bash commands are available for your reference in the appendix. 
Before starting, make sure you understand how to navigate the file system with \Verb{pwd}, \Verb{ls}, and \Verb{cd}, and that you can copy, move, and remove a sample file using the \Verb{cp}, \Verb{mv}, and \Verb{rm} commands. 
Also make sure you understand the difference between \emph{relative} and \emph{absolute} file paths.\sidenote{Hint: read the appendix!}

It is a good practice to perform each Gaussian calculation a separate directory (this is true for other programs as well).
The calculations are initialized by calling the \Verb{g16} command on an input file:\sidenote{
  This input represents commands typed into the shell.
  The leading characters in blue (everything up to the dollar sign) represent the user prompt and should not be typed.
}
\begin{bashinput}
  g16 input &
\end{bashinput}
	
By convention, the input file is named \Verb{input.com}, though any name and extension will work.
The basic input file is shown below.\sidenote{
  Content wrapped in angle brackets (\Verb{<*>}) should be \textbf{replaced} with the desired value and the angle brackets should be \textbf{removed} (they are not a recognized Gaussian input), e.g., \Verb{<basis>} \( \Rightarrow \) \Verb{STO-3G}.
}
This file starts a calculation on two processor cores using \qty{400}{\mega\byte} of memory.
The output will be redirected to the \Verb{input.log} output file. 
This file contains the basic information and results of the calculation such as the total energy, atomic forces, and so forth. 
Additional output files might be generated according to the specified settings. 
Individual components of the input file are described below. 
\begin{gaussinput}[title=Structure of a simple input file, 
label=lst:gauss_samp]
%nproc=2
%mem=400MB
#<method> <basis-set>
#sp scf=tight
    <empty line>
<title information>
    <empty line>
<charge> <multiplicity>
<atom1>   <x1>	<y1>	<z1>
<atom2>   <x2>	<y2>	<z2>
 ... 
<atomN>   <xN>	<yN>	<zN>
    <empty line>
\end{gaussinput}

\begin{description}[font=\ttfamily\upshape]
  \item[\%nproc=2] This keyword specifies the number of processing cores that will be employed for calculations.
  \item[\%mem=400MB] This keyword manages the maximum memory usage during the calculations.
    The most efficient memory specification is beyond this tutorial.
  \item[method] Substitute \Verb{<method>} with the method of choice for electron--electron interactions.
    In this tutorial, we will use several methods, namely Hartree-Fock (HF), Møller-Plesset second order perturbation theory  (MP2) and a few density-functionals.
    The details of each method will be covered in detail during the lecture.
  \item[basis-set] Substitute the \Verb{<basis-set>} word with the desired basis set.
    This specifies a set of basis functions (for instance atomic orbitals, gaussian-type orbitals, plane waves) that will be used to express an electronic configuration.
    The recommended basis sets are specified in each exercise.
  \item[sp] The \Verb{sp} command orders Gaussian to perform single-point calculations, i.e., energy evaluation of a specified structure using \Verb{method} and \Verb{basis-set}
  \item[scf=tight] The Schrödinger equation is solved in self-consistent manner.
    The \Verb{scf=tight} option specifies tight convergence criteria for the self-consistent cycle.
  \item[your-comment] This line, surrounded by two empty lines, holds your comment, usually a description of the molecule and/or calculation to be performed. 
  \item[charge] The \Verb{charge} keyword should be substituted with the total (integer-valued) charge of your system. 
  \item[multiplicity] The \Verb{multiplicity} keyword should be substituted with the multiplicity of your system (\( 2\mtext{S}+1 \), where \( \mtext{S} \) is total spin).
    This should always be an integer.
  \item[atomX <X> <Y> <Z>] This block specifies the geometry of the system.
    You can use either the atomic symbol (e.g., \Verb{C}) or the atomic number (e.g., \Verb{6}) to specify the atom type, followed by it's cartesian coordinates in units of Angstroms (\unit{\angstrom}).\sidenote{
      Use a decimal-valued coordinate, even if it is a whole number (\Verb{0.0}, not \Verb{0}).
    }
    This block must be followed by an empty line.
\end{description}

Remember, there are no bonds (sticks) in quantum chemistry.
The bonding is the result of the respective positions of atoms in space.
The `stick' visible in visualization programs is simply a rendering for more intuitive display.
A sample input for a square-planar molecule can be found at the end of the section (with all values filled in). 


\subsection{Additional tools and programs}

\begin{description}[font=\bfseries]
  \item[Bash shell] A short list of the basic bash (command line) commands is given in the appendix. 
  \item[Text editor] JupyterLab comes with a built-in text editor.
    Most files can be opened and edited by double-clicking their icon in the file browser panel on the left side of the window.
    Occasionally, you will run into files with an extension that opens a different program (e.g., CSV files open the CSV viewer, which can't edit by default).
    In these cases, you'll have to click ``File > Open from Path…'' and give the full path to the file.
    New files can be created by clicking ``File > New > Text File'' or clicking on the Text File icon in the launcher.
    Another option is the command line text editor \Verb{vim}.
    When working from the command line, this program is (almost) always available.
    If you plan on continuing with computer work, it would behoove you to learn the basics of Vim.
    A number of introductions to Vim are available online. Two such examples are \url{https://www.openvim.com} and \url{https://vim-adventures.com/}.
  \item[Scripts] For some exercises, scripts are required for dedicated tasks.
    All scripts you will need for this tutorial can be found in their respective directories.
\end{description}

\begin{gaussinput}[title=A sample \Verb{input.com} file for the \ch{XeF4} molecule.]
%nproc=2
%mem=400MB
# b3lyp 6-31g
# sp scf=tight

Xenon tetrafluoride single point DFT calculation

0 1
Xe  0.0   0.0   0.0
F   1.0   0.0   0.0
F   0.0   1.0   0.0
F  -1.0   0.0   0.0
F   0.0  -1.0   0.0

\end{gaussinput}