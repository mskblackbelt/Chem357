%!TEX root = ./Intro_to_CC.tex

\section{Problem 1: The hydrogen atom} % (fold)
\label{sec:problemI}

In this exercise, we will look at different basis sets using the hydrogen atom. The hydrogen atom is the only non-trivial system for which the exact analytic solution is known. 
By the end of the first exercise, we will see how various computational methods compare to each other and to the exact solution.  
From a technical perspective, we will learn how to compose input files, run basic Gaussian calculations, search for energy in the Gaussian output, and perform basis set convergence tests.

\subsection*{Getting started - the hydrogen atom}
%\task{
%\textbf{Tasks:}
%  \begin{itemize}
%  \item Generate a simple \geometryin{} and \controlin{} file and run FHI-aims.
%  \item Test the convergence of the total energy with basis size.
%  \item Compare the total energy of the hydrogen atom computed with different methods implemented in FHI-aims. Do all methods converge to the same result?
%  \end{itemize}%
%}
%
%\noindent
%\task{
%\textbf{Educational Objectives:}
  %\begin{itemize}
  %\item Become aquainted with running FHI-aims calculations
  %\item Generate a \controlin and \geometryin file.
  %\item Learn how to do systematic basis set convergence
  %\end{itemize}
%}



\textbf{Tasks}
\begin{enumerate}
  \item First, go to the \Verb{Problem_1} directory by typing in the terminal \bashinline{cd ~/pchem_comp-chem_template/Problem_1}. 
  There, create a test directory (\bashinline{mkdir "name_of_directory"}) with a name of your choosing. 
  Inside it, generate a simple \Verb{input.com} file (\menu{File>New>Text File}) which contains only a single hydrogen atom, using the example shown in the introduction. 
  This corresponds to a single hydrogen atom in a hypothetical ideal gas phase. 
  It is located at the origin of the coordinate system, although its position does not matter here.
 
  \item For the method, use \Verb{HF} (Hartree-Fock method) and minimal \Verb{STO-3G} basis set which represents each available atomic orbital with three contracted gaussian functions.\sidenote{Gaussian commands are not case-sensitive, so \texttt{HF} is the same as \texttt{hf}, \texttt{Hf}, or \texttt{hF}.}

  \item Now, inside the directory, run G16 using the command:
    \begin{bashinput}
      g16 input.com &
    \end{bashinput}
    Once the calculation has finished, open the \Verb{input.log} file with a text editor (You may click it in the file browser or, for instance, type \bashinline{less input.log} in the terminal).\sidenote[][-3\baselineskip]{
      If your calculation results in an error, check your input file. 
      The editor in JupyterLab automatically strips off the last empty line of a file, so you need to add \textbf{two} empty lines before saving in that program. 
      If you're missing the empty line, an easy fix is to run \bashinline{echo "\n" >> input.com}, then run \bashinline{g16 input.com &} again. 
      \Verb{\n} is the representation for ``newline'', and this appends an empty line to the end of your file.
      If you still can't get your input file to run, try comparing it to the sample input file and make sure you have all of the correct elements (and have replaced the appropriate placeholders with valid text).} 
    You may need to right-click (or \keys{\Alt}+right-click in Safari) to open the contextual menu in JupyterLab. 
    In that menu, click \menu{Open with>Editor}.
    If you find (\menu{Edit>Find…} or \keys{\cmd/\ctrl + F}) the line ``Normal termination of Gaussian'' near the end, then your calculation converged. 
    We are now interested in the total energy. 
    Search for ``SCF Done:'' inside the output file. 
    You should find a following line: \\
    \Verb{SCF Done: E(UHF) = #####  A.U. after X cycles}
  
    This is the computed electronic energy of the H atom using Hartree-Fock theory in the STO-3G basis set. 
    Compare it with the exact result for the hydrogen atom (\(\SI{0.5}{\hartree} \approx  \SI{13.6057}{\eV} \approx \SI{313.7545}{\kcal\per\mol}\)).\sidenote{
      \textbf{TIP}: In later exercises, to find this value quickly and efficiently, use the command \bashinline[fontsize=\footnotesize]{grep "Done" input.log}. 
      The \Verb{grep} command searches the \Verb{input.log} file looking for the phrase ``Done''  and outputs each line containing that phrase. 
      Since the file contains only one such a phrase (it solved the electron-theory problem only once), there is only one such line. 
      Please note that the capitalization matters (you can use the \Verb{-i} flag to perform a case-insensitive search).}

  \item Redo the calculation with different basis sets (\Verb{cc-pVDZ}, \Verb{ cc-pVTZ}, \Verb{ cc-pVQZ}) by creating a new directory, copying the input file into the new directory, and changing the respective keyword in the input file. 
  Search the output file to find out how many basis functions are actually used in the calculations. 
  Then, in your Jupyter notebook, plot the total energy as function of the basis set size. 
  At which basis set does the energy converge to the exact solution? 
\end{enumerate}

\subsection*{Method performance}

Repeat the calculations with different methods using the prepared bash script \Verb{performance.sh}. 
In the script, on the line that says \Verb{for m in METHODS}, replace the word \Verb{METHODS} with the following list of density functionals (including spaces):\sidenote{Edit the script file with the built-in editor or with Vim in the terminal.} \\
  \Verb{SVWN PBEPBE PBE1PBE}\\
You can add in the \Verb{HF} method if you like, to check the results against your previous step. 
Next, execute the script by typing: 
\begin{bashinput}
  bash performance.sh
\end{bashinput}
The script will iterate over the specified methods and tested basis sets (STO-3G, cc-pV\emph{x}Z, where \( x = \)D, T, Q) and create nested directories for each method/basis set pair. 
Next, it will execute the calculations. 
Finally, it creates a \Verb{performance.dat} file which contains a list of basis sets, number of basis functions in the set, and the computed energy for different methods. 
Use this data to prepare a plot in your Jupyter notebook showing the convergence of different methods to the exact value of \SI{0.5}{Ha}.
Do all of them converge correctly to the same solution? 
The details of the listed theoretical methods to evaluate electron--electron interactions and why they converge to different values for the apparently trivial one-electron system are beyond this tutorial and will be covered in lecture later this semester. 

% section problemI (end)