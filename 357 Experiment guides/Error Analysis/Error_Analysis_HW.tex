\documentclass[nobib,nofonts,nols,nohyper]{tufte-handout}
%% Header file cloned from https://github.com/wickles/latex-base

%%%%%%%%%%%%%%%%%%
%% CONTENTS
%%%%%%%%%%%%%%%%%%
% To-do / issues
% Packages
% Commands
% Special Symbols
% Environments
% More commands: Resizable delimiters
% More commands: Derivatives
% Useful templates
% Notes
%%%%%%%%%%%%%%%%%%
%%%%%%%%%%%%%%%%%%



%%%%%%%%%%%%%%%%%%
%% TO-DO / ISSUES
%%%%%%%%%%%%%%%%%%

%% packages
% Replace physics package with alternatives
% - Physics replacements
% 	- braket
% 	- derivative (need tectonic to support TeXLive 2021)
% 	- vectors?
% Update commands for siunitx (requires TeXLive 2021)




%%%%%%%%%%%%%%%%%%
%% PACKAGES
%%%%%%%%%%%%%%%%%%


%% debugging / diagnostics
\RequirePackage[l2tabu,orthodox]{nag} % nags user about obsolete and improper syntax

\usepackage{xparse} % provides high-level interface for producing document-level commands
	% via \[Declare/New/Renew/Provide/etc]DocumentCommand
	% allows for more than one optional argument in commands

\usepackage{iftex,ifluatex,ifxetex,ifdraft} % check if a document is being processed with pdfTeX, or XeTEX, or LuaTEX
\newif\ifxetexorluatex % a new conditional starts as false, true if using XeTeX OR LuaTeX
\ifnum 0\ifxetex 1\fi\ifluatex 1\fi>0
   \xetexorluatextrue
\fi

%% fonts and encoding 

% standard and structural packages

% \newcommand\bmmax{4} % increase max number of bm font allocations. default 4?
% \newcommand\hmmax{1} % increase max number of hm font allocations. default 3?
\usepackage{bm} % provides \bm command for robustly bolding math characters

%%%%%%%%%%%%%%%%%%%%%% Additional math fonts %%%%%%%%%%%%%%%%%%%%%%%%%%%%%%%%%%%
\usepackage{amssymb} % for \mathbb (upper case only), \mathfrak fonts
%%%%%%%%%%%%%%%%%%%%%%%%%%%%%%%%%%%%%%%%%%%%%%%%%%%%%%%%%%%%%%%%%%%%%%%%%%%%%%%%%

\usepackage{microtype} % improves kerning in certain cases. 
	% recommended to disable protrusion in table of contents!

%% Latex interface 

\usepackage{letltxmacro} % provides \LetLtxMacro command for correct renaming of commands
\usepackage{etoolbox} % provides many useful programming tools, 
	% e.g. \ifdefempty{cs}{true}{false}



%% media interface

\usepackage{graphicx} % support the \includegraphics command and options
\usepackage{subfig} % Support for subfigures and subcaptions
	\captionsetup[subfloat]{position=bottom}
% \makeatletter%
% \@ifclassloaded{tufte-handout}% xcolor is pre-loaded with the tufte-latex package
%   {
	% \ifxetexorluatex % if lua- or xelatex http://tex.stackexchange.com/a/140164/1913
		\newcommand{\textools}[2][5]{%
			\begingroup\addfontfeatures{LetterSpace=#1}#2\endgroup
		}
		\renewcommand{\allcapsspacing}[1]{\textools[15]{#1}}
		\renewcommand{\smallcapsspacing}[1]{\textools[10]{#1}}
		\renewcommand{\allcaps}[1]{\textools[15]{\MakeTextUppercase{#1}}}
		\renewcommand{\smallcaps}[1]{\smallcapsspacing{\scshape\MakeTextLowercase{#1}}}
		\renewcommand{\textsc}[1]{\smallcapsspacing{\textsmallcaps{#1}}}
% 	\else
% 	\fi
% }%
%   {\usepackage[dvipsnames,svgnames,table,hyperref]{xcolor}
% 	% provides access to large number of colors and related features
% 	% see end notes for lists of available colors
% 	}%
% \makeatother%
\usepackage{svg} % provides \includesvg command for svg figures. 
\usepackage{pgfplots} % for plotting in tikzpicture environment
	\pgfplotsset{compat=1.16} % required to select newest version
\usepackage{tikzscale} % allows \includegraphics{*.tikz} and scaling of TiKZ images


%% math interface

\usepackage{amsmath} % for nice math commands and environments
\usepackage{mathtools} % extends amsmath with bug fixes and useful commands, e.g.
	% \shortintertext for short interjections in align environment,
	% \prescript{t}{b}{X} for simple, nicely aligned math pre-(super/sub)scripts
	% \Aboxed{...} for boxing full lines in 'align' environment
\usepackage{array} % improves array support, esp. in tabular env. 
	% see also xtab.sty
\usepackage{booktabs} % allows for improved spacing in tabular env. 
	% use \toprule, \*midrule, \bottomrule instead of \hline
	% see also ctable.sty

\usepackage{derivative} % provides \odv, \pdv, \odif, \pdif
\usepackage{bropd} % provides \br command which simplifies nesting of bracketed terms 
	% e.g. \br{\br{x-a}^2+\br{y-b}^2} produces \left[ \left( x-a \right)^2 + \left( y-b \right)^2 \right]


%% Science and programming packages
\usepackage{fvextra} % for verbatim and comment environments with \Verb
\usepackage{chemmacros} % for writing chemical formulas with \ch, e.g. \ch{AgCl2-} or \ch{^{227}_{90}Th+}
	\usechemmodule{
		spectroscopy, % provides 
    thermodynamics, % provides state variables and equations
    units, % provides \[mM]olar, \Torr, \atm, \cal, \cmc, \MolMass
		} % also loads siunitx and chemformula
	\DeclareSIUnit\ppm{ppm}
  \sisetup{% siunit package options
			per-mode = symbol,%
			inter-unit-product=\ensuremath{{}\!\cdot\!{}},%
			separate-uncertainty,%
			multi-part-units = single,%
			retain-explicit-plus,%
			list-final-separator={, and },%
			math-celsius = °\text{C}, % for temperatures
			text-celsius = °C,
			math-degree = °, % for angles
			text-degree = °}%


\usepackage{physics} % provides streamlined interface with many commands to simplify
	% writing standard physics notation (bra-kets, derivatives, etc.)
	% differentials and derivatives: \dd[n]{x}, \dv[n]{f}{x}, \dv{x}, \pdv{f}{x}{y}, \var{Q}, \fdv{F}{g}
	% bra-ket notation: \bra, \ket, \braket, \dyad, \matrixel 
	% \qty(x) for delimited quantities
	% \abs, \norm, \eval, \order ,\comm, \acomm
	% vectors: \vb, \va, vu, \vdot, \cross, \grad, \div, \curl, \laplacian
	% largely replaces and adds basic math functions with auto-delimiters: trig, linear algebra, etc.
		% no inverse hyperbolic trig? 
	% in particular adds \tr, \Tr, \rank, \erf, \Res, \pv / \PV, \Re, \Im
	% auto padding text: \qq{string}
	% matrix quantities: \mqty(a & b \\ c & d) or \mqty[x \\ y], Pauli \pmat{n}, diagonal matrices \mqty(\dmat{1,2&3\\4&5}), anti-diagonal \admat

%% misc packages

\usepackage{datetime} % allows easy formatting of dates, e.g. \formatdate{dd}{mm}{yyyy}
%\usepackage[inline]{enumitem} % allows for custom labels on enumerated lists
	% e.g. \begin{enumerate}[label=\textbf{(\alph*)}]
	% label options are: \alph, \Alph, \arabic, \roman, and \Roman
	% inline option creates '*' versions of enumerate, itemize, description 
		% which can be inlined within the text of a paragraph. 
% \usepackage{outlines} % provides 'outline[style]' environment, allowing for subitems in lists
	% e.g. \begin{outline} \1 item \2 subitem \3[A)] subsubitem \1 item \end{outline}
	% or with other style: \begin{outline}[enumerate], etc

% \usepackage{rotating}
	% provides environments for rotating arbitrary objects, e.g. sideways, turn[ang], rotate[ang]
	% also provides macro \turnbox{ang}{stuff}
%\usepackage{ctable} % allows for footnotes under table and properly spaced caption above 
	% must be loaded after tikz
	% incorporates (..?)
\usepackage{framed} % provides boxed 'framed' environment for easily boxing text 
\usepackage{tcolorbox}
	\tcbuselibrary{skins, breakable, xparse, minted}
	% provides fancier boxes than regular \makebox, \fbox, etc.
	% e.g. \doublebox, \ovalbox, \shadowbox
	% Can use `\tcbuselibrary{listings}` to use the listings library, 
		% doesn't require a language to be defined. 
\usepackage{empheq} % provides 'empheq' environment 
	% for improved control over shape, size, color of framed boxes, e.g. 
\newcommand{\boxedeq}[2]{
	\begin{empheq}[box={\fboxsep=6pt\fbox}]{align}\label{#1}#2\end{empheq}
}
\newcommand{\coloredeq}[2]{
	\begin{empheq}[box=\colorbox{lightgreen}]{align}\label{#1}#2\end{empheq}
}


%% document interface 

\usepackage{footnote} % 
\usepackage{hyperref} % adds hyperlinks and outline to PDF documents
	\hypersetup{%
		pdfencoding=auto,%
		psdextra,%
		pdfusetitle,%
		colorlinks=true,%
		linkcolor=BrickRed, %
		citecolor=Green, %
		filecolor=Mulberry, %
		urlcolor=NavyBlue, %
		menucolor=BrickRed, %
		runcolor=Mulberry, %
		linkbordercolor=BrickRed, %
		citebordercolor=Green, %
		filebordercolor=Mulberry, %
		urlbordercolor=NavyBlue, %
		menubordercolor=BrickRed, %
		runbordercolor=Mulberry %
		} %
	% options enable enhanced unicode and math support in PDF outlines [causes conflict with \C command?]
\usepackage{cleveref} % provides \cref command which inserts contextually correct word in front of ref.
	% e.g. \cref{eq:myeq} --> Equation 1.2, or so
\usepackage{bookmark} % improves package hyperref's bookmarking. 
	% properties such as style and color can be set. Generates bookmarks in first run. 

%% font packages -- load fontenc, then inputenc, then lmodern. 
% see http://tex.stackexchange.com/a/44699
\usepackage{fontspec}
\usepackage[math-style=ISO]{unicode-math} 
\ifdraft{}{
	\setmainfont{STIX2Text}[
		Extension={.otf},
		UprightFont={*-Regular},
		BoldFont={*-Bold},
		ItalicFont={*-Italic},
		BoldItalicFont={*-BoldItalic},]
	\setmathfont{STIX2Math}[
		Extension={.otf}]
}
	\renewcommand{\vb}[1]{\symbf{#1}} % fix vector command from physics package to work with unicode-math


%% load later packages
\usepackage{lineno} % provides line numbers in main text for reference and peer review
	% activated by calling \linenumbers in document
\usepackage[textsize=footnotesize]{todonotes}





%%%%%%%%%%%%%%%%%%
%% COMMANDS
%%%%%%%%%%%%%%%%%%

% \newcommand{\mtext}[1]{{\textnormal{#1}}} % for writing text within math mode, e.g. for subscripts
\LetLtxMacro{\mtext}{\text} % legacy alias for \mtext
% \LetLtxMacro{\opname}{\operatorname} % custom operator names
% %\newcommand{\tr}{\opname{tr}} % for trace
% %\newcommand{\rank}{\opname{rank}} % for rank
% \newcommand{\diag}{\opname{diag}} % i.e. \diag(\lambda_1, \dots, \lambda_n)
% \LetLtxMacro{\fancyRe}{\real} % already renamed from physics.sty
% \LetLtxMacro{\fancyIm}{\imaginary} % see above
% \renewcommand{\Re}{\opname{Re}}
% \renewcommand{\Im}{\opname{Im}}
% \renewcommand{\Res}{\opname*{Res}} % for residue function (handles limits properly)
% \newcommand{\inv}{^{-1}}
% \newcommand{\sgn}{\opname{sgn}} % sign/signum function
\DeclareMathOperator{\sgn}{sgn}
\DeclareMathOperator{\erfc}{erfc}
\DeclareMathOperator{\GammaFunc}{\symup{\Gamma}}
\newcommand{\iu}{\TextOrMath{$\mtext{i}$}{\mtext{i}\mkern1mu}}
%
% \newcommand{\laplacian}{\nabla^2}

%% pre-defined colors
% standard: black, blue, brown, cyan, darkgray, gray, green, lightgray, lime, magenta, olive, orange, pink, purple, red, teal, violet, white, yellow
%
% dvips: Apricot, Aquamarine, Bittersweet, Black, Blue, BlueGreen, BlueViolet, BrickRed, Brown, BurntOrange, CadetBlue, CarnationPink, Cerulean, CornflowerBlue, Cyan, Dandelion, DarkOrchid, Emerald, ForestGreen, Fuchsia, Goldenrod, Gray, Green, GreenYellow, JungleGreen, Lavender, LimeGreen, Magenta, Mahogany, Maroon, Melon, MidnightBlue, Mulberry, NavyBlue, OliveGreen, Orange, OrangeRed, Orchid, Peach, Periwinkle, PineGreen, Plum, ProcessBlue, Purple, RawSienna, Red, RedOrange, RedViolet, Rhodamine, RoyalBlue, RoyalPurple, RubineRed, Salmon, SeaGreen, Sepia, SkyBlue, SpringGreen, Tan, TealBlue, Thistle, Turquoise, Violet, VioletRed, White, WildStrawberry, Yellow, YellowGreen, YellowOrange

\usepackage{pgfplotstable}
  \usetikzlibrary{tikzmark}
  \pgfplotsset{
    label style={font=\Large},
    tick label style={font=\Large}
  }
  
\crefformat{problem}{prob.~(#2#1#3)}

%%%%%%% Bibliography options %%%%%%%%%%%%%%%%%%%%%%%%%%%%%%%%%%%%%%%%%%%%%%%%%%
% (fold)
\usepackage[%
	style=numeric-comp,%
  sortcites,%
  sorting=none,%
  defernumbers=true,%
  hyperref,
  backend=biber,
	]{biblatex}
	\addbibresource{../pchem_bib.bib}

	% Filter bibliography file to only include entries matching keyword.
	\DeclareSourcemap{%
	  \maps[datatype=bibtex]{%
	    \map{%
	      \step[nocited, final]%
	      \step[fieldsource=keywords, notmatch=errors, final]%
	      \step[entrynull]%
	    }%
	  }%
	}
	
	% Add a section for cited bibliography entries. 
  % At the end, create a second bibliography for uncited items (Further Reading).
	\DeclareBibliographyCategory{cited}
	\AtEveryCitekey{\addtocategory{cited}{\thefield{entrykey}}}
	
  % Define a style for the Further Reading section
  \defbibenvironment{nolabelbib}
    {\list
       {}
       {\setlength{\leftmargin}{2\bibhang}%
        \setlength{\itemindent}{-\bibhang}%
        \setlength{\itemsep}{\bibitemsep}%
        \setlength{\parsep}{\bibparsep}}}
    {\endlist}
    {\item}
  
  % Remove URL from citations if DOI is present.
	\AtEveryBibitem{%
	  \iffieldundef{doi}{ % do nothing if true
		}
		{ % otherwise, clear the URL
			\clearfield{url}
		}%
	}
% (end)
%%%%%%%%%%%%%%%%%%%%%%%%%%%%%%%%%%%%%%%%%%%%%%%%%%%%%%%%%%%%%%%%%%%%%%%%%%%%%%%

\title{Introduction to Error Analysis}

\author{Dustin Wheeler}

\begin{document}

\maketitle

\section{Homework Problems: Error Propagation} % (fold)
\label{sec:homework_problems_error_propagation}

\begin{enumerate}
	\item In each of the following cases, determine the answer and its error, assuming that the errors to the relevant quantities involved in the calculation are uncorrelated (\emph{i.e.,} the errors add in quadrature).\marginnote{This problem is from \textcite{lyons1991data}.}
	\begin{enumerate}
		\item Determine the distance between the points \( \br{\num{0.0(2)}, \num{0.0(3)}} \) and \( \br{\num{3.0(3)}, \num{4.0(2)}} \), and the angle that the line joining them makes with the \( x \) axis. 
		\item The number, \( N \), of particles surviving a distance, \( x \), in a medium is given by \[ 
		N_0 \exp\br{-x/\lambda} \, ,
		\] 
		where \( N_0 \) is the number of particles at \( x = 0 \) and \( \lambda \) is the mean free path of a particle. 
		What is \( N \) if \( N_0=\num{1000(5)e6} \), \( x = \SI{1.00(1)}{\m} \), and \( \lambda = \SI{0.25(6)}{\m} \)?
		
		\item A particle travels along a straight-line trajectory given by \[
		 y = a + bx \, .
		 \] 
		If \( a = \SI{2.5(3)}{\m} \) and \( b = \num{5.0(1)e-2} \), what is the value of \( y \) at:
		\begin{enumerate}
			\item \( x = \SI{4}{\m} \) and
			
			\item \( x = \SI{4.0(1)}{\m} \)?
		\end{enumerate}
		
		\item The molar specific heat, \( c_{\textsc{m}} \), of a metal at low temperature, \( T \), is given by \[
			c_{\textsc{m}} = a T + b T^3 \, .
		\]
		If \( a = \SI{1.35(5)}{\mJ\per\mol\K^2} \), \( b = \SI{0.021(1)}{\mJ\per\mol\K^4} \), and \( T = \SI{5.0(5)}{\K} \), what is the value of \( c_{\textsc{m}} \)?
	\end{enumerate}
	
	\item\marginnote{The following problems come from \textcite{taylor1997error}.} Evaluate each of the following:
	\begin{enumerate}
		\item \( \br{\num{5.6(7)} + \num{3.70(3)}} \)
		
		\item \( \br{\num{5.6(7)} + \num{2.3(1)}} \)
		
		\item \( \br{\num{5.6(7)} + \num{4.1(2)}} \)
		
		\item \( \br{\num{5.6(7)} + \num{1.9(3)}} \)
	\end{enumerate}
	For each sum, consider both the case that the original uncertainties are independent and random (``errors add in quadrature'') and that they are not (``errors add directly''). 
	Assuming the uncertainties are needed with only one significant figure, identify those cases in which the second of the original uncertainties can be ignored entirely. 
	If you decide to do the additions in quadrature on a calculator, note that the conversion from rectangular to polar coordinates automatically calculates \( \sqrt{x^2 + y^2} \) for a given \( x \) and \( y \). 
	
	\item A student makes the following measurements:
		\begin{align*}
			a &= \SI{5(1)}{\cm} \\
			b &= \SI{18(2)}{\cm} \\
			c &= \SI{12(1)}{\cm} \\
			t &= \SI{2.0(5)}{\s} \\
			m &= \SI{18(1)}{\g} 
		\end{align*}
	\begin{enumerate}
		\item Use the provisional rules (eqs. 3.4 and 3.8 in the text) to compute the following quantities with their absolute and relative uncertainties (assuming the errors are correlated, \emph{i.e., not} independent and random):
		\begin{enumerate}
			\item \( a + b + c \)
			\item \( a + b - c \)
			\item \( ct \)
			\item \( mb/t \)
		\end{enumerate}
	
		\item Repeat the problem assuming that the original uncertainties \emph{are} independent and random. 
		Arrange your answers in a table to compare the two different methods of propagating errors. 
	\end{enumerate}
	
	\begin{figure*}[tb]
		\centering
		\pgfdeclarelayer{background}
		\pgfdeclarelayer{foreground}
		\pgfsetlayers{background,main,foreground}
			\begin{tikzpicture}
				\begin{pgfonlayer}{background}
					\begin{axis}[
						name=calibCurve,
						axis line style=semithick,
						width=5in,
						height=3.5in,
						xlabel={Deflection \( \theta \) (degrees)},
						xmin=50.01, xmax=55,
						xtick={50,51,52,53,54,55},
						extra x ticks={51,52,53,54},
						xtick align=inside,
						ylabel={Wavelength \( \lambda \) (\si{\nm})},
						ymin=400, ymax=700,
						ytick align=inside,
						ytick={400,450,500,550,600,650,700},
						extra y ticks={500,600},
						minor tick num=4,
						extra tick style={grid=major}
						]
						\addplot [black, very thick, domain=50:55] {12 * (exp(-0.62 * (x-55)) - 1.75 * (x-54.5)) + 400};
						\coordinate (insetPosition) at (rel axis cs:0.97,0.95);
					\end{axis}
				\end{pgfonlayer}

				\begin{pgfonlayer}{foreground}
					\begin{axis}[
						name=insetAxis,
						at={(insetPosition)},
						anchor={outer north east},
						axis background/.style={fill=white},
						width=2.5in,
						height=2in,
						xtick={52.4,52.6,52.8,53.0},
						extra x ticks={52.5,52.6,52.7,52.8},
						extra x tick labels={},
						xmin=52.4, xmax=52.9,
						ytick={475,480,485,490,495,500,505},
						yticklabel={\pgfmathparse{mod(\ticknum+1,2)==0?int(5*\ticknum+475):}\pgfmathresult},
						extra y ticks={480,490,500},
						ymin=475, ymax=505,
						anchor=north east,
						minor tick num=9,
						tick label style={font=\small},
						extra tick style={grid=major},
						extra y tick labels={},
						]
						\addplot [black, very thick, domain=52.4:52.9] {12 * (exp(-0.62 * (x-55)) - 1.75 * (x-54.5)) + 400};
					\end{axis}
				\end{pgfonlayer}

				\begin{pgfonlayer}{main}
					\fill [black!0] ([shift={(+2pt,+2pt)}] insetAxis.outer south west)
						rectangle	([shift={(+2pt,+2pt)}] insetAxis.outer north east);
				\end{pgfonlayer}

			\end{tikzpicture}
		\caption{Calibration curve of wavelength \( \lambda \) against deflection \( \theta \) for \cref{prob:spec}.}
		\label{fig:calib}
	\end{figure*}
	\item \label[problem]{prob:spec} A spectrometer is a device for separating the different wavelengths in a beam of light and measuring the wavelengths. 
	It deflects the different wavelengths through different angles \( \theta \), and, if the relation between the angle \( \theta \) and the wavelength \( \lambda \) is known, the experimenter can find \( \lambda \) by measuring \( \theta \). 
	Careful measurements with a certain spectrometer have established the calibration curve shown in \cref{fig:calib}; this figure is simply a graph of \( \lambda \) (in nanometers, \si{\nm}) agains \( \theta \), obtained by measuring \( \theta \) for several accurately known wavelengths \( \lambda \). 
	A student directs a narrow beam of light from a hydrogen lamp through this spectrometer and finds that the light consists of just three well-defined wavelengths; that is, she sees three narrow beams (one red, one turquoise, and one violet) emerging at three different angles. 
	She measures those angles as
	\begin{align*}
		\theta_1 &= \SI{51.0(2)}{\degree} \\
		\theta_2 &= \SI{52.6(2)}{\degree} \\
		\theta_3 &= \SI{54.0(2)}{\degree} \\
	\end{align*}
	\begin{enumerate}
		\item \label[problem]{prob:calib} Use the calibration curve in \cref{fig:calib} to find the corresponding wavelengths \( \lambda_1 \), \( \lambda_2 \), and \( \lambda_3 \) with their uncertainties. 
		\item According to theory, these wavelengths should be \SIlist{656;486;434}{\nm}. 
		Are the student's measurements in satisfactory agreement with these theoretical values?
		\item \label[problem]{prob:zoomed} If the spectrometer has a vernier scale to read the angles, the angles can be measured with an uncertainty of \ang{0.05} or even less. 
		Let us suppose the three measurements above have uncertainties of \SI{\pm0.05}{\degree}. Given this new, smaller uncertainty in the angles and \emph{without drawing any more lines on the graph,} use your answers from \cref{prob:calib} to find the new uncertainties in the three wavelengths, explaining clearly how you do it.\footnote{Hint: the calibration curve is nearly straight in the vicinity of any one measurement.}
		\item To take advantage of more accurate measurements, an experimenter may need to enlarge the calibration curve. 
		The inset in \cref{fig:calib} is an enlargement in the vicinity of the angle \( \theta_2 \). 
		Use this graph to find the wavelength \( \lambda_2 \) if \( \theta_2 \) has been measured as \SI{52.72(5)}{\degree}; check that your prediction for the uncertainty of \( \lambda_2 \) in \cref{prob:zoomed} was correct. 
	\end{enumerate}
	
	\item If you measure two independent variables as \[
		x = \num{6.0(1)} \quad \text{and} \quad y = \num{3.0(1)} \, ,
	\]
	and use these values to calculate \( q = xy + x^2/y \), what will be your answer and its uncertainty?\marginnote[-4\baselineskip]{You must use the general rule (eq. 3.47 in \textcite{taylor1997error} to find \( \delta q \). To simplify your calculation, do it first by finding the two separate contributions \( \delta q_x \) and \( \delta q_y \) as defined in rules 3.50 and 3.51, then combine them in quadrature.}
	
	\item If an object is placed at a distance \( p \) from a lens and an image is formed at a distance \( q \) from the lens, the lens's focal length can be found as\sidenote{This equation follows from the \emph{lens equation}, \( 1/f = (1/p) + (1/q) \).} 
	\begin{equation}
		f = \frac{pq}{p + q} \, . 
		\label[equation]{eq:lens}
	\end{equation}
	\begin{enumerate}
		\item \label[problem]{prob:lens_a} Use the general rule (eq. 3.47) to derive a formula for the uncertainty \( \delta f \) in terms of \( p \), \( q \), and their uncertainties. 
		
		\item Starting from \cref{eq:lens} directly, you cannot find \( \delta f \) in steps because \( p \) and \( q \) both appear in numerator and denominator. Show, however, that \( f \) can be rewritten as \[
			f = \frac{1}{1/p + 1/q} \, .
		\]
		Starting from this form, you \emph{can} evaluate \( \delta f \) in steps. Do so, and verify that you get the same answer as in \cref{prob:lens_a}.
	\end{enumerate}
\end{enumerate}


% section homework_problems_error_propagation (end)

\nocite{*}
\printbibliography[category=cited]% default title for `article` class: "References"

\printbibliography[
  env=nolabelbib,
	title={Further Reading},
	resetnumbers,
	notcategory=cited,
	]

\end{document}
