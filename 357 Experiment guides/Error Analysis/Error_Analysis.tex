\documentclass[nobib,nofonts,nols,nohyper]{tufte-handout}
% \RequirePackage[cache=false]{minted}
%% Header file cloned from https://github.com/wickles/latex-base

%%%%%%%%%%%%%%%%%%
%% CONTENTS
%%%%%%%%%%%%%%%%%%
% To-do / issues
% Packages
% Commands
% Special Symbols
% Environments
% More commands: Resizable delimiters
% More commands: Derivatives
% Useful templates
% Notes
%%%%%%%%%%%%%%%%%%
%%%%%%%%%%%%%%%%%%



%%%%%%%%%%%%%%%%%%
%% TO-DO / ISSUES
%%%%%%%%%%%%%%%%%%

%% packages
% Replace physics package with alternatives
% - Physics replacements
% 	- braket
% 	- derivative (need tectonic to support TeXLive 2021)
% 	- vectors?
% Update commands for siunitx (requires TeXLive 2021)




%%%%%%%%%%%%%%%%%%
%% PACKAGES
%%%%%%%%%%%%%%%%%%


%% debugging / diagnostics
\RequirePackage[l2tabu,orthodox]{nag} % nags user about obsolete and improper syntax

\usepackage{xparse} % provides high-level interface for producing document-level commands
	% via \[Declare/New/Renew/Provide/etc]DocumentCommand
	% allows for more than one optional argument in commands

\usepackage{iftex,ifluatex,ifxetex,ifdraft} % check if a document is being processed with pdfTeX, or XeTEX, or LuaTEX
\newif\ifxetexorluatex % a new conditional starts as false, true if using XeTeX OR LuaTeX
\ifnum 0\ifxetex 1\fi\ifluatex 1\fi>0
   \xetexorluatextrue
\fi

%% fonts and encoding 

% standard and structural packages

% \newcommand\bmmax{4} % increase max number of bm font allocations. default 4?
% \newcommand\hmmax{1} % increase max number of hm font allocations. default 3?
\usepackage{bm} % provides \bm command for robustly bolding math characters

%%%%%%%%%%%%%%%%%%%%%% Additional math fonts %%%%%%%%%%%%%%%%%%%%%%%%%%%%%%%%%%%
\usepackage{amssymb} % for \mathbb (upper case only), \mathfrak fonts
%%%%%%%%%%%%%%%%%%%%%%%%%%%%%%%%%%%%%%%%%%%%%%%%%%%%%%%%%%%%%%%%%%%%%%%%%%%%%%%%%

\usepackage{microtype} % improves kerning in certain cases. 
	% recommended to disable protrusion in table of contents!

%% Latex interface 

\usepackage{letltxmacro} % provides \LetLtxMacro command for correct renaming of commands
\usepackage{etoolbox} % provides many useful programming tools, 
	% e.g. \ifdefempty{cs}{true}{false}



%% media interface

\usepackage{graphicx} % support the \includegraphics command and options
\usepackage{subfig} % Support for subfigures and subcaptions
	\captionsetup[subfloat]{position=bottom}
% \makeatletter%
% \@ifclassloaded{tufte-handout}% xcolor is pre-loaded with the tufte-latex package
%   {
	% \ifxetexorluatex % if lua- or xelatex http://tex.stackexchange.com/a/140164/1913
		\newcommand{\textools}[2][5]{%
			\begingroup\addfontfeatures{LetterSpace=#1}#2\endgroup
		}
		\renewcommand{\allcapsspacing}[1]{\textools[15]{#1}}
		\renewcommand{\smallcapsspacing}[1]{\textools[10]{#1}}
		\renewcommand{\allcaps}[1]{\textools[15]{\MakeTextUppercase{#1}}}
		\renewcommand{\smallcaps}[1]{\smallcapsspacing{\scshape\MakeTextLowercase{#1}}}
		\renewcommand{\textsc}[1]{\smallcapsspacing{\textsmallcaps{#1}}}
% 	\else
% 	\fi
% }%
%   {\usepackage[dvipsnames,svgnames,table,hyperref]{xcolor}
% 	% provides access to large number of colors and related features
% 	% see end notes for lists of available colors
% 	}%
% \makeatother%
\usepackage{svg} % provides \includesvg command for svg figures. 
\usepackage{pgfplots} % for plotting in tikzpicture environment
	\pgfplotsset{compat=1.16} % required to select newest version
\usepackage{tikzscale} % allows \includegraphics{*.tikz} and scaling of TiKZ images


%% math interface

\usepackage{amsmath} % for nice math commands and environments
\usepackage{mathtools} % extends amsmath with bug fixes and useful commands, e.g.
	% \shortintertext for short interjections in align environment,
	% \prescript{t}{b}{X} for simple, nicely aligned math pre-(super/sub)scripts
	% \Aboxed{...} for boxing full lines in 'align' environment
\usepackage{array} % improves array support, esp. in tabular env. 
	% see also xtab.sty
\usepackage{booktabs} % allows for improved spacing in tabular env. 
	% use \toprule, \*midrule, \bottomrule instead of \hline
	% see also ctable.sty

\usepackage{derivative} % provides \odv, \pdv, \odif, \pdif
\usepackage{bropd} % provides \br command which simplifies nesting of bracketed terms 
	% e.g. \br{\br{x-a}^2+\br{y-b}^2} produces \left[ \left( x-a \right)^2 + \left( y-b \right)^2 \right]


%% Science and programming packages
\usepackage{fvextra} % for verbatim and comment environments with \Verb
\usepackage{chemmacros} % for writing chemical formulas with \ch, e.g. \ch{AgCl2-} or \ch{^{227}_{90}Th+}
	\usechemmodule{
		spectroscopy, % provides 
    thermodynamics, % provides state variables and equations
    units, % provides \[mM]olar, \Torr, \atm, \cal, \cmc, \MolMass
		} % also loads siunitx and chemformula
	\DeclareSIUnit\ppm{ppm}
  \sisetup{% siunit package options
			per-mode = symbol,%
			inter-unit-product=\ensuremath{{}\!\cdot\!{}},%
			separate-uncertainty,%
			multi-part-units = single,%
			retain-explicit-plus,%
			list-final-separator={, and },%
			math-celsius = °\text{C}, % for temperatures
			text-celsius = °C,
			math-degree = °, % for angles
			text-degree = °}%


\usepackage{physics} % provides streamlined interface with many commands to simplify
	% writing standard physics notation (bra-kets, derivatives, etc.)
	% differentials and derivatives: \dd[n]{x}, \dv[n]{f}{x}, \dv{x}, \pdv{f}{x}{y}, \var{Q}, \fdv{F}{g}
	% bra-ket notation: \bra, \ket, \braket, \dyad, \matrixel 
	% \qty(x) for delimited quantities
	% \abs, \norm, \eval, \order ,\comm, \acomm
	% vectors: \vb, \va, vu, \vdot, \cross, \grad, \div, \curl, \laplacian
	% largely replaces and adds basic math functions with auto-delimiters: trig, linear algebra, etc.
		% no inverse hyperbolic trig? 
	% in particular adds \tr, \Tr, \rank, \erf, \Res, \pv / \PV, \Re, \Im
	% auto padding text: \qq{string}
	% matrix quantities: \mqty(a & b \\ c & d) or \mqty[x \\ y], Pauli \pmat{n}, diagonal matrices \mqty(\dmat{1,2&3\\4&5}), anti-diagonal \admat

%% misc packages

\usepackage{datetime} % allows easy formatting of dates, e.g. \formatdate{dd}{mm}{yyyy}
%\usepackage[inline]{enumitem} % allows for custom labels on enumerated lists
	% e.g. \begin{enumerate}[label=\textbf{(\alph*)}]
	% label options are: \alph, \Alph, \arabic, \roman, and \Roman
	% inline option creates '*' versions of enumerate, itemize, description 
		% which can be inlined within the text of a paragraph. 
% \usepackage{outlines} % provides 'outline[style]' environment, allowing for subitems in lists
	% e.g. \begin{outline} \1 item \2 subitem \3[A)] subsubitem \1 item \end{outline}
	% or with other style: \begin{outline}[enumerate], etc

% \usepackage{rotating}
	% provides environments for rotating arbitrary objects, e.g. sideways, turn[ang], rotate[ang]
	% also provides macro \turnbox{ang}{stuff}
%\usepackage{ctable} % allows for footnotes under table and properly spaced caption above 
	% must be loaded after tikz
	% incorporates (..?)
\usepackage{framed} % provides boxed 'framed' environment for easily boxing text 
\usepackage{tcolorbox}
	\tcbuselibrary{skins, breakable, xparse, minted}
	% provides fancier boxes than regular \makebox, \fbox, etc.
	% e.g. \doublebox, \ovalbox, \shadowbox
	% Can use `\tcbuselibrary{listings}` to use the listings library, 
		% doesn't require a language to be defined. 
\usepackage{empheq} % provides 'empheq' environment 
	% for improved control over shape, size, color of framed boxes, e.g. 
\newcommand{\boxedeq}[2]{
	\begin{empheq}[box={\fboxsep=6pt\fbox}]{align}\label{#1}#2\end{empheq}
}
\newcommand{\coloredeq}[2]{
	\begin{empheq}[box=\colorbox{lightgreen}]{align}\label{#1}#2\end{empheq}
}


%% document interface 

\usepackage{footnote} % 
\usepackage{hyperref} % adds hyperlinks and outline to PDF documents
	\hypersetup{%
		pdfencoding=auto,%
		psdextra,%
		pdfusetitle,%
		colorlinks=true,%
		linkcolor=BrickRed, %
		citecolor=Green, %
		filecolor=Mulberry, %
		urlcolor=NavyBlue, %
		menucolor=BrickRed, %
		runcolor=Mulberry, %
		linkbordercolor=BrickRed, %
		citebordercolor=Green, %
		filebordercolor=Mulberry, %
		urlbordercolor=NavyBlue, %
		menubordercolor=BrickRed, %
		runbordercolor=Mulberry %
		} %
	% options enable enhanced unicode and math support in PDF outlines [causes conflict with \C command?]
\usepackage{cleveref} % provides \cref command which inserts contextually correct word in front of ref.
	% e.g. \cref{eq:myeq} --> Equation 1.2, or so
\usepackage{bookmark} % improves package hyperref's bookmarking. 
	% properties such as style and color can be set. Generates bookmarks in first run. 

%% font packages -- load fontenc, then inputenc, then lmodern. 
% see http://tex.stackexchange.com/a/44699
\usepackage{fontspec}
\usepackage[math-style=ISO]{unicode-math} 
\ifdraft{}{
	\setmainfont{STIX2Text}[
		Extension={.otf},
		UprightFont={*-Regular},
		BoldFont={*-Bold},
		ItalicFont={*-Italic},
		BoldItalicFont={*-BoldItalic},]
	\setmathfont{STIX2Math}[
		Extension={.otf}]
}
	\renewcommand{\vb}[1]{\symbf{#1}} % fix vector command from physics package to work with unicode-math


%% load later packages
\usepackage{lineno} % provides line numbers in main text for reference and peer review
	% activated by calling \linenumbers in document
\usepackage[textsize=footnotesize]{todonotes}





%%%%%%%%%%%%%%%%%%
%% COMMANDS
%%%%%%%%%%%%%%%%%%

% \newcommand{\mtext}[1]{{\textnormal{#1}}} % for writing text within math mode, e.g. for subscripts
\LetLtxMacro{\mtext}{\text} % legacy alias for \mtext
% \LetLtxMacro{\opname}{\operatorname} % custom operator names
% %\newcommand{\tr}{\opname{tr}} % for trace
% %\newcommand{\rank}{\opname{rank}} % for rank
% \newcommand{\diag}{\opname{diag}} % i.e. \diag(\lambda_1, \dots, \lambda_n)
% \LetLtxMacro{\fancyRe}{\real} % already renamed from physics.sty
% \LetLtxMacro{\fancyIm}{\imaginary} % see above
% \renewcommand{\Re}{\opname{Re}}
% \renewcommand{\Im}{\opname{Im}}
% \renewcommand{\Res}{\opname*{Res}} % for residue function (handles limits properly)
% \newcommand{\inv}{^{-1}}
% \newcommand{\sgn}{\opname{sgn}} % sign/signum function
\DeclareMathOperator{\sgn}{sgn}
\DeclareMathOperator{\erfc}{erfc}
\DeclareMathOperator{\GammaFunc}{\symup{\Gamma}}
\newcommand{\iu}{\TextOrMath{$\mtext{i}$}{\mtext{i}\mkern1mu}}
%
% \newcommand{\laplacian}{\nabla^2}

%% pre-defined colors
% standard: black, blue, brown, cyan, darkgray, gray, green, lightgray, lime, magenta, olive, orange, pink, purple, red, teal, violet, white, yellow
%
% dvips: Apricot, Aquamarine, Bittersweet, Black, Blue, BlueGreen, BlueViolet, BrickRed, Brown, BurntOrange, CadetBlue, CarnationPink, Cerulean, CornflowerBlue, Cyan, Dandelion, DarkOrchid, Emerald, ForestGreen, Fuchsia, Goldenrod, Gray, Green, GreenYellow, JungleGreen, Lavender, LimeGreen, Magenta, Mahogany, Maroon, Melon, MidnightBlue, Mulberry, NavyBlue, OliveGreen, Orange, OrangeRed, Orchid, Peach, Periwinkle, PineGreen, Plum, ProcessBlue, Purple, RawSienna, Red, RedOrange, RedViolet, Rhodamine, RoyalBlue, RoyalPurple, RubineRed, Salmon, SeaGreen, Sepia, SkyBlue, SpringGreen, Tan, TealBlue, Thistle, Turquoise, Violet, VioletRed, White, WildStrawberry, Yellow, YellowGreen, YellowOrange

% \tcbuselibrary{minted}

%%%%%%% Bibliography options %%%%%%%%%%%%%%%%%%%%%%%%%%%%%%%%%%%%%%%%%%%%%%%%%%
\usepackage[%
	bibstyle=numeric,%
	citestyle=numeric-comp,%
	sortcites,%
	sorting=none,%
	defernumbers=true%
	]{biblatex}

%% TODO: Troubleshoot this section… useful to only maintain one bibliography.
	% % Filter bibliography file to only include entries matching keyword.
	% \DeclareSourcemap{
	%   \maps[datatype=bibtex]{
	%     \map{
	%       \step[cited, final]
	%       \step[fieldsource=keywords, match=errors, final]
	%       \step[entrynull]
	%     }
	%   }
	% }
	
	% Add a section for cited bibliography entries. At the end, 
	% create a second bibliography for uncited items.
	\DeclareBibliographyCategory{cited}
	\AtEveryCitekey{\addtocategory{cited}{\thefield{entrykey}}}
	% Remove URL from citations if DOI is present.
	\AtEveryBibitem{%
	  \iffieldundef{doi}{
		% do nothing if true
		}
		{ % otherwise, clear the URL
			\clearfield{url}
		}%
	}
	\addbibresource{prop_err.bib}
%%%%%%%%%%%%%%%%%%%%%%%%%%%%%%%%%%%%%%%%%%%%%%%%%%%%%%%%%%%%%%%%%%%%%%%%%%%%%%%

\title{Introduction to Error Analysis}

\author{Dustin Wheeler}

\begin{document}

\maketitle

\begin{abstract}
	\noindent
	In this exercise, students will be introduced to basic concepts in scientific error analysis and the propagation of error in calculations. 
	Upon completion, you should be familiar with the sources of experimental error, be able to track the impact of those errors through a series of calculations, and accurately report the uncertainty in your final reported values. % \thanks{Transcribed (with corrections) from \cite{AuthorA}} 
\end{abstract}

\section{Introduction} % (fold)
\label{sec:intro}

This exercise assumes that the reader is familiar with the concepts of \emph{significant figures} and \emph{accuracy} and \emph{precision} as they apply to experimental results. 
Additionally, the reader should be comfortable with introductory statistical analysis, simple algebra, and (partial) differentiation. 


% section intro (end)

\section{Review} % (fold)
\label{sec:review}

\subsection{Uncertainty} % (fold)
\label{sub:uncertainty}

Uncertainty, often referred to as \emph{error} in scientific measurements, is an unavoidable fact of life. 
While it cannot be removed, it can be understood and minimized. 
To see why it cannot be removed, we have only to ask multiple people to measure the same object with the same ruler. 
Given a ruler with demarcations down to \SI{1}{\cm} and an object \SI{\sim8.3}{\cm} in length, three students might measure lengths of \SIlist{8.5;8.0;8.2}{\cm}. 
Naively, we might suggest that we would get better agreement by giving the students a ruler with \si{\mm}-demarcations. 
But now, the students might measure \SIlist{8.35;8.30;8.28}{\cm} for the length… we still see disagreement in the value! 
Eventually, we will get to a point where the demarcations on the ruler are too small for the students to see and have spent a lot of time gaining precision that we might not need.
Perhaps all we desired was a quick estimate to know if we could place the object in a \SI{10}{\cm} tray. 
In that case, all we needed was that first measurement and the estimate of \SI{\sim8}{\cm} for the object's length. 

This exercise shows that more precision isn't always necessary and that there is no such thing as \emph{infinite} precision. 
Uncertainty is a part of the process and we would do best to accept it and move forward. 

% subsection uncertainty (end)

\subsection{Estimation of Measurements} % (fold)
\label{sub:estimation_of_measurements}

Now that we can see that uncertainty exists, we must find a way to estimate measurements. 
In the simplest case, where we might take a single measurement, we need a basis from which to start. 
A good rule of thumb when reading from a scale is to assume an uncertainty of \num{0.2} times the smallest marking on the scale. 
In this case, the length readings in the previous section should be presented as \SI{8.5(2)}{\cm}. 
This formatting gives us a way of presenting both our measured value \emph{and} the uncertainty in that measurement. 
In this way, we can be sure that we are reporting our results in the most complete and informative way possible. 
If measurements are being read from a digital instrument, estimation of the reading is unnecessary. 
This does \emph{not} mean that there is no uncertainty! 
It only means that the uncertainty is not determined by your ability to estimate spacing on a scale, but instead is determined by the electronics that make up the instrument. 
Manufacturers make their uncertainty values available in most cases\footnote{If they don't make it available, it's time to ask, ``Why not?''}, so you only need to find the reported uncertainty for the instrument and report that value in your measurements. 

% subsection estimation_of_measurements (end)

\subsection{Statistical Analysis of Multiple Measurements} % (fold)
\label{sub:statistical_analysis_of_multiple_measurements}

In the previous section, we discussed methods to estimate the error of a single measurement. 
Clearly, we can increase the certainty in our measurement by making multiple measurements. 

Suppose we measure the length of our \SI{8.3}{\cm} object again and have each student repeat the measurement twice. 
Now our list of numbers is \SIlist{8.5;8.0;8.2;8.3;8.4;8.4}{\cm}. 
As you will recall from quantitative analysis, we can get the \emph{mean}, \emph{variance}, and \emph{standard deviation} of this measurement. 
As we see in \cref{eq:avg}, the mean, \( \overline{x} \), of a measurement is simply the sum of all the measurements, \( x_i \), divided by the number of measurements.
\begin{equation}
	\overline{x} = \frac{1}{n}\sum_{i=1}^n{x_i}
	\label{eq:avg}
\end{equation}

This gives us a single value to use for our estimate, but no indication of the uncertainty in the measurement. 
The next value we can compute is how far each of our measurements varies from the mean, or the \emph{variance}. 
The variance is calculated by averaging the square of the distance from each measurement to the mean, shown in \cref{eq:var}. 
\begin{equation}
	\sigma^2 = \frac{1}{n}\sum_{i=1}^{n}{\br{x_i - \overline{x}}^2}
	\label{eq:var}
\end{equation}



% subsection statistical_analysis_of_multiple_measurements (end)

\subsection{Sources of Uncertainty} % (fold)
\label{sub:sources_of_uncertainty}

Now that we've established what uncertainty is and how to report it, let's examine some sources and types of uncertainty. 
Recall that the terms \emph{precision} and \emph{accuracy} have very specific definitions to the experimentalist. 
If we go back to the picture of the dartboard, we recall that a precise player will have all of their darts clustered in a small area of the board, while an accurate player will have their darts evenly distributed around the center of the board. 
Only the precise \emph{and} accurate player will have a small cluster of darts near the center of the board. 

In class, we are often repeating well-known experiments or probing a quantity that is both accurately and precisely known. 
In the research lab, this is rarely the case, and when we are making a measurement, we often don't know what the true value of the measurement should be. 
In terms of the dartboard, it's as though the board and wall are covered by a black cloth while the throwing occurs. 
When this is the case, we can't judge our experimental technique based on accuracy and can only rely on the precision of the measurement to guide us. 

With this in mind, we begin to examine factors that will affect the precision and accuracy of our measurements. 


% subsection sources_of_uncertainty (end)

Types/sources of error
Random (random processes)
Systematic (method errors)
Instrumental (Instrumental errors)

Examples



\subsection{Basic Statistics} % (fold)
\label{sub:basic_statistics}

Mean, 
Variance, 
Standard deviation

Probability distributions
- Uniform
- Binomial
- Normal (Gaussian)
- Poisson

% subsection basic_statistics (end)

% section review (end)

\section{Propagation of Error} % (fold)
\label{sec:propagation_of_error}

Simple calculations (rate, area)
% section propagation_of_error (end)

\nocite{*}
\printbibliography[category=cited]% default title for `article` class: "References"

\printbibliography[title={Further Reading},notcategory=cited]

\end{document}
