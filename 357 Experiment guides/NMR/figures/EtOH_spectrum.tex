\pgfplotsset{compat=1.16}
\usetikzlibrary{math}

\pgfplotsset{
  every axis legend/.append style =
    {
      % Change the text alignment so the end of the text (rather than the
      % start) lines up.
      cells = { anchor = east },
      % The standard pgfplots settings use a box around legends:
      % I prefer without this.
      draw  = none
    }
}

\pgfplotsset{
  nmr/.style =
    {
      % Using a cycle list just altering colour means that there are no
      % marks: that is normal for this sort of plot.
      cycle list name = color list , 
      % Ensure that the x-axis values always have the same number of 
      % decimal places, to avoid e.g. "1" but "1.2".
      every x tick label/.append style  =
        { 
          /pgf/number format/.cd ,
           precision = 1 , 
           fixed         ,
           zerofill
        },
      % The labels apply to all plots of this type.
      % Notice that in this case the zero is TMS, but that
      % will depend on the experiment.
      xlabel = $\delta \, (\si{ppm})$,
    },
}

% Not everyone likes the 'axis box' effect which is the pgfplots default.
% Here, we'll set up to use 'Tufte-like' settings: see
% https://www.tug.org/members/TUGboat/tb34-2/tb107dugge.pdf for more on
% this.
\makeatletter
\pgfplotsset{
  tufte axes/.style =
    {
			% after end axis/.code =
        % {
        %   \draw ({rel axis cs:0,0} -| {axis cs:\pgfplots@data@xmin,0})
        %     -- ({rel axis cs:0,0}  -| {axis cs:\pgfplots@data@xmax,0});
        % },
			axis x line = bottom,
			axis y line = none,
			every inner x axis line/.append style={-|},
      tick align      = outside,
      tick pos        = left,
			% ytick						= \empty,
			x dir						= reverse,
    }
}
\makeatother

\begin{figure}[htb]
	\begin{tikzpicture}
		\pgfdeclarelayer{background}
		\pgfdeclarelayer{foreground}
		\pgfsetlayers{background,main,foreground}
	  \begin{pgfonlayer}{background}
	  	\begin{axis}%
	  		    [
	  		      % Choose the general settings
	  		      nmr,
	  		      % and the Tufte style
	  		      tufte axes,
	  		      % Place the legend 'out of the way': this needs a bit of
	  		      % experimentation!
	  		      scale only axis,
							ymin=-1,ymax=18,
							% enlarge y limits={rel=0.2},
	  		      width = 0.95\textwidth
	  		    ]
	  		    % Loop for each scan rate
	  		    \foreach \datafile in {EtOH_nmr_simsc}
	  		      {
	  		        % For each case, add a plot
	  		        \addplot
	  		          table
	  		            [
	  		              % This plot extracts data directly from the instrument files.
	  		              % To do that, we skip the first couple of lines.
	  		              	% skip first n = 2 ,
	  		              % The x-axis has a correction for the zero: that is done using
	  		              % a simple expression
	  		              % x expr       = \thisrowno{0} + 0.412,
	  		              % The y-axis is more complicated! There is a scaling so the
	  		              % currents are in microamperes, and also a division by the
	  		              % scan rate. The data files themselves are named using the
	  		              % scan rate in millivolts per second: that is converted to 
	  		              % volts per second before doing the square root.
	  		              % y expr       = {}
	  		            ]
	  		          from {Figures/\datafile.txt}; 
								\coordinate (insetPosition) at (axis cs:3.687,3);
								\node [coordinate,pin=above:{\ch{-OH}}]
						        at (axis cs:4.8,5.1)   {};
						    \node [coordinate,pin=above:{\ch{-CH3}}]
						        at (axis cs:1.226,15) {};
	  		      };  
	  		\end{axis}
	  \end{pgfonlayer}
		\tikzmath{
			\pkctr=3.688;
			\jCH=6.3/60;
			\jOH=3.7/60;
			\pkcta=\pkctr+\jOH/2;
			\pkcte=\pkctr-\jOH/2;
			\treeroot=7.5;
			\leva=6.75;
			\levb=6.5;
			\levc=5.75;
			\levd=5.5;
			\leve=5.0;
			\levf=4.75;
			\apk=\pkctr+\jOH/2+3/2*\jCH;
			\bpk=\pkctr+\jOH/2+1/2*\jCH;
			\cpk=\pkctr+\jOH/2-1/2*\jCH;
			\dpk=\pkctr+\jOH/2-3/2*\jCH;
			\epk=\pkctr-\jOH/2+3/2*\jCH;
			\fpk=\pkctr-\jOH/2+1/2*\jCH;
			\gpk=\pkctr-\jOH/2-1/2*\jCH;
			\hpk=\pkctr-\jOH/2-3/2*\jCH;
			\jCHlbl=\pkctr-\jOH/2-\jCH;
		}
	  \begin{pgfonlayer}{foreground}
	  	\begin{axis}%
				[
					% Choose the general settings
					nmr,
					% and the Tufte style
					axis x line = none,
					axis y line = none,
					x dir=reverse,
					xlabel={},
					ymax=8,
					scale only axis,
					name=insetAxis,
					at={(insetPosition)},
					anchor={south},
					width=0.5\textwidth,
					height=2.4in,
				]
				{
					\addplot
						table from {Figures/EtOH_nmr_simsc_zoom.txt}; 
					\node (A) at (axis cs:\pkctr,\treeroot) {\ch{-CH2-{}}};
					\coordinate (Ag) at (axis cs:\apk,1.16);
					\coordinate (Bg) at (axis cs:\bpk,3.64);
					\coordinate (Cg) at (axis cs:\cpk,3.90);
					\coordinate (Dg) at (axis cs:\dpk,1.43);
					\coordinate (Eg) at (axis cs:\epk,0.96);
					\coordinate (Fg) at (axis cs:\fpk,3.17);
					\coordinate (Gg) at (axis cs:\gpk,3.56);
					\coordinate (Hg) at (axis cs:\hpk,1.36);
					\coordinate (Af) at (axis cs:\apk,\levf);
					\coordinate (Bf) at (axis cs:\bpk,\levf);
					\coordinate (Cf) at (axis cs:\cpk,\levf);
					\coordinate (Df) at (axis cs:\dpk,\levf);
					\coordinate (Ed) at (axis cs:\epk,\levd);
					\coordinate (Fd) at (axis cs:\fpk,\levd);
					\coordinate (Gd) at (axis cs:\gpk,\levd);
					\coordinate (Hd) at (axis cs:\hpk,\levd);
					\coordinate (Ae) at (axis cs:\pkcta,\leve);
					\coordinate (Ec) at (axis cs:\pkcte,\levc);
					\coordinate (Ad) at (axis cs:\pkcta,\levb);
					\coordinate (Eb) at (axis cs:\pkcte,\levb);
					\coordinate (Aa) at (axis cs:\pkctr,\leva);
					\coordinate (Oa) at (axis cs:\pkcta,6.15);
					\coordinate (Ob) at (axis cs:\pkcte,6.15);
					\coordinate (Ha) at (axis cs:\gpk,5);
					\coordinate (Hb) at (axis cs:\hpk,5);
				};
				\draw (Ag) -- (Af) -- (Ae) -- (Ad) -- (Aa) -- (A);
				\draw (Bg) -- (Bf) -- (Ae);
				\draw (Cg) -- (Cf) -- (Ae);
				\draw (Dg) -- (Df) -- (Ae);
				\draw (Eg) -- (Ed) -- (Ec) -- (Eb) -- (Aa);
				\draw (Fg) -- (Fd) -- (Ec);
				\draw (Gg) -- (Gd) -- (Ec);
				\draw (Hg) -- (Hd) -- (Ec);
				\draw[stealth-stealth] (Oa) -- (Ob);
				\draw[stealth-stealth] (Ha) -- (Hb);
		    \node [coordinate,pin=above left:{$J_\textup{\ch{CH2,OH}}$}]
		        at (axis cs:\pkctr,6.15) {};
		    \node [coordinate,pin=70:{$J_\textup{\ch{CH2,CH3}}$}]
		        at (axis cs:\jCHlbl,5) {};
			\end{axis}
	  \end{pgfonlayer}
	\end{tikzpicture}
	\caption{Simulated NMR spectrum of highly purified ethanol at \SI{90}{\MHz}.}\
	\label{fig:EtOH_spectrum}
\end{figure}