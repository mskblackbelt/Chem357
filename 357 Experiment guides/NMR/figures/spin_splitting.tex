\documentclass{standalone}
% %% Header file cloned from https://github.com/wickles/latex-base

%%%%%%%%%%%%%%%%%%
%% CONTENTS
%%%%%%%%%%%%%%%%%%
% To-do / issues
% Packages
% Commands
% Special Symbols
% Environments
% Notes
%%%%%%%%%%%%%%%%%%
%%%%%%%%%%%%%%%%%%



%%%%%%%%%%%%%%%%%%
%% TO-DO / ISSUES
%%%%%%%%%%%%%%%%%%

% Fix header format in tufte-latex (bad spacing, no small-caps font in Charter). 
% Replace \Molar unit, check for \textsc




%%%%%%%%%%%%%%%%%%
%% PACKAGES
%%%%%%%%%%%%%%%%%%


%% debugging / diagnostics
\RequirePackage[l2tabu,orthodox]{nag} % nags user about obsolete and improper syntax

\usepackage{xparse} % provides high-level interface for producing document-level commands
	% via \[Declare/New/Renew/Provide/etc]DocumentCommand
	% allows for more than one optional argument in commands

\usepackage{ifdraft}

%% fonts and encoding 

\newcommand{\textools}[2][5]{%
	\begingroup\addfontfeatures{LetterSpace=#1}#2\endgroup
	}
\renewcommand{\allcapsspacing}[1]{\textools[15]{#1}}
\renewcommand{\smallcapsspacing}[1]{\textools[10]{#1}}
\renewcommand{\allcaps}[1]{\textools[15]{\MakeTextUppercase{#1}}}
\renewcommand{\smallcaps}[1]{\textit{#1}} 
	% Version of Charter font included with macOS 13 doesn't work with \textsc
	% It uses individual smallcaps glyphs instead. 
% \renewcommand{\textsc}[1]{\smallcapsspacing{\textsmallcaps{#1}}}

%% font packages -- load fontspec, then select fonts and features. 

\usepackage{fontspec}
\usepackage[math-style=ISO,mathrm=sym]{unicode-math} 
\ifdraft{}{
	\setmainfont{Charter}
	\setmathfont{STIX Two Math}
	\setmonofont{Consolas}
}


% standard and structural packages

\usepackage{bm} % provides \bm command for robustly bolding math characters
\usepackage{microtype} % improves kerning in certain cases. 
	% recommended to disable protrusion in table of contents!

%% Latex interface 

\usepackage{letltxmacro} % provides \LetLtxMacro command for correct renaming of commands
\usepackage{etoolbox} % provides many useful programming tools, 
	% e.g. \ifdefempty{cs}{true}{false}



%% media interface

\usepackage{graphicx} % support the \includegraphics command and options
\usepackage{subfig} % Support for subfigures and subcaptions
	\captionsetup[subfloat]{position=bottom}
\usepackage{pgfplots} % for plotting in tikzpicture environment
	\pgfplotsset{compat=1.16} % required to select newest version
\usepackage{tikzscale} % allows \includegraphics{*.tikz} and scaling of TiKZ images


%% math interface

\usepackage{amsmath} % for nice math commands and environments
\usepackage{mathtools} % extends amsmath with bug fixes and useful commands, e.g.
	% \shortintertext for short interjections in align environment,
	% \prescript{t}{b}{X} for simple, nicely aligned math pre-(super/sub)scripts
	% \Aboxed{...} for boxing full lines in 'align' environment
\usepackage{derivative} % provides \odv, \pdv, \odif, \pdif
\usepackage{bropd} % provides \br command which simplifies nesting of bracketed terms 
	% e.g. \br{\br{x-a}^2+\br{y-b}^2} produces \left[ \left( x-a \right)^2 + \left( y-b \right)^2 \right]
\usepackage{array} % improves array support, esp. in tabular env. 
	% see also xtab.sty
\usepackage{booktabs} % allows for improved spacing in tabular env. 
	% use \toprule, \*midrule, \bottomrule instead of \hline


%% Science and programming packages
\usepackage{fvextra} % for verbatim and comment environments with \Verb
\usepackage{chemmacros} % for writing chemical formulas with \ch, e.g. \ch{AgCl2-} or \ch{^{227}_{90}Th+}
	\usechemmodule{
		spectroscopy, % loads formula and siunitx modules, provides \NMR command.  
	    thermodynamics, % provides state variables and equations
    	units, % provides \[mM]olar, \Torr, \atm, \cal, \cmc, \MolMass
		} % also loads siunitx and chemformula
	\sisetup{% siunit package options
		per-mode = symbol,%
		inter-unit-product=\ensuremath{{}\!\cdot\!{}},%
		separate-uncertainty,%
		multi-part-units = single,%
		retain-explicit-plus,%
		list-final-separator={, and },%
		math-celsius = °\text{C}, % for temperatures
		text-celsius = °C,
		math-degree = °, % for angles
		text-degree = °,%
		input-digits = 0123456789 \pi \mitpi% necessary to use \pi in SI entries, affects rounding of digits.
		}%
	\DeclareSIUnit\ppm{ppm}
	\DeclareSIUnit\angstrom{\text{Å}} % Symbol doesn't exist in STIXTwoMath, need to force text font. 
	\DeclareSIUnit\wn{cm^{-1}}
	

%% misc packages

\usepackage{framed} % provides boxed 'framed' environment for easily boxing text 
\usepackage{tcolorbox}
	\tcbuselibrary{skins, breakable, xparse, minted}
	% provides fancier boxes than regular \makebox, \fbox, etc.
	% e.g. \doublebox, \ovalbox, \shadowbox
	% Can use `\tcbuselibrary{listings}` to use the listings library, 
		% doesn't require a language to be defined. 
\usepackage{empheq} % provides 'empheq' environment 
	% for improved control over shape, size, color of framed boxes, e.g. 
\newcommand{\boxedeq}[2]{
	\begin{empheq}[box={\fboxsep=6pt\fbox}]{align}\label{#1}#2\end{empheq}
}
\newcommand{\coloredeq}[2]{
	\begin{empheq}[box=\colorbox{lightgreen}]{align}\label{#1}#2\end{empheq}
}


%% document interface 

\usepackage{footnote} % 
\usepackage{hyperref} % adds hyperlinks and outline to PDF documents
	\hypersetup{%
		pdfencoding=auto,%
		psdextra,%
		pdfusetitle,%
		colorlinks=true,%
		linkcolor=BrickRed, %
		citecolor=Green, %
		filecolor=Mulberry, %
		urlcolor=NavyBlue, %
		menucolor=BrickRed, %
		runcolor=Mulberry, %
		linkbordercolor=BrickRed, %
		citebordercolor=Green, %
		filebordercolor=Mulberry, %
		urlbordercolor=NavyBlue, %
		menubordercolor=BrickRed, %
		runbordercolor=Mulberry %
		} %
	% options enable enhanced unicode and math support in PDF outlines [causes conflict with \C command?]
\usepackage{cleveref} % provides \cref command which inserts contextually correct word in front of ref.
	% e.g. \cref{eq:myeq} --> Equation 1.2, or so
\usepackage{bookmark} % improves package hyperref's bookmarking. 
	% properties such as style and color can be set. Generates bookmarks in first run. 


%% load later packages
\usepackage[textsize=footnotesize]{todonotes}



%%%%%%%%%%%%%%%%%%
%% COMMANDS
%%%%%%%%%%%%%%%%%%

\newcommand{\mtext}[1]{{\mathup{#1}}}
\DeclareMathOperator{\sgn}{sgn}
\DeclareMathOperator{\erf}{erf}
\DeclareMathOperator{\erfc}{erfc}
\DeclareMathOperator{\GammaFunc}{\symup{\Gamma}}
\DeclareMathOperator{\laplacian}{\nabla^2}
\DeclarePairedDelimiter\abs{\lvert}{\rvert}
\newcommand{\vb}[1]{\symbfit{#1}}

%% pre-defined colors
% standard: black, blue, brown, cyan, darkgray, gray, green, lightgray, lime, magenta, olive, orange, pink, purple, red, teal, violet, white, yellow
%
% dvips: Apricot, Aquamarine, Bittersweet, Black, Blue, BlueGreen, BlueViolet, BrickRed, Brown, BurntOrange, CadetBlue, CarnationPink, Cerulean, CornflowerBlue, Cyan, Dandelion, DarkOrchid, Emerald, ForestGreen, Fuchsia, Goldenrod, Gray, Green, GreenYellow, JungleGreen, Lavender, LimeGreen, Magenta, Mahogany, Maroon, Melon, MidnightBlue, Mulberry, NavyBlue, OliveGreen, Orange, OrangeRed, Orchid, Peach, Periwinkle, PineGreen, Plum, ProcessBlue, Purple, RawSienna, Red, RedOrange, RedViolet, Rhodamine, RoyalBlue, RoyalPurple, RubineRed, Salmon, SeaGreen, Sepia, SkyBlue, SpringGreen, Tan, TealBlue, Thistle, Turquoise, Violet, VioletRed, White, WildStrawberry, Yellow, YellowGreen, YellowOrange

\usepackage{pgfplots}
\pgfplotsset{compat=newest}
\usepgfplotslibrary{groupplots}
\usetikzlibrary{math}
\usetikzlibrary{calc}
\usepackage{siunitx,chemmacros}
\begin{document}

%% Create single x label for grouped plots  
\makeatletter
\pgfplotsset{
    groupplot xlabel/.initial={},
    every groupplot x label/.style={
        at={
            ($(
                {\pgfplots@group@name\space c1r\pgfplots@group@rows.west} |- 
                {\pgfplots@group@name\space c1r\pgfplots@group@rows.outer south}
            )!0.5!(
                {\pgfplots@group@name\space c\pgfplots@group@columns r\pgfplots@group@rows.east} |- 
                {\pgfplots@group@name\space c\pgfplots@group@columns r\pgfplots@group@rows.outer south}
            ) -(0.0,0.5)$)
            },
        anchor=north,
        align=center,
    },
    groupplot ylabel/.initial={},
    every groupplot y label/.style={
            rotate=90,
        at={
            ($(
                {\pgfplots@group@name\space c1r1.north} -|
                {\pgfplots@group@name\space c1r1.outer west}
            )!0.5!(
                {\pgfplots@group@name\space c1r\pgfplots@group@rows.south} -|
                {\pgfplots@group@name\space c1r\pgfplots@group@rows.outer west}
            )$)
        },
        anchor=south
    },
    execute at end groupplot/.code={%
      \node [/pgfplots/every groupplot x label]
        {\pgfkeysvalueof{/pgfplots/groupplot xlabel}};  
      \node [/pgfplots/every groupplot y label]
        {\pgfkeysvalueof{/pgfplots/groupplot ylabel}};
    }
}

\def\endpgfplots@environment@groupplot{%
    \endpgfplots@environment@opt%
    \pgfkeys{/pgfplots/execute at end groupplot}%
    \endgroup%
}
\makeatother

%% Create style for NMR spectra (color cycle, x-axis label, tick precision)
\pgfplotsset{
  nmr/.style={
  % Using a cycle list just altering colour means that there are no
  % marks: that is normal for this sort of plot.
  cycle list name=color list, 
  % Ensure that the x-axis values always have the same number of 
  % decimal places, to avoid e.g. "1" but "1.2".
  every x tick label/.append style={ 
    /pgf/number format/.cd,
     precision=2, 
     fixed,
     zerofill,
  },
  % The labels apply to all plots of this type.
  % Notice that in this case the zero is TMS, but that
  % will depend on the experiment.
  % xlabel = $\delta \, (\si{ppm})$,
  },
}

% Not everyone likes the 'axis box' effect which is the pgfplots default.
% Here, we'll set up to use 'Tufte-like' settings: see
% https://www.tug.org/members/TUGboat/tb34-2/tb107dugge.pdf for more on
% this.
% \makeatletter
\pgfplotsset{
  tufte axes/.style={
	% after end axis/.code={
    % \draw ({rel axis cs:0,0} -| {axis cs:\pgfplots@data@xmin,0})
    %   -- ({rel axis cs:0,0}  -| {axis cs:\pgfplots@data@xmax,0});
    % },
	axis x line = bottom,
  axis y line = none,
  every inner x axis line/.append style={-|},
  tick align = outside,
  tick pos   = left,
	ytick			 = \empty,
  x dir			 = reverse,
  }
}
% \makeatother

\pgfplotsset{
    width=0.95\textwidth,
    height=1.4\textwidth,
    scale only axis,}
    
\begin{tikzpicture}
  \begin{groupplot}[group style={
      group size=2 by 1,
      xticklabels at=edge bottom,
      horizontal sep=0pt,
      every plot/.style={nmr}
    },
    groupplot xlabel={
      $\delta \, (\si{ppm})$ \\ $\Leftarrow \nu, \, B_\textup{loc}$},
    ymin=-0.08, ymax=6.75,
  ]
  
  \nextgroupplot[
    xmin=9.67,xmax=9.87,
    % ymin=-0.08, ymax=1.8,
    axis x line   = bottom,
    axis y line   = none,
    every inner x axis line/.append style={-|},
    tick align    = outside,
    tick pos      = left,
    ytick     = \empty,
    x dir      = reverse,
    axis x discontinuity=crunch,
    width=0.55\textwidth,
  ]
    
    \tikzmath{
      \pkctr=9.789;
      \lvctr=2.5;
      \lvwidth=0.02/2;
      \lvscale=\pkctr;
      \jCHO=2.9/100.0;
      %
      \pkcta=\pkctr+(3/2)*\jCHO;
      \pkctb=\pkctr+(1/2)*\jCHO;
      \pkctc=\pkctr-(1/2)*\jCHO;
      \pkctd=\pkctr-(3/2)*\jCHO;
      %
      \lev1=\lvctr-1/2*(0.2)*\lvscale;
      \lev2=\lvctr+1/2*(0.2)*\lvscale;
      %
      \levag=\lvctr-1/2*(\pkcta-\pkctr+.2)*\lvscale;
      \levae=\lvctr+1/2*(\pkcta-\pkctr+.2)*\lvscale;
      \aleft=\pkcta+\lvwidth;
      \aright=\pkcta-\lvwidth;
      %
      \levbg=\lvctr-1/2*(\pkctb-\pkctr+.2)*\lvscale;
      \levbe=\lvctr+1/2*(\pkctb-\pkctr+.2)*\lvscale;
      \bleft=\pkctb+\lvwidth;
      \bright=\pkctb-\lvwidth;
      %
      \levcg=\lvctr-1/2*(\pkctc-\pkctr+.2)*\lvscale;
      \levce=\lvctr+1/2*(\pkctc-\pkctr+.2)*\lvscale;
      \cleft=\pkctc+\lvwidth;
      \cright=\pkctc-\lvwidth;
      %
      \levdg=\lvctr-1/2*(\pkctd-\pkctr+.2)*\lvscale;
      \levde=\lvctr+1/2*(\pkctd-\pkctr+.2)*\lvscale;
      \dleft=\pkctd+\lvwidth;
      \dright=\pkctd-\lvwidth;
    }
    
    {\addplot table from {Figures/CH3CHO_nmr_simsc2.dat};
      \node (fragment) at (\pkctr,5.0) {\ch{-CHO}};
      \node[align=center,left=-4pt] (CHO) at (\aleft,4.5) [red]{\ch{CHO}\\spin};
      \node[align=center] (CH3) at (\pkctr,4.5) [blue]{\ch{CH3}\\spins};
      \coordinate (01) at (\dright,\lev1);
      \coordinate (02) at (\aleft,\lev1);
      \coordinate (11) at (\dright,\lev2);
      \coordinate (12) at (\aleft,\lev2);
      \coordinate (Ag1) at (\aright,\levag);
      \coordinate (Ag2) at (\aleft,\levag);
      \coordinate (Ag3) at (\pkcta, \levag);
      \coordinate (Ae1) at (\aright,\levae);
      \coordinate (Ae2) at (\aleft,\levae);
      \coordinate (Ae3) at (\pkcta, \levae);
      \coordinate (Bg1) at (\bright,\levbg);
      \coordinate (Bg2) at (\bleft,\levbg);
      \coordinate (Bg3) at (\pkctb, \levbg);
      \coordinate (Be1) at (\bright,\levbe);
      \coordinate (Be2) at (\bleft,\levbe);
      \coordinate (Be3) at (\pkctb, \levbe);
      \coordinate (Cg1) at (\cright,\levcg);
      \coordinate (Cg2) at (\cleft,\levcg);
      \coordinate (Cg3) at (\pkctc, \levcg);
      \coordinate (Ce1) at (\cright,\levce);
      \coordinate (Ce2) at (\cleft,\levce);
      \coordinate (Ce3) at (\pkctc, \levce);
      \coordinate (Dg1) at (\dright,\levdg);
      \coordinate (Dg2) at (\dleft,\levdg);
      \coordinate (Dg3) at (\pkctd, \levdg);
      \coordinate (De1) at (\dright,\levde);
      \coordinate (De2) at (\dleft,\levde);
      \coordinate (De3) at (\pkctd, \levde);
    };
    \begin{scope}[thick] 
      \draw[thin,dashed] (01) -- (02) node[left=1pt] 
        [red]{$\uparrow$};
      \draw[thin,dashed] (11) -- (12) node[left=1pt] 
        [red]{$\downarrow$};     
      \draw (Ag1) -- (Ag2);
      \draw (Ae1) -- (Ae2);
      \draw[-Stealth] (Ag3) -- (Ae3) node[above] 
        [blue]{$\downarrow\downarrow\downarrow$};
      \draw (Bg1) -- (Bg2);
      \draw (Be1) -- (Be2);
      \draw[-Stealth] (Bg3) -- (Be3) node[above] 
        [align=center,blue]{ 
          $\downarrow\downarrow\uparrow$ \\
          $\downarrow\uparrow\downarrow$ \\
          $\uparrow\downarrow\downarrow$};
      \draw (Cg1) -- (Cg2);
      \draw (Ce1) -- (Ce2);
      \draw[-Stealth] (Cg3) -- (Ce3) node[above] 
        [align=center,blue]{ 
          $\downarrow\uparrow\uparrow$ \\
          $\uparrow\downarrow\uparrow$ \\
          $\uparrow\uparrow\downarrow$};
      \draw (Dg1) -- (Dg2);
      \draw (De1) -- (De2);
      \draw[-Stealth] (Dg3) -- (De3) node[above] 
        [blue]{$\uparrow\uparrow\uparrow$};
    \end{scope}

  \nextgroupplot[
    xmin=2.15,xmax=2.26,
    % ymin=-0.08, ymax=1.8,
  	axis x line = bottom,
	  axis y line = none,
    x axis line style={|-},
    tick align      = outside,
    tick pos        = left,
  	ytick						= \empty,
	  x dir						= reverse,
    width=0.40\textwidth,]
    
    \tikzmath{
      \pkctr=2.206;
      \lvctr=2.5;
      \lvwidth=0.02/2;
      \lvscale=\pkctr;
      \jCHO=2.9/100.0;
      %
      \pkcta=\pkctr+(1/2)*\jCHO;
      \pkctb=\pkctr-(1/2)*\jCHO;
      %
      \lev1=\lvctr-3/2*(.25)*\lvscale;
      \lev2=\lvctr-1/2*(.25)*\lvscale;
      \lev3=\lvctr+1/2*(.25)*\lvscale;
      \lev4=\lvctr+3/2*(.25)*\lvscale;
      %
      \leva1=\lvctr-3/2*(\pkcta-\pkctr+.25)*\lvscale;
      \leva2=\lvctr-1/2*(\pkcta-\pkctr+.25)*\lvscale;
      \leva3=\lvctr+1/2*(\pkcta-\pkctr+.25)*\lvscale;
      \leva4=\lvctr+3/2*(\pkcta-\pkctr+.25)*\lvscale;
      \aleft=\pkcta+\lvwidth;
      \aright=\pkcta-\lvwidth;
      %
      \levb1=\lvctr-3/2*(\pkctb-\pkctr+.25)*\lvscale;
      \levb2=\lvctr-1/2*(\pkctb-\pkctr+.25)*\lvscale;
      \levb3=\lvctr+1/2*(\pkctb-\pkctr+.25)*\lvscale;
      \levb4=\lvctr+3/2*(\pkctb-\pkctr+.25)*\lvscale;
      \bleft=\pkctb+\lvwidth;
      \bright=\pkctb-\lvwidth;
    }
    
    {\addplot table from {Figures/CH3CHO_nmr_simsc1.dat};
      \node (fragment) at (\pkctr,5.0) {\ch{-CH3}};
      \node[align=center,left=-2pt] (CH3) at (\aleft,4.0) 
        [red]{\ch{CH3}\\spins};
      \node[align=center] (CHO) at (\pkctr,4.0) [blue]{\ch{CHO}\\spin};
      \coordinate (011) at (\bright,\lev1);
      \coordinate (012) at (\aleft,\lev1);
      \coordinate (021) at (\bright,\lev2);
      \coordinate (022) at (\aleft,\lev2);
      \coordinate (031) at (\bright,\lev3);
      \coordinate (032) at (\aleft,\lev3);
      \coordinate (041) at (\bright,\lev4);
      \coordinate (042) at (\aleft,\lev4);
      \coordinate (A11) at (\aright,\leva1);
      \coordinate (A12) at (\aleft,\leva1);
      \coordinate (A13) at (\pkcta, \leva1);
      \coordinate (A21) at (\aright,\leva2);
      \coordinate (A22) at (\aleft,\leva2);
      \coordinate (A23) at (\pkcta, \leva2);
      \coordinate (A31) at (\aright,\leva3);
      \coordinate (A32) at (\aleft,\leva3);
      \coordinate (A33) at (\pkcta, \leva3);
      \coordinate (A41) at (\aright,\leva4);
      \coordinate (A42) at (\aleft,\leva4);
      \coordinate (A43) at (\pkcta, \leva4);
      \coordinate (B11) at (\bright,\levb1);
      \coordinate (B12) at (\bleft,\levb1);
      \coordinate (B13) at (\pkctb, \levb1);
      \coordinate (B21) at (\bright,\levb2);
      \coordinate (B22) at (\bleft,\levb2);
      \coordinate (B23) at (\pkctb, \levb2);
      \coordinate (B31) at (\bright,\levb3);
      \coordinate (B32) at (\bleft,\levb3);
      \coordinate (B33) at (\pkctb, \levb3);
      \coordinate (B41) at (\bright,\levb4);
      \coordinate (B42) at (\bleft,\levb4);
      \coordinate (B43) at (\pkctb, \levb4);
    };
    \begin{scope}[thick] 
      \draw[dashed,thin] (011) -- (012) node[left=1pt] 
        [red]{$\uparrow\uparrow\uparrow$};
      \draw[dashed,thin] (021) -- (022) node[left=1pt] 
        [red,align=center]{
          $\downarrow\downarrow\uparrow$ \\
          $\downarrow\uparrow\downarrow$ \\
          $\uparrow\downarrow\downarrow$};
      \draw[dashed,thin] (031) -- (032) node[left=1pt] 
        [red,align=center]{
          $\downarrow\uparrow\uparrow$ \\
          $\uparrow\downarrow\uparrow$ \\
          $\uparrow\uparrow\downarrow$};
      \draw[dashed,thin] (041) -- (042) node[left=1pt] 
        [red]{$\downarrow\downarrow\downarrow$};
      \draw (A11) -- (A12);
      \draw (A21) -- (A22);
      \draw (A31) -- (A32);
      \draw (A41) -- (A42);
      \draw[-Stealth] (A13) -- (A23);
      \draw[-Stealth] (A23) -- (A33);
      \draw[-Stealth] (A33) -- (A43) node[above] 
        [blue]{$\downarrow$};
      \draw (B11) -- (B12);
      \draw (B21) -- (B22);
      \draw (B31) -- (B32);
      \draw (B41) -- (B42);
      \draw[-Stealth] (B13) -- (B23);
      \draw[-Stealth] (B23) -- (B33);
      \draw[-Stealth] (B33) -- (B43) node[above] 
        [blue]{$\uparrow$};
    \end{scope}
    
  \end{groupplot}
\end{tikzpicture}
	
\end{document}
