\documentclass[nobib,nofonts,nols,nohyper]{tufte-handout}
% \RequirePackage{minted}
%% Header file cloned from https://github.com/wickles/latex-base

%%%%%%%%%%%%%%%%%%
%% CONTENTS
%%%%%%%%%%%%%%%%%%
% To-do / issues
% Packages
% Commands
% Special Symbols
% Environments
% More commands: Resizable delimiters
% More commands: Derivatives
% Useful templates
% Notes
%%%%%%%%%%%%%%%%%%
%%%%%%%%%%%%%%%%%%



%%%%%%%%%%%%%%%%%%
%% TO-DO / ISSUES
%%%%%%%%%%%%%%%%%%

%% packages
% Replace physics package with alternatives
% - Physics replacements
% 	- braket
% 	- derivative (need tectonic to support TeXLive 2021)
% 	- vectors?
% Update commands for siunitx (requires TeXLive 2021)




%%%%%%%%%%%%%%%%%%
%% PACKAGES
%%%%%%%%%%%%%%%%%%


%% debugging / diagnostics
\RequirePackage[l2tabu,orthodox]{nag} % nags user about obsolete and improper syntax

\usepackage{xparse} % provides high-level interface for producing document-level commands
	% via \[Declare/New/Renew/Provide/etc]DocumentCommand
	% allows for more than one optional argument in commands

\usepackage{iftex,ifluatex,ifxetex,ifdraft} % check if a document is being processed with pdfTeX, or XeTEX, or LuaTEX
\newif\ifxetexorluatex % a new conditional starts as false, true if using XeTeX OR LuaTeX
\ifnum 0\ifxetex 1\fi\ifluatex 1\fi>0
   \xetexorluatextrue
\fi

%% fonts and encoding 

% standard and structural packages

% \newcommand\bmmax{4} % increase max number of bm font allocations. default 4?
% \newcommand\hmmax{1} % increase max number of hm font allocations. default 3?
\usepackage{bm} % provides \bm command for robustly bolding math characters

%%%%%%%%%%%%%%%%%%%%%% Additional math fonts %%%%%%%%%%%%%%%%%%%%%%%%%%%%%%%%%%%
\usepackage{amssymb} % for \mathbb (upper case only), \mathfrak fonts
%%%%%%%%%%%%%%%%%%%%%%%%%%%%%%%%%%%%%%%%%%%%%%%%%%%%%%%%%%%%%%%%%%%%%%%%%%%%%%%%%

\usepackage{microtype} % improves kerning in certain cases. 
	% recommended to disable protrusion in table of contents!

%% Latex interface 

\usepackage{letltxmacro} % provides \LetLtxMacro command for correct renaming of commands
\usepackage{etoolbox} % provides many useful programming tools, 
	% e.g. \ifdefempty{cs}{true}{false}



%% media interface

\usepackage{graphicx} % support the \includegraphics command and options
\usepackage{subfig} % Support for subfigures and subcaptions
	\captionsetup[subfloat]{position=bottom}
% \makeatletter%
% \@ifclassloaded{tufte-handout}% xcolor is pre-loaded with the tufte-latex package
%   {
	% \ifxetexorluatex % if lua- or xelatex http://tex.stackexchange.com/a/140164/1913
		\newcommand{\textools}[2][5]{%
			\begingroup\addfontfeatures{LetterSpace=#1}#2\endgroup
		}
		\renewcommand{\allcapsspacing}[1]{\textools[15]{#1}}
		\renewcommand{\smallcapsspacing}[1]{\textools[10]{#1}}
		\renewcommand{\allcaps}[1]{\textools[15]{\MakeTextUppercase{#1}}}
		\renewcommand{\smallcaps}[1]{\smallcapsspacing{\scshape\MakeTextLowercase{#1}}}
		\renewcommand{\textsc}[1]{\smallcapsspacing{\textsmallcaps{#1}}}
% 	\else
% 	\fi
% }%
%   {\usepackage[dvipsnames,svgnames,table,hyperref]{xcolor}
% 	% provides access to large number of colors and related features
% 	% see end notes for lists of available colors
% 	}%
% \makeatother%
\usepackage{svg} % provides \includesvg command for svg figures. 
\usepackage{pgfplots} % for plotting in tikzpicture environment
	\pgfplotsset{compat=1.16} % required to select newest version
\usepackage{tikzscale} % allows \includegraphics{*.tikz} and scaling of TiKZ images


%% math interface

\usepackage{amsmath} % for nice math commands and environments
\usepackage{mathtools} % extends amsmath with bug fixes and useful commands, e.g.
	% \shortintertext for short interjections in align environment,
	% \prescript{t}{b}{X} for simple, nicely aligned math pre-(super/sub)scripts
	% \Aboxed{...} for boxing full lines in 'align' environment
\usepackage{array} % improves array support, esp. in tabular env. 
	% see also xtab.sty
\usepackage{booktabs} % allows for improved spacing in tabular env. 
	% use \toprule, \*midrule, \bottomrule instead of \hline
	% see also ctable.sty

\usepackage{derivative} % provides \odv, \pdv, \odif, \pdif
\usepackage{bropd} % provides \br command which simplifies nesting of bracketed terms 
	% e.g. \br{\br{x-a}^2+\br{y-b}^2} produces \left[ \left( x-a \right)^2 + \left( y-b \right)^2 \right]


%% Science and programming packages
\usepackage{fvextra} % for verbatim and comment environments with \Verb
\usepackage{chemmacros} % for writing chemical formulas with \ch, e.g. \ch{AgCl2-} or \ch{^{227}_{90}Th+}
	\usechemmodule{
		spectroscopy, % provides 
    thermodynamics, % provides state variables and equations
    units, % provides \[mM]olar, \Torr, \atm, \cal, \cmc, \MolMass
		} % also loads siunitx and chemformula
	\DeclareSIUnit\ppm{ppm}
  \sisetup{% siunit package options
			per-mode = symbol,%
			inter-unit-product=\ensuremath{{}\!\cdot\!{}},%
			separate-uncertainty,%
			multi-part-units = single,%
			retain-explicit-plus,%
			list-final-separator={, and },%
			math-celsius = °\text{C}, % for temperatures
			text-celsius = °C,
			math-degree = °, % for angles
			text-degree = °}%


\usepackage{physics} % provides streamlined interface with many commands to simplify
	% writing standard physics notation (bra-kets, derivatives, etc.)
	% differentials and derivatives: \dd[n]{x}, \dv[n]{f}{x}, \dv{x}, \pdv{f}{x}{y}, \var{Q}, \fdv{F}{g}
	% bra-ket notation: \bra, \ket, \braket, \dyad, \matrixel 
	% \qty(x) for delimited quantities
	% \abs, \norm, \eval, \order ,\comm, \acomm
	% vectors: \vb, \va, vu, \vdot, \cross, \grad, \div, \curl, \laplacian
	% largely replaces and adds basic math functions with auto-delimiters: trig, linear algebra, etc.
		% no inverse hyperbolic trig? 
	% in particular adds \tr, \Tr, \rank, \erf, \Res, \pv / \PV, \Re, \Im
	% auto padding text: \qq{string}
	% matrix quantities: \mqty(a & b \\ c & d) or \mqty[x \\ y], Pauli \pmat{n}, diagonal matrices \mqty(\dmat{1,2&3\\4&5}), anti-diagonal \admat

%% misc packages

\usepackage{datetime} % allows easy formatting of dates, e.g. \formatdate{dd}{mm}{yyyy}
%\usepackage[inline]{enumitem} % allows for custom labels on enumerated lists
	% e.g. \begin{enumerate}[label=\textbf{(\alph*)}]
	% label options are: \alph, \Alph, \arabic, \roman, and \Roman
	% inline option creates '*' versions of enumerate, itemize, description 
		% which can be inlined within the text of a paragraph. 
% \usepackage{outlines} % provides 'outline[style]' environment, allowing for subitems in lists
	% e.g. \begin{outline} \1 item \2 subitem \3[A)] subsubitem \1 item \end{outline}
	% or with other style: \begin{outline}[enumerate], etc

% \usepackage{rotating}
	% provides environments for rotating arbitrary objects, e.g. sideways, turn[ang], rotate[ang]
	% also provides macro \turnbox{ang}{stuff}
%\usepackage{ctable} % allows for footnotes under table and properly spaced caption above 
	% must be loaded after tikz
	% incorporates (..?)
\usepackage{framed} % provides boxed 'framed' environment for easily boxing text 
\usepackage{tcolorbox}
	\tcbuselibrary{skins, breakable, xparse, minted}
	% provides fancier boxes than regular \makebox, \fbox, etc.
	% e.g. \doublebox, \ovalbox, \shadowbox
	% Can use `\tcbuselibrary{listings}` to use the listings library, 
		% doesn't require a language to be defined. 
\usepackage{empheq} % provides 'empheq' environment 
	% for improved control over shape, size, color of framed boxes, e.g. 
\newcommand{\boxedeq}[2]{
	\begin{empheq}[box={\fboxsep=6pt\fbox}]{align}\label{#1}#2\end{empheq}
}
\newcommand{\coloredeq}[2]{
	\begin{empheq}[box=\colorbox{lightgreen}]{align}\label{#1}#2\end{empheq}
}


%% document interface 

\usepackage{footnote} % 
\usepackage{hyperref} % adds hyperlinks and outline to PDF documents
	\hypersetup{%
		pdfencoding=auto,%
		psdextra,%
		pdfusetitle,%
		colorlinks=true,%
		linkcolor=BrickRed, %
		citecolor=Green, %
		filecolor=Mulberry, %
		urlcolor=NavyBlue, %
		menucolor=BrickRed, %
		runcolor=Mulberry, %
		linkbordercolor=BrickRed, %
		citebordercolor=Green, %
		filebordercolor=Mulberry, %
		urlbordercolor=NavyBlue, %
		menubordercolor=BrickRed, %
		runbordercolor=Mulberry %
		} %
	% options enable enhanced unicode and math support in PDF outlines [causes conflict with \C command?]
\usepackage{cleveref} % provides \cref command which inserts contextually correct word in front of ref.
	% e.g. \cref{eq:myeq} --> Equation 1.2, or so
\usepackage{bookmark} % improves package hyperref's bookmarking. 
	% properties such as style and color can be set. Generates bookmarks in first run. 

%% font packages -- load fontenc, then inputenc, then lmodern. 
% see http://tex.stackexchange.com/a/44699
\usepackage{fontspec}
\usepackage[math-style=ISO]{unicode-math} 
\ifdraft{}{
	\setmainfont{STIX2Text}[
		Extension={.otf},
		UprightFont={*-Regular},
		BoldFont={*-Bold},
		ItalicFont={*-Italic},
		BoldItalicFont={*-BoldItalic},]
	\setmathfont{STIX2Math}[
		Extension={.otf}]
}
	\renewcommand{\vb}[1]{\symbf{#1}} % fix vector command from physics package to work with unicode-math


%% load later packages
\usepackage{lineno} % provides line numbers in main text for reference and peer review
	% activated by calling \linenumbers in document
\usepackage[textsize=footnotesize]{todonotes}





%%%%%%%%%%%%%%%%%%
%% COMMANDS
%%%%%%%%%%%%%%%%%%

% \newcommand{\mtext}[1]{{\textnormal{#1}}} % for writing text within math mode, e.g. for subscripts
\LetLtxMacro{\mtext}{\text} % legacy alias for \mtext
% \LetLtxMacro{\opname}{\operatorname} % custom operator names
% %\newcommand{\tr}{\opname{tr}} % for trace
% %\newcommand{\rank}{\opname{rank}} % for rank
% \newcommand{\diag}{\opname{diag}} % i.e. \diag(\lambda_1, \dots, \lambda_n)
% \LetLtxMacro{\fancyRe}{\real} % already renamed from physics.sty
% \LetLtxMacro{\fancyIm}{\imaginary} % see above
% \renewcommand{\Re}{\opname{Re}}
% \renewcommand{\Im}{\opname{Im}}
% \renewcommand{\Res}{\opname*{Res}} % for residue function (handles limits properly)
% \newcommand{\inv}{^{-1}}
% \newcommand{\sgn}{\opname{sgn}} % sign/signum function
\DeclareMathOperator{\sgn}{sgn}
\DeclareMathOperator{\erfc}{erfc}
\DeclareMathOperator{\GammaFunc}{\symup{\Gamma}}
\newcommand{\iu}{\TextOrMath{$\mtext{i}$}{\mtext{i}\mkern1mu}}
%
% \newcommand{\laplacian}{\nabla^2}

%% pre-defined colors
% standard: black, blue, brown, cyan, darkgray, gray, green, lightgray, lime, magenta, olive, orange, pink, purple, red, teal, violet, white, yellow
%
% dvips: Apricot, Aquamarine, Bittersweet, Black, Blue, BlueGreen, BlueViolet, BrickRed, Brown, BurntOrange, CadetBlue, CarnationPink, Cerulean, CornflowerBlue, Cyan, Dandelion, DarkOrchid, Emerald, ForestGreen, Fuchsia, Goldenrod, Gray, Green, GreenYellow, JungleGreen, Lavender, LimeGreen, Magenta, Mahogany, Maroon, Melon, MidnightBlue, Mulberry, NavyBlue, OliveGreen, Orange, OrangeRed, Orchid, Peach, Periwinkle, PineGreen, Plum, ProcessBlue, Purple, RawSienna, Red, RedOrange, RedViolet, Rhodamine, RoyalBlue, RoyalPurple, RubineRed, Salmon, SeaGreen, Sepia, SkyBlue, SpringGreen, Tan, TealBlue, Thistle, Turquoise, Violet, VioletRed, White, WildStrawberry, Yellow, YellowGreen, YellowOrange

\usepackage[mackeys=text]{menukeys}
\tcbuselibrary{minted}

%%%%%%% Bibliography options %%%%%%%%%%%%%%%%%%%%%%%%%%%%%%%%%%%%%%%%%%%%%%%%%%
% (fold)
\usepackage[%
	style=numeric-comp,%
  sortcites,%
  sorting=none,%
  defernumbers=true,%
  hyperref,
  backend=biber,
	]{biblatex}
	\addbibresource{../pchem_bib.bib}

	% Filter bibliography file to only include entries matching keyword.
	\DeclareSourcemap{%
	  \maps[datatype=bibtex]{%
	    \map{%
	      \step[nocited, final]%
	      \step[fieldsource=keywords, notmatch=plot, final]%
	      \step[entrynull]%
	    }%
	  }%
	}

	% Add a section for cited bibliography entries.
  % At the end, create a second bibliography for uncited items (Further Reading).
	\DeclareBibliographyCategory{cited}
	\AtEveryCitekey{\addtocategory{cited}{\thefield{entrykey}}}

  % Define a style for the Further Reading section
  \defbibenvironment{nolabelbib}
    {\list
       {}
       {\setlength{\leftmargin}{2\bibhang}%
        \setlength{\itemindent}{-\bibhang}%
        \setlength{\itemsep}{\bibitemsep}%
        \setlength{\parsep}{\bibparsep}}}
    {\endlist}
    {\item}

  % Remove URL from citations if DOI is present.
	\AtEveryBibitem{%
	  \iffieldundef{doi}{ % do nothing if true
		}
		{ % otherwise, clear the URL
			\clearfield{url}
		}%
	}
% (end)
%%%%%%%%%%%%%%%%%%%%%%%%%%%%%%%%%%%%%%%%%%%%%%%%%%%%%%%%%%%%%%%%%%%%%%%%%%%%%%%

%%%%%% Listing settings %%%%%%%%%%%%%%%%%%%%%%%%%%%%%%%%%%%%%%%%%%%%%%%%%%%%%%%
\tcbset{
  colback=blue!4!white,
  colframe=blue!85!black,
  listing only,
  left=1.5em,
  enhanced jigsaw,
  breakable,
  overlay={
    \begin{tcbclipinterior}
      \fill[red!20!blue!20!white]
      (frame.south west)  rectangle
      ([xshift=1.5em]frame.north west);
    \end{tcbclipinterior}}
}
\setminted{
  style=tango,
  fontsize=\small,
  breaklines,
  breakafter=|,
  autogobble,
  linenos,
  numberfirstline=true,
  firstnumber=1,
  stepnumber=2,
  numbersep=1em
}

\newmintinline[pyinline]{python}{python3}

\newtcblisting{pyinput}[1][]{
  listing engine=minted,
  minted language=python,
  listing only,
  #1
}

%%%%%%%%%%%%%%%%%%%%%%%%%%%%%%%%%%%%%%%%%%%%%%%%%%%%%%%%%%%%%%%%%%%%%%%%%%%%%%%

\title{Introduction to Data Analysis and Plotting \\with Jupyter}

\author{Dustin Wheeler}

\begin{document}

\maketitle

\begin{abstract}
	\noindent
	The goal of this exercise is to familiarize the student with basic data manipulation and plotting using Jupyter notebooks.
	Many of the concepts are general and can be translated to other software applications.
	By the end of this exercise, you should be able to
	\begin{itemize}
		\item perform basic manipulations on data columns (formatting, transformations)
		\item perform simple mathematical operations over a set of data
		\item plot columns of data in a clear and visually pleasing manner
		\item perform simple statistical operations on a set of data
		\item get a line-of-best-fit for multiple datasets
		\item import data from plain text files (CSV, TSV, DAT, etc.), like those created by many spectrometers and logging instruments
	\end{itemize}
\end{abstract}

\section{Introduction} % (fold)
\label{sec:intro}

Why do we plot data? Edward Tufte may have said it best:

\begin{quote}
	At their best, graphics are instruments for reasoning about quantitative information.
	Often the most effective way to describe, explore, and summarize a set of numbers---even a very large set---is to look at pictures of those numbers.
\end{quote}
% section intro (end)

\section{A Little Bit About Plots} % (fold)
\label{sec:a-little-bit-about-plots}

Plots (also called graphs or charts, though these are not \emph{quite} the same thing) come in a variety of styles.
The can be used to explore data as your are performing analyses or to convey results to a reader as part of a report.
There are a variety of ``standard'' types which should cover your needs in this course.
The main types we will use are:

\begin{description}
	\item[Scatter plot] Used to show correlation between two variables, each point generally represents an \textbf{independent} measurement.

	Lines are \textbf{not} used to connect points, lines are instead used to represent a mathematical model fitting the data.

	\item[Line graph] Indicates continuity in the \( x \) dimension, often suggests a single collection session for all data on the ``line'' (e.g., a UV-Vis or NMR spectrum).
	Individual points are generally not shown, though the data generally consists
  of individual collection points.
	Lines should never be smoothed (or ``splined''), as this treatment no longer accurately represents the data.

	Really, a line graph is just a special case of a scatter plot where the
  independent variable is continuous (like a fluorescence or NMR spectrum).

	\item[Histogram] Represents the distribution of data in a series of ``bins'', shows the frequency of a measured event.
	Closely related to density plots, in the same way scatter plots are related to
  line plots.
\end{description}

Here are the important parts of a plot, along with a brief description and guidelines for formatting.

\begin{description}
	\item[Title] Should be descriptive and represent \emph{all} data in the plot.
	\item[Axes] Should be clearly labeled (category, units, tick numbers).
	Should have an appropriate range (from slightly less than your minimum value to slightly more than your maximum value).
	\item[Fit line] Should cover the whole spread of the data, but should not predict beyond the data range (you don't know anything about measurements in that region).
	\item[Legend] Should clearly show which set of data is which, trend lines should clearly correlate to a given data set.
\end{description}

% section a-little-bit-about-plots (end)

\section{Basic Plotting and Linear Fits} % (fold)
\label{basic-plotting-and-linear-fits}

Your first task is to take the data in the file \texttt{data.txt} and import it into a new Jupyter notebook.\sidenote{
Note: prior to performing the following exercises, you should run through the \href{https://jupyter-notebook.readthedocs.io/en/stable/notebook.html}{introductory documentation to Jupyter} to familiarize yourself with the concepts and interface.}

To get started with this task, please use this link to clone the files into your Jupyter directory:\\ \href{http:///sugarcube.hunter.cuny.edu/hub/user-redirect/git-pull?repo=https%3A%2F%2Fgithub.com%2Fmskblackbelt%2F357_plot_intro&urlpath=lab%2Ftree%2F357_plot_intro&branch=main}{\texttt{https://tinyurl.com/chem357-plotting}}\\
This link will direct your browser to the class JupyterHub and download a folder containing a number of resources you'll need for this exercise.

The folder \Verb{raw_data} contains a few datasets in various formats.
The file extension doesn't always tell you what type of information is in a file and can sometimes
lead you astray.
For this exercise, the file \Verb{data.txt} contains a set of ``tab-delimited'' data, meaning that the values in a row are separated by tab characters.
Another common format is ``comma-delimited'' or ``comma-separated'' data, such as the dataset in \Verb{anscombe.csv}.
You will find that different instruments have different defaults for exporting data, but these two are the most common.
Sometimes, a \Verb{.dat} extension might be used for a plain-text file to indicate `data' without indicating the underlying format or delimiter type.

Make sure that you are located in the top folder for this lesson (i.e., inside
of ``357\_plot\_intro'') and open a new Jupyter template by
\begin{enumerate}
  \item clicking on the ``New Launcher'' button at the top of the file browser
    panel, or
  \item Selecting ``File > New Launcher'',
  \end{enumerate}
then clicking on the Template icon, select ``pchem\_templates'' in the top box
and ``pchem\_report.ipynb'' in the second box.
You can also create a file from a new template directly by clicking on ``File >
New > Template''.
These methods are all equivalent, pick whichever you're most
comfortable using.
This will open a new Jupyter notebook with a series of cells pre-populated with information to get you started.
Give this notebook a title (as instructed in the first cell), then click on ``File
> Rename Notebook…'' and give the notebook a title following the instructions in the first cell.

In the notebook, execute the code cell under the ``Library Import'' section.
``Libraries'' are just collections of useful Python code that individuals or groups have created.
They generally cover a specific area or hold to a theme.
In our notebook template, we use the following:
\begin{description}
  \item[\href{https://docs.python.org/3/library/pathlib.html}{pathlib}] A library
  for searching and working with paths on multiple filesystems.
  \item[\href{https://docs.python.org/3/library/shutil.html}{shutil}] A library
    for high-level file manipulation.
    We will mostly use it for working with ZIP archives.
  \item[\href{https://docs.python.org/3/library/re.html}{re}] A library that
    allows for regular expression use in Python, nearly the same as those used
    in the Perl programming language.
    Regular expressions allow for complex string matching, enabling the
    extraction of very specific text from large files.
  \item[\href{http://numpy.org/}{Numpy}] A general purpose numerical library
    that speeds up number based operations, allows functions to operate on lists
    of numbers, and gives a large collection of useful mathematical functions
    (things like polynomial fitting, matrix operations, and common algebraic and
    trigonometric functions).
    The core functionality is based around Numpy arrays, very fast \emph{n}-dimensional arrays of data (a fundamental requirement for scientific computation).
  \item[\href{https://scipy.org/scipylib/index.html}{Scipy}] A fundamental library for scientific computing, builds on the Numpy library but includes numeric integration routines, non-linear curve fitting routines, interpolation, numerical optimization, and statistics.
  \item[\href{http://pandas.pydata.org/}{Pandas}] A data management library that offers functions for importing, tidying, organizing, and filtering data using the DataFrame construct. Excellent resource for two-dimensional data, starts breaking down with more dimensions (look at the \href{http://xarray.pydata.org/}{xarray} package, which is built on top of Pandas, for this purpose).
  \item[\href{http://matplotlib.org/}{Matplotlib}] A comprehensive library for creating static, animated, and interactive visualizations in Python. It has become the de-facto standard library for plot visualizations in Python.
  \item[\href{https://seaborn.pydata.org}{Seaborn}] A collection of routines and formats built on top of Matplotlib that provides a high-level interface for drawing attractive and informative statistical graphics.
\end{description}

Now, create a new cell  under the Data import section.
Import the data from \Verb{data.txt} by using the import function of the ``Pandas'' (\Verb{pd}) library:

\begin{pyinput}
data = pd.read_csv("raw_data/data.txt", sep="\t")
\end{pyinput}

This will take the contents of \Verb{data.txt} and turn it into an object called a ``DataFrame'' with the name \pyinline{data}.\sidenote[][-3\baselineskip]{For more information, type \pyinline{help(pd.DataFrame)} to get more information on the DataFrame object. Alternately, in the Jupyter notebook, you can press \keys{Shift + Tab} at the end of a command to get help on the object (i.e., when your cursor is at the end of \pyinline{pd.DataFrame} or your new object \pyinline{data}.)}
The \pyinline{sep="\t"} argument is required because our data is ``tab-separated'', rather than ``comma-separated'', which is the default (as suggested by the function name \Verb{read_csv}).
When in doubt, specify the field separator when you import data with Python and the Pandas library.

To view your new object, just type \pyinline{data} on a new line and press \keys{Shift + Enter} to evaluate the cell.
By default code cells will display the results of the final line of the cell (if there is anything to display).\sidenote{They only display the results of intermediate lines of a cell if the \pyinline{print()} command is used.}
You should end up with two columns of data with 21 data points in each column.
Once you have the data columns, create a new scatter plot from the data.
While the Pandas library automatically incorporates basic plotting functions from the Matplotlib library, customizing the plot can often be difficult.\sidenote{
	\emph{n.b.:}   Pandas is designed first as a tool for finance and business  operations.
	The default settings for plots are often inappropriate for  scientific plots so some tweaking may be necessary.}
It works well for quick plots, but we'll generally be better off using Matplotlib directly.
To quickly plot your data, you only need to type \pyinline{plt.plot('X',
  'Y', data=data)}.
While you are exploring data, the default appearance is fine, but you'll probably want to start customizing the appearance of plots when you get them ready for publication.

Adding data to the plot function is done in the following manner:
\begin{pyinput}
  plt.plot(x1, y1, ['fmt1'], x2, y2, ['fmt2'], ..., data=data-source)
\end{pyinput}
The variables \Verb{x1}, \Verb{y1}, etc. represent the \(x\) and \(y\) data you want to plot, while the string \Verb{'fmt#'} is a format string for the plot data, composed of a letter representing the color of the data and a character to represent the symbol or line.
If you're using labeled data, such as columns in a dataframe, you can use the label strings as your \Verb{x1} and \Verb{y1} information.
The default format is \Verb{'b-'}, which is a solid blue line.
The square brackets arount \Verb{'fmt#'} indicates that the parameter is optional.
In addition to the parameter being optional, each part is also optional, so \Verb{':'} would keep the default color and change the line style to a dotted line, while \Verb{'r'} would create a red solid line.
Adding information like title, axes labels, etc. can be done by using the appropriate command.\sidenote{
  If any of this confuses you, or you'd like more examples, see the \href{https://matplotlib.org/stable/tutorials/introductory/pyplot.html}{Pyplot Tutorial} on the Matplotlib website.
}
Think of a plot as an object exclusive to its own cell (\emph{i.e.,} it has an implicit identity like ``plot-742'').
Any calls to \pyinline{plt} within that cell will operate on that particular plot object.
So using functions like \pyinline{plt.title()} or \pyinline{plt.xlabel()} will change the title and \(x\)-axis label of that particular plot.
There are a slew of methods available to \Verb{matplotlib.pyplot} and you can find out more information on the \href{https://matplotlib.org/stable/api/_as_gen/matplotlib.pyplot.html}{associated help page}.

With this in mind, create a scatter plot (symbol type \Verb{'o'}) of Y as a function of X, give the plot a title (``X vs. Y'' is fine) and label each axis (``X'', ``Y'').

\subsection{Polishing Plot Appearance} % (fold)
\label{sub:polishing_plot_appearance}

When creating more complex plots, once the basics are complete, remove any unnecessary ``chart junk'' to prepare your plot for publication.
This includes things like background shading (Matplotlib is pretty good about this), grid lines (only necessary if you're trying to manually read data off the chart; we don't need this with the ubiquity of computers), and the border surrounding the full plot area (should only need axes to indicate the origin or provide an anchor for tick marks).
If there is a legend (useful when there are multiple datasets present), it should usually be a part of the plot area, not pushed off to the side.
Don't waste valuable data area in your papers on something like a legend than can be nestled in one of the empty parts of your plot.
Finally, while color \emph{seems} like a nice feature, plots shouldn't rely on color to convey meaning.
There's always a chance that readers won't have access to color printouts, or that the reader may be colorblind.
To keep plots accessible, use symbols, line weights, and dashing to differentiate your datasets.
The lesson to learn from this paragraph is that simplicity is key to conveying your message.
The sole purpose of your figures is to help the reader understand the message.
Anything that distracts from that is harmful to your cause.

Now that the plot has been cleaned up, we'd like to add a trendline.
The Seaborn library is a visual analysis package built on top of Matplotlib with an emphasis on data exploration and friendly figures.
To plot our data with Seaborn, input the following:
\begin{pyinput}
  sns.lmplot(data=data, x="X", y="Y", ci=None)
\end{pyinput}
This creates a scatterplot using the X and Y columns from \Verb{data}, fits it with a linear model (``lm''), and disables the confidence interval (CI) in the plot.\sidenote{``ci'' can be set to a numeric value to represent the confidence interval of the associated fit, typically \qty{95}{\percent} (\pyinline{ci=95}).}
A linear fit should be sufficient for this data.
The Seaborn library is really designed for \emph{exploratory} data analysis, not for complex curve fits and edge cases of data.
It is generally inappropriate for spectral data, but very good for visual and statistical explorations of data.

% subsection polishing_plot_appearance (end)

% section basic-plotting-and-linear-fits (end)

\subsection{Fitting Data} % (fold)
\label{fitting-data}

% TODO: Change polyfit() to Polynomial.fit(), need covariance matrix first
Now that you have a trendline, it's time to get some information about it.
The Numpy library is a fundamental library in the numeric Python community.
It contains a vast array of functions and structures that are used by \emph{many} other libraries.\sidenote{The Numpy library (shortened to \Verb{np} by convention) has a host of useful functions, including a number of common mathematical functions.
  The square root function is one of these (\Verb{np.sqrt(num)}).
  Another useful one (that you'll need later) is \Verb{np.log10(num)}.}
One useful function in Numpy is the \href{https://numpy.org/doc/stable/reference/generated/numpy.polyfit.html}{\pyinline{polyfit()}} function, which calculates a polynomial of best fit using a linear least-squares method.
Examples can be seen on the linked page, but we will run through the simplest example, a straight line.

To use the \pyinline{polyfit()} function, you need to specify the data (``X'' and ``Y'') and the degree of the polynomial (\num{1}).
All other arguments are optional, and just giving these three arguments will return the coefficients of the polynomial from the highest power to the lowest (\( c_n, c_{n-1}, \ldots, c_1, c_0 \)).
The optional argument \pyinline{cov=True} will return the \href{https://en.wikipedia.org/wiki/Covariance_matrix}{\emph{covariance matrix}}, the diagonal values of which are the variances for each coefficient.

These (or, more precisely, their square roots) help you to see how ``good'' the returned fit is.
Please read the help page to learn what statistics are returned and in what form.\sidenote[][-2\baselineskip]{If you're interested in doing more statistical modeling in Python, the \href{https://www.statsmodels.org/}{Statsmodel} package offers a number of descriptive statistics and flexible models.}

Implement this using the following code:\sidenote{A couple of notes on Python features:
\begin{itemize}
  \item everything following the number sign (\emph{\#}) on a line is considered a \emph{comment}, meaning that it isn't evaluated.
  This is a great way to document your code as you work.
  \item The \href{https://docs.python.org/3.6/reference/lexical_analysis.html#f-strings}{``f-strings''} feature was introduced in Python 3.6.
  They are a far more convenient way to included code in printed strings, and result in much more readable code.
  Use them if you have the option, I won't bother teaching you older methods of formatting strings.
  \item Line 3 assigns multiple variables at once by using \href{https://eng.libretexts.org/Bookshelves/Computer_Science/Book\%3A_Python_for_Everybody_(Severance)/10\%3A_Tuples/10.03\%3A_Tuple_Assignment}{tuple assignment}.
  Don't worry too much about the mechanics of this, just know that it's a handy tool when the right side of your assignment puts out multiple items.
\end{itemize}
}
\begin{pyinput}
# Fitting using the `numpy.polyfit()` function
# Function returns fit coefficiences (1d array of values) and an n x n covariance matrix
fit_coeff, fit_cov = np.polyfit(data["X"], data["Y"], 1, cov=True)

# The standard error of each coefficient is just the square root of that diagonal element.
fit_err = np.sqrt(np.diag(fit_cov))

# Use formatted strings (f-strings) to display values with useful labels
print(f"slope: {fit_coeff[0]}, std. err.: {fit_err[0]}\n\
intercept: {fit_coeff[1]}, std. err.: {fit_err[1]}")
\end{pyinput}

% section fitting-data (end)

\section{Anscombe's Quartet} % (fold)
\label{anscombes-quartet}

Repeat the process of importing data with the file \Verb{anscombe.csv}.\sidenote{There are two options here: you will need to change the delimiter to a comma (\pyinline{delim=","}) for this file, or, since the default delimiter for \pyinline{pd.read_csv()} is a comma, you can omit the \pyinline{delim=","} argument altogether (it is implicitly defined).}
Name the new dataframe \Verb{anscombe} and display it by calling it alone in the last line of your cell.\sidenote{Just like we did with the dataframe \Verb{data} in the previous section.}

Now we want to calculate and print a linear fit and corresponding error for each of the four sets of data in \Verb{anscombe}.
The columns in \Verb{anscombe} should have the names ``x1'', ``y1'', ``x2'', ``y2'', etc.
For convenience, I've created a small loop for you to use on your data.
This allows us to perform repetitive operations without worrying about being consistent in our typing from operation to operation.\sidenote{If it's correct for the first set of data, it will be correct for each subsequent one}
\begin{pyinput}
# We will make a loop to make this less repetitive (and reduce our chances of making an error)
fits = []
for x, y in (('x1','y1'),('x2','y2'),('x3','y3'),('x4','y4')):
    fit_coeff, fit_cov = np.polyfit(anscombe[x], anscombe[y], 1, cov=True)
    fit_err = np.sqrt(np.diag(fit_cov))
    fits.append([fit_coeff,fit_err])

# This operation makes my point regarding this data more clear, as you'll see
with np.printoptions(precision=2):
    print("Fit param., Std. Err.")
    for item in fits:
        print(item)
\end{pyinput}

Clearly, the data sets are different… You can see that from the raw data in the table.
So what's going on?
To learn more, plot each of the datasets in a separate plot.
Discuss your observations with your lab partner, and figure out the best way to deal with information like this.

The key takeaway from this exercise is that you should always plot your data, even if it's only a quick glance on the side of your workspace!
This data was taken from \href{https://doi.org/10.1080/00031305.1973.10478966}{an article written by F.J. Anscombe in 1973} arguing for the necessity of graphs in statistical analysis.
It is such a valuable dataset that the \Verb{seaborn} library is installed with a copy of the data (\pyinline{sns.load_dataset("anscombe")}).

% section anscombes-quartet (end)

\section{Column Manipulation and Plotting Spectra} % (fold)
\label{column-manipulation-and-plotting-spectra}

Start a new section titled ``Plotting Spectra'' and import the file \Verb{ftir_data.csv}.
Print out the top of the dataframe using the command \pyinline{df.head()}, where \Verb{df} is the name of your dataframe.
Notice that we have data in the form ``Wavenumber (cm\textasciicircum-1}})'', ``Transmittance (\%)''.

There is a very simple syntax in Pandas for making a new column:
\begin{pyinput}
df["new col name"] = func(df["old col name"])
\end{pyinput}
In this case, \pyinline{func()} just represents the transformation that we are applying to the original column (multiplication, addition, inversion, etc.).
Columns in dataframes will apply functions across every item in the column by default (a process called ``broadcasting''), so making a line-by-line transformation (e.g., doubling a column) is as easy as \pyinline{df["B"] = 2 * df["A"]}.
Using this syntax, create new calculated columns in our dataframe so we can plot the same spectrum as ``Wavelength'' vs ``Absorbance (A)''.\sidenote{What are the relevant units for \%T and A?}

To get you started converting wavenumber to wavelength, make the following transformation on the ``Wavenumber'' column\sidenote{Notice that Python requires the use of ``**'' for exponentiation. This is because the caret character (\textasciicircum) is often used for other commands.
This convention is common in programming languages.}:
\begin{pyinput}
df["Wavelength (??)"] = 1/(df["Wavenumber (cm^-1)"]/10**(-2)) * 10**(6)
\end{pyinput}
You should be able to derive this equation quickly on your own…\sidenote{Hint: label the units and work through the dimensional analysis. Fill in the question marks with the appropriate units.}

Perform the same steps to take the column for ``Transmittance'' and transform it into Absorbance values.
Remember that Absorbance is \(\log_{10}(1/T)\) while Transmittance is \(I / I_0\).

Once you have created the new columns, create two plots.
The first should plot wavenumber agains transmittance, while the second will plot wavelength against absorbance.
Because these are spectra (data collected in a single experiment), it is appropriate to plot them as line plots, \emph{without} plotting the individual points.
Make sure that your plot reflects this.

Make sure your plots are labeled in all the appropriate locations and that your spectra look clean and presentable.

Congratulations, you have the basics of plotting and calculation using Jupyter!
You are expected to take this information and build upon it as you write reports through the semester.
All of these techniques can be translated to other applications, some of which might be more appropriate for the type of research being performed in a research group.
You should be able to transfer your basic skills to any other software with ease, and the tips on presentability are universally applicable.

%section column-manipulation-and-plotting-spectra (end)

% TODO: Integrate exercises from MolSSI plotting tutorial https://education.molssi.org/python-visualization/introduction.html
\section{Alternative Software} % (fold)
\label{alternative-software}

While Microsoft Excel may be the most common piece of software for plotting,  that doesn't mean it's the best.
You have seen that Python (by way of Jupyter) is capable of doing just about anything you can imagine, often in only a few lines.
In the future, should you find that Jupyter doesn't meet your needs, here is a list of alternative software for creating scientific plots (I've left out general purpose spreadsheet software).
\begin{description}
	\item[\href{https://www.graphpad.com}{GraphPad Prism} (CUNY site license available)] Statistical analysis software for Windows and macOS.
	Capable of repeated analysis on huge datasets of life/social science data.
	Falls short in spectral analysis tools.

	\url{https://www.graphpad.com}

	\item[\href{https://www.originlab.com}{Origin by OriginLab} (no site license available)] Professional data analysis and plotting software for scientists and engineers.
	Heavy focus on physical sciences and engineering.
	Only available on Windows.

	\url{https://www.originlab.com}

	\item[\href{https://www.wavemetrics.com/products/igorpro}{Igor Pro by WaveMetrics} (no site license available)] Very similar to Origin, but available for macOS and Windows.

	\url{https://www.wavemetrics.com/products/igorpro}

	\item[\href{http://www.wolfram.com/mathematica/}{Mathematica} (site license),
	\href{https://www.mathworks.com/products/matlab.html}{Matlab} (site license),
	\href{https://jupyter.org}{JupyterLab} (free)]
	These are really programming environments, though they are designed for use by scientists. Mathematica and Matlab are commercial products, while Jupyter is an open-source option that uses the \href{https://julialang.org}{JUlia},
  \href{https://www.python.org}{PYThon}, and \href{https://www.r-project.org}{R} languages.
	They all have entry-output interfaces so that live interaction can be done with data.
	While the learning curve is steep, the payout for learning a programming language will last a lifetime.

	\url{http://www.wolfram.com/mathematica/}

	\url{https://www.mathworks.com/products/matlab.html}

	\url{https://jupyter.org}

\end{description}

\nocite{*}
\printbibliography[category=cited]% default title for `article` class: "References"

\DeclareFieldFormat{labelnumberwidth}{\textbullet}
\printbibliography[%
  title={Further Reading},%
  resetnumbers,%
  omitnumbers,%
	notcategory=cited,%
	]

\end{document}
