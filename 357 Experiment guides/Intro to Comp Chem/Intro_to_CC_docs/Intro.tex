	\section{A quick summary of the exercises}

	\subsection{A guideline through the tutorial}
	This tutorial aims to give a basic introduction to electronic structure calculations for very simple systems. 
	As every quantum chemistry code has its own philosophy, this tutorial should familiarize you with general-purpose Gaussian16 (hereafter abbreviated as G16) software and GausView6 visualization tool.
	The experiments will also demonstrate predictive power of quantum-chemical calculations. 	
	
	First, the basic structure of an input file to the G16 software will be explained.
	The second part will introduce scanning along the binding curve and computing observables. 
	The third part introduces geometric optimization of a small molecule and how to assesse reliability of the result. 
	%The third part is dedicated to visualizing electronic structure derived data. 

	\noindent
	
	%\begin{enumerate}
	%\begin{enumerate}[label=Part \Roman*:,
	%  align=left,itemindent=\parindent]  
	%  \item \hyperref[sec:partI]{\quad Basic electronic-structure calculations with G16}
	  
	      \begin{enumerate}[label=Prob. \Roman*:,
	    align=left,leftmargin=*,itemindent=\parsep]
	      \item \hyperref[sec:problemI]{The hydrogen atom}
	      \item \hyperref[sec:problemII]{Hydrofluoric acid: bond length and dipole moment}	      
	      \item \hyperref[sec:problemIII]{Hydronium cation: geometry relaxation, vibrations, and PES}
	      %\item \hyperref[sec:problemIII]{Molecular oxygen : spin state}
	      \end{enumerate}
	   

	  %\item \hyperref[sec:partII]{\quad Local structure optimization of\texorpdfstring{ H$_3$O$^+$}{hydronium cation}}
	  
	 %    \begin{enumerate}[label=Problem \Roman*:,align=left,
	 %   leftmargin=*,itemindent=\parindent,resume]
	 %     \item \hyperref[sec:problemIII]{Planar hydronium cation}
	 %     \item \hyperref[sec:problemIV]{Pyramidal hydronium cation}
	 %     \item \hyperref[sec:problemV]{Potential-Energy Surface scan}
%	 %     \item \hyperref[sec:problemIX]{Limits of the harmonic approximation \texorpdfstring{\textcolor{red}{\textbf{!}}}{!}}
	 %     \end{enumerate}

	  %\item \hyperref[sec:partIII]{\qquad Electron density mixing and visualising electron densities and eigenstates}
	  %    \begin{enumerate}[label=Problem \Roman*:,align=left,
	  %  leftmargin=*,itemindent=\parindent,resume]
	  %    \item \hyperref[sec:problemVI]{Visualizing density differences of \textit{para}-benzoquinone}
	  %    \end{enumerate}
	%\end{enumerate}
\noindent
	%\textbf{As the first step, please copy the folder \texttt{experiment_5 } from \texttt{\$PChem } to your working directory. }\\

	%For every exercise, we also provide solutions and sample input files. They can be found in \tutorialonedir/solutions and \referencedir/exercise\_XX/templates, respectively. However, we strongly recommend to use the provided input files only in case of time shortage. You will maximize your learning progress by trying to generate the input files on your own. In case you get stuck with a particular problem, do not hesitate to ask one of the tutors. For the tutorials, an executable of FHI-aims will be provided on your workstation.\\


	%\textbf{Note: Please do not copy and paste the description in this pdf into your input files. Typically, invisible characters from the formatting are copied, too. FHI-aims will revoke these characters and your calculations will not start. }

	\subsection{First look at G16}

	A good practice is to perform each calculations a separate directory. The calcuations are initialized by calling the g16 command on an input file:
	
	\texttt{g16 input \&}
	
	\noindent By convention, always name the \texttt{input} file as \texttt{input.com}. The basic input file is composed as follow:

	\begin{tip}
	\textbf{A structure of a simple input file}:
	\newline
	\%nproc=4 \newline
	\%mem=400MB \newline
	\# \texttt{method} \texttt{basis-set}  \newline
	\# sp scf=tight  \newline
	[ empty line ] \newline
	your-comment \newline
	[ empty line ] \newline
	\texttt{charge} \texttt{multiplicity} \newline
	\texttt{atom1 X Y Z}  \newline
	\texttt{atom2 X Y Z}  \newline
	$\ldots$           \newline
	\texttt{atomN X Y Z}  \newline
	[ empty line ]
	\end{tip}
	\noindent{} The above starts a calculation on two processors using 400 megabytes of memory. The output will be redirected to the \texttt{ input.log} oputput file. This file contains the basic information and results of the calculation such as the total energy, atomic forces, and so forth. Additional output files might be generated according to the specified settings. Next paragraph explines the file step by step. 


\begin{description}
  \item[\texttt{ \%nproc=4}] This keyword specifies the number of processors that will be employed for calculations. The laptops you use have 4 processors and you should use them all. 
  \item[\texttt{ \%mem=400MB}] This keyword manages the maximum memory usage during the calculations. The efficient memory specification is beyond this tutorial. 
  \item[\texttt{ method}]Substitute the \texttt{method} with the method of choice for electron - electron interactions. In this tutorial, we will use several methods, namely Hartree-Fock (HF), Møller-Plesset second order perturbation theory  (MP2) and few density-functionals. The details of each method will be covered in details during the lecture. 
  \item[\texttt{ basis-set}] Substitute the \texttt{basis-set} word with the desired basis set which specifies a set of basis functions (for instance atomic orbitals, gaussian-type orbitals, plane waves) that will be used to express electronic configuration. The recommended basis sets are specified in each excercise. 
  \item[\texttt{ sp}] \texttt{sp} orders gaussian to perform single-point calculations, i.e. energy evaluation of a specified structure using \texttt{method} and \texttt{basis-set}
  \item[\texttt{ scf=tight}] The Schrödinger equation is solved in self-consistent manner. The \texttt{scf=tight} specifies tight convergence criteria for the self-consistent cycle. 
  \item[\texttt{ your-comment}] This line, surrounded by two empty lines, holds your comment. 
  \item[\texttt{ charge}] The \texttt{charge} keyword should be substituted with a total charge of your system. 
  \item[\texttt{ multiplicity}] The \texttt{multiplicity} keyword should be substituted with multiplicity of your system (2S+1, where S is total spin)
  \item[\texttt{ atomX X Y Z}] This block specifies the geometry of the system. You can use either symbol or atomic number for specifing the atom type, followed by it's cartesian coordinates in units of Angstoms [\AA]. This block must be followed by an empty line. 
\end{description}

Remember, there is no bonds (sticks) in quantum chemistry. The bonding is the result of the respective positions of atoms in space. The `stick' visible in visualization program is simply a rendering for more intuitive display. 


\subsection{Additional tools and programs}

\noindent
\textbf{Bash shell:}\\
A short list of the basic bash (command line) commands is given in Appendix~\hyperref[app:bash]{I}.

\noindent 
\textbf{vi:} \\
% A short summary of the vi editor is given in Appendix~\hyperref[app:vi]{II}.
A number of introductions to the \texttt{vi} editor are available online. Two such examples are \url{https://www.openvim.com} and \url{https://vim-adventures.com/}.


\noindent
\textbf{Scripts:}\\
For some exercises, scripts are required for dedicated tasks.
All scripts you will need for this tutorial can be found in respective directories. 

