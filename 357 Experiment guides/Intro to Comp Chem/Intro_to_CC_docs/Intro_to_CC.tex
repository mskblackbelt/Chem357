% TODO: Convert code over to Psi4 so all work can be done inside Jupyter notebooks
% Tutorial prepared by Igor Y. Zhang, Hagen-Henrik Kowalski and Tonghao Shen 2017.
% Transcribed and modified by Dustin D. Wheeler
% Updated 2021
\documentclass[nobib,nofonts,nols,nohyper]{tufte-handout}
%% Header file cloned from https://github.com/wickles/latex-base

%%%%%%%%%%%%%%%%%%
%% CONTENTS
%%%%%%%%%%%%%%%%%%
% To-do / issues
% Packages
% Commands
% Special Symbols
% Environments
% Notes
%%%%%%%%%%%%%%%%%%
%%%%%%%%%%%%%%%%%%



%%%%%%%%%%%%%%%%%%
%% TO-DO / ISSUES
%%%%%%%%%%%%%%%%%%

% Fix header format in tufte-latex (bad spacing, no small-caps font in Charter). 
% Replace \Molar unit, check for \textsc




%%%%%%%%%%%%%%%%%%
%% PACKAGES
%%%%%%%%%%%%%%%%%%


%% debugging / diagnostics
\RequirePackage[l2tabu,orthodox]{nag} % nags user about obsolete and improper syntax

\usepackage{xparse} % provides high-level interface for producing document-level commands
	% via \[Declare/New/Renew/Provide/etc]DocumentCommand
	% allows for more than one optional argument in commands

\usepackage{ifdraft}

%% fonts and encoding 

\newcommand{\textools}[2][5]{%
	\begingroup\addfontfeatures{LetterSpace=#1}#2\endgroup
	}
\renewcommand{\allcapsspacing}[1]{\textools[15]{#1}}
\renewcommand{\smallcapsspacing}[1]{\textools[10]{#1}}
\renewcommand{\allcaps}[1]{\textools[15]{\MakeTextUppercase{#1}}}
\renewcommand{\smallcaps}[1]{\textit{#1}} 
	% Version of Charter font included with macOS 13 doesn't work with \textsc
	% It uses individual smallcaps glyphs instead. 
% \renewcommand{\textsc}[1]{\smallcapsspacing{\textsmallcaps{#1}}}

%% font packages -- load fontspec, then select fonts and features. 

\usepackage{fontspec}
\usepackage[math-style=ISO,mathrm=sym]{unicode-math} 
\ifdraft{}{
	\setmainfont{Charter}
	\setmathfont{STIX Two Math}
	\setmonofont{Consolas}
}


% standard and structural packages

\usepackage{bm} % provides \bm command for robustly bolding math characters
\usepackage{microtype} % improves kerning in certain cases. 
	% recommended to disable protrusion in table of contents!

%% Latex interface 

\usepackage{letltxmacro} % provides \LetLtxMacro command for correct renaming of commands
\usepackage{etoolbox} % provides many useful programming tools, 
	% e.g. \ifdefempty{cs}{true}{false}



%% media interface

\usepackage{graphicx} % support the \includegraphics command and options
\usepackage{subfig} % Support for subfigures and subcaptions
	\captionsetup[subfloat]{position=bottom}
\usepackage{pgfplots} % for plotting in tikzpicture environment
	\pgfplotsset{compat=1.16} % required to select newest version
\usepackage{tikzscale} % allows \includegraphics{*.tikz} and scaling of TiKZ images


%% math interface

\usepackage{amsmath} % for nice math commands and environments
\usepackage{mathtools} % extends amsmath with bug fixes and useful commands, e.g.
	% \shortintertext for short interjections in align environment,
	% \prescript{t}{b}{X} for simple, nicely aligned math pre-(super/sub)scripts
	% \Aboxed{...} for boxing full lines in 'align' environment
\usepackage{derivative} % provides \odv, \pdv, \odif, \pdif
\usepackage{bropd} % provides \br command which simplifies nesting of bracketed terms 
	% e.g. \br{\br{x-a}^2+\br{y-b}^2} produces \left[ \left( x-a \right)^2 + \left( y-b \right)^2 \right]
\usepackage{array} % improves array support, esp. in tabular env. 
	% see also xtab.sty
\usepackage{booktabs} % allows for improved spacing in tabular env. 
	% use \toprule, \*midrule, \bottomrule instead of \hline


%% Science and programming packages
\usepackage{fvextra} % for verbatim and comment environments with \Verb
\usepackage{chemmacros} % for writing chemical formulas with \ch, e.g. \ch{AgCl2-} or \ch{^{227}_{90}Th+}
	\usechemmodule{
		spectroscopy, % loads formula and siunitx modules, provides \NMR command.  
	    thermodynamics, % provides state variables and equations
    	units, % provides \[mM]olar, \Torr, \atm, \cal, \cmc, \MolMass
		} % also loads siunitx and chemformula
	\sisetup{% siunit package options
		per-mode = symbol,%
		inter-unit-product=\ensuremath{{}\!\cdot\!{}},%
		separate-uncertainty,%
		multi-part-units = single,%
		retain-explicit-plus,%
		list-final-separator={, and },%
		math-celsius = °\text{C}, % for temperatures
		text-celsius = °C,
		math-degree = °, % for angles
		text-degree = °,%
		input-digits = 0123456789 \pi \mitpi% necessary to use \pi in SI entries, affects rounding of digits.
		}%
	\DeclareSIUnit\ppm{ppm}
	\DeclareSIUnit\angstrom{\text{Å}} % Symbol doesn't exist in STIXTwoMath, need to force text font. 
	\DeclareSIUnit\wn{cm^{-1}}
	

%% misc packages

\usepackage{framed} % provides boxed 'framed' environment for easily boxing text 
\usepackage{tcolorbox}
	\tcbuselibrary{skins, breakable, xparse, minted}
	% provides fancier boxes than regular \makebox, \fbox, etc.
	% e.g. \doublebox, \ovalbox, \shadowbox
	% Can use `\tcbuselibrary{listings}` to use the listings library, 
		% doesn't require a language to be defined. 
\usepackage{empheq} % provides 'empheq' environment 
	% for improved control over shape, size, color of framed boxes, e.g. 
\newcommand{\boxedeq}[2]{
	\begin{empheq}[box={\fboxsep=6pt\fbox}]{align}\label{#1}#2\end{empheq}
}
\newcommand{\coloredeq}[2]{
	\begin{empheq}[box=\colorbox{lightgreen}]{align}\label{#1}#2\end{empheq}
}


%% document interface 

\usepackage{footnote} % 
\usepackage{hyperref} % adds hyperlinks and outline to PDF documents
	\hypersetup{%
		pdfencoding=auto,%
		psdextra,%
		pdfusetitle,%
		colorlinks=true,%
		linkcolor=BrickRed, %
		citecolor=Green, %
		filecolor=Mulberry, %
		urlcolor=NavyBlue, %
		menucolor=BrickRed, %
		runcolor=Mulberry, %
		linkbordercolor=BrickRed, %
		citebordercolor=Green, %
		filebordercolor=Mulberry, %
		urlbordercolor=NavyBlue, %
		menubordercolor=BrickRed, %
		runbordercolor=Mulberry %
		} %
	% options enable enhanced unicode and math support in PDF outlines [causes conflict with \C command?]
\usepackage{cleveref} % provides \cref command which inserts contextually correct word in front of ref.
	% e.g. \cref{eq:myeq} --> Equation 1.2, or so
\usepackage{bookmark} % improves package hyperref's bookmarking. 
	% properties such as style and color can be set. Generates bookmarks in first run. 


%% load later packages
\usepackage[textsize=footnotesize]{todonotes}



%%%%%%%%%%%%%%%%%%
%% COMMANDS
%%%%%%%%%%%%%%%%%%

\newcommand{\mtext}[1]{{\mathup{#1}}}
\DeclareMathOperator{\sgn}{sgn}
\DeclareMathOperator{\erf}{erf}
\DeclareMathOperator{\erfc}{erfc}
\DeclareMathOperator{\GammaFunc}{\symup{\Gamma}}
\DeclareMathOperator{\laplacian}{\nabla^2}
\DeclarePairedDelimiter\abs{\lvert}{\rvert}
\newcommand{\vb}[1]{\symbfit{#1}}

%% pre-defined colors
% standard: black, blue, brown, cyan, darkgray, gray, green, lightgray, lime, magenta, olive, orange, pink, purple, red, teal, violet, white, yellow
%
% dvips: Apricot, Aquamarine, Bittersweet, Black, Blue, BlueGreen, BlueViolet, BrickRed, Brown, BurntOrange, CadetBlue, CarnationPink, Cerulean, CornflowerBlue, Cyan, Dandelion, DarkOrchid, Emerald, ForestGreen, Fuchsia, Goldenrod, Gray, Green, GreenYellow, JungleGreen, Lavender, LimeGreen, Magenta, Mahogany, Maroon, Melon, MidnightBlue, Mulberry, NavyBlue, OliveGreen, Orange, OrangeRed, Orchid, Peach, Periwinkle, PineGreen, Plum, ProcessBlue, Purple, RawSienna, Red, RedOrange, RedViolet, Rhodamine, RoyalBlue, RoyalPurple, RubineRed, Salmon, SeaGreen, Sepia, SkyBlue, SpringGreen, Tan, TealBlue, Thistle, Turquoise, Violet, VioletRed, White, WildStrawberry, Yellow, YellowGreen, YellowOrange

\usepackage[mackeys=text]{menukeys}
  \usetikzlibrary{arrows.meta}
  \tikzset{menukeys key symbol/.style={>=Triangle}} % Change Arrow headers for key symbols
  \DeclareSIUnit\kcal{kcal}
  \DeclareSIUnit\wn{\raiseto{-1}\cm}
  \DeclareSIUnit\hartree{Ha}

%%%%%%% Bibliography options %%%%%%%%%%%%%%%%%%%%%%%%%%%%%%%%%%%%%%%%%%%%%%%%%
% (fold)
\usepackage[%
  style=numeric-comp,%
  sortcites,%
  sorting=none,%
  defernumbers=true,%
  hyperref,
  backend=biber,
  ]{biblatex}
  \addbibresource{../../pchem_bib.bib}

  % Filter bibliography file to only include entries matching keyword.
  \DeclareSourcemap{%
    \maps[datatype=bibtex]{%
      \map{%
        \step[nocited, final]%
        \step[fieldsource=keywords, notmatch=compchem, final]%
        \step[entrynull]%
      }%
    }%
  }

  % Add a section for cited bibliography entries.
  % At the end, create a second bibliography for uncited items (Further Reading).
  \DeclareBibliographyCategory{cited}
  \AtEveryCitekey{\addtocategory{cited}{\thefield{entrykey}}}

  % Define a style for the Further Reading section
  \defbibenvironment{nolabelbib}
    {\list
       {}
       {\setlength{\leftmargin}{2\bibhang}%
        \setlength{\itemindent}{-\bibhang}%
        \setlength{\itemsep}{\bibitemsep}%
        \setlength{\parsep}{\bibparsep}}}
    {\endlist}
    {\item}

  % Remove URL from citations if DOI is present.
  \AtEveryBibitem{%
    \iffieldundef{doi}{ % do nothing if true
    }
    { % otherwise, clear the URL
      \clearfield{url}
    }%
  }
% (end)

%%%%%%% Source Code options %%%%%%%%%%%%%%%%%%%%%%%%%%%%%%%%%%%%%%%%%%%%%%%%%%

% Command to insert $PROMPT before shell console input
\newcommand{\BashFancyFormatLine}{%
  \def\FancyVerbFormatLine##1{\textcolor{blue}{\small user:\char`\~\$}\ ##1}%
}

% Set up Gaussian input settings
\tcbuselibrary{listings} % Use listings library for dummy Gaussian input files.
	\lstset{morecomment=[s]{<}{>}} % Set comment style for dummy Gaussian input

% tcolorbox settings for code blocks
\tcbset{
  colback=blue!4!white,
  colframe=blue!85!black,
  listing only,
  left=1.5em,
  enhanced jigsaw,
  breakable,
  overlay={
    \begin{tcbclipinterior}
      \fill[red!20!blue!20!white] 
      (frame.south west)  rectangle 
      ([xshift=1.5em]frame.north west);
    \end{tcbclipinterior}}
}
\setminted{
  style=tango,
  fontsize=\small,
  breaklines,
  breakafter=|,
  autogobble,
  linenos,
  numberfirstline=true,
  firstnumber=1,
  stepnumber=2,
  numbersep=1em
}

% Settings for all code blocks
\lstset{
  breaklines=true,
  basicstyle=\small,
  numberstyle=\tiny,
  numbers=left,
  numberfirstline=true,
  firstnumber=1,
  stepnumber=2,
  numbersep=2em,
}

% Settings for Gaussian code blocks
\lstdefinestyle{gaussian}{
  language=Clean,
  % keywordstyle=\bfseries,
  commentstyle=\slshape,
  % stringstyle=\ttfamily,
  numbers=left,
  firstnumber=1,
  stepnumber=2,
  numberfirstline=true,
  numbersep=2em,
  numberstyle=\tiny,
  breaklines=true,
  breakautoindent=true,
  tabsize=2,
  basicstyle=\ttfamily\small,
  showspaces=false,
  showstringspaces=false,
  morecomment=[s]{<}{>}
}

% Minted settings for inline bash code
\newmintinline{bash}{}

% Hack to force linebreaks at `¬` in bashinputs. From
% https://tex.stackexchange.com/a/461455/27032
\AtBeginEnvironment{bashinput}{%
  \catcode`¬\active
  \begingroup\lccode`~=`\¬\lowercase{\endgroup\def~{\linebreak}}%
}

% Settings for bash code blocks
\newtcblisting{bashinput}[1][]{
  listing engine=minted,
  minted language=bash, 
  listing only,
  minted options={,
    formatcom=\BashFancyFormatLine
  },
  #1
}

% Settings for Gaussian input file blocks
\newtcblisting{gaussinput}[1][]{
	listing engine=listings,
	listing only,
  listing options={
    style=gaussian
    },
  #1
}
%%%%%%%%%%%%%%%%%%%%%%%%%%%%%%%%%%%%%%%%%%%%%%%%%%%%%%%%%%%%%%%%%%%%%%%%%%%%%%%

\begin{document}

\title{Introduction to Computational Chemistry}

\author{Dustin Wheeler, Mateusz Marianski}

\begin{document}

\maketitle

\begin{abstract}
  This tutorial aims to give a basic introduction to electronic structure calculations for very simple systems.\thanks{Adapted from \textcite{marianski19}.} 
  As every quantum chemistry code has its own philosophy, this tutorial should familiarize you with the general-purpose Gaussian16 (hereafter abbreviated as g16) software.
  The experiments will also demonstrate the predictive power of quantum-chemical calculations. 	
	
  First, the basic structure of an input file to the G16 software will be explained.
  The second part will introduce scanning along the binding curve and computing observables. 
  The third part introduces geometric optimization of a small molecule and how to assess reliability of the result. 

  \begin{enumerate}[
      label=Prob. \Roman*:,
      align=left,
      leftmargin=*,
      itemindent=\parsep
    ]
    \item \hyperref[sec:problemI]{The hydrogen atom}
    \item \hyperref[sec:problemII]{Hydrofluoric acid: bond length and dipole moment}	      
    \item \hyperref[sec:problemIII]{Hydronium cation: geometry relaxation, vibrations, and PES}
  \end{enumerate}

As the first step, please use this link to clone the files into your Jupyter directory:\\  \href{https://sugarcube.hunter.cuny.edu/hub/user-redirect/git-pull?repo=https%3A%2F%2Fgithub.com%2Fmskblackbelt%2Fpchem_comp-chem_template&urlpath=lab%2Ftree%2Fpchem_comp-chem_template%2FCompChem_template.ipynb&branch=main}{\texttt{https://tinyurl.com/chem357-compchem}}
\marginnote[-4\baselineskip]{While working through this lab, do \textbf{not} copy and paste the code in this PDF into your input files. Invisible formatting characters are often copied from the PDF. Gaussian will not understand these characters and your calculations will not start.}

This guide is meant to be used in tandem with the Jupyter notebook contained in the cloned folder (``pchem\_comp-chem\_template/CompChem\_template.ipynb''). 
That notebook has a number of cells pre-populated for your convenience. Make sure you read the notes and comments in the notebook as you go through the lab and fill in any blanks before executing the cells. 
\end{abstract}



	\section{A quick summary of the exercises}

	\subsection{A guideline through the tutorial}
	This tutorial aims to give a basic introduction to electronic structure calculations for very simple systems. 
	As every quantum chemistry code has its own philosophy, this tutorial should familiarize you with general-purpose Gaussian16 (hereafter abbreviated as G16) software and GausView6 visualization tool.
	The experiments will also demonstrate predictive power of quantum-chemical calculations. 	
	
	First, the basic structure of an input file to the G16 software will be explained.
	The second part will introduce scanning along the binding curve and computing observables. 
	The third part introduces geometric optimization of a small molecule and how to assesse reliability of the result. 
	%The third part is dedicated to visualizing electronic structure derived data. 

	\noindent
	
	%\begin{enumerate}
	%\begin{enumerate}[label=Part \Roman*:,
	%  align=left,itemindent=\parindent]  
	%  \item \hyperref[sec:partI]{\quad Basic electronic-structure calculations with G16}
	  
	      \begin{enumerate}[label=Prob. \Roman*:,
	    align=left,leftmargin=*,itemindent=\parsep]
	      \item \hyperref[sec:problemI]{The hydrogen atom}
	      \item \hyperref[sec:problemII]{Hydrofluoric acid: bond length and dipole moment}	      
	      \item \hyperref[sec:problemIII]{Hydronium cation: geometry relaxation, vibrations, and PES}
	      %\item \hyperref[sec:problemIII]{Molecular oxygen : spin state}
	      \end{enumerate}
	   

	  %\item \hyperref[sec:partII]{\quad Local structure optimization of\texorpdfstring{ H$_3$O$^+$}{hydronium cation}}
	  
	 %    \begin{enumerate}[label=Problem \Roman*:,align=left,
	 %   leftmargin=*,itemindent=\parindent,resume]
	 %     \item \hyperref[sec:problemIII]{Planar hydronium cation}
	 %     \item \hyperref[sec:problemIV]{Pyramidal hydronium cation}
	 %     \item \hyperref[sec:problemV]{Potential-Energy Surface scan}
%	 %     \item \hyperref[sec:problemIX]{Limits of the harmonic approximation \texorpdfstring{\textcolor{red}{\textbf{!}}}{!}}
	 %     \end{enumerate}

	  %\item \hyperref[sec:partIII]{\qquad Electron density mixing and visualising electron densities and eigenstates}
	  %    \begin{enumerate}[label=Problem \Roman*:,align=left,
	  %  leftmargin=*,itemindent=\parindent,resume]
	  %    \item \hyperref[sec:problemVI]{Visualizing density differences of \textit{para}-benzoquinone}
	  %    \end{enumerate}
	%\end{enumerate}
\noindent
	%\textbf{As the first step, please copy the folder \texttt{experiment_5 } from \texttt{\$PChem } to your working directory. }\\

	%For every exercise, we also provide solutions and sample input files. They can be found in \tutorialonedir/solutions and \referencedir/exercise\_XX/templates, respectively. However, we strongly recommend to use the provided input files only in case of time shortage. You will maximize your learning progress by trying to generate the input files on your own. In case you get stuck with a particular problem, do not hesitate to ask one of the tutors. For the tutorials, an executable of FHI-aims will be provided on your workstation.\\


	%\textbf{Note: Please do not copy and paste the description in this pdf into your input files. Typically, invisible characters from the formatting are copied, too. FHI-aims will revoke these characters and your calculations will not start. }

	\subsection{First look at G16}

	A good practice is to perform each calculations a separate directory. The calcuations are initialized by calling the g16 command on an input file:
	
	\texttt{g16 input \&}
	
	\noindent By convention, always name the \texttt{input} file as \texttt{input.com}. The basic input file is composed as follow:

	\begin{tip}
	\textbf{A structure of a simple input file}:
	\newline
	\%nproc=4 \newline
	\%mem=400MB \newline
	\# \texttt{method} \texttt{basis-set}  \newline
	\# sp scf=tight  \newline
	[ empty line ] \newline
	your-comment \newline
	[ empty line ] \newline
	\texttt{charge} \texttt{multiplicity} \newline
	\texttt{atom1 X Y Z}  \newline
	\texttt{atom2 X Y Z}  \newline
	$\ldots$           \newline
	\texttt{atomN X Y Z}  \newline
	[ empty line ]
	\end{tip}
	\noindent{} The above starts a calculation on two processors using 400 megabytes of memory. The output will be redirected to the \texttt{ input.log} oputput file. This file contains the basic information and results of the calculation such as the total energy, atomic forces, and so forth. Additional output files might be generated according to the specified settings. Next paragraph explines the file step by step. 


\begin{description}
  \item[\texttt{ \%nproc=4}] This keyword specifies the number of processors that will be employed for calculations. The laptops you use have 4 processors and you should use them all. 
  \item[\texttt{ \%mem=400MB}] This keyword manages the maximum memory usage during the calculations. The efficient memory specification is beyond this tutorial. 
  \item[\texttt{ method}]Substitute the \texttt{method} with the method of choice for electron - electron interactions. In this tutorial, we will use several methods, namely Hartree-Fock (HF), Møller-Plesset second order perturbation theory  (MP2) and few density-functionals. The details of each method will be covered in details during the lecture. 
  \item[\texttt{ basis-set}] Substitute the \texttt{basis-set} word with the desired basis set which specifies a set of basis functions (for instance atomic orbitals, gaussian-type orbitals, plane waves) that will be used to express electronic configuration. The recommended basis sets are specified in each excercise. 
  \item[\texttt{ sp}] \texttt{sp} orders gaussian to perform single-point calculations, i.e. energy evaluation of a specified structure using \texttt{method} and \texttt{basis-set}
  \item[\texttt{ scf=tight}] The Schrödinger equation is solved in self-consistent manner. The \texttt{scf=tight} specifies tight convergence criteria for the self-consistent cycle. 
  \item[\texttt{ your-comment}] This line, surrounded by two empty lines, holds your comment. 
  \item[\texttt{ charge}] The \texttt{charge} keyword should be substituted with a total charge of your system. 
  \item[\texttt{ multiplicity}] The \texttt{multiplicity} keyword should be substituted with multiplicity of your system (2S+1, where S is total spin)
  \item[\texttt{ atomX X Y Z}] This block specifies the geometry of the system. You can use either symbol or atomic number for specifing the atom type, followed by it's cartesian coordinates in units of Angstoms [\AA]. This block must be followed by an empty line. 
\end{description}

Remember, there is no bonds (sticks) in quantum chemistry. The bonding is the result of the respective positions of atoms in space. The `stick' visible in visualization program is simply a rendering for more intuitive display. 


\subsection{Additional tools and programs}

\noindent
\textbf{Bash shell:}\\
A short list of the basic bash (command line) commands is given in Appendix~\hyperref[app:bash]{I}.

\noindent 
\textbf{vi:} \\
% A short summary of the vi editor is given in Appendix~\hyperref[app:vi]{II}.
A number of introductions to the \texttt{vi} editor are available online. Two such examples are \url{https://www.openvim.com} and \url{https://vim-adventures.com/}.


\noindent
\textbf{Scripts:}\\
For some exercises, scripts are required for dedicated tasks.
All scripts you will need for this tutorial can be found in respective directories. 


%!TEX root = ./Intro_to_CC.tex

\section{Problem I: The hydrogen atom} % (fold)
\label{sec:problemI}

In this exercise, we will look at different basis sets using the hydrogen atom. 
The hydrogen atom is the only non-trivial system for which the exact analytic solution is known. 
By the end of the first exercise, we will see how various computational methods compare to each other and to the exact solution.  
From a technical perspective, we will learn how to compose input files, run basic Psi4 calculations, search for energy in the Gaussian output, and perform basis set convergence tests.

\subsection*{Getting started}

\textbf{Tasks}
\begin{enumerate}
  \item We'll begin by setting some options for the calculation. 
  Because we're dealing with a hydrogen atom, we have to use an ``unrestricted Hartree-Fock''  method (the unpaired electron must be properly accounted for), hence \Verb{"uhf"}. 
  We'll set the basis set to a very minimal set called \href{https://en.wikipedia.org/wiki/STO-nG_basis_sets}{``STO-3G''} (each available atomic orbital is represented by three contracted gaussian functions). 
  Finally, we define the molecule geometry and assign it a name. 
  For this \emph{very} simple system, setting the basis, method, and geometry is as simple as: 
\begin{pyinput}
h_atom = psi4.geometry("H")
basis = 'sto-3g'
psi4.set_options({
                  'reference': 'uhf'
                })
\end{pyinput}
  With a single atom, there's no need to identify connectivity or location of the atom in space.
  We use the \Verb{set_options()} method and create a dictionary of options we'd like to set.  
  \item Now that we've defined a molecule and basis, we'll tell Psi4 what sort of calculation we'd like to run. 
  To begin, we'll perform a simple energy calculation and ask Psi4 to return the energy and the wavefunction for the system. 
\begin{pyinput}
energy, wfn = psi4.energy('hf/'+basis, return_wfn=True)
nbf = wfn.basisset().nbf()
print(f'{energy=}\t{nbf=}')
\end{pyinput}
    
    The option \pyinline{return_wfn=True} tells Psi4 to return information about the wavefunction for the system. 
    From that, we pull out the number of basis functions (\Verb{nbf}) being used in our calculation and print out the energy and number of basis functions. 
    
    This is the computed electronic energy of the H atom using Hartree-Fock theory in the STO-3G basis set. 
    Compare it with the exact result for the hydrogen atom (\(\qty{0.5}{\hartree} \approx  \qty{13.6057}{\eV} \approx \qty{313.7545}{\kcal\per\mol}\)).
    
  \item Redo the calculation with your original basis set (STO-3G) and the following basis sets: cc-pVDZ, cc-pVTZ, cc-pVQZ by creating a list of the basis set names (as strings), then setting up a loop to perform the calculation and save the energy and number of basis sets to a pair of lists.\sidenote{
    You can find a rough overview of \href{https://en.wikipedia.org/wiki/Basis_set_\%28chemistry\%29}{basis sets at Wikipedia} and a \href{https://psicode.org/psi4manual/master/basissets.html}{Psi4-specific overview} is also available.
    } 
  Then, plot the total energy as function of the basis set size. 
  At which basis set does the energy converge to the exact solution? 
\end{enumerate}

\subsection*{Method performance}

In this step, we'll repeat the previous calculations with different methods by wrapping your previous work in another loop. This loop should run over the following list of methods: HF, SVWN, PBE, and PBE0. 
As before, we create a list containing these methods (as strings), and run a loop inside a loop. 
In order to save our information in a nice structure of recalling later (as we want to keep the methods separate for plotting), we're going to store our results in a dictionary object. 
This means we need to initialize an empty dictionary in which to save our data. This has been set up for you in the notebook template. 
Spend some time looking over the nested loops to make sure you see what's being done in each loop. 

In the next cell, we set up a small loop to plot the data showing the convergence of different methods to the exact value of \qty{0.5}{Ha}.
Do all of them converge correctly to the same solution? 
The details of the listed theoretical methods to evaluate electron--electron interactions and why they converge to different values for the apparently trivial one-electron system are beyond this tutorial and will be covered in lecture later this semester. 

% section problemI (end)
%!TEX root = ./Intro_to_CC.tex

\section{Problem II: Hydrofluoric acid (HF) -- bond length and dipole moment} \label{sec:problemII}

\subsection*{The hydrogen fluoride molecule (HF)}

In the exercise, we will calculate the binding curve, atomization energy (\( \increment H_\mtext{at} \)), and dipole moment for the hydrogen fluoride (HF) molecule with two methods. 
From a technical perspective, this exercise teaches how to use Python loops to perform repeated computations with Psi4. 

\begin{enumerate} 
  \item The first task of this exercise will be to find learn a bit about the output data from a calculation
  Start by defining a new hydrogen fluoride molecule. 
  This is again done for you in the first cell. 
  The input block also has space for you to specify the method and the basis set used for computation. 
  In this exercise, use \Verb{HF} (Hartree-Fock) and the \Verb{6-31G(d,p)} basis set. 
  
  Notice the geometry is defined in terms of the atomic connectivities (the first atom is the "center", labeled 1, other atoms are labeled sequentially with connectivity assigned by number. The line ``\Verb{H 1 r}'' means that a hydrogen atom is placed distance \(r\) away from atom \num{1}.) The geometry could alternatively be defined explicitly in Cartesian coordinates like so: 
  \begin{geominput}[title=Cartesian geometry definition]
  F 0.0 0.0 0.0
  H 0.0 0.0 r
  \end{geominput}
  We use the variable \Verb{r} as a placeholder so we can define this distance later in our code.\sidenote{This variable can be anything we like, as long as it's a valid variable name in Python (\emph{i.e.,} \Verb{interatomic_distance} would be perfectly acceptable).}
  Before we start making multiple calculations, let's make a single one and figure out which computational variables we'd like to collect. 
  We can set the distance variable to a value with \pyinline{hf_bond.r = 1.0}, then perform our energy calculation (again, making sure to return and store the wavefunction data, just as before.)
  
  Once we have this, you can print out all of the data saved in the wavefunction variable with \pyinline{wfn.variables()}.\sidenote{Assuming you named the wavefunction variable \Verb{wfn}.}
  This will print out all of the properties returned by the current calculation. 
  The data are returned as dictionary items, with the dictionary key in capital letters and the value given after the colon. 
  An individual property can be recalled using \pyinline{wfn.variable('name')}, where \Verb{'name'} is the (case-insensitive) name of the key. 
  Go ahead and print out the value for the property \Verb{'current dipole'}. 
  Notice that the value is returned as a vector: a three-element array of \(x\), \(y\), and \(z\) components. 
  Recall that, by convention, the principle bond axis of a molecule is defined as \(z\). In a simple diatomic, this is the \emph{only} bond, and so the entirety of the dipole should lie along the \(z\)-axis. 
  You should verify this by making sure the \(x\) and \(y\) components of the dipole array are zero. 
  If they are, you can take the following easy shortcut: rather than needing to calculate the total length of the vector (``taking the norm''),\sidenote{Easily done using the NumPy function \Verb{np.linalg.norm}.} you can just grab the \(z\) component as the full magnitude of the dipole vector. 
  In the rest of this exercise, we'll be using the energy and dipole values from this calculation. 

  \item The next step of this exercise will be to find the equilibrium bond distance of hydrogen fluoride (HF) from a series of calculations. 
  Create a loop to calculate the energy of the molecule at a range of  \(r\) values between \qtyrange{0.7}{1.3}{\bohr}.\sidenote{
    The easiest way to generate this list of values is to use the \pyinline{np.arange(start, stop, step)} function from NumPy. 
    This function takes (up to) three arguments: start, stop, and step. 
    If only a single argument is given, it acts just like the \pyinline{range()} function and generates integer values from 0 up to (but not including) the (stop) value given. 
    If two values are given, it goes from the start to stop in unit steps, and providing all three values goes from start to stop in the requested step size.}
  A good step size is \num{0.02} (this is the atomic unit of length, \(\unit{\bohr} \simeq \qty{52.9}{\pm}\)).
  Save the energy of each calculation as an element in a list, then plot the list of \(r\) values against the list of energies. 
  The plot should have a distinct minimum. 
  To extract the minimum value from the energy list, use the \pyinline{np.min()} and \pyinline{np.argmin()} functions from NumPy. 
  Both functions take a list as the input. 
  The first returns the minimum value of the list, while the second returns the index of the minimum value. 
  Use the index of the minimum to find the corresponding value in the list of \(r\) values. 
  Which bond length corresponds to the lowest energy? How does the bond length compare to the experimental bond length of \qty{0.917}{\angstrom}?

  \item To compare with experimental values, we compute the atomization energy (\( \increment H_\mtext{at} \)).
  In order to calculate \( \increment H_\mtext{at} \), we will also need the total energy of the isolated H and F atoms. Compute the total energies for the single atoms using the methods \Verb{HF} and \Verb{6-31G(d,p)} basis set. 
  You can look back to our work on the lone hydrogen atom in Problem~I to guide your work. 

  %\begin{tip}
  %\textbf{Note:}\newline
  %Atoms are highly symmetric systems, often with multiple degenerate solutions. In the case of fluorine, for example, the unpaired p-electron might sit in the $p_x$, $p_y$, or $p_z$ orbital. All three solutions are equivalent. If the calculation is started unbiased, it might converge to a superposition of these three cases, which is a saddle point on the potential energy surface and results in partial electron occupations. Although in DFT non-integer occupations are in principle allowed, one should be very suspicious when obtaining such a solution for non-metallic systems. Typically, solutions exist that are lower in energy. They can be found by breaking the inherent symmetry of the problem, for example by applying a small external field at the beginning of the SCF cycle.
  %\end{tip}
  
  %To break the inherent symmetry of an atom and ensure integer occupation, set the keyword \texttt{switch\_external\_pert 10 safe}. This means that for 10 iterations, a small external field in the z-direction is applied and then switched off. Usually, this is sufficient to perturb the SCF out of the symmetric solution and towards the correct electronic structure.
  
  Next, calculate the atomization energy (\( \increment H_\mtext{at} \)) of HF by subtracting the free-atom energies from the predicted total energy of HF (\emph{i.e.,} the minimum total energy found when varying bond distances).
  
  \begin{equation}
   \increment H_\mtext{at}= E^\mtext{HF}_\mtext{tot} - E^\mtext{H}_\mtext{atom} - E^\mtext{F}_\mtext{atom}
  \end{equation}
  
  How does this compare to the experimental value of \(\increment H_\mtext{at}=\qty{135.2}{\kcal\per\mol}\) (\qty{5.86}{\eV})? 
  
  \item Now, let us look at the dipole moment. 
  How does the dipole at the equilibrium distance compare with the experimental value of \qty{1.82}{\debye}?\sidenote[][-5\baselineskip]{
    Psi4 returns values in terms of atomic units. 
    The debye (\unit{\debye}) is the most commonly used dipole unit, and is one of the few remaining relics in use from the cgs system. 
    The SI unit of dipole is the coulomb-meter (\unit{\coulomb\meter}), but atomic dipoles are on the order of \qty{1e-30}{\coulomb\meter}, or a quectocoulomb-meter \unit{\quecto\coulomb\meter} (no, I'm not making that prefix up, it's one of the four newest SI prefixes), making it an unwieldy quantity. 
    The true \emph{atomic} unit of the dipole is \unit{\dipole}, or the product of the charge on one electron multiplied by the Bohr radius.
  } 
  Plot the dipole moment vs. the bond distance. You will find a (mostly) linear correspondence. 
  Do you expect this trend to continue at large distances?  Why or why not?
  
  \item Next, repeat the bond length determination using \Verb{PBE0} method and same basis set. 
  In addition, you need to compute energies for hydrogen (H) and fluorine (F) again using new method. 
  How does the optimal bond length, atomization energy and dipole moment change?
  In the lab report, prepare a plot of both dissociation curves, a plot of both dipole moments and a comparison of the two computed atomization energies (and the experimental value). 
  
\end{enumerate}

%\subsection*{Optional: Charge partition schemes.}
%Chemical reactivity and many physical properties are often explained in terms of atomic charges. However, atomic charges are not physical observables, since no unique operator exists to determine this quantity. They rather depend on the chosen charge partition scheme. The charge partition schemes that is probably most commonly used is Mulliken \cite{mulliken1955epa}. In FHI-aims, you can request it by specifying
%\texttt{output mulliken}  
%in \controlin{}.
%
%For the equilibrium structure of HF, compute the atomic charges with this scheme using \texttt{pbe0}. Use the charges to calculate the dipole moment $\mu$ in the point dipole approximation. In this approximation, for a two atom system, the dipole moment is defined as:
%\begin{equation}
%\mu = q \cdot \left|  \vec{r_H} - \vec{r_F} \right|
%\end{equation}
%where $q$ is the atomic partial charge and $\vec{r_H}$ and $\vec{r_F}$ are the atomic positions of the atoms. The absolute value of the difference $ \left|  \vec{r_H} - \vec{r_F} \right|$ is the distance between hydrogen and fluorine.  Compare the dipole moment to the one computed by FHI-aims. How do they compare?
%
%\section{Problem III: Molecular oxygen - a critical look \textcolor{red}{!}} \label{sec:problemIII}
%
%An important part of every calculation is to always look critically at the output and ensure that the result is reasonable. For some systems, defaults may not be adequate, or assumptions which commonly work well may prove to be wrong. A prime example is the the treatment of spin in systems with degenerate orbitals, such as O$_2$.
%
%\task{[\textbf{Educational Objectives:}] 
%  \begin{itemize}
%\item  Do not trust default settings blindly
%\item See the effect of an incorrect spin-treatment
%  \end{itemize}%
%}
%%
%\noindent
%
%\textbf{Tasks}
%\begin{enumerate}
%\item Set up a calculation for O$_2$ similar to the previous exercise, but leave out the \texttt{ spin} keyword in \controlin{}.  Look at the output file to find out which spin treatment is used by default.
%\item Look at the output file and find the occupation numbers of the Kohn-Sham orbitals.  Does this make sense here?
%\item Using \texttt{pbe} and \texttt{pbe0}, calculate the binding curve in the interval  $[0.8 , 1.6 ]$~\AA~ with a stepwidth of 0.1~\AA~ as before. Also, calculate the atomization energy ($\increment H_{at}$). 
%\item Repeat the calculation with \texttt{spin collinear} and \texttt{default\_initial\_moment 2.0} - is Hund's rule now fulfilled? Compare the results: Do both spin settings yield the same equilibrium bond length? Calculate the difference in the total energy at the equilibrium bond length. Which one is lower? How does it compare to the experimental value of 1.21 \AA? How does the atomization energy ($\increment H_{at}$) compare to the experimental value of 5.18 eV?
%\end{enumerate}
%
%Note: The oxygen atom is a notoriously difficult case to converge. If you have problems with it, try using the linear mixer for the SCF by including  the keyword \texttt{mixer linear} in \controlin{}. The linear mixer is guaranteed to converge, but usually requires much more iterations than the default (Pulay) mixing scheme.

%!TEX root = ./Intro_to_CC.tex

\section{Problem 3: Hydronium cation}
\label{sec:problemIII}
\subsection*{Planar hydronium cation}

This exercise covers how to perform geometry optimizations.
Specifically, we will relax the \ch{H3O^+} molecule starting from an initial planar guess for the geometry. 

\begin{enumerate}

  \item The planar \ch{H3O+} geometry has been provided in the file \Verb{geom_planar.xyz}.
  \begin{gaussinput}[title=Contents of \Verb{geom_planar.xyz}]
O    0.00    0.00   0.00
H    0.92   -0.53   0.00
H   -0.92   -0.53   0.00
H    0.00    1.06   0.00

  \end{gaussinput}
  
  \item Create an \Verb{input.com} file, using the template provided in the first problem. 
  Use the \Verb{HF} level of theory and the \Verb{6-31G(d,p)} basis set. 
  We want to relax the geometry and perform the vibrational analysis of the ion. 
  Therefore, replace the \Verb{sp} keyword (`single-point') from the template with \Verb{opt freq} (`optimization' and `frequency'). 
  After removing the information for the previous molecule, add the geometry of the cation at the end of the input file\sidenote{copy by hand or, in the terminal, type \bashinline{cat geom_planar.xyz >> input.com} to append the contents to then end of the file}.
  
  \item Run Gaussian.
  \begin{bashinput}
      g16 input.com &
  \end{bashinput} 
  
  \item Once the calculation is complete, visualize the results, we will use the notebook in JupyterLab. 
  Execute the first few cells of the section until you get an output that shows you the molecule. 
  This cell also outputs the total energy of the ion. 
  Note the command that outputs this value. 
  Also note that the molecule only shows bonds to two of the hydrogen atoms. 
  What does the fully relaxed structure look like? 
  Do you think that this is the structure of \ch{H3O+} in the gas phase? 
  % Save a picture of the ion.
  
  \item   We'll use two additional cells to define bonds between the oxygen and all three hydrogens and to reformat the vibrational information so we can visualize it. 
  The final cells in this section of the notebook will show the corrected structure and list the normal modes/vibrations with their IR intensity, sorted by wavenumber (\unit{\wn}). 
  Click through the spectrum to animate some of the vibrations. 
  You can select all of the vibrations by clicking the menu icon \( (\vdots) \) in the upper right corner of the output view window and using the dropdown menu for ``Normal Mode''. 
  You should see that one of the frequencies is negative -- select the normal mode to which it corresponds. 
  In the discussion section (at the end of the notebook), include this table and indicate what kind of molecular motion each vibration corresponds to. 

\end{enumerate}


\subsection*{Pyramidal hydronium cation}

Next, repeat the calculations for a pyramidal hydronium cation: 
\begin{gaussinput}[title=Contents of \Verb{geom_pyramidal.xyz}]
O    0.00   0.00   0.00 
H    0.92  -0.53  -0.66 
H   -0.92  -0.53  -0.66 
H    0.00   1.06  -0.66 

\end{gaussinput}
Visualize the results again using the steps from the previous section. 
You should see the \ch{H3O+} in a pyramidal conformation now.  
Note again the total energy of the ion and compare it with the planar structure in your discussion. 
Which conformation has lower energy? 
Next, view the vibrations table and spectrum. 
If calculations were done properly, all vibrations should have positive wavenumbers. 
Describe again in the discussion section the motion the vibrations correspond to. 


\subsection*{Potential-Energy Surface Scan}

In the final problem, we are going to inspect the potential-energy surface of the hydronium ion along its umbrella mode. In the planar cation problem, you have seen that the negative\sidenote{In fact, it is imaginary, \(i\) is dropped by convention} frequency corresponds to such an `umbrella' mode. 

\begin{itemize}
  \item In a terminal window, change to the \Verb{PES} directory. 
  You will an template input file already prepared. 
  If you open it, you should see that the \( xyz \)-cartesian coordinates have been replaced with a \( z \)-matrix. 
  The \( z \)-matrix allows precise control of the geometry within single calculations.

    \begin{gaussinput}[title=\(z\)-matrix in the PES input file]
O   	                       
X   1 a                   
H   1 a   2 HOX            
H   1 a   2 HOX   3  120.0 
H   1 a   2 HOX   3 -120.0 

a=OH 
HOX= 135. -1. 45 

    \end{gaussinput}
  
  
  \item The first column shows the bonding, second shows the angles between atoms and the third column specifies the dihedral angle. 
  X is a dummy (non-existent) atom that enables control of the umbrella motion. 
  \Cref{fig:internal_coords} explains the \( z \)-matrix graphically. 
  First, calculate the \ch{O-H} distance using the functions defined in the first cell of this section and the optimized pyramidal geometry from the previous section. Replace the \Verb{OH}\sidenote{On the line that says \Verb{a=OH}} in the input file with this value. 
  Next, run the calculations using Gaussian. 
  The calculations will perform a scan along the \ch{X-O-H} angle, performing a single point calculations every \ang{1} from \SIrange{135}{90}{\degree}. 
  When the calculations are finished, import the results to Jupyter and plot the resulting potential-energy surface. 
  Discuss these results in your report.  
  The PES for angles below \ang{90} is the mirror image. Use \unit{\kcal\per\mol} instead of \unit{\hartree} in the report. 
  
  \begin{marginfigure}[-12\baselineskip]
    \centering
      \includegraphics[width=0.75\textwidth]{pics/zmat.png}
    \caption{Definition of hydronium ion internal coordinates. 
      The calculations perform a scan along the \ch{X-O-H} coordinates for all three hydrogens from \SIrange{135}{90}{\degree}. 
      The \ang{120} dihedral angle indicates the relative position of hydrogen atoms.}
    \label{fig:internal_coords}
  \end{marginfigure}

  \item Localize the lowest-energy and transition structure along the PES and calculate the reaction barrier of the internal flip of the hydronium ion. 
  Whereas the energy of the pyramidal ion is comparable with the minimum on the PES, the energy of the optimized planar cation and the local maximum on the PES is different. 
  Why? Compare the geometries. 

  \item In the final step, rerun the calculations using the \texttt{MP2} method and compare your results. 
  You'll need to rerun the geometry optimization (for the pyramidal molecule) using the MP2 level of theory so you get the right bond length to run the PES scan of the umbrella mode. 
  Try plotting both PES scan curves in one plot to compare methods. 
  Look at the difference in final energies and the difference in the barrier energy for the two methods. 


\end{itemize}

\section*{Lab report}

For the lab report, please prepare following data.\sidenote{This is the bare minimum; there are couple of open questions in the text that you should try to assess.}
All of this can be done in the Discussion section of the Jupyter notebook. 
When you are finished, export the finished document as a PDF file  by clicking \menu{File>Export Notebook As…>Export Notebook to PDF} and email that PDF to me. 

\begin{enumerate}
  \item Plot the Total Energy vs. number of basis functions for a hydrogen atom for all methods used in Problem 1. 

  \item Plot the binding curves for HF using Hartree-Fock and PBE1PBE methods. Find the minimum distance and compute the atomization energy. 
  Plot the dipole moment as a function of the bond distance. 
  Plot both methods in the same image. You can do this with the \Verb{plt.plot()} method from the Matplotlib library. 
  If you made the distances into the index for your dataframe, you can call those values with \Verb{df.index.values}. 
  Multiple sets of \emph{x-y} data can be plotted with \Verb{plt.plot(x1, y1, x2, y2)}, where \Verb{x1}, ... are the various lists of \emph{x} and \emph{y} data. 

  \item Prepare tables listing molecular vibrations in a hydronium ion in planar and pyramidal geometries. 
  Make sure the wavenumbers are shown for all molecular vibrations. 

  \item Plot the PES along the HOX coordinate using HF and MP2 methods. 
  Use \unit{\kcal\per\mol} for the \( y \)-axis. 
  Compute the heigh of the barrier separating two pyramidal structures (the energy required to pass over the planar intermediate on the PES). 
  
\end{enumerate}



\clearpage

\appendix
%!TEX root = ./Intro_to_CC.tex 

\subsection{Appendix I: Bash and vi}
\label{app:bash}

Bash is a Unix shell and command language for the GNU Project and the default shell on Linux and OS X systems. We will use it to execute most programs and exercises. Below you find a list of the most import commands. Items in quotes indicate user-selected input (a directory/file name, a string of text, etc.). Bash furthermore offers a full programming language (often implemented via shell scripts) to automate tasks, e.g., via loops.
\marginnote[-4\baselineskip]{
A note on \emph{relative} and \emph{absolute} paths: Say you were giving directions to a location. You have two methods you can describe getting to the location:
\begin{itemize}
  \item Relative to where you stand, or
  \item Relative to a landmark.
\end{itemize}
Both descriptions get you to the same location, but the former only works from your current location (``take a left, then a right, go through two lights then take another right'' wouldn't necessarily work from the next town over, but works from where you stand).

In file systems, if you have \Verb{/home/user/documents/data.txt}, that's an absolute path (the starting \Verb{/} indicates the root of the drive). If you have \Verb{documents/data.txt}, it will only work so long as you're starting from \Verb{/home/user}. If you start in \Verb{/home/user/documents} you would need a \Verb{../} to get there correctly using the relative path. 

However, no matter where you are on the hard drive, \Verb{/home/user/documents/data.txt} is a definitive way to get to that file.} 

\begin{itemize}
  \item \textbf{Basic navigation:}
  \begin{description}[
    font=\ttfamily\upshape,
    align=left,
    style=nextline,
    itemindent=*]
    \item[ls] list all files and folders 
    \item[ls "dir-name"] list files in the directory.
    \item[ls -lh] Detailed (\textbf{l}ong) list, \textbf{h}uman readable
    \item[ls -l mypics/*.jpg] list only the jpeg files in the ``mypics'' directory
    \item[cd "folderName"] change directory
    \item[cd ..] go up one folder, tip: string together multiple folders \Verb{../../..}
  \end{description}
  
  \item \textbf{Basic file operations:}
  \begin{description}[
    font=\ttfamily\upshape,
    align=left,
    style=nextline,
    itemindent=*]
    \item[cat "file"] show all contents of a file
    \item[head "file"] show the top 10 lines of a file
    \item[tail -n5 "file"] show the last 5 lines of a file

    \item[mkdir "dir-name"] creates a new directory entitled ``dir-name'' (called folder in Windows and macOS) 
    \item[cp "file1" "file2"] - copy ``file1'' to ``file2'' 
    \item[cp image.jpg mypics/] - copy the file ``image.jpg'' to the ``mypics'' directory 
    \item[cp *.txt stuff/] copy all of files ending with ``.txt'' to the directory ``stuff''

    \item[mv "file1" "file2"] move (rename) ``file1'' to ``file2''
    \item[mv "file1" "dir-name>/"] move ``file1'' to directory ``dir-name'' 
    \item[mv "folderName/" ..] move directory up one level

    \item[rm "file1"] delete ``file1'' 
    \item[rm -r "junk\_stuff"] delete directory ``junk\_stuff'' and all files contained in it
  \end{description}
  
  \item \textbf{Extract, sort and filter data:}
  \begin{description}[
    font=\ttfamily\upshape,
    align=left,
    style=nextline,
    itemindent=*]
    \item[grep "someText" "file1"] search for the text ``someText'' in ``file1''.\sidenote{If your input has spaces, enclosing the input in double quotes (\Verb{"}) will preserve the spaces, e.g.,\Verb{"}Some quoted text\Verb{"}.} The \Verb{-i} flag tells grep to \textbf{i}gnore letter case (upper/lower).
    \item[grep -r "text" "folderName/"] return a list of lines in files contained in ``folderName'' with occurrences of ``text''
  \end{description}
  
  \item \textbf{Flow redirection and chain commands -- redirecting results of commands:}
  \begin{description}[
    font=\ttfamily\upshape,
    align=left,
    style=nextline,
    itemindent=*]
    \item[>] at the end of a command to redirect the result to a file (overwrites the contents of the file)
    \item[>>] at the end of a command to \textbf{append} the result to the end of a file
    \item[|] at the end of a command to send the output to another command 
    \item[\&] run the command in the background 
  \end{description}
  
  \item \textbf{Basic control:}
  \begin{description}[
    font=\upshape,
    align=left,
    style=nextline,
    itemindent=*]
    \item[\keys{\tab}] auto completion of file or command
    \item[\keys{\arrowkeyup}/\keys{\arrowkeydown}] See previous/next commands
    \item[\keys{\ctrl+R}] reverse search history
    \item[\keys{\ctrl+L}] clear the terminal
    \item[\Verb{!!}] repeat last command 
  \end{description}
\end{itemize}

\textbf{vi} is a terminal-based file edit program. By typing \Verb{vi} you open the program and create a new file that can be save later. By typing \Verb{vi "fileName"}, you open ``fileName'' to edit it. If ``fileName'' doesn't exist, you will create the new file and edit it immediately with this program. 

The editor, despite its simplicity in appearance, is a very powerful terminal-based tool with numerous key-bindings. Therefore, be careful what you press. In order to start editing the file, you first need to press \keys{I} (`\textbf{i}nsert') and then you can start typing. In order to save the file, press the \keys{Esc} key to exit the editing mode, then type \Verb{:} (\keys{\shift+;}) to enter the command mode in the bottom of the editor and type \Verb{wq} (for `\textbf{w}rite \textbf{q}uit'). Confirm with \keys{\return}. If you want to quit without saving the file, type \Verb{q!} in command mode. 







\vfill

\nocite{*}
\printbibliography[category=cited]% default title for `article` class: "References"

\DeclareFieldFormat{labelnumberwidth}{\textbullet}
\printbibliography[%
  title={Further Reading},%
  resetnumbers,%
  omitnumbers,%
  notcategory=cited,%
  ]

\end{document}
